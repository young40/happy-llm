% Options for packages loaded elsewhere
\PassOptionsToPackage{unicode}{hyperref}
\PassOptionsToPackage{hyphens}{url}
\documentclass[
]{article}
\usepackage{xcolor}
\usepackage{amsmath,amssymb}
\setcounter{secnumdepth}{5}
\usepackage{iftex}
\ifPDFTeX
  \usepackage[T1]{fontenc}
  \usepackage[utf8]{inputenc}
  \usepackage{textcomp} % provide euro and other symbols
\else % if luatex or xetex
  \usepackage{unicode-math} % this also loads fontspec
  \defaultfontfeatures{Scale=MatchLowercase}
  \defaultfontfeatures[\rmfamily]{Ligatures=TeX,Scale=1}
\fi
\usepackage{lmodern}
\ifPDFTeX\else
  % xetex/luatex font selection
\fi
% Use upquote if available, for straight quotes in verbatim environments
\IfFileExists{upquote.sty}{\usepackage{upquote}}{}
\IfFileExists{microtype.sty}{% use microtype if available
  \usepackage[]{microtype}
  \UseMicrotypeSet[protrusion]{basicmath} % disable protrusion for tt fonts
}{}
\makeatletter
\@ifundefined{KOMAClassName}{% if non-KOMA class
  \IfFileExists{parskip.sty}{%
    \usepackage{parskip}
  }{% else
    \setlength{\parindent}{0pt}
    \setlength{\parskip}{6pt plus 2pt minus 1pt}}
}{% if KOMA class
  \KOMAoptions{parskip=half}}
\makeatother
\setlength{\emergencystretch}{3em} % prevent overfull lines
\providecommand{\tightlist}{%
  \setlength{\itemsep}{0pt}\setlength{\parskip}{0pt}}
\usepackage{bookmark}
\IfFileExists{xurl.sty}{\usepackage{xurl}}{} % add URL line breaks if available
\urlstyle{same}
\hypersetup{
  hidelinks,
  pdfcreator={LaTeX via pandoc}}

\author{}
\date{}

\begin{document}

{
\setcounter{tocdepth}{3}
\tableofcontents
}
🚀 Happy-LLM 扩展内容

社区驱动的大语言模型学习资源

\begin{center}\rule{0.5\linewidth}{0.5pt}\end{center}

\subsection{📖 为什么会有 Extra
Chapter?}\label{ux4e3aux4ec0ux4e48ux4f1aux6709-extra-chapter}

  在 Happy-LLM
主教程的基础上,我们发现社区中有许多优秀的学习者和实践者,他们在学习和使用大语言模型的过程中积累了宝贵的经验、独到的见解和实用的技巧。这些内容虽然不属于主教程的核心知识体系,但对于深入理解和应用大语言模型具有重要价值。

\textbf{Extra Chapter 的设立目的:}

\begin{itemize}
\tightlist
\item
  🌟 \textbf{汇聚智慧}:收集社区成员的优秀学习笔记、实践经验和技术博客
\item
  🔄 \textbf{持续更新}:保持内容的时效性,跟上大语言模型领域的快速发展
\item
  🤝 \textbf{促进交流}:为社区成员提供分享和交流的平台
\item
  📚 \textbf{补充完善}:对主教程内容进行有益的补充和扩展
\item
  💡 \textbf{启发思考}:通过不同视角和实践案例,启发读者的深度思考
\end{itemize}

\textbf{Extra Chapter 包含的内容类型:}

\begin{itemize}
\tightlist
\item
  📝 \textbf{学习笔记}:深度学习心得和知识总结
\item
  🛠️ \textbf{实践案例}:真实项目中的应用经验
\item
  🔬 \textbf{技术探索}:前沿技术的研究和实验
\item
  💭 \textbf{思考感悟}:对大语言模型发展的独特见解
\item
  🎯 \textbf{专题研究}:特定领域或问题的深入分析
\end{itemize}

\begin{center}\rule{0.5\linewidth}{0.5pt}\end{center}

\subsection{📋 PR 贡献规范}\label{pr-ux8d21ux732eux89c4ux8303}

  我们热烈欢迎社区成员为 Extra Chapter
贡献优质内容!为了保证内容质量和项目的整体性,请遵循以下规范:

\subsubsection{🗂️
目录结构规范}\label{ux76eeux5f55ux7ed3ux6784ux89c4ux8303}

每个贡献的内容应按照以下目录结构组织:

\begin{verbatim}
Extra-Chapter/
├── your-topic-name/                    # 你的主题文件夹
│   ├── readme.md                       # 主要内容文件(必需)
│   ├── images/                         # 图片资源文件夹(可选)
│   │   ├── figure1.png
│   │   └── figure2.jpg
│   ├── code/                           # 代码文件夹(可选)
│   │   ├── example.py
│   │   └── requirements.txt
│   ├── data/                           # 数据文件夹(可选)
│   │   └── sample_data.json
│   └── notebook.ipynb                  # Jupyter Notebook(如涉及代码必选)
└── Readme.md                           # 本文件
\end{verbatim}

\subsubsection{📝
文件命名规范}\label{ux6587ux4ef6ux547dux540dux89c4ux8303}

\begin{enumerate}
\def\labelenumi{\arabic{enumi}.}
\tightlist
\item
  \textbf{主题文件夹命名}:

  \begin{itemize}
  \tightlist
  \item
    使用小写字母和连字符
  \item
    名称要简洁明了,能够概括主题内容
  \item
    例如:\texttt{why-fine-tune-small-large-language-models}、\texttt{rag-optimization-techniques}
  \end{itemize}
\item
  \textbf{主要内容文件}:

  \begin{itemize}
  \tightlist
  \item
    必须命名为 \texttt{readme.md}
  \item
    使用 Markdown 格式编写
  \end{itemize}
\item
  \textbf{图片文件}:

  \begin{itemize}
  \tightlist
  \item
    统一放在 \texttt{images/} 文件夹下
  \item
    使用描述性的文件名
  \item
    支持格式:\texttt{.png}、\texttt{.jpg}、\texttt{.jpeg}、\texttt{.gif}、\texttt{.svg}
  \end{itemize}
\item
  \textbf{代码文件}:

  \begin{itemize}
  \tightlist
  \item
    如涉及代码,请尽量提供可直接运行的 Jupyter Notebook 文件
  \item
    统一放在 \texttt{code/} 文件夹下
  \item
    使用标准的文件扩展名
  \item
    如有依赖,请提供 \texttt{requirements.txt}
  \item
    如有 Jupyter Notebook 文件,请放在主文件夹下
  \end{itemize}
\end{enumerate}

\subsubsection{✍️
内容质量要求}\label{ux5185ux5bb9ux8d28ux91cfux8981ux6c42}

\begin{enumerate}
\def\labelenumi{\arabic{enumi}.}
\tightlist
\item
  \textbf{原创性}:

  \begin{itemize}
  \tightlist
  \item
    内容必须是原创或经过授权的
  \item
    如引用他人内容,请注明出处
  \end{itemize}
\item
  \textbf{技术准确性}:

  \begin{itemize}
  \tightlist
  \item
    确保技术内容的准确性
  \item
    代码示例应能正常运行
  \item
    提供必要的环境说明
  \end{itemize}
\item
  \textbf{结构清晰}:

  \begin{itemize}
  \tightlist
  \item
    使用清晰的标题层次
  \item
    合理使用列表、表格等格式
  \item
    重要内容使用适当的强调
  \end{itemize}
\item
  \textbf{语言规范}:

  \begin{itemize}
  \tightlist
  \item
    使用规范的中文表达
  \item
    技术术语使用准确
  \item
    避免错别字和语法错误
  \end{itemize}
\end{enumerate}

\subsubsection{PR commit messgae
内容}\label{pr-commit-messgae-ux5185ux5bb9}

请在 PR commit message 中 包含以下内容:

\begin{itemize}
\tightlist
\item
  新增的主题文件夹名称
\item
  贡献内容的概述
\item
  贡献内容的详细描述
\item
  你的 Github 个人主页链接,及你的个人介绍
\item
  个人 title 及工作经历 or 学校 or 研究方向
\end{itemize}

如以下所示:

\begin{verbatim}
Extra Chapter: 你的主题名称

详细描述你的贡献内容,包括新增的主题文件夹、文件内容和目录结构。

- 新增的主题文件夹名称:your-topic-name
- 贡献内容的概述:详细介绍你的贡献内容
- 贡献内容的详细描述:详细描述你的贡献内容,包括新增的主题文件夹、文件内容和目录结构。
- 你的 Github 个人主页链接及个人介绍:[你的个人主页链接](https://example.com),介绍你的研究方向、技术专长等。
- 个人 title 及工作经历 or 学校 or 研究方向:内容贡献者-xxxx学校,研究方向为自然语言处理。
\end{verbatim}

\end{document}
