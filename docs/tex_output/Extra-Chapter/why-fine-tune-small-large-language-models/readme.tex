% Options for packages loaded elsewhere
\PassOptionsToPackage{unicode}{hyperref}
\PassOptionsToPackage{hyphens}{url}
\documentclass[
]{article}
\usepackage{xcolor}
\usepackage{amsmath,amssymb}
\setcounter{secnumdepth}{5}
\usepackage{iftex}
\ifPDFTeX
  \usepackage[T1]{fontenc}
  \usepackage[utf8]{inputenc}
  \usepackage{textcomp} % provide euro and other symbols
\else % if luatex or xetex
  \usepackage{unicode-math} % this also loads fontspec
  \defaultfontfeatures{Scale=MatchLowercase}
  \defaultfontfeatures[\rmfamily]{Ligatures=TeX,Scale=1}
\fi
\usepackage{lmodern}
\ifPDFTeX\else
  % xetex/luatex font selection
\fi
% Use upquote if available, for straight quotes in verbatim environments
\IfFileExists{upquote.sty}{\usepackage{upquote}}{}
\IfFileExists{microtype.sty}{% use microtype if available
  \usepackage[]{microtype}
  \UseMicrotypeSet[protrusion]{basicmath} % disable protrusion for tt fonts
}{}
\makeatletter
\@ifundefined{KOMAClassName}{% if non-KOMA class
  \IfFileExists{parskip.sty}{%
    \usepackage{parskip}
  }{% else
    \setlength{\parindent}{0pt}
    \setlength{\parskip}{6pt plus 2pt minus 1pt}}
}{% if KOMA class
  \KOMAoptions{parskip=half}}
\makeatother
\usepackage{color}
\usepackage{fancyvrb}
\newcommand{\VerbBar}{|}
\newcommand{\VERB}{\Verb[commandchars=\\\{\}]}
\DefineVerbatimEnvironment{Highlighting}{Verbatim}{commandchars=\\\{\}}
% Add ',fontsize=\small' for more characters per line
\newenvironment{Shaded}{}{}
\newcommand{\AlertTok}[1]{\textcolor[rgb]{1.00,0.00,0.00}{\textbf{#1}}}
\newcommand{\AnnotationTok}[1]{\textcolor[rgb]{0.38,0.63,0.69}{\textbf{\textit{#1}}}}
\newcommand{\AttributeTok}[1]{\textcolor[rgb]{0.49,0.56,0.16}{#1}}
\newcommand{\BaseNTok}[1]{\textcolor[rgb]{0.25,0.63,0.44}{#1}}
\newcommand{\BuiltInTok}[1]{\textcolor[rgb]{0.00,0.50,0.00}{#1}}
\newcommand{\CharTok}[1]{\textcolor[rgb]{0.25,0.44,0.63}{#1}}
\newcommand{\CommentTok}[1]{\textcolor[rgb]{0.38,0.63,0.69}{\textit{#1}}}
\newcommand{\CommentVarTok}[1]{\textcolor[rgb]{0.38,0.63,0.69}{\textbf{\textit{#1}}}}
\newcommand{\ConstantTok}[1]{\textcolor[rgb]{0.53,0.00,0.00}{#1}}
\newcommand{\ControlFlowTok}[1]{\textcolor[rgb]{0.00,0.44,0.13}{\textbf{#1}}}
\newcommand{\DataTypeTok}[1]{\textcolor[rgb]{0.56,0.13,0.00}{#1}}
\newcommand{\DecValTok}[1]{\textcolor[rgb]{0.25,0.63,0.44}{#1}}
\newcommand{\DocumentationTok}[1]{\textcolor[rgb]{0.73,0.13,0.13}{\textit{#1}}}
\newcommand{\ErrorTok}[1]{\textcolor[rgb]{1.00,0.00,0.00}{\textbf{#1}}}
\newcommand{\ExtensionTok}[1]{#1}
\newcommand{\FloatTok}[1]{\textcolor[rgb]{0.25,0.63,0.44}{#1}}
\newcommand{\FunctionTok}[1]{\textcolor[rgb]{0.02,0.16,0.49}{#1}}
\newcommand{\ImportTok}[1]{\textcolor[rgb]{0.00,0.50,0.00}{\textbf{#1}}}
\newcommand{\InformationTok}[1]{\textcolor[rgb]{0.38,0.63,0.69}{\textbf{\textit{#1}}}}
\newcommand{\KeywordTok}[1]{\textcolor[rgb]{0.00,0.44,0.13}{\textbf{#1}}}
\newcommand{\NormalTok}[1]{#1}
\newcommand{\OperatorTok}[1]{\textcolor[rgb]{0.40,0.40,0.40}{#1}}
\newcommand{\OtherTok}[1]{\textcolor[rgb]{0.00,0.44,0.13}{#1}}
\newcommand{\PreprocessorTok}[1]{\textcolor[rgb]{0.74,0.48,0.00}{#1}}
\newcommand{\RegionMarkerTok}[1]{#1}
\newcommand{\SpecialCharTok}[1]{\textcolor[rgb]{0.25,0.44,0.63}{#1}}
\newcommand{\SpecialStringTok}[1]{\textcolor[rgb]{0.73,0.40,0.53}{#1}}
\newcommand{\StringTok}[1]{\textcolor[rgb]{0.25,0.44,0.63}{#1}}
\newcommand{\VariableTok}[1]{\textcolor[rgb]{0.10,0.09,0.49}{#1}}
\newcommand{\VerbatimStringTok}[1]{\textcolor[rgb]{0.25,0.44,0.63}{#1}}
\newcommand{\WarningTok}[1]{\textcolor[rgb]{0.38,0.63,0.69}{\textbf{\textit{#1}}}}
\setlength{\emergencystretch}{3em} % prevent overfull lines
\providecommand{\tightlist}{%
  \setlength{\itemsep}{0pt}\setlength{\parskip}{0pt}}
\usepackage{bookmark}
\IfFileExists{xurl.sty}{\usepackage{xurl}}{} % add URL line breaks if available
\urlstyle{same}
\hypersetup{
  hidelinks,
  pdfcreator={LaTeX via pandoc}}

\author{}
\date{}

\begin{document}

{
\setcounter{tocdepth}{3}
\tableofcontents
}
\section{大模型都这么厉害了,微调0.6B的小模型有什么意义?}\label{ux5927ux6a21ux578bux90fdux8fd9ux4e48ux5389ux5bb3ux4e86ux5faeux8c030.6bux7684ux5c0fux6a21ux578bux6709ux4ec0ux4e48ux610fux4e49}

大家在日常使用Deepseek-R1或者是阿里新发布的Qwen3模型,他们的模型都是能力很强,所提供的API服也都可以满足大家的日常或者是公司开发所需。但大家也可以想一个简单的问题几个简单的问题,如下:

\begin{enumerate}
\def\labelenumi{\arabic{enumi}.}
\tightlist
\item
  公司的数据是够敏感,是否需要保密?
\item
  日常使用大模型的任务是否很困难,对推理链是否刚需?
\item
  任务调用的大模型API并发量是多少?每日资金消耗有多少?
\end{enumerate}

对于问题1,如果公司数据敏感,那我建议不要调用供应商提供的大模型API。就算供应商保证不会拿你们数据做训练,但你们的数据还是泄漏了(会有不必要的风险),建议本地部署大模型。

对于问题2,如果使用大模型的场景问题很困难并且刚需推理链,那可以使用供应商的API,这样可以保证推理链的上下文不会爆显存。如果问题很简单,没有刚需推理链,那建议本地部署小模型即可。

对于问题3,如果任务很简单,且调用的大模型API并发量很高,那我建议微调一个特定任务的小模型,本地部署。这样可以满足高并发,并且可以减少资金消耗。(本地部署,默认硬件环境单卡4090)

看到这里,想必大家已经思考完了以上三个问题,心中有了答案。那我给出一个小小的案例。

\subsection{微调模型的需求性}\label{ux5faeux8c03ux6a21ux578bux7684ux9700ux6c42ux6027}

假如你的公司有一个从投诉的文本中抽取用户信息的任务。比如,你需要从以下文本中抽取用户姓名、住址、邮箱、投诉的问题等等。

\begin{quote}
这只是一个小小的案例,数据也是我用大模型批量制造的。真正的投诉数据不会这么``干净、整洁''。
\end{quote}

INPUT:

\begin{Shaded}
\begin{Highlighting}[]
\NormalTok{龙琳,宁夏回族自治区璐市城东林街g座 955491,邮箱 nafan@example.com。小区垃圾堆积成山,晚上噪音扰人清梦,停车难上加难,简直无法忍受!}
\end{Highlighting}
\end{Shaded}

OUTPUT:

\begin{Shaded}
\begin{Highlighting}[]
\FunctionTok{\{}
    \DataTypeTok{"name"}\FunctionTok{:} \StringTok{"龙琳"}\FunctionTok{,}
    \DataTypeTok{"address"}\FunctionTok{:} \StringTok{"宁夏回族自治区璐市城东林街g座 955491"}\FunctionTok{,}
    \DataTypeTok{"email"}\FunctionTok{:} \StringTok{"nafan@example.com"}\FunctionTok{,}
    \DataTypeTok{"question"}\FunctionTok{:} \StringTok{"小区垃圾堆积成山,晚上噪音扰人清梦,停车难上加难,简直无法忍受!"}
\FunctionTok{\}}
\end{Highlighting}
\end{Shaded}

那你当然可以调用
Deepseek最强大的模型R1,也可以调用阿里最新发布最强大的模型
Qwen3-235B-A22B等等,这些模型的信息抽取效果也很非常的棒。

但有个问题,如果你有几百万条这样的数据要处理,全部调用最新的,最好的大模型可能需要消耗几万块钱。并且,如果这些投诉数据,比如电信投诉数据,电网投诉数据,这些数据是敏感的不可以直接放到外网的。

所以,综合数据敏感,和资金消耗。最好的选择就是微调一个小模型(如Qwen3-0.6B),既可以保证高并发,可以保证数据不泄漏,保证模型抽取的效果,还可以省钱!!!

那下面,用一个小案例带大家实操一下,微调Qwen3-0.6B小模型完成文本信息抽取任务。

\subsection{配置环境
下载数据}\label{ux914dux7f6eux73afux5883-ux4e0bux8f7dux6570ux636e}

\begin{quote}
Colab
文件地址:https://colab.research.google.com/drive/18ByY11KVhIy6zWx1uKUjSzqeHTme-TtU?usp=drive\_link
\end{quote}

\begin{Shaded}
\begin{Highlighting}[]
\OperatorTok{!}\NormalTok{pip install datasets swanlab }\OperatorTok{{-}}\NormalTok{q}
\end{Highlighting}
\end{Shaded}

\begin{Shaded}
\begin{Highlighting}[]
\OperatorTok{!}\NormalTok{wget }\OperatorTok{{-}{-}}\NormalTok{no}\OperatorTok{{-}}\NormalTok{check}\OperatorTok{{-}}\NormalTok{certificate }\StringTok{\textquotesingle{}https://docs.google.com/uc?export=download\&id=1a0sf5C209CLW5824TJkUM4olMy0zZWpg\textquotesingle{}} \OperatorTok{{-}}\NormalTok{O fake\_sft.json}
\end{Highlighting}
\end{Shaded}

\subsection{处理数据}\label{ux5904ux7406ux6570ux636e}

\begin{Shaded}
\begin{Highlighting}[]
\ImportTok{from}\NormalTok{ datasets }\ImportTok{import}\NormalTok{ Dataset}
\ImportTok{import}\NormalTok{ pandas }\ImportTok{as}\NormalTok{ pd}
\ImportTok{from}\NormalTok{ transformers }\ImportTok{import}\NormalTok{ AutoTokenizer, AutoModelForCausalLM, DataCollatorForSeq2Seq, TrainingArguments, Trainer, GenerationConfig}
\ImportTok{from}\NormalTok{ peft }\ImportTok{import}\NormalTok{ LoraConfig, TaskType, get\_peft\_model}
\ImportTok{import}\NormalTok{ torch}
\end{Highlighting}
\end{Shaded}

\begin{Shaded}
\begin{Highlighting}[]
\CommentTok{\# 将JSON文件转换为CSV文件}
\NormalTok{df }\OperatorTok{=}\NormalTok{ pd.read\_json(}\StringTok{\textquotesingle{}fake\_sft.json\textquotesingle{}}\NormalTok{)}
\NormalTok{ds }\OperatorTok{=}\NormalTok{ Dataset.from\_pandas(df)}
\NormalTok{ds[:}\DecValTok{3}\NormalTok{]}
\end{Highlighting}
\end{Shaded}

\begin{Shaded}
\begin{Highlighting}[]
\NormalTok{model\_id }\OperatorTok{=} \StringTok{"Qwen/Qwen3{-}0.6B"}
\end{Highlighting}
\end{Shaded}

\begin{Shaded}
\begin{Highlighting}[]
\NormalTok{tokenizer }\OperatorTok{=}\NormalTok{ AutoTokenizer.from\_pretrained(model\_id, use\_fast}\OperatorTok{=}\VariableTok{False}\NormalTok{)}
\NormalTok{tokenizer}
\end{Highlighting}
\end{Shaded}

对大语言模型进行
\texttt{supervised-finetuning}(\texttt{sft},有监督微调)的数据格式如下:

\begin{Shaded}
\begin{Highlighting}[]
\FunctionTok{\{}
  \DataTypeTok{"instruction"}\FunctionTok{:} \StringTok{"回答以下用户问题,仅输出答案。"}\FunctionTok{,}
  \DataTypeTok{"input"}\FunctionTok{:} \StringTok{"1+1等于几?"}\FunctionTok{,}
  \DataTypeTok{"output"}\FunctionTok{:} \StringTok{"2"}
\FunctionTok{\}}
\end{Highlighting}
\end{Shaded}

其中,\texttt{instruction}
是用户指令,告知模型其需要完成的任务;\texttt{input}
是用户输入,是完成用户指令所必须的输入内容;\texttt{output}
是模型应该给出的输出。

有监督微调的目标是让模型具备理解并遵循用户指令的能力。因此,在构建数据集时,我们应针对我们的目标任务,针对性构建数据。比如,如果我们的目标是通过大量人物的对话数据微调得到一个能够
role-play 甄嬛对话风格的模型,因此在该场景下的数据示例如下:

\begin{Shaded}
\begin{Highlighting}[]
\FunctionTok{\{}
  \DataTypeTok{"instruction"}\FunctionTok{:} \StringTok{"你父亲是谁?"}\FunctionTok{,}
  \DataTypeTok{"input"}\FunctionTok{:} \StringTok{""}\FunctionTok{,}
  \DataTypeTok{"output"}\FunctionTok{:} \StringTok{"家父是大理寺少卿甄远道。"}
\FunctionTok{\}}
\end{Highlighting}
\end{Shaded}

\texttt{Qwen3} 采用的 \texttt{Chat\ Template}格式如下:

由于 \texttt{Qwen3} 是混合推理模型,因此可以手动选择开启思考模式

不开启 \texttt{thinking\ mode}

\begin{Shaded}
\begin{Highlighting}[]
\NormalTok{messages }\OperatorTok{=}\NormalTok{ [}
\NormalTok{    \{}\StringTok{"role"}\NormalTok{: }\StringTok{"system"}\NormalTok{, }\StringTok{"content"}\NormalTok{: }\StringTok{"You are a helpful AI"}\NormalTok{\},}
\NormalTok{    \{}\StringTok{"role"}\NormalTok{: }\StringTok{"user"}\NormalTok{, }\StringTok{"content"}\NormalTok{: }\StringTok{"How are you?"}\NormalTok{\},}
\NormalTok{    \{}\StringTok{"role"}\NormalTok{: }\StringTok{"assistant"}\NormalTok{, }\StringTok{"content"}\NormalTok{: }\StringTok{"I\textquotesingle{}m fine, think you. and you?"}\NormalTok{\},}
\NormalTok{]}

\NormalTok{text }\OperatorTok{=}\NormalTok{ tokenizer.apply\_chat\_template(}
\NormalTok{    messages,}
\NormalTok{    tokenize}\OperatorTok{=}\VariableTok{False}\NormalTok{,}
\NormalTok{    add\_generation\_prompt}\OperatorTok{=}\VariableTok{True}\NormalTok{,}
\NormalTok{    enable\_thinking}\OperatorTok{=}\VariableTok{False}
\NormalTok{)}
\BuiltInTok{print}\NormalTok{(text)}
\end{Highlighting}
\end{Shaded}

\begin{verbatim}
<|im_start|>system
You are a helpful AI<|im_end|>
<|im_start|>user
How are you?<|im_end|>
<|im_start|>assistant
<think>

</think>

I'm fine, think you. and you?<|im_end|>
<|im_start|>assistant
<think>

</think>
\end{verbatim}

\texttt{LoRA}(\texttt{Low-Rank\ Adaptation})训练的数据是需要经过格式化、编码之后再输入给模型进行训练的,我们需要先将输入文本编码为
\texttt{input\_ids},将输出文本编码为
\texttt{labels},编码之后的结果是向量。我们首先定义一个预处理函数,这个函数用于对每一个样本,同时编码其输入、输出文本并返回一个编码后的字典:

\begin{Shaded}
\begin{Highlighting}[]
\KeywordTok{def}\NormalTok{ process\_func(example):}
\NormalTok{    MAX\_LENGTH }\OperatorTok{=} \DecValTok{1024} \CommentTok{\# 设置最大序列长度为1024个token}
\NormalTok{    input\_ids, attention\_mask, labels }\OperatorTok{=}\NormalTok{ [], [], [] }\CommentTok{\# 初始化返回值}
    \CommentTok{\# 适配chat\_template}
\NormalTok{    instruction }\OperatorTok{=}\NormalTok{ tokenizer(}
        \SpecialStringTok{f"\textless{}s\textgreater{}\textless{}|im\_start|\textgreater{}system}\CharTok{\textbackslash{}n}\SpecialCharTok{\{}\NormalTok{example[}\StringTok{\textquotesingle{}system\textquotesingle{}}\NormalTok{]}\SpecialCharTok{\}}\SpecialStringTok{\textless{}|im\_end|\textgreater{}}\CharTok{\textbackslash{}n}\SpecialStringTok{"}
        \SpecialStringTok{f"\textless{}|im\_start|\textgreater{}user}\CharTok{\textbackslash{}n}\SpecialCharTok{\{}\NormalTok{example[}\StringTok{\textquotesingle{}instruction\textquotesingle{}}\NormalTok{] }\OperatorTok{+}\NormalTok{ example[}\StringTok{\textquotesingle{}input\textquotesingle{}}\NormalTok{]}\SpecialCharTok{\}}\SpecialStringTok{\textless{}|im\_end|\textgreater{}}\CharTok{\textbackslash{}n}\SpecialStringTok{"}
        \SpecialStringTok{f"\textless{}|im\_start|\textgreater{}assistant}\CharTok{\textbackslash{}n}\SpecialStringTok{\textless{}think\textgreater{}}\CharTok{\textbackslash{}n\textbackslash{}n}\SpecialStringTok{\textless{}/think\textgreater{}}\CharTok{\textbackslash{}n\textbackslash{}n}\SpecialStringTok{"}\NormalTok{,}
\NormalTok{        add\_special\_tokens}\OperatorTok{=}\VariableTok{False}
\NormalTok{    )}
\NormalTok{    response }\OperatorTok{=}\NormalTok{ tokenizer(}\SpecialStringTok{f"}\SpecialCharTok{\{}\NormalTok{example[}\StringTok{\textquotesingle{}output\textquotesingle{}}\NormalTok{]}\SpecialCharTok{\}}\SpecialStringTok{"}\NormalTok{, add\_special\_tokens}\OperatorTok{=}\VariableTok{False}\NormalTok{)}
    \CommentTok{\# 将instructio部分和response部分的input\_ids拼接,并在末尾添加eos token作为标记结束的token}
\NormalTok{    input\_ids }\OperatorTok{=}\NormalTok{ instruction[}\StringTok{"input\_ids"}\NormalTok{] }\OperatorTok{+}\NormalTok{ response[}\StringTok{"input\_ids"}\NormalTok{] }\OperatorTok{+}\NormalTok{ [tokenizer.pad\_token\_id]}
    \CommentTok{\# 注意力掩码,表示模型需要关注的位置}
\NormalTok{    attention\_mask }\OperatorTok{=}\NormalTok{ instruction[}\StringTok{"attention\_mask"}\NormalTok{] }\OperatorTok{+}\NormalTok{ response[}\StringTok{"attention\_mask"}\NormalTok{] }\OperatorTok{+}\NormalTok{ [}\DecValTok{1}\NormalTok{]}
    \CommentTok{\# 对于instruction,使用{-}100表示这些位置不计算loss(即模型不需要预测这部分)}
\NormalTok{    labels }\OperatorTok{=}\NormalTok{ [}\OperatorTok{{-}}\DecValTok{100}\NormalTok{] }\OperatorTok{*} \BuiltInTok{len}\NormalTok{(instruction[}\StringTok{"input\_ids"}\NormalTok{]) }\OperatorTok{+}\NormalTok{ response[}\StringTok{"input\_ids"}\NormalTok{] }\OperatorTok{+}\NormalTok{ [tokenizer.pad\_token\_id]}
    \ControlFlowTok{if} \BuiltInTok{len}\NormalTok{(input\_ids) }\OperatorTok{\textgreater{}}\NormalTok{ MAX\_LENGTH:  }\CommentTok{\# 超出最大序列长度截断}
\NormalTok{        input\_ids }\OperatorTok{=}\NormalTok{ input\_ids[:MAX\_LENGTH]}
\NormalTok{        attention\_mask }\OperatorTok{=}\NormalTok{ attention\_mask[:MAX\_LENGTH]}
\NormalTok{        labels }\OperatorTok{=}\NormalTok{ labels[:MAX\_LENGTH]}
    \ControlFlowTok{return}\NormalTok{ \{}
        \StringTok{"input\_ids"}\NormalTok{: input\_ids,}
        \StringTok{"attention\_mask"}\NormalTok{: attention\_mask,}
        \StringTok{"labels"}\NormalTok{: labels}
\NormalTok{    \}}
\end{Highlighting}
\end{Shaded}

\begin{Shaded}
\begin{Highlighting}[]
\NormalTok{tokenized\_id }\OperatorTok{=}\NormalTok{ ds.}\BuiltInTok{map}\NormalTok{(process\_func, remove\_columns}\OperatorTok{=}\NormalTok{ds.column\_names)}
\NormalTok{tokenized\_id}
\end{Highlighting}
\end{Shaded}

\begin{Shaded}
\begin{Highlighting}[]
\NormalTok{tokenizer.decode(tokenized\_id[}\DecValTok{0}\NormalTok{][}\StringTok{\textquotesingle{}input\_ids\textquotesingle{}}\NormalTok{])}
\end{Highlighting}
\end{Shaded}

\begin{Shaded}
\begin{Highlighting}[]
\NormalTok{tokenizer.decode(}\BuiltInTok{list}\NormalTok{(}\BuiltInTok{filter}\NormalTok{(}\KeywordTok{lambda}\NormalTok{ x: x }\OperatorTok{!=} \OperatorTok{{-}}\DecValTok{100}\NormalTok{, tokenized\_id[}\DecValTok{1}\NormalTok{][}\StringTok{"labels"}\NormalTok{])))}
\end{Highlighting}
\end{Shaded}

\subsection{加载模型}\label{ux52a0ux8f7dux6a21ux578b}

加载模型并配置LoraConfig

\begin{Shaded}
\begin{Highlighting}[]
\NormalTok{model }\OperatorTok{=}\NormalTok{ AutoModelForCausalLM.from\_pretrained(model\_id, device\_map}\OperatorTok{=}\StringTok{"auto"}\NormalTok{,torch\_dtype}\OperatorTok{=}\NormalTok{torch.bfloat16)}
\NormalTok{model}
\end{Highlighting}
\end{Shaded}

\begin{verbatim}
Qwen3ForCausalLM(
  (model): Qwen3Model(
    (embed_tokens): Embedding(151936, 1024)
    (layers): ModuleList(
      (0-27): 28 x Qwen3DecoderLayer(
        (self_attn): Qwen3Attention(
          (q_proj): Linear(in_features=1024, out_features=2048, bias=False)
          (k_proj): Linear(in_features=1024, out_features=1024, bias=False)
          (v_proj): Linear(in_features=1024, out_features=1024, bias=False)
          (o_proj): Linear(in_features=2048, out_features=1024, bias=False)
          (q_norm): Qwen3RMSNorm((128,), eps=1e-06)
          (k_norm): Qwen3RMSNorm((128,), eps=1e-06)
        )
        (mlp): Qwen3MLP(
          (gate_proj): Linear(in_features=1024, out_features=3072, bias=False)
          (up_proj): Linear(in_features=1024, out_features=3072, bias=False)
          (down_proj): Linear(in_features=3072, out_features=1024, bias=False)
          (act_fn): SiLU()
        )
        (input_layernorm): Qwen3RMSNorm((1024,), eps=1e-06)
        (post_attention_layernorm): Qwen3RMSNorm((1024,), eps=1e-06)
      )
    )
    (norm): Qwen3RMSNorm((1024,), eps=1e-06)
    (rotary_emb): Qwen3RotaryEmbedding()
  )
  (lm_head): Linear(in_features=1024, out_features=151936, bias=False)
)
\end{verbatim}

\begin{Shaded}
\begin{Highlighting}[]
\NormalTok{model.enable\_input\_require\_grads() }\CommentTok{\# 开启梯度检查点时,要执行该方法}
\end{Highlighting}
\end{Shaded}

\subsection{Lora Config}\label{lora-config}

\texttt{LoraConfig}这个类中可以设置很多参数,比较重要的如下

\begin{itemize}
\tightlist
\item
  \texttt{task\_type}:模型类型,现在绝大部分 \texttt{decoder\_only}
  的模型都是因果语言模型 \texttt{CAUSAL\_LM}
\item
  \texttt{target\_modules}:需要训练的模型层的名字,主要就是
  \texttt{attention}部分的层,不同的模型对应的层的名字不同
\item
  \texttt{r}:\texttt{LoRA} 的秩,决定了低秩矩阵的维度,较小的
  \texttt{r} 意味着更少的参数
\item
  \texttt{lora\_alpha}:缩放参数,与 \texttt{r} 一起决定了 \texttt{LoRA}
  更新的强度。实际缩放比例为\texttt{lora\_alpha/r},在当前示例中是
  \texttt{32\ /\ 8\ =\ 4} 倍
\item
  \texttt{lora\_dropout}:应用于 \texttt{LoRA} 层的
  \texttt{dropout\ rate},用于防止过拟合
\end{itemize}

\begin{Shaded}
\begin{Highlighting}[]
\ImportTok{from}\NormalTok{ peft }\ImportTok{import}\NormalTok{ LoraConfig, TaskType, get\_peft\_model}

\NormalTok{config }\OperatorTok{=}\NormalTok{ LoraConfig(}
\NormalTok{    task\_type}\OperatorTok{=}\NormalTok{TaskType.CAUSAL\_LM,}
\NormalTok{    target\_modules}\OperatorTok{=}\NormalTok{[}\StringTok{"q\_proj"}\NormalTok{, }\StringTok{"k\_proj"}\NormalTok{, }\StringTok{"v\_proj"}\NormalTok{, }\StringTok{"o\_proj"}\NormalTok{, }\StringTok{"gate\_proj"}\NormalTok{, }\StringTok{"up\_proj"}\NormalTok{, }\StringTok{"down\_proj"}\NormalTok{],}
\NormalTok{    inference\_mode}\OperatorTok{=}\VariableTok{False}\NormalTok{, }\CommentTok{\# 训练模式}
\NormalTok{    r}\OperatorTok{=}\DecValTok{8}\NormalTok{, }\CommentTok{\# Lora 秩}
\NormalTok{    lora\_alpha}\OperatorTok{=}\DecValTok{32}\NormalTok{, }\CommentTok{\# Lora alaph,具体作用参见 Lora 原理}
\NormalTok{    lora\_dropout}\OperatorTok{=}\FloatTok{0.1}\CommentTok{\# Dropout 比例}
\NormalTok{)}
\NormalTok{config}
\end{Highlighting}
\end{Shaded}

\begin{Shaded}
\begin{Highlighting}[]
\NormalTok{model }\OperatorTok{=}\NormalTok{ get\_peft\_model(model, config)}
\NormalTok{config}
\end{Highlighting}
\end{Shaded}

\begin{Shaded}
\begin{Highlighting}[]
\NormalTok{model.print\_trainable\_parameters()  }\CommentTok{\# 模型参数训练量只有0.8395\%}
\end{Highlighting}
\end{Shaded}

\begin{quote}
trainable params: 5,046,272 \textbar\textbar{} all params: 601,096,192
\textbar\textbar{} trainable\%: 0.8395
\end{quote}

\subsection{Training Arguments}\label{training-arguments}

\begin{itemize}
\tightlist
\item
  \texttt{output\_dir}:模型的输出路径
\item
  \texttt{per\_device\_train\_batch\_size}:每张卡上的
  \texttt{batch\_size}
\item
  \texttt{gradient\_accumulation\_steps}: 梯度累计
\item
  \texttt{num\_train\_epochs}:顾名思义 \texttt{epoch}
\end{itemize}

\begin{Shaded}
\begin{Highlighting}[]
\NormalTok{args }\OperatorTok{=}\NormalTok{ TrainingArguments(}
\NormalTok{    output\_dir}\OperatorTok{=}\StringTok{"Qwen3\_instruct\_lora"}\NormalTok{,}
\NormalTok{    per\_device\_train\_batch\_size}\OperatorTok{=}\DecValTok{4}\NormalTok{,}
\NormalTok{    gradient\_accumulation\_steps}\OperatorTok{=}\DecValTok{4}\NormalTok{,}
\NormalTok{    logging\_steps}\OperatorTok{=}\DecValTok{1}\NormalTok{,}
\NormalTok{    num\_train\_epochs}\OperatorTok{=}\DecValTok{3}\NormalTok{,}
\NormalTok{    save\_steps}\OperatorTok{=}\DecValTok{50}\NormalTok{,}
\NormalTok{    learning\_rate}\OperatorTok{=}\FloatTok{1e{-}4}\NormalTok{,}
\NormalTok{    save\_on\_each\_node}\OperatorTok{=}\VariableTok{True}\NormalTok{,}
\NormalTok{    gradient\_checkpointing}\OperatorTok{=}\VariableTok{True}\NormalTok{,}
\NormalTok{    report\_to}\OperatorTok{=}\StringTok{"none"}\NormalTok{,}
\NormalTok{)}
\end{Highlighting}
\end{Shaded}

\subsection{SwanLab 简介}\label{swanlab-ux7b80ux4ecb}

\href{https://github.com/swanhubx/swanlab}{SwanLab}
是一个开源的模型训练记录工具,面向 AI
研究者,提供了训练可视化、自动日志记录、超参数记录、实验对比、多人协同等功能。在
\texttt{SwanLab}
上,研究者能基于直观的可视化图表发现训练问题,对比多个实验找到研究灵感,并通过在线链接的分享与基于组织的多人协同训练,打破团队沟通的壁垒。

\textbf{为什么要记录训练}

相较于软件开发,模型训练更像一个实验科学。一个品质优秀的模型背后,往往是成千上万次实验。研究者需要不断尝试、记录、对比,积累经验,才能找到最佳的模型结构、超参数与数据配比。在这之中,如何高效进行记录与对比,对于研究效率的提升至关重要。

\texttt{(2)\ Use\ an\ existing\ SwanLab\ account} 并使用 private API Key
登录

\begin{Shaded}
\begin{Highlighting}[]
\ImportTok{import}\NormalTok{ swanlab}
\ImportTok{from}\NormalTok{ swanlab.integration.transformers }\ImportTok{import}\NormalTok{ SwanLabCallback}

\CommentTok{\# 实例化SwanLabCallback}
\NormalTok{swanlab\_callback }\OperatorTok{=}\NormalTok{ SwanLabCallback(}
\NormalTok{    project}\OperatorTok{=}\StringTok{"Qwen3{-}Lora"}\NormalTok{,  }\CommentTok{\# 注意修改}
\NormalTok{    experiment\_name}\OperatorTok{=}\StringTok{"Qwen3{-}8B{-}LoRA{-}experiment"}  \CommentTok{\# 注意修改}
\NormalTok{)}
\end{Highlighting}
\end{Shaded}

\begin{Shaded}
\begin{Highlighting}[]
\ImportTok{import}\NormalTok{ swanlab}
\ImportTok{from}\NormalTok{ swanlab.integration.transformers }\ImportTok{import}\NormalTok{ SwanLabCallback}

\CommentTok{\# 实例化SwanLabCallback}
\NormalTok{swanlab\_callback }\OperatorTok{=}\NormalTok{ SwanLabCallback(}
\NormalTok{    project}\OperatorTok{=}\StringTok{"Qwen3{-}Lora"}\NormalTok{,}
\NormalTok{    experiment\_name}\OperatorTok{=}\StringTok{"Qwen3{-}0.6B{-}extarct{-}lora{-}2"}
\NormalTok{)}
\end{Highlighting}
\end{Shaded}

\begin{Shaded}
\begin{Highlighting}[]
\NormalTok{trainer }\OperatorTok{=}\NormalTok{ Trainer(}
\NormalTok{    model}\OperatorTok{=}\NormalTok{model,}
\NormalTok{    args}\OperatorTok{=}\NormalTok{args,}
\NormalTok{    train\_dataset}\OperatorTok{=}\NormalTok{tokenized\_id,}
\NormalTok{    data\_collator}\OperatorTok{=}\NormalTok{DataCollatorForSeq2Seq(tokenizer}\OperatorTok{=}\NormalTok{tokenizer, padding}\OperatorTok{=}\VariableTok{True}\NormalTok{),}
\NormalTok{    callbacks}\OperatorTok{=}\NormalTok{[swanlab\_callback]}
\NormalTok{)}
\end{Highlighting}
\end{Shaded}

\begin{Shaded}
\begin{Highlighting}[]
\NormalTok{trainer.train()}
\end{Highlighting}
\end{Shaded}

\subsection{测试文本}\label{ux6d4bux8bd5ux6587ux672c}

\begin{Shaded}
\begin{Highlighting}[]
\NormalTok{prompt }\OperatorTok{=} \StringTok{"龙琳   ,宁夏回族自治区璐市城东林街g座 955491,nafan@example.com。小区垃圾堆积成山,晚上噪音扰人清梦,停车难上加难,简直无法忍受!太插件了阿萨德看见啊啥的健康仨都会撒娇看到撒谎的、"}

\NormalTok{messages }\OperatorTok{=}\NormalTok{ [}
\NormalTok{    \{}\StringTok{"role"}\NormalTok{: }\StringTok{"system"}\NormalTok{, }\StringTok{"content"}\NormalTok{: }\StringTok{"将文本中的name、address、email、question提取出来,以json格式输出,字段为name、address、email、question,值为文本中提取出来的内容。"}\NormalTok{\},}
\NormalTok{    \{}\StringTok{"role"}\NormalTok{: }\StringTok{"user"}\NormalTok{, }\StringTok{"content"}\NormalTok{: prompt\}}
\NormalTok{]}

\NormalTok{inputs }\OperatorTok{=}\NormalTok{ tokenizer.apply\_chat\_template(messages,}
\NormalTok{                                       add\_generation\_prompt}\OperatorTok{=}\VariableTok{True}\NormalTok{,}
\NormalTok{                                       tokenize}\OperatorTok{=}\VariableTok{True}\NormalTok{,}
\NormalTok{                                       return\_tensors}\OperatorTok{=}\StringTok{"pt"}\NormalTok{,}
\NormalTok{                                       return\_dict}\OperatorTok{=}\VariableTok{True}\NormalTok{,}
\NormalTok{                                       enable\_thinking}\OperatorTok{=}\VariableTok{False}\NormalTok{).to(}\StringTok{\textquotesingle{}cuda\textquotesingle{}}\NormalTok{)}

\NormalTok{gen\_kwargs }\OperatorTok{=}\NormalTok{ \{}\StringTok{"max\_length"}\NormalTok{: }\DecValTok{2500}\NormalTok{, }\StringTok{"do\_sample"}\NormalTok{: }\VariableTok{True}\NormalTok{, }\StringTok{"top\_k"}\NormalTok{: }\DecValTok{1}\NormalTok{\}}
\ControlFlowTok{with}\NormalTok{ torch.no\_grad():}
\NormalTok{    outputs }\OperatorTok{=}\NormalTok{ model.generate(}\OperatorTok{**}\NormalTok{inputs, }\OperatorTok{**}\NormalTok{gen\_kwargs)}
\NormalTok{    outputs }\OperatorTok{=}\NormalTok{ outputs[:, inputs[}\StringTok{\textquotesingle{}input\_ids\textquotesingle{}}\NormalTok{].shape[}\DecValTok{1}\NormalTok{]:]}
    \BuiltInTok{print}\NormalTok{(tokenizer.decode(outputs[}\DecValTok{0}\NormalTok{], skip\_special\_tokens}\OperatorTok{=}\VariableTok{True}\NormalTok{))}
\end{Highlighting}
\end{Shaded}

\begin{Shaded}
\begin{Highlighting}[]
\FunctionTok{\{}
    \DataTypeTok{"name"}\FunctionTok{:} \StringTok{"龙琳"}\FunctionTok{,}
    \DataTypeTok{"address"}\FunctionTok{:} \StringTok{"宁夏回族自治区璐市城东林街g座 955491"}\FunctionTok{,}
    \DataTypeTok{"email"}\FunctionTok{:} \StringTok{"nafan@example.com"}\FunctionTok{,}
    \DataTypeTok{"question"}\FunctionTok{:} \StringTok{"小区垃圾堆积成山,晚上噪音扰人清梦,停车难上加难,简直无法忍受!太插件了阿萨德看见啊啥的健康仨都会撒娇看到撒谎的、"}
\FunctionTok{\}}
\end{Highlighting}
\end{Shaded}


\end{document}
