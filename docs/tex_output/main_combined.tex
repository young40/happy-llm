\documentclass[12pt,a4paper]{book}
\usepackage[UTF8]{ctex}
\usepackage{geometry}
\usepackage{graphicx}
\usepackage{hyperref}
\usepackage{bookmark}
\usepackage{amsmath,amssymb}
\usepackage{listings}
\usepackage{xcolor}
\usepackage{fancyhdr}
\usepackage{titlesec}
\usepackage{tocloft}

% 页面设置
\geometry{left=2.5cm,right=2.5cm,top=2.5cm,bottom=2.5cm}

% 代码块设置
\lstset{
    basicstyle=\ttfamily\small,
    breaklines=true,
    frame=single,
    numbers=left,
    numberstyle=\tiny,
    keywordstyle=\color{blue},
    commentstyle=\color{green!60!black},
    stringstyle=\color{red},
    backgroundcolor=\color{gray!10},
    showspaces=false,
    showstringspaces=false,
    showtabs=false,
    tabsize=2
}

% 超链接设置
\hypersetup{
    colorlinks=true,
    linkcolor=blue,
    filecolor=magenta,
    urlcolor=cyan,
    citecolor=green,
    bookmarks=true,
    bookmarksopen=true,
    pdfstartview=FitH
}

% 页眉页脚设置
\pagestyle{fancy}
\fancyhf{}
\fancyhead[LE,RO]{\thepage}
\fancyhead[RE]{\leftmark}
\fancyhead[LO]{\rightmark}
\renewcommand{\headrulewidth}{0.4pt}

% 章节标题格式
\titleformat{\chapter}{\Large\bfseries}{\thechapter}{1em}{}
\titleformat{\section}{\large\bfseries}{\thesection}{1em}{}
\titleformat{\subsection}{\normalsize\bfseries}{\thesubsection}{1em}{}

% 目录格式
\renewcommand{\cftchapfont}{\bfseries}
\renewcommand{\cftsecfont}{\normalsize}

\begin{document}

% 标题页
\begin{titlepage}
    \centering
    \vspace*{2cm}
    
    {\Huge\bfseries 动手学大语言模型}\\[2cm]
    
    {\Large 从理论到实践}\\[1cm]
    
    {\large Datawhale 开源学习社区}\\[2cm]
    
    \vfill
    
    {\large \today}
\end{titlepage}

% 版权页
\newpage
\thispagestyle{empty}
\vspace*{2cm}
\begin{center}
    \textbf{版权声明}
    
    \vspace{1cm}
    
    本书由 Datawhale 开源学习社区编写,采用开源协议发布。
    
    欢迎读者在遵守开源协议的前提下自由使用、修改和分发本书内容。
    
    \vspace{1cm}
    
    GitHub: https://github.com/datawhalechina/happy-llm
\end{center}

% 目录
\tableofcontents
\newpage

% 前言
\chapter{前言}
\input{前言}

% 第一章
\chapter{NLP基础概念}
\input{chapter1/第一章 NLP基础概念}

% 第二章
\chapter{Transformer架构}
% Options for packages loaded elsewhere
\PassOptionsToPackage{unicode}{hyperref}
\PassOptionsToPackage{hyphens}{url}
\documentclass[
]{article}
\usepackage{xcolor}
\usepackage{amsmath,amssymb}
\setcounter{secnumdepth}{5}
\usepackage{iftex}
\ifPDFTeX
  \usepackage[T1]{fontenc}
  \usepackage[utf8]{inputenc}
  \usepackage{textcomp} % provide euro and other symbols
\else % if luatex or xetex
  \usepackage{unicode-math} % this also loads fontspec
  \defaultfontfeatures{Scale=MatchLowercase}
  \defaultfontfeatures[\rmfamily]{Ligatures=TeX,Scale=1}
\fi
\usepackage{lmodern}
\ifPDFTeX\else
  % xetex/luatex font selection
\fi
% Use upquote if available, for straight quotes in verbatim environments
\IfFileExists{upquote.sty}{\usepackage{upquote}}{}
\IfFileExists{microtype.sty}{% use microtype if available
  \usepackage[]{microtype}
  \UseMicrotypeSet[protrusion]{basicmath} % disable protrusion for tt fonts
}{}
\makeatletter
\@ifundefined{KOMAClassName}{% if non-KOMA class
  \IfFileExists{parskip.sty}{%
    \usepackage{parskip}
  }{% else
    \setlength{\parindent}{0pt}
    \setlength{\parskip}{6pt plus 2pt minus 1pt}}
}{% if KOMA class
  \KOMAoptions{parskip=half}}
\makeatother
\usepackage{color}
\usepackage{fancyvrb}
\newcommand{\VerbBar}{|}
\newcommand{\VERB}{\Verb[commandchars=\\\{\}]}
\DefineVerbatimEnvironment{Highlighting}{Verbatim}{commandchars=\\\{\}}
% Add ',fontsize=\small' for more characters per line
\newenvironment{Shaded}{}{}
\newcommand{\AlertTok}[1]{\textcolor[rgb]{1.00,0.00,0.00}{\textbf{#1}}}
\newcommand{\AnnotationTok}[1]{\textcolor[rgb]{0.38,0.63,0.69}{\textbf{\textit{#1}}}}
\newcommand{\AttributeTok}[1]{\textcolor[rgb]{0.49,0.56,0.16}{#1}}
\newcommand{\BaseNTok}[1]{\textcolor[rgb]{0.25,0.63,0.44}{#1}}
\newcommand{\BuiltInTok}[1]{\textcolor[rgb]{0.00,0.50,0.00}{#1}}
\newcommand{\CharTok}[1]{\textcolor[rgb]{0.25,0.44,0.63}{#1}}
\newcommand{\CommentTok}[1]{\textcolor[rgb]{0.38,0.63,0.69}{\textit{#1}}}
\newcommand{\CommentVarTok}[1]{\textcolor[rgb]{0.38,0.63,0.69}{\textbf{\textit{#1}}}}
\newcommand{\ConstantTok}[1]{\textcolor[rgb]{0.53,0.00,0.00}{#1}}
\newcommand{\ControlFlowTok}[1]{\textcolor[rgb]{0.00,0.44,0.13}{\textbf{#1}}}
\newcommand{\DataTypeTok}[1]{\textcolor[rgb]{0.56,0.13,0.00}{#1}}
\newcommand{\DecValTok}[1]{\textcolor[rgb]{0.25,0.63,0.44}{#1}}
\newcommand{\DocumentationTok}[1]{\textcolor[rgb]{0.73,0.13,0.13}{\textit{#1}}}
\newcommand{\ErrorTok}[1]{\textcolor[rgb]{1.00,0.00,0.00}{\textbf{#1}}}
\newcommand{\ExtensionTok}[1]{#1}
\newcommand{\FloatTok}[1]{\textcolor[rgb]{0.25,0.63,0.44}{#1}}
\newcommand{\FunctionTok}[1]{\textcolor[rgb]{0.02,0.16,0.49}{#1}}
\newcommand{\ImportTok}[1]{\textcolor[rgb]{0.00,0.50,0.00}{\textbf{#1}}}
\newcommand{\InformationTok}[1]{\textcolor[rgb]{0.38,0.63,0.69}{\textbf{\textit{#1}}}}
\newcommand{\KeywordTok}[1]{\textcolor[rgb]{0.00,0.44,0.13}{\textbf{#1}}}
\newcommand{\NormalTok}[1]{#1}
\newcommand{\OperatorTok}[1]{\textcolor[rgb]{0.40,0.40,0.40}{#1}}
\newcommand{\OtherTok}[1]{\textcolor[rgb]{0.00,0.44,0.13}{#1}}
\newcommand{\PreprocessorTok}[1]{\textcolor[rgb]{0.74,0.48,0.00}{#1}}
\newcommand{\RegionMarkerTok}[1]{#1}
\newcommand{\SpecialCharTok}[1]{\textcolor[rgb]{0.25,0.44,0.63}{#1}}
\newcommand{\SpecialStringTok}[1]{\textcolor[rgb]{0.73,0.40,0.53}{#1}}
\newcommand{\StringTok}[1]{\textcolor[rgb]{0.25,0.44,0.63}{#1}}
\newcommand{\VariableTok}[1]{\textcolor[rgb]{0.10,0.09,0.49}{#1}}
\newcommand{\VerbatimStringTok}[1]{\textcolor[rgb]{0.25,0.44,0.63}{#1}}
\newcommand{\WarningTok}[1]{\textcolor[rgb]{0.38,0.63,0.69}{\textbf{\textit{#1}}}}
\setlength{\emergencystretch}{3em} % prevent overfull lines
\providecommand{\tightlist}{%
  \setlength{\itemsep}{0pt}\setlength{\parskip}{0pt}}
\usepackage{bookmark}
\IfFileExists{xurl.sty}{\usepackage{xurl}}{} % add URL line breaks if available
\urlstyle{same}
\hypersetup{
  hidelinks,
  pdfcreator={LaTeX via pandoc}}

\author{}
\date{}

\begin{document}

{
\setcounter{tocdepth}{3}
\tableofcontents
}
\section{第二章 Transformer
架构}\label{ux7b2cux4e8cux7ae0-transformer-ux67b6ux6784}

\subsection{2.1 注意力机制}\label{ux6ce8ux610fux529bux673aux5236}

\subsubsection{2.1.1
什么是注意力机制}\label{ux4ec0ux4e48ux662fux6ce8ux610fux529bux673aux5236}

随着 NLP 从统计机器学习向深度学习迈进,作为 NLP
核心问题的文本表示方法也逐渐从统计学习向深度学习迈进。正如我们在第一章所介绍的,文本表示从最初的通过统计学习模型进行计算的向量空间模型、语言模型,通过
Word2Vec 的单层神经网络进入到通过神经网络学习文本表示的时代。但是,从
计算机视觉(Computer
Vision,CV)为起源发展起来的神经网络,其核心架构有三种:

\begin{itemize}
\tightlist
\item
  前馈神经网络(Feedforward Neural
  Network,FNN),即每一层的神经元都和上下两层的每一个神经元完全连接,如图2.1所示:
\end{itemize}

图2.1 前馈神经网络

\begin{itemize}
\tightlist
\item
  卷积神经网络(Convolutional Neural
  Network,CNN),即训练参数量远小于前馈神经网络的卷积层来进行特征提取和学习,如图2.2所示:
\end{itemize}

图2.2 卷积神经网络

\begin{itemize}
\tightlist
\item
  循环神经网络(Recurrent Neural
  Network,RNN),能够使用历史信息作为输入、包含环和自重复的网络,如图2.3所示:
\end{itemize}

图2.3 循环神经网络

由于 NLP 任务所需要处理的文本往往是序列,因此专用于处理序列、时序数据的
RNN 往往能够在 NLP
任务上取得最优的效果。事实上,在注意力机制横空出世之前,RNN 以及 RNN
的衍生架构 LSTM 是 NLP
领域当之无愧的霸主。例如,我们在第一章讲到过的开创了预训练思想的文本表示模型
ELMo,就是使用的双向 LSTM 作为网络架构。

但 RNN 及 LSTM
虽然具有捕捉时序信息、适合序列生成的优点,却有两个难以弥补的缺陷:

\begin{enumerate}
\def\labelenumi{\arabic{enumi}.}
\item
  序列依序计算的模式能够很好地模拟时序信息,但限制了计算机并行计算的能力。由于序列需要依次输入、依序计算,图形处理器(Graphics
  Processing Unit,GPU)并行计算的能力受到了极大限制,导致 RNN
  为基础架构的模型虽然参数量不算特别大,但计算时间成本却很高;
\item
  RNN 难以捕捉长序列的相关关系。在 RNN
  架构中,距离越远的输入之间的关系就越难被捕捉,同时 RNN
  需要将整个序列读入内存依次计算,也限制了序列的长度。虽然 LSTM
  中通过门机制对此进行了一定优化,但对于较远距离相关关系的捕捉,RNN
  依旧是不如人意的。
\end{enumerate}

针对这样的问题,Vaswani 等学者参考了在 CV 领域被提出、被经常融入到 RNN
中使用的注意力机制(Attention)(注意,虽然注意力机制在 NLP
被发扬光大,但其确实是在 CV
领域被提出的),创新性地搭建了完全由注意力机制构成的神经网络------Transformer,也就是大语言模型(Large
Language
Model,LLM)的鼻祖及核心架构,从而让注意力机制一跃成为深度学习最核心的架构之一。

那么,究竟什么是注意力机制?

注意力机制最先源于计算机视觉领域,其核心思想为当我们关注一张图片,我们往往无需看清楚全部内容而仅将注意力集中在重点部分即可。而在自然语言处理领域,我们往往也可以通过将重点注意力集中在一个或几个
token,从而取得更高效高质的计算效果。

注意力机制有三个核心变量:\textbf{Query}(查询值)、\textbf{Key}(键值)和
\textbf{Value}(真值)。我们可以通过一个案例来理解每一个变量所代表的含义。例如,当我们有一篇新闻报道,我们想要找到这个报道的时间,那么,我们的
Query
可以是类似于``时间''、``日期''一类的向量(为了便于理解,此处使用文本来表示,但其实际是稠密的向量),Key
和 Value 会是整个文本。通过对 Query 和 Key
进行运算我们可以得到一个权重,这个权重其实反映了从 Query
出发,对文本每一个 token 应该分布的注意力相对大小。通过把权重和 Value
进行运算,得到的最后结果就是从 Query 出发计算整个文本注意力得到的结果。

\hspace{0pt}具体而言,注意力机制的特点是通过计算 \textbf{Query}
与\textbf{Key}的相关性为真值加权求和,从而拟合序列中每个词同其他词的相关关系。

\subsubsection{2.1.2
深入理解注意力机制}\label{ux6df1ux5165ux7406ux89e3ux6ce8ux610fux529bux673aux5236}

刚刚我们说到,注意力机制有三个核心变量:查询值 Query,键值 Key 和 真值
Value。接下来我们以字典为例,逐步分析注意力机制的计算公式是如何得到的,从而帮助读者深入理解注意力机制。首先,我们有这样一个字典:

\begin{Shaded}
\begin{Highlighting}[]
\FunctionTok{\{}
    \DataTypeTok{"apple"}\FunctionTok{:}\DecValTok{10}\FunctionTok{,}
    \DataTypeTok{"banana"}\FunctionTok{:}\DecValTok{5}\FunctionTok{,}
    \DataTypeTok{"chair"}\FunctionTok{:}\DecValTok{2}
\FunctionTok{\}}
\end{Highlighting}
\end{Shaded}

此时,字典的键就是注意力机制中的键值 Key,而字典的值就是真值
Value。字典支持我们进行精确的字符串匹配,例如,如果我们想要查找的值也就是查询值
Query 为``apple'',那么我们可以直接通过将 Query 与 Key
做匹配来得到对应的 Value。

但是,如果我们想要匹配的 Query 是一个包含多个 Key
的概念呢?例如,我们想要查找``fruit'',此时,我们应该将 apple 和 banana
都匹配到,但不能匹配到 chair。因此,我们往往会选择将 Key 对应的 Value
进行组合得到最终的 Value。

例如,当我们的 Query 为``fruit'',我们可以分别给三个 Key
赋予如下的权重:

\begin{Shaded}
\begin{Highlighting}[]
\FunctionTok{\{}
    \DataTypeTok{"apple"}\FunctionTok{:}\FloatTok{0.6}\FunctionTok{,}
    \DataTypeTok{"banana"}\FunctionTok{:}\FloatTok{0.4}\FunctionTok{,}
    \DataTypeTok{"chair"}\FunctionTok{:}\DecValTok{0}
\FunctionTok{\}}
\end{Highlighting}
\end{Shaded}

那么,我们最终查询到的值应该是:

\[
value = 0.6 * 10 + 0.4 * 5 + 0 * 2 = 8
\]

给不同 Key 所赋予的不同权重,就是我们所说的注意力分数,也就是为了查询到
Query,我们应该赋予给每一个 Key 多少注意力。但是,如何针对每一个
Query,计算出对应的注意力分数呢?从直观上讲,我们可以认为 Key 与 Query
相关性越高,则其所应该赋予的注意力权重就越大。但是,我们如何能够找到一个合理的、能够计算出正确的注意力分数的方法呢?

在第一章中,我们有提到词向量的概念。通过合理的训练拟合,词向量能够表征语义信息,从而让语义相近的词在向量空间中距离更近,语义较远的词在向量空间中距离更远。我们往往用欧式距离来衡量词向量的相似性,但我们同样也可以用点积来进行度量:

\[
v·w = \sum_{i}v_iw_i
\]

根据词向量的定义,语义相似的两个词对应的词向量的点积应该大于0,而语义不相似的词向量点积应该小于0。

那么,我们就可以用点积来计算词之间的相似度。假设我们的 Query
为``fruit'',对应的词向量为 \(q\) ;我们的 Key 对应的词向量为
\(k = [v_{apple} v_{banana} v_{chair}]\) ,则我们可以计算 Query
和每一个键的相似程度:

\[
x = qK^T
\]

此处的 K 即为将所有 Key
对应的词向量堆叠形成的矩阵。基于矩阵乘法的定义,x 即为 q 与每一个 k
值的点积。现在我们得到的 x 即反映了 Query 和每一个 Key
的相似程度,我们再通过一个 Softmax 层将其转化为和为 1 的权重:

\[
\text{softmax}(x)_i = \frac{e^{xi}}{\sum_{j}e^{x_j}}
\]

这样,得到的向量就能够反映 Query 和每一个 Key
的相似程度,同时又相加权重为
1,也就是我们的注意力分数了。最后,我们再将得到的注意力分数和值向量做对应乘积即可。根据上述过程,我们就可以得到注意力机制计算的基本公式:

\[
attention(Q,K,V) = softmax(qK^T)v
\]

不过,此时的值还是一个标量,同时,我们此次只查询了一个
Query。我们可以将值转化为维度为 \(d_v\) 的向量,同时一次性查询多个
Query,同样将多个 Query 对应的词向量堆叠在一起形成矩阵 Q,得到公式:

\[
attention(Q,K,V) = softmax(QK^T)V
\]

目前,我们离标准的注意力机制公式还差最后一步。在上一个公式中,如果 Q 和
K 对应的维度 \(d_k\) 比较大,softmax
放缩时就非常容易受影响,使不同值之间的差异较大,从而影响梯度的稳定性。因此,我们要将
Q 和 K 乘积的结果做一个放缩:

\[
attention(Q,K,V) = softmax(\frac{QK^T}{\sqrt{d_k}})V
\]

这也就是注意力机制的核心计算公式了。

\subsubsection{2.1.3
注意力机制的实现}\label{ux6ce8ux610fux529bux673aux5236ux7684ux5b9eux73b0}

基于上文,我们可以很简单地使用 Pytorch 来实现注意力机制的代码:

\begin{Shaded}
\begin{Highlighting}[]
\CommentTok{\textquotesingle{}\textquotesingle{}\textquotesingle{}注意力计算函数\textquotesingle{}\textquotesingle{}\textquotesingle{}}
\KeywordTok{def}\NormalTok{ attention(query, key, value, dropout}\OperatorTok{=}\VariableTok{None}\NormalTok{):}
    \CommentTok{\textquotesingle{}\textquotesingle{}\textquotesingle{}}
\CommentTok{    args:}
\CommentTok{    query: 查询值矩阵}
\CommentTok{    key: 键值矩阵}
\CommentTok{    value: 真值矩阵}
\CommentTok{    \textquotesingle{}\textquotesingle{}\textquotesingle{}}
    \CommentTok{\# 获取键向量的维度,键向量的维度和值向量的维度相同}
\NormalTok{    d\_k }\OperatorTok{=}\NormalTok{ query.size(}\OperatorTok{{-}}\DecValTok{1}\NormalTok{) }
    \CommentTok{\# 计算Q与K的内积并除以根号dk}
    \CommentTok{\# transpose——相当于转置}
\NormalTok{    scores }\OperatorTok{=}\NormalTok{ torch.matmul(query, key.transpose(}\OperatorTok{{-}}\DecValTok{2}\NormalTok{, }\OperatorTok{{-}}\DecValTok{1}\NormalTok{)) }\OperatorTok{/}\NormalTok{ math.sqrt(d\_k)}
    \CommentTok{\# Softmax}
\NormalTok{    p\_attn }\OperatorTok{=}\NormalTok{ scores.softmax(dim}\OperatorTok{={-}}\DecValTok{1}\NormalTok{)}
    \ControlFlowTok{if}\NormalTok{ dropout }\KeywordTok{is} \KeywordTok{not} \VariableTok{None}\NormalTok{:}
\NormalTok{        p\_attn }\OperatorTok{=}\NormalTok{ dropout(p\_attn)}
        \CommentTok{\# 采样}
     \CommentTok{\# 根据计算结果对value进行加权求和}
    \ControlFlowTok{return}\NormalTok{ torch.matmul(p\_attn, value), p\_attn}
\end{Highlighting}
\end{Shaded}

注意,在上文代码中,我们假设输入的 q、k、v
是已经经过转化的词向量矩阵,也就是公式中的
Q、K、V。我们仅需要通过上述几行代码,就可以实现核心的注意力机制计算。

\subsubsection{2.1.4 自注意力}\label{ux81eaux6ce8ux610fux529b}

根据上文的分析,我们可以发现,注意力机制的本质是对两段序列的元素依次进行相似度计算,寻找出一个序列的每个元素对另一个序列的每个元素的相关度,然后基于相关度进行加权,即分配注意力。而这两段序列即是我们计算过程中
Q、K、V 的来源。

但是,在我们的实际应用中,我们往往只需要计算 Query 和 Key
之间的注意力结果,很少存在额外的真值
Value。也就是说,我们其实只需要拟合两个文本序列。\hspace{0pt}在经典的
注意力机制中,Q 往往来自于一个序列,K 与 V
来自于另一个序列,都通过参数矩阵计算得到,从而可以拟合这两个序列之间的关系。例如在
Transformer 的 Decoder 结构中,Q 来自于 Decoder 的输入,K 与 V 来自于
Encoder
的输出,从而拟合了编码信息与历史信息之间的关系,便于综合这两种信息实现未来的预测。

\hspace{0pt}但在 Transformer 的 Encoder 结构中,使用的是
注意力机制的变种 ------
自注意力(self-attention,自注意力)机制。所谓自注意力,即是计算本身序列中每个元素对其他元素的注意力分布,即在计算过程中,Q、K、V
都由同一个输入通过不同的参数矩阵计算得到。在 Encoder 中,Q、K、V
分别是输入对参数矩阵 \(W_q、W_k、W_v\)
做积得到,从而拟合输入语句中每一个 token 对其他所有 token 的关系。

通过自注意力机制,我们可以找到一段文本中每一个 token 与其他所有 token
的相关关系大小,从而建模文本之间的依赖关系。\hspace{0pt}在代码中的实现,self-attention
机制其实是通过给 Q、K、V 的输入传入同一个参数实现的:

\begin{Shaded}
\begin{Highlighting}[]
\CommentTok{\# attention 为上文定义的注意力计算函数}
\NormalTok{attention(x, x, x)}
\end{Highlighting}
\end{Shaded}

\subsubsection{2.1.5
掩码自注意力}\label{ux63a9ux7801ux81eaux6ce8ux610fux529b}

掩码自注意力,即 Mask
Self-Attention,是指使用注意力掩码的自注意力机制。掩码的作用是遮蔽一些特定位置的
token,模型在学习的过程中,会忽略掉被遮蔽的 token。

使用注意力掩码的核心动机是让模型只能使用历史信息进行预测而不能看到未来信息。使用注意力机制的
Transformer 模型也是通过类似于 n-gram
的语言模型任务来学习的,也就是对一个文本序列,不断根据之前的 token
来预测下一个 token,直到将整个文本序列补全。

例如,如果待学习的文本序列是 【BOS】I like
you【EOS】,那么,模型会按如下顺序进行预测和学习:

\begin{verbatim}
Step 1:输入 【BOS】,输出 I
Step 2:输入 【BOS】I,输出 like
Step 3:输入 【BOS】I like,输出 you
Step 4:输入 【BOS】I like you,输出 【EOS】
\end{verbatim}

理论上来说,只要学习的语料足够多,通过上述的过程,模型可以学会任意一种文本序列的建模方式,也就是可以对任意的文本进行补全。

但是,我们可以发现,上述过程是一个串行的过程,也就是需要先完成 Step
1,才能做 Step
2,接下来逐步完成整个序列的补全。我们在一开始就说过,Transformer 相对于
RNN
的核心优势之一即在于其可以并行计算,具有更高的计算效率。如果对于每一个训练语料,模型都需要串行完成上述过程才能完成学习,那么很明显没有做到并行计算,计算效率很低。

针对这个问题,Transformer
就提出了掩码自注意力的方法。掩码自注意力会生成一串掩码,来遮蔽未来信息。例如,我们待学习的文本序列仍然是
【BOS】I like
you【EOS】,我们使用的注意力掩码是【MASK】,那么模型的输入为:

\begin{verbatim}
<BOS> 【MASK】【MASK】【MASK】【MASK】
<BOS>    I   【MASK】 【MASK】【MASK】
<BOS>    I     like  【MASK】【MASK】
<BOS>    I     like    you  【MASK】
<BoS>    I     like    you   </EOS>
\end{verbatim}

在每一行输入中,模型仍然是只看到前面的 token,预测下一个
token。但是注意,上述输入不再是串行的过程,而可以一起并行地输入到模型中,模型只需要每一个样本根据未被遮蔽的
token 来预测下一个 token 即可,从而实现了并行的语言模型。

观察上述的掩码,我们可以发现其实则是一个和文本序列等长的上三角矩阵。我们可以简单地通过创建一个和输入同等长度的上三角矩阵作为注意力掩码,再使用掩码来遮蔽掉输入即可。也就是说,当输入维度为
(batch\_size, seq\_len, hidden\_size)时,我们的 Mask 矩阵维度一般为
(1, seq\_len, seq\_len)(通过广播实现同一个 batch 中不同样本的计算)。

在具体实现中,我们通过以下代码生成 Mask 矩阵:

\begin{Shaded}
\begin{Highlighting}[]
\CommentTok{\# 创建一个上三角矩阵,用于遮蔽未来信息。}
\CommentTok{\# 先通过 full 函数创建一个 1 * seq\_len * seq\_len 的矩阵}
\NormalTok{mask }\OperatorTok{=}\NormalTok{ torch.full((}\DecValTok{1}\NormalTok{, args.max\_seq\_len, args.max\_seq\_len), }\BuiltInTok{float}\NormalTok{(}\StringTok{"{-}inf"}\NormalTok{))}
\CommentTok{\# triu 函数的功能是创建一个上三角矩阵}
\NormalTok{mask }\OperatorTok{=}\NormalTok{ torch.triu(mask, diagonal}\OperatorTok{=}\DecValTok{1}\NormalTok{)}
\end{Highlighting}
\end{Shaded}

生成的 Mask 矩阵会是一个上三角矩阵,上三角位置的元素均为
-inf,其他位置的元素置为0。

在注意力计算时,我们会将计算得到的注意力分数与这个掩码做和,再进行
Softmax 操作:

\begin{Shaded}
\begin{Highlighting}[]
\CommentTok{\# 此处的 scores 为计算得到的注意力分数,mask 为上文生成的掩码矩阵}
\NormalTok{scores }\OperatorTok{=}\NormalTok{ scores }\OperatorTok{+}\NormalTok{ mask[:, :seqlen, :seqlen]}
\NormalTok{scores }\OperatorTok{=}\NormalTok{ F.softmax(scores.}\BuiltInTok{float}\NormalTok{(), dim}\OperatorTok{={-}}\DecValTok{1}\NormalTok{).type\_as(xq)}
\end{Highlighting}
\end{Shaded}

通过做求和,上三角区域(也就是应该被遮蔽的 token
对应的位置)的注意力分数结果都变成了
\texttt{-inf},而下三角区域的分数不变。再做 Softmax 操作,\texttt{-inf}
的值在经过 Softmax 之后会被置为
0,从而忽略了上三角区域计算的注意力分数,从而实现了注意力遮蔽。

\subsubsection{2.1.6 多头注意力}\label{ux591aux5934ux6ce8ux610fux529b}

注意力机制可以实现并行化与长期依赖关系拟合,但一次注意力计算只能拟合一种相关关系,单一的注意力机制很难全面拟合语句序列里的相关关系。因此
Transformer 使用了多头注意力机制(Multi-Head
Attention),即同时对一个语料进行多次注意力计算,每次注意力计算都能拟合不同的关系,将最后的多次结果拼接起来作为最后的输出,即可更全面深入地拟合语言信息。

在原论文中,作者也通过实验证实,多头注意力计算中,每个不同的注意力头能够拟合语句中的不同信息,如图2.4所示:

图2.4 多头注意力机制

\hspace{0pt}上层与下层分别是两个注意力头对同一段语句序列进行自注意力计算的结果,可以看到,对于不同的注意力头,能够拟合不同层次的相关信息。通过多个注意力头同时计算,能够更全面地拟合语句关系。

事实上,所谓的多头注意力机制其实就是将原始的输入序列进行多组的自注意力处理;然后再将每一组得到的自注意力结果拼接起来,再通过一个线性层进行处理,得到最终的输出。我们用公式可以表示为:

\[
\mathrm{MultiHead}(Q, K, V) = \mathrm{Concat}(\mathrm{head_1}, ...,
\mathrm{head_h})W^O    \\
    \text{where}~\mathrm{head_i} = \mathrm{Attention}(QW^Q_i, KW^K_i, VW^V_i)
\]

其最直观的代码实现并不复杂,即 n 个头就有 n
组3个参数矩阵,每一组进行同样的注意力计算,但由于是不同的参数矩阵从而通过反向传播实现了不同的注意力结果,然后将
n 个结果拼接起来输出即可。

但上述实现时空复杂度均较高,我们可以通过矩阵运算巧妙地实现并行的多头计算,其核心逻辑在于使用三个组合矩阵来代替了n个参数矩阵的组合,也就是矩阵内积再拼接其实等同于拼接矩阵再内积。具体实现可以参考下列代码:

\begin{Shaded}
\begin{Highlighting}[]
\ImportTok{import}\NormalTok{ torch.nn }\ImportTok{as}\NormalTok{ nn}
\ImportTok{import}\NormalTok{ torch}

\CommentTok{\textquotesingle{}\textquotesingle{}\textquotesingle{}多头自注意力计算模块\textquotesingle{}\textquotesingle{}\textquotesingle{}}
\KeywordTok{class}\NormalTok{ MultiHeadAttention(nn.Module):}

    \KeywordTok{def} \FunctionTok{\_\_init\_\_}\NormalTok{(}\VariableTok{self}\NormalTok{, args: ModelArgs, is\_causal}\OperatorTok{=}\VariableTok{False}\NormalTok{):}
        \CommentTok{\# 构造函数}
        \CommentTok{\# args: 配置对象}
        \BuiltInTok{super}\NormalTok{().}\FunctionTok{\_\_init\_\_}\NormalTok{()}
        \CommentTok{\# 隐藏层维度必须是头数的整数倍,因为后面我们会将输入拆成头数个矩阵}
        \ControlFlowTok{assert}\NormalTok{ args.dim }\OperatorTok{\%}\NormalTok{ args.n\_heads }\OperatorTok{==} \DecValTok{0}
        \CommentTok{\# 模型并行处理大小,默认为1。}
\NormalTok{        model\_parallel\_size }\OperatorTok{=} \DecValTok{1}
        \CommentTok{\# 本地计算头数,等于总头数除以模型并行处理大小。}
        \VariableTok{self}\NormalTok{.n\_local\_heads }\OperatorTok{=}\NormalTok{ args.n\_heads }\OperatorTok{//}\NormalTok{ model\_parallel\_size}
        \CommentTok{\# 每个头的维度,等于模型维度除以头的总数。}
        \VariableTok{self}\NormalTok{.head\_dim }\OperatorTok{=}\NormalTok{ args.dim }\OperatorTok{//}\NormalTok{ args.n\_heads}

        \CommentTok{\# Wq, Wk, Wv 参数矩阵,每个参数矩阵为 n\_embd x n\_embd}
        \CommentTok{\# 这里通过三个组合矩阵来代替了n个参数矩阵的组合,其逻辑在于矩阵内积再拼接其实等同于拼接矩阵再内积,}
        \CommentTok{\# 不理解的读者可以自行模拟一下,每一个线性层其实相当于n个参数矩阵的拼接}
        \VariableTok{self}\NormalTok{.wq }\OperatorTok{=}\NormalTok{ nn.Linear(args.dim, args.n\_heads }\OperatorTok{*} \VariableTok{self}\NormalTok{.head\_dim, bias}\OperatorTok{=}\VariableTok{False}\NormalTok{)}
        \VariableTok{self}\NormalTok{.wk }\OperatorTok{=}\NormalTok{ nn.Linear(args.dim, args.n\_heads }\OperatorTok{*} \VariableTok{self}\NormalTok{.head\_dim, bias}\OperatorTok{=}\VariableTok{False}\NormalTok{)}
        \VariableTok{self}\NormalTok{.wv }\OperatorTok{=}\NormalTok{ nn.Linear(args.dim, args.n\_heads }\OperatorTok{*} \VariableTok{self}\NormalTok{.head\_dim, bias}\OperatorTok{=}\VariableTok{False}\NormalTok{)}
        \CommentTok{\# 输出权重矩阵,维度为 dim x n\_embd(head\_dim = n\_embeds / n\_heads)}
        \VariableTok{self}\NormalTok{.wo }\OperatorTok{=}\NormalTok{ nn.Linear(args.n\_heads }\OperatorTok{*} \VariableTok{self}\NormalTok{.head\_dim, args.dim, bias}\OperatorTok{=}\VariableTok{False}\NormalTok{)}
        \CommentTok{\# 注意力的 dropout}
        \VariableTok{self}\NormalTok{.attn\_dropout }\OperatorTok{=}\NormalTok{ nn.Dropout(args.dropout)}
        \CommentTok{\# 残差连接的 dropout}
        \VariableTok{self}\NormalTok{.resid\_dropout }\OperatorTok{=}\NormalTok{ nn.Dropout(args.dropout)}
         
        \CommentTok{\# 创建一个上三角矩阵,用于遮蔽未来信息}
        \CommentTok{\# 注意,因为是多头注意力,Mask 矩阵比之前我们定义的多一个维度}
        \ControlFlowTok{if}\NormalTok{ is\_causal:}
\NormalTok{           mask }\OperatorTok{=}\NormalTok{ torch.full((}\DecValTok{1}\NormalTok{, }\DecValTok{1}\NormalTok{, args.max\_seq\_len, args.max\_seq\_len), }\BuiltInTok{float}\NormalTok{(}\StringTok{"{-}inf"}\NormalTok{))}
\NormalTok{           mask }\OperatorTok{=}\NormalTok{ torch.triu(mask, diagonal}\OperatorTok{=}\DecValTok{1}\NormalTok{)}
           \CommentTok{\# 注册为模型的缓冲区}
           \VariableTok{self}\NormalTok{.register\_buffer(}\StringTok{"mask"}\NormalTok{, mask)}

    \KeywordTok{def}\NormalTok{ forward(}\VariableTok{self}\NormalTok{, q: torch.Tensor, k: torch.Tensor, v: torch.Tensor):}

        \CommentTok{\# 获取批次大小和序列长度,[batch\_size, seq\_len, dim]}
\NormalTok{        bsz, seqlen, \_ }\OperatorTok{=}\NormalTok{ q.shape}

        \CommentTok{\# 计算查询(Q)、键(K)、值(V),输入通过参数矩阵层,维度为 (B, T, n\_embed) x (n\_embed, n\_embed) {-}\textgreater{} (B, T, n\_embed)}
\NormalTok{        xq, xk, xv }\OperatorTok{=} \VariableTok{self}\NormalTok{.wq(q), }\VariableTok{self}\NormalTok{.wk(k), }\VariableTok{self}\NormalTok{.wv(v)}

        \CommentTok{\# 将 Q、K、V 拆分成多头,维度为 (B, T, n\_head, C // n\_head),然后交换维度,变成 (B, n\_head, T, C // n\_head)}
        \CommentTok{\# 因为在注意力计算中我们是取了后两个维度参与计算}
        \CommentTok{\# 为什么要先按B*T*n\_head*C//n\_head展开再互换1、2维度而不是直接按注意力输入展开,是因为view的展开方式是直接把输入全部排开,}
        \CommentTok{\# 然后按要求构造,可以发现只有上述操作能够实现我们将每个头对应部分取出来的目标}
\NormalTok{        xq }\OperatorTok{=}\NormalTok{ xq.view(bsz, seqlen, }\VariableTok{self}\NormalTok{.n\_local\_heads, }\VariableTok{self}\NormalTok{.head\_dim)}
\NormalTok{        xk }\OperatorTok{=}\NormalTok{ xk.view(bsz, seqlen, }\VariableTok{self}\NormalTok{.n\_local\_heads, }\VariableTok{self}\NormalTok{.head\_dim)}
\NormalTok{        xv }\OperatorTok{=}\NormalTok{ xv.view(bsz, seqlen, }\VariableTok{self}\NormalTok{.n\_local\_heads, }\VariableTok{self}\NormalTok{.head\_dim)}
\NormalTok{        xq }\OperatorTok{=}\NormalTok{ xq.transpose(}\DecValTok{1}\NormalTok{, }\DecValTok{2}\NormalTok{)}
\NormalTok{        xk }\OperatorTok{=}\NormalTok{ xk.transpose(}\DecValTok{1}\NormalTok{, }\DecValTok{2}\NormalTok{)}
\NormalTok{        xv }\OperatorTok{=}\NormalTok{ xv.transpose(}\DecValTok{1}\NormalTok{, }\DecValTok{2}\NormalTok{)}


        \CommentTok{\# 注意力计算}
        \CommentTok{\# 计算 QK\^{}T / sqrt(d\_k),维度为 (B, nh, T, hs) x (B, nh, hs, T) {-}\textgreater{} (B, nh, T, T)}
\NormalTok{        scores }\OperatorTok{=}\NormalTok{ torch.matmul(xq, xk.transpose(}\DecValTok{2}\NormalTok{, }\DecValTok{3}\NormalTok{)) }\OperatorTok{/}\NormalTok{ math.sqrt(}\VariableTok{self}\NormalTok{.head\_dim)}
        \CommentTok{\# 掩码自注意力必须有注意力掩码}
        \ControlFlowTok{if} \VariableTok{self}\NormalTok{.is\_causal:}
            \ControlFlowTok{assert} \BuiltInTok{hasattr}\NormalTok{(}\VariableTok{self}\NormalTok{, }\StringTok{\textquotesingle{}mask\textquotesingle{}}\NormalTok{)}
            \CommentTok{\# 这里截取到序列长度,因为有些序列可能比 max\_seq\_len 短}
\NormalTok{            scores }\OperatorTok{=}\NormalTok{ scores }\OperatorTok{+} \VariableTok{self}\NormalTok{.mask[:, :, :seqlen, :seqlen]}
        \CommentTok{\# 计算 softmax,维度为 (B, nh, T, T)}
\NormalTok{        scores }\OperatorTok{=}\NormalTok{ F.softmax(scores.}\BuiltInTok{float}\NormalTok{(), dim}\OperatorTok{={-}}\DecValTok{1}\NormalTok{).type\_as(xq)}
        \CommentTok{\# 做 Dropout}
\NormalTok{        scores }\OperatorTok{=} \VariableTok{self}\NormalTok{.attn\_dropout(scores)}
        \CommentTok{\# V * Score,维度为(B, nh, T, T) x (B, nh, T, hs) {-}\textgreater{} (B, nh, T, hs)}
\NormalTok{        output }\OperatorTok{=}\NormalTok{ torch.matmul(scores, xv)}

        \CommentTok{\# 恢复时间维度并合并头。}
        \CommentTok{\# 将多头的结果拼接起来, 先交换维度为 (B, T, n\_head, C // n\_head),再拼接成 (B, T, n\_head * C // n\_head)}
        \CommentTok{\# contiguous 函数用于重新开辟一块新内存存储,因为Pytorch设置先transpose再view会报错,}
        \CommentTok{\# 因为view直接基于底层存储得到,然而transpose并不会改变底层存储,因此需要额外存储}
\NormalTok{        output }\OperatorTok{=}\NormalTok{ output.transpose(}\DecValTok{1}\NormalTok{, }\DecValTok{2}\NormalTok{).contiguous().view(bsz, seqlen, }\OperatorTok{{-}}\DecValTok{1}\NormalTok{)}

        \CommentTok{\# 最终投影回残差流。}
\NormalTok{        output }\OperatorTok{=} \VariableTok{self}\NormalTok{.wo(output)}
\NormalTok{        output }\OperatorTok{=} \VariableTok{self}\NormalTok{.resid\_dropout(output)}
        \ControlFlowTok{return}\NormalTok{ output}
\end{Highlighting}
\end{Shaded}

\subsection{2.2 Encoder-Decoder}\label{encoder-decoder}

在上一节,我们详细介绍了 Transformer
的核心------注意力机制。在《Attention is All You
Need》一文中,作者通过仅使用注意力机制而抛弃传统的 RNN、CNN 架构搭建出
Transformer 模型,从而带来了 NLP 领域的大变革。在 Transformer
中,使用注意力机制的是其两个核心组件------Encoder(编码器)和
Decoder(解码器)。事实上,后续基于 Transformer
架构而来的预训练语言模型基本都是对 Encoder-Decoder
部分进行改进来构建新的模型架构,例如只使用 Encoder 的 BERT、只使用
Decoder 的 GPT 等。

在本节中,我们将以上一节所介绍的 注意力机制为基础,从 Transformer
所针对的 Seq2Seq 任务出发,解析 Transformer 的 Encoder-Decoder 结构。

\subsubsection{2.2.1 Seq2Seq 模型}\label{seq2seq-ux6a21ux578b}

Seq2Seq,即序列到序列,是一种经典 NLP
任务。具体而言,是指模型输入的是一个自然语言序列
\(input = (x_1, x_2, x_3...x_n)\) ,输出的是一个可能不等长的自然语言序列
\(output = (y_1, y_2, y_3...y_m)\) 。事实上,Seq2Seq 是 NLP
最经典的任务,几乎所有的 NLP 任务都可以视为 Seq2Seq
任务。例如文本分类任务,可以视为输出长度为 1 的目标序列(如在上式中
\(m\) =
1);词性标注任务,可以视为输出与输入序列等长的目标序列(如在上式中
\(m\) = \(n\) )。

机器翻译任务即是一个经典的 Seq2Seq
任务,例如,我们的输入可能是``今天天气真好'',输出是``Today is a good
day.''。Transformer 是一个经典的 Seq2Seq
模型,即模型的输入为文本序列,输出为另一个文本序列。事实上,Transformer
一开始正是应用在机器翻译任务上的。

对于 Seq2Seq
任务,一般的思路是对自然语言序列进行编码再解码。所谓编码,就是将输入的自然语言序列通过隐藏层编码成能够表征语义的向量(或矩阵),可以简单理解为更复杂的词向量表示。而解码,就是对输入的自然语言序列编码得到的向量或矩阵通过隐藏层输出,再解码成对应的自然语言目标序列。通过编码再解码,就可以实现
Seq2Seq 任务。

Transformer 中的 Encoder,就是用于上述的编码过程;Decoder
则用于上述的解码过程。Transformer 结构,如图2.5所示:

图2.5 编码器-解码器结构

Transformer 由 Encoder 和 Decoder 组成,每一个 Encoder(Decoder)又由
6个 Encoder(Decoder)Layer 组成。输入源序列会进入 Encoder 进行编码,到
Encoder Layer 的最顶层再将编码结果输出给 Decoder Layer 的每一层,通过
Decoder 解码后就可以得到输出目标序列了。

接下来,我们将首先介绍 Encoder 和 Decoder
内部传统神经网络的经典结构------前馈神经网络(FNN)、层归一化(Layer
Norm)和残差连接(Residual Connection),然后进一步分析 Encoder 和
Decoder 的内部结构。

\subsubsection{2.2.2
前馈神经网络}\label{ux524dux9988ux795eux7ecfux7f51ux7edc}

前馈神经网络(Feed Forward Neural Network,下简称
FFN),也就是我们在上一节提过的每一层的神经元都和上下两层的每一个神经元完全连接的网络结构。每一个
Encoder Layer
都包含一个上文讲的注意力机制和一个前馈神经网络。前馈神经网络的实现是较为简单的:

\begin{Shaded}
\begin{Highlighting}[]
\KeywordTok{class}\NormalTok{ MLP(nn.Module):}
    \CommentTok{\textquotesingle{}\textquotesingle{}\textquotesingle{}前馈神经网络\textquotesingle{}\textquotesingle{}\textquotesingle{}}
    \KeywordTok{def} \FunctionTok{\_\_init\_\_}\NormalTok{(}\VariableTok{self}\NormalTok{, dim: }\BuiltInTok{int}\NormalTok{, hidden\_dim: }\BuiltInTok{int}\NormalTok{, dropout: }\BuiltInTok{float}\NormalTok{):}
        \BuiltInTok{super}\NormalTok{().}\FunctionTok{\_\_init\_\_}\NormalTok{()}
        \CommentTok{\# 定义第一层线性变换,从输入维度到隐藏维度}
        \VariableTok{self}\NormalTok{.w1 }\OperatorTok{=}\NormalTok{ nn.Linear(dim, hidden\_dim, bias}\OperatorTok{=}\VariableTok{False}\NormalTok{)}
        \CommentTok{\# 定义第二层线性变换,从隐藏维度到输入维度}
        \VariableTok{self}\NormalTok{.w2 }\OperatorTok{=}\NormalTok{ nn.Linear(hidden\_dim, dim, bias}\OperatorTok{=}\VariableTok{False}\NormalTok{)}
        \CommentTok{\# 定义dropout层,用于防止过拟合}
        \VariableTok{self}\NormalTok{.dropout }\OperatorTok{=}\NormalTok{ nn.Dropout(dropout)}

    \KeywordTok{def}\NormalTok{ forward(}\VariableTok{self}\NormalTok{, x):}
        \CommentTok{\# 前向传播函数}
        \CommentTok{\# 首先,输入x通过第一层线性变换和RELU激活函数}
        \CommentTok{\# 然后,结果乘以输入x通过第三层线性变换的结果}
        \CommentTok{\# 最后,通过第二层线性变换和dropout层}
        \ControlFlowTok{return} \VariableTok{self}\NormalTok{.dropout(}\VariableTok{self}\NormalTok{.w2(F.relu(}\VariableTok{self}\NormalTok{.w1(x))))}
    
\end{Highlighting}
\end{Shaded}

注意,Transformer 的前馈神经网络是由两个线性层中间加一个 RELU
激活函数组成的,以及前馈神经网络还加入了一个 Dropout 层来防止过拟合。

\subsubsection{2.2.3 层归一化}\label{ux5c42ux5f52ux4e00ux5316}

层归一化,也就是 Layer
Norm,是深度学习中经典的归一化操作。神经网络主流的归一化一般有两种,批归一化(Batch
Norm)和层归一化(Layer Norm)。

归一化核心是为了让不同层输入的取值范围或者分布能够比较一致。由于深度神经网络中每一层的输入都是上一层的输出,因此多层传递下,对网络中较高的层,之前的所有神经层的参数变化会导致其输入的分布发生较大的改变。也就是说,随着神经网络参数的更新,各层的输出分布是不相同的,且差异会随着网络深度的增大而增大。但是,需要预测的条件分布始终是相同的,从而也就造成了预测的误差。

因此,在深度神经网络中,往往需要归一化操作,将每一层的输入都归一化成标准正态分布。批归一化是指在一个
mini-batch 上进行归一化,相当于对一个 batch
对样本拆分出来一部分,首先计算样本的均值:

\[
\mu_j = \frac{1}{m}\sum^{m}_{i=1}Z_j^{i}
\]

其中,\(Z_j^{i}\) 是样本 i 在第 j 个维度上的值,m 就是 mini-batch
的大小。

再计算样本的方差:

\[
\sigma^2 = \frac{1}{m}\sum^{m}_{i=1}(Z_j^i - \mu_j)^2
\]

最后,对每个样本的值减去均值再除以标准差来将这一个 mini-batch
的样本的分布转化为标准正态分布:

\[
\widetilde{Z_j} = \frac{Z_j - \mu_j}{\sqrt{\sigma^2 + \epsilon}}
\]

此处加上 \(\epsilon\) 这一极小量是为了避免分母为0。

但是,批归一化存在一些缺陷,例如:

\begin{itemize}
\tightlist
\item
  当显存有限,mini-batch 较小时,Batch Norm
  取的样本的均值和方差不能反映全局的统计分布信息,从而导致效果变差;
\item
  对于在时间维度展开的 RNN,不同句子的同一分布大概率不同,所以 Batch
  Norm 的归一化会失去意义;
\item
  在训练时,Batch Norm 需要保存每个 step
  的统计信息(均值和方差)。在测试时,由于变长句子的特性,测试集可能出现比训练集更长的句子,所以对于后面位置的
  step,是没有训练的统计量使用的;
\item
  应用 Batch Norm,每个 step 都需要去保存和计算 batch 统计量,耗时又耗力
\end{itemize}

因此,出现了在深度神经网络中更常用、效果更好的层归一化(Layer
Norm)。相较于 Batch Norm 在每一层统计所有样本的均值和方差,Layer Norm
在每个样本上计算其所有层的均值和方差,从而使每个样本的分布达到稳定。Layer
Norm 的归一化方式其实和 Batch Norm
是完全一样的,只是统计统计量的维度不同。

基于上述进行归一化的公式,我们可以简单地实现一个 Layer Norm 层:

\begin{Shaded}
\begin{Highlighting}[]
\KeywordTok{class}\NormalTok{ LayerNorm(nn.Module):}
    \CommentTok{\textquotesingle{}\textquotesingle{}\textquotesingle{} Layer Norm 层\textquotesingle{}\textquotesingle{}\textquotesingle{}}
    \KeywordTok{def} \FunctionTok{\_\_init\_\_}\NormalTok{(}\VariableTok{self}\NormalTok{, features, eps}\OperatorTok{=}\FloatTok{1e{-}6}\NormalTok{):}
    \BuiltInTok{super}\NormalTok{(LayerNorm, }\VariableTok{self}\NormalTok{).}\FunctionTok{\_\_init\_\_}\NormalTok{()}
    \CommentTok{\# 线性矩阵做映射}
    \VariableTok{self}\NormalTok{.a\_2 }\OperatorTok{=}\NormalTok{ nn.Parameter(torch.ones(features))}
    \VariableTok{self}\NormalTok{.b\_2 }\OperatorTok{=}\NormalTok{ nn.Parameter(torch.zeros(features))}
    \VariableTok{self}\NormalTok{.eps }\OperatorTok{=}\NormalTok{ eps}
    
    \KeywordTok{def}\NormalTok{ forward(}\VariableTok{self}\NormalTok{, x):}
    \CommentTok{\# 在统计每个样本所有维度的值,求均值和方差}
\NormalTok{    mean }\OperatorTok{=}\NormalTok{ x.mean(}\OperatorTok{{-}}\DecValTok{1}\NormalTok{, keepdim}\OperatorTok{=}\VariableTok{True}\NormalTok{) }\CommentTok{\# mean: [bsz, max\_len, 1]}
\NormalTok{    std }\OperatorTok{=}\NormalTok{ x.std(}\OperatorTok{{-}}\DecValTok{1}\NormalTok{, keepdim}\OperatorTok{=}\VariableTok{True}\NormalTok{) }\CommentTok{\# std: [bsz, max\_len, 1]}
    \CommentTok{\# 注意这里也在最后一个维度发生了广播}
    \ControlFlowTok{return} \VariableTok{self}\NormalTok{.a\_2 }\OperatorTok{*}\NormalTok{ (x }\OperatorTok{{-}}\NormalTok{ mean) }\OperatorTok{/}\NormalTok{ (std }\OperatorTok{+} \VariableTok{self}\NormalTok{.eps) }\OperatorTok{+} \VariableTok{self}\NormalTok{.b\_2}
\end{Highlighting}
\end{Shaded}

注意,在我们上文实现的 Layer Norm 层中,有两个线性矩阵进行映射。

\subsubsection{2.2.4 残差连接}\label{ux6b8bux5deeux8fdeux63a5}

由于 Transformer
模型结构较复杂、层数较深,\hspace{0pt}为了避免模型退化,Transformer
采用了残差连接的思想来连接每一个子层。残差连接,即下一层的输入不仅是上一层的输出,还包括上一层的输入。残差连接允许最底层信息直接传到最高层,让高层专注于残差的学习。

\hspace{0pt}例如,在 Encoder
中,在第一个子层,输入进入多头自注意力层的同时会直接传递到该层的输出,然后该层的输出会与原输入相加,再进行标准化。在第二个子层也是一样。即:

\[
x = x + MultiHeadSelfAttention(LayerNorm(x))
\]

\[
output = x + FNN(LayerNorm(x))
\]

我们在代码实现中,通过在层的 forward 计算中加上原值来实现残差连接:

\begin{Shaded}
\begin{Highlighting}[]
\CommentTok{\# 注意力计算}
\NormalTok{h }\OperatorTok{=}\NormalTok{ x }\OperatorTok{+} \VariableTok{self}\NormalTok{.attention.forward(}\VariableTok{self}\NormalTok{.attention\_norm(x))}
\CommentTok{\# 经过前馈神经网络}
\NormalTok{out }\OperatorTok{=}\NormalTok{ h }\OperatorTok{+} \VariableTok{self}\NormalTok{.feed\_forward.forward(}\VariableTok{self}\NormalTok{.fnn\_norm(h))}
\end{Highlighting}
\end{Shaded}

在上文代码中,self.attention\_norm 和 self.fnn\_norm 都是 LayerNorm
层,self.attn 是注意力层,而 self.feed\_forward 是前馈神经网络。

\subsubsection{2.2.5 Encoder}\label{encoder}

在实现上述组件之后,我们可以搭建起 Transformer 的 Encoder。Encoder 由 N
个 Encoder Layer 组成,每一个 Encoder Layer
包括一个注意力层和一个前馈神经网络。因此,我们可以首先实现一个 Encoder
Layer:

\begin{Shaded}
\begin{Highlighting}[]
\KeywordTok{class}\NormalTok{ EncoderLayer(nn.Module):}
  \CommentTok{\textquotesingle{}\textquotesingle{}\textquotesingle{}Encoder层\textquotesingle{}\textquotesingle{}\textquotesingle{}}
    \KeywordTok{def} \FunctionTok{\_\_init\_\_}\NormalTok{(}\VariableTok{self}\NormalTok{, args):}
        \BuiltInTok{super}\NormalTok{().}\FunctionTok{\_\_init\_\_}\NormalTok{()}
        \CommentTok{\# 一个 Layer 中有两个 LayerNorm,分别在 Attention 之前和 MLP 之前}
        \VariableTok{self}\NormalTok{.attention\_norm }\OperatorTok{=}\NormalTok{ LayerNorm(args.n\_embd)}
        \CommentTok{\# Encoder 不需要掩码,传入 is\_causal=False}
        \VariableTok{self}\NormalTok{.attention }\OperatorTok{=}\NormalTok{ MultiHeadAttention(args, is\_causal}\OperatorTok{=}\VariableTok{False}\NormalTok{)}
        \VariableTok{self}\NormalTok{.fnn\_norm }\OperatorTok{=}\NormalTok{ LayerNorm(args.n\_embd)}
        \VariableTok{self}\NormalTok{.feed\_forward }\OperatorTok{=}\NormalTok{ MLP(args)}

    \KeywordTok{def}\NormalTok{ forward(}\VariableTok{self}\NormalTok{, x):}
        \CommentTok{\# Layer Norm}
\NormalTok{        norm\_x }\OperatorTok{=} \VariableTok{self}\NormalTok{.attention\_norm(x)}
        \CommentTok{\# 自注意力}
\NormalTok{        h }\OperatorTok{=}\NormalTok{ x }\OperatorTok{+} \VariableTok{self}\NormalTok{.attention.forward(norm\_x, norm\_x, norm\_x)}
        \CommentTok{\# 经过前馈神经网络}
\NormalTok{        out }\OperatorTok{=}\NormalTok{ h }\OperatorTok{+} \VariableTok{self}\NormalTok{.feed\_forward.forward(}\VariableTok{self}\NormalTok{.fnn\_norm(h))}
        \ControlFlowTok{return}\NormalTok{ out}
\end{Highlighting}
\end{Shaded}

然后我们搭建一个 Encoder,由 N 个 Encoder Layer 组成,在最后会加入一个
Layer Norm 实现规范化:

\begin{Shaded}
\begin{Highlighting}[]
\KeywordTok{class}\NormalTok{ Encoder(nn.Module):}
    \CommentTok{\textquotesingle{}\textquotesingle{}\textquotesingle{}Encoder 块\textquotesingle{}\textquotesingle{}\textquotesingle{}}
    \KeywordTok{def} \FunctionTok{\_\_init\_\_}\NormalTok{(}\VariableTok{self}\NormalTok{, args):}
        \BuiltInTok{super}\NormalTok{(Encoder, }\VariableTok{self}\NormalTok{).}\FunctionTok{\_\_init\_\_}\NormalTok{() }
        \CommentTok{\# 一个 Encoder 由 N 个 Encoder Layer 组成}
        \VariableTok{self}\NormalTok{.layers }\OperatorTok{=}\NormalTok{ nn.ModuleList([EncoderLayer(args) }\ControlFlowTok{for}\NormalTok{ \_ }\KeywordTok{in} \BuiltInTok{range}\NormalTok{(args.n\_layer)])}
        \VariableTok{self}\NormalTok{.norm }\OperatorTok{=}\NormalTok{ LayerNorm(args.n\_embd)}

    \KeywordTok{def}\NormalTok{ forward(}\VariableTok{self}\NormalTok{, x):}
        \CommentTok{"分别通过 N 层 Encoder Layer"}
        \ControlFlowTok{for}\NormalTok{ layer }\KeywordTok{in} \VariableTok{self}\NormalTok{.layers:}
\NormalTok{            x }\OperatorTok{=}\NormalTok{ layer(x)}
        \ControlFlowTok{return} \VariableTok{self}\NormalTok{.norm(x)}
\end{Highlighting}
\end{Shaded}

通过 Encoder 的输出,就是输入编码之后的结果。

\subsubsection{2.2.6 Decoder}\label{decoder}

类似的,我们也可以先搭建 Decoder Layer,再将 N 个 Decoder Layer 组装为
Decoder。但是和 Encoder 不同的是,Decoder
由两个注意力层和一个前馈神经网络组成。第一个注意力层是一个掩码自注意力层,即使用
Mask 的注意力计算,保证每一个 token 只能使用该 token
之前的注意力分数;第二个注意力层是一个多头注意力层,该层将使用第一个注意力层的输出作为
query,使用 Encoder 的输出作为 key 和
value,来计算注意力分数。最后,再经过前馈神经网络:

\begin{Shaded}
\begin{Highlighting}[]
\KeywordTok{class}\NormalTok{ DecoderLayer(nn.Module):}
  \CommentTok{\textquotesingle{}\textquotesingle{}\textquotesingle{}解码层\textquotesingle{}\textquotesingle{}\textquotesingle{}}
    \KeywordTok{def} \FunctionTok{\_\_init\_\_}\NormalTok{(}\VariableTok{self}\NormalTok{, args):}
        \BuiltInTok{super}\NormalTok{().}\FunctionTok{\_\_init\_\_}\NormalTok{()}
        \CommentTok{\# 一个 Layer 中有三个 LayerNorm,分别在 Mask Attention 之前、Self Attention 之前和 MLP 之前}
        \VariableTok{self}\NormalTok{.attention\_norm\_1 }\OperatorTok{=}\NormalTok{ LayerNorm(args.n\_embd)}
        \CommentTok{\# Decoder 的第一个部分是 Mask Attention,传入 is\_causal=True}
        \VariableTok{self}\NormalTok{.mask\_attention }\OperatorTok{=}\NormalTok{ MultiHeadAttention(args, is\_causal}\OperatorTok{=}\VariableTok{True}\NormalTok{)}
        \VariableTok{self}\NormalTok{.attention\_norm\_2 }\OperatorTok{=}\NormalTok{ LayerNorm(args.n\_embd)}
        \CommentTok{\# Decoder 的第二个部分是 类似于 Encoder 的 Attention,传入 is\_causal=False}
        \VariableTok{self}\NormalTok{.attention }\OperatorTok{=}\NormalTok{ MultiHeadAttention(args, is\_causal}\OperatorTok{=}\VariableTok{False}\NormalTok{)}
        \VariableTok{self}\NormalTok{.ffn\_norm }\OperatorTok{=}\NormalTok{ LayerNorm(args.n\_embd)}
        \CommentTok{\# 第三个部分是 MLP}
        \VariableTok{self}\NormalTok{.feed\_forward }\OperatorTok{=}\NormalTok{ MLP(args)}

    \KeywordTok{def}\NormalTok{ forward(}\VariableTok{self}\NormalTok{, x, enc\_out):}
        \CommentTok{\# Layer Norm}
\NormalTok{        norm\_x }\OperatorTok{=} \VariableTok{self}\NormalTok{.attention\_norm\_1(x)}
        \CommentTok{\# 掩码自注意力}
\NormalTok{        x }\OperatorTok{=}\NormalTok{ x }\OperatorTok{+} \VariableTok{self}\NormalTok{.mask\_attention.forward(norm\_x, norm\_x, norm\_x)}
        \CommentTok{\# 多头注意力}
\NormalTok{        norm\_x }\OperatorTok{=} \VariableTok{self}\NormalTok{.attention\_norm\_2(x)}
\NormalTok{        h }\OperatorTok{=}\NormalTok{ x }\OperatorTok{+} \VariableTok{self}\NormalTok{.attention.forward(norm\_x, enc\_out, enc\_out)}
        \CommentTok{\# 经过前馈神经网络}
\NormalTok{        out }\OperatorTok{=}\NormalTok{ h }\OperatorTok{+} \VariableTok{self}\NormalTok{.feed\_forward.forward(}\VariableTok{self}\NormalTok{.ffn\_norm(h))}
        \ControlFlowTok{return}\NormalTok{ out}
\end{Highlighting}
\end{Shaded}

然后同样的,我们搭建一个 Decoder 块:

\begin{Shaded}
\begin{Highlighting}[]
\KeywordTok{class}\NormalTok{ Decoder(nn.Module):}
    \CommentTok{\textquotesingle{}\textquotesingle{}\textquotesingle{}解码器\textquotesingle{}\textquotesingle{}\textquotesingle{}}
    \KeywordTok{def} \FunctionTok{\_\_init\_\_}\NormalTok{(}\VariableTok{self}\NormalTok{, args):}
        \BuiltInTok{super}\NormalTok{(Decoder, }\VariableTok{self}\NormalTok{).}\FunctionTok{\_\_init\_\_}\NormalTok{() }
        \CommentTok{\# 一个 Decoder 由 N 个 Decoder Layer 组成}
        \VariableTok{self}\NormalTok{.layers }\OperatorTok{=}\NormalTok{ nn.ModuleList([DecoderLayer(args) }\ControlFlowTok{for}\NormalTok{ \_ }\KeywordTok{in} \BuiltInTok{range}\NormalTok{(args.n\_layer)])}
        \VariableTok{self}\NormalTok{.norm }\OperatorTok{=}\NormalTok{ LayerNorm(args.n\_embd)}

    \KeywordTok{def}\NormalTok{ forward(}\VariableTok{self}\NormalTok{, x, enc\_out):}
        \CommentTok{"Pass the input (and mask) through each layer in turn."}
        \ControlFlowTok{for}\NormalTok{ layer }\KeywordTok{in} \VariableTok{self}\NormalTok{.layers:}
\NormalTok{            x }\OperatorTok{=}\NormalTok{ layer(x, enc\_out)}
        \ControlFlowTok{return} \VariableTok{self}\NormalTok{.norm(x)}
\end{Highlighting}
\end{Shaded}

完成上述 Encoder、Decoder 的搭建,就完成了 Transformer
的核心部分,接下来将 Encoder、Decoder 拼接起来再加入 Embedding
层就可以搭建出完整的 Transformer 模型啦。

\subsection{2.3 搭建一个
Transformer}\label{ux642dux5efaux4e00ux4e2a-transformer}

在前两章,我们分别深入剖析了 Attention 机制和 Transformer
的核心------Encoder、Decoder
结构,接下来,我们就可以基于上一章实现的组件,搭建起一个完整的
Transformer 模型。

\subsubsection{2.3.1 Embedding 层}\label{embedding-ux5c42}

正如我们在第一章所讲过的,在 NLP
任务中,我们往往需要将自然语言的输入转化为机器可以处理的向量。在深度学习中,承担这个任务的组件就是
Embedding 层。

Embedding
层其实是一个存储固定大小的词典的嵌入向量查找表。也就是说,在输入神经网络之前,我们往往会先让自然语言输入通过分词器
tokenizer,分词器的作用是把自然语言输入切分成 token 并转化成一个固定的
index。例如,如果我们将词表大小设为
4,输入``我喜欢你'',那么,分词器可以将输入转化成:

\begin{verbatim}
input: 我
output: 0

input: 喜欢
output: 1

input:你
output: 2
\end{verbatim}

当然,在实际情况下,tokenizer
的工作会比这更复杂。例如,分词有多种不同的方式,可以切分成词、切分成子词、切分成字符等,而词表大小则往往高达数万数十万。此处我们不赘述
tokenizer 的详细情况,在后文会详细介绍大模型的 tokenizer
是如何运行和训练的。

因此,Embedding 层的输入往往是一个形状为
(batch\_size,seq\_len,1)的矩阵,第一个维度是一次批处理的数量,第二个维度是自然语言序列的长度,第三个维度则是
token 经过 tokenizer 转化成的 index 值。例如,对上述输入,Embedding
层的输入会是:

\begin{verbatim}
[[[0],[1],[2]]]
\end{verbatim}

其 batch\_size 为1,seq\_len 为3,转化出来的 index 如上。

而 Embedding
内部其实是一个可训练的(Vocab\_size,embedding\_dim)的权重矩阵,词表里的每一个值,都对应一行维度为
embedding\_dim
的向量。对于输入的值,会对应到这个词向量,然后拼接成(batch\_size,seq\_len,embedding\_dim)的矩阵输出。

上述实现并不复杂,我们可以直接使用 torch 中的 Embedding 层:

\begin{Shaded}
\begin{Highlighting}[]
\VariableTok{self}\NormalTok{.tok\_embeddings }\OperatorTok{=}\NormalTok{ nn.Embedding(args.vocab\_size, args.dim)}
\end{Highlighting}
\end{Shaded}

\subsubsection{2.3.2 位置编码}\label{ux4f4dux7f6eux7f16ux7801}

注意力机制可以实现良好的并行计算,但同时,其注意力计算的方式也导致序列中相对位置的丢失。在
RNN、LSTM
中,输入序列会沿着语句本身的顺序被依次递归处理,因此输入序列的顺序提供了极其重要的信息,这也和自然语言的本身特性非常吻合。

但从上文对注意力机制的分析我们可以发现,在注意力机制的计算过程中,对于序列中的每一个
token,其他各个位置对其来说都是平等的,即``我喜欢你''和``你喜欢我''在注意力机制看来是完全相同的,但无疑这是注意力机制存在的一个巨大问题。因此,为使用序列顺序信息,保留序列中的相对位置信息,Transformer
采用了位置编码机制,该机制也在之后被多种模型沿用。

\hspace{0pt}位置编码,即根据序列中 token
的相对位置对其进行编码,再将位置编码加入词向量编码中。位置编码的方式有很多,Transformer
使用了正余弦函数来进行位置编码(绝对位置编码Sinusoidal),其编码方式为:

\[
PE(pos, 2i) = sin(pos/10000^{2i/d_{model}})\\
PE(pos, 2i+1) = cos(pos/10000^{2i/d_{model}})
\]

\hspace{0pt}上式中,pos 为 token 在句子中的位置,2i 和 2i+1 则是指示了
token 是奇数位置还是偶数位置,从上式中我们可以看出对于奇数位置的 token
和偶数位置的 token,Transformer 采用了不同的函数进行编码。

我们以一个简单的例子来说明位置编码的计算过程:假如我们输入的是一个长度为
4 的句子''I like to code'',我们可以得到下面的词向量矩阵 \(\rm x\)
,其中每一行代表的就是一个词向量, \(\rm x_0=[0.1,0.2,0.3,0.4]\)
对应的就是``I''的词向量,它的pos就是为0,以此类推,第二行代表的是``like''的词向量,它的pos就是1:

\[
\rm x = \begin{bmatrix} 0.1 & 0.2 & 0.3 & 0.4 \\ 0.2 & 0.3 & 0.4 & 0.5 \\ 0.3 & 0.4 & 0.5 & 0.6 \\ 0.4 & 0.5 & 0.6 & 0.7 \end{bmatrix}
\]

\hspace{0pt}则经过位置编码后的词向量为:

\[
\rm x_{PE} = \begin{bmatrix} 0.1 & 0.2 & 0.3 & 0.4 \\ 0.2 & 0.3 & 0.4 & 0.5 \\ 0.3 & 0.4 & 0.5 & 0.6 \\ 0.4 & 0.5 & 0.6 & 0.7 \end{bmatrix} + \begin{bmatrix} \sin(\frac{0}{10000^0}) & \cos(\frac{0}{10000^0}) & \sin(\frac{0}{10000^{2/4}}) & \cos(\frac{0}{10000^{2/4}}) \\ \sin(\frac{1}{10000^0}) & \cos(\frac{1}{10000^0}) & \sin(\frac{1}{10000^{2/4}}) & \cos(\frac{1}{10000^{2/4}}) \\ \sin(\frac{2}{10000^0}) & \cos(\frac{2}{10000^0}) & \sin(\frac{2}{10000^{2/4}}) & \cos(\frac{2}{10000^{2/4}}) \\ \sin(\frac{3}{10000^0}) & \cos(\frac{3}{10000^0}) & \sin(\frac{3}{10000^{2/4}}) & \cos(\frac{3}{10000^{2/4}}) \end{bmatrix} = \begin{bmatrix} 0.1 & 1.2 & 0.3 & 1.4 \\ 1.041 & 0.84 & 0.41 & 1.49 \\ 1.209 & -0.016 & 0.52 & 1.59 \\ 0.541 & -0.489 & 0.895 & 1.655 \end{bmatrix}
\]

我们可以使用如下的代码来获取上述例子的位置编码:

\begin{Shaded}
\begin{Highlighting}[]
\ImportTok{import}\NormalTok{ numpy }\ImportTok{as}\NormalTok{ np}
\ImportTok{import}\NormalTok{ matplotlib.pyplot }\ImportTok{as}\NormalTok{ plt}
\KeywordTok{def}\NormalTok{ PositionEncoding(seq\_len, d\_model, n}\OperatorTok{=}\DecValTok{10000}\NormalTok{):}
\NormalTok{    P }\OperatorTok{=}\NormalTok{ np.zeros((seq\_len, d\_model))}
    \ControlFlowTok{for}\NormalTok{ k }\KeywordTok{in} \BuiltInTok{range}\NormalTok{(seq\_len):}
        \ControlFlowTok{for}\NormalTok{ i }\KeywordTok{in}\NormalTok{ np.arange(}\BuiltInTok{int}\NormalTok{(d\_model}\OperatorTok{/}\DecValTok{2}\NormalTok{)):}
\NormalTok{            denominator }\OperatorTok{=}\NormalTok{ np.power(n, }\DecValTok{2}\OperatorTok{*}\NormalTok{i}\OperatorTok{/}\NormalTok{d\_model)}
\NormalTok{            P[k, }\DecValTok{2}\OperatorTok{*}\NormalTok{i] }\OperatorTok{=}\NormalTok{ np.sin(k}\OperatorTok{/}\NormalTok{denominator)}
\NormalTok{            P[k, }\DecValTok{2}\OperatorTok{*}\NormalTok{i}\OperatorTok{+}\DecValTok{1}\NormalTok{] }\OperatorTok{=}\NormalTok{ np.cos(k}\OperatorTok{/}\NormalTok{denominator)}
    \ControlFlowTok{return}\NormalTok{ P}

\NormalTok{P }\OperatorTok{=}\NormalTok{ PositionEncoding(seq\_len}\OperatorTok{=}\DecValTok{4}\NormalTok{, d\_model}\OperatorTok{=}\DecValTok{4}\NormalTok{, n}\OperatorTok{=}\DecValTok{100}\NormalTok{)}
\BuiltInTok{print}\NormalTok{(P)}
\end{Highlighting}
\end{Shaded}

\begin{Shaded}
\begin{Highlighting}[]
\NormalTok{[[ }\FloatTok{0.}          \FloatTok{1.}          \FloatTok{0.}          \FloatTok{1.}\NormalTok{        ]}
\NormalTok{ [ }\FloatTok{0.84147098}  \FloatTok{0.54030231}  \FloatTok{0.09983342}  \FloatTok{0.99500417}\NormalTok{]}
\NormalTok{ [ }\FloatTok{0.90929743} \OperatorTok{{-}}\FloatTok{0.41614684}  \FloatTok{0.19866933}  \FloatTok{0.98006658}\NormalTok{]}
\NormalTok{ [ }\FloatTok{0.14112001} \OperatorTok{{-}}\FloatTok{0.9899925}   \FloatTok{0.29552021}  \FloatTok{0.95533649}\NormalTok{]]}
\end{Highlighting}
\end{Shaded}

这样的位置编码主要有两个好处:

\begin{enumerate}
\def\labelenumi{\arabic{enumi}.}
\tightlist
\item
  使 PE
  能够适应比训练集里面所有句子更长的句子,假设训练集里面最长的句子是有
  20 个单词,突然来了一个长度为 21
  的句子,则使用公式计算的方法可以计算出第 21 位的 Embedding。
\item
  可以让模型容易地计算出相对位置,对于固定长度的间距 k,PE(pos+k) 可以用
  PE(pos) 计算得到。因为 Sin(A+B) = Sin(A)Cos(B) + Cos(A)Sin(B),
  Cos(A+B) = Cos(A)Cos(B) - Sin(A)Sin(B)。
\end{enumerate}

我们也可以通过严谨的数学推导证明该编码方式的优越性。原始的 Transformer
Embedding 可以表示为:

\[
\begin{equation}f(\cdots,\boldsymbol{x}_m,\cdots,\boldsymbol{x}_n,\cdots)=f(\cdots,\boldsymbol{x}_n,\cdots,\boldsymbol{x}_m,\cdots)\end{equation}
\]

很明显,这样的函数是不具有不对称性的,也就是无法表征相对位置信息。我们想要得到这样一种编码方式:

\[
\begin{equation}\tilde{f}(\cdots,\boldsymbol{x}_m,\cdots,\boldsymbol{x}_n,\cdots)=f(\cdots,\boldsymbol{x}_m + \boldsymbol{p}_m,\cdots,\boldsymbol{x}_n + \boldsymbol{p}_n,\cdots)\end{equation}
\]

这里加上的 \(p_m\), \(p_n\) 就是位置编码。接下来我们将
\(f(...,x_m+p_m,...,x_n+p_n)\) 在 m,n 两个位置上做泰勒展开:

\[
\begin{equation}\tilde{f}\approx f + \boldsymbol{p}_m^{\top} \frac{\partial f}{\partial \boldsymbol{x}_m} + \boldsymbol{p}_n^{\top} \frac{\partial f}{\partial \boldsymbol{x}_n} + \frac{1}{2}\boldsymbol{p}_m^{\top} \frac{\partial^2 f}{\partial \boldsymbol{x}_m^2}\boldsymbol{p}_m + \frac{1}{2}\boldsymbol{p}_n^{\top} \frac{\partial^2 f}{\partial \boldsymbol{x}_n^2}\boldsymbol{p}_n + \underbrace{\boldsymbol{p}_m^{\top} \frac{\partial^2 f}{\partial \boldsymbol{x}_m \partial \boldsymbol{x}_n}\boldsymbol{p}_n}_{\boldsymbol{p}_m^{\top} \boldsymbol{\mathcal{H}} \boldsymbol{p}_n}\end{equation}
\]

可以看到第1项与位置无关,2~5项仅依赖单一位置,第6项(f 分别对 m、n
求偏导)与两个位置有关,所以我们希望第六项( \(p_m^THp_n\)
)表达相对位置信息,即求一个函数 g 使得:

\[
p_m^THp_n = g(m-n)
\]

我们假设 \(H\) 是一个单位矩阵,则:

\[
p_m^THp_n = p_m^Tp_n = \langle\boldsymbol{p}_m, \boldsymbol{p}_n\rangle = g(m-n)
\]

通过将向量 {[}x,y{]} 视为复数 x+yi,基于复数的运算法则构建方程:

\[
\begin{equation}\langle\boldsymbol{p}_m, \boldsymbol{p}_n\rangle = \text{Re}[\boldsymbol{p}_m \boldsymbol{p}_n^*]\end{equation}
\]

再假设存在复数 \(q_{m-n}\) 使得:

\[
\begin{equation}\boldsymbol{p}_m \boldsymbol{p}_n^* = \boldsymbol{q}_{m-n}\end{equation}
\]

使用复数的指数形式求解这个方程,得到二维情形下位置编码的解:

\[
\begin{equation}\boldsymbol{p}_m = e^{\text{i}m\theta}\quad\Leftrightarrow\quad \boldsymbol{p}_m=\begin{pmatrix}\cos m\theta \\ \sin m\theta\end{pmatrix}\end{equation}
\]

由于内积满足线性叠加性,所以更高维的偶数维位置编码,我们可以表示为多个二维位置编码的组合:

\[
\begin{equation}\boldsymbol{p}_m = \begin{pmatrix}e^{\text{i}m\theta_0} \\ e^{\text{i}m\theta_1} \\ \vdots \\ e^{\text{i}m\theta_{d/2-1}}\end{pmatrix}\quad\Leftrightarrow\quad \boldsymbol{p}_m=\begin{pmatrix}\cos m\theta_0 \\ \sin m\theta_0 \\ \cos m\theta_1 \\ \sin m\theta_1 \\ \vdots \\ \cos m\theta_{d/2-1} \\ \sin m\theta_{d/2-1}  \end{pmatrix}\end{equation}
\]

再取
\(\theta_i = 10000^{-2i/d}\)(该形式可以使得随着\textbar m−n\textbar 的增大,⟨pm,pn⟩有着趋于零的趋势,这一点可以通过对位置编码做积分来证明,而
base 取为 10000 是实验结果),就得到了上文的编码方式。

当 \(H\) 不是一个单位矩阵时,因为模型的 Embedding 层所形成的 d
维向量之间任意两个维度的相关性比较小,满足一定的解耦性,我们可以将其视作对角矩阵,那么使用上述编码:

\[
\begin{equation}\boldsymbol{p}_m^{\top} \boldsymbol{\mathcal{H}} \boldsymbol{p}_n=\sum_{i=1}^{d/2} \boldsymbol{\mathcal{H}}_{2i,2i} \cos m\theta_i \cos n\theta_i + \boldsymbol{\mathcal{H}}_{2i+1,2i+1} \sin m\theta_i \sin n\theta_i\end{equation}
\]

通过积化和差:

\[
\begin{equation}\sum_{i=1}^{d/2} \frac{1}{2}\left(\boldsymbol{\mathcal{H}}_{2i,2i} + \boldsymbol{\mathcal{H}}_{2i+1,2i+1}\right) \cos (m-n)\theta_i + \frac{1}{2}\left(\boldsymbol{\mathcal{H}}_{2i,2i} - \boldsymbol{\mathcal{H}}_{2i+1,2i+1}\right) \cos (m+n)\theta_i \end{equation}
\]

说明该编码仍然可以表示相对位置。

上述\hspace{0pt}编码结果,如图2.6所示:

图2.6 编码结果

基于上述原理,我们实现一个\hspace{0pt}位置编码层:

\begin{Shaded}
\begin{Highlighting}[]

\KeywordTok{class}\NormalTok{ PositionalEncoding(nn.Module):}
    \CommentTok{\textquotesingle{}\textquotesingle{}\textquotesingle{}位置编码模块\textquotesingle{}\textquotesingle{}\textquotesingle{}}

    \KeywordTok{def} \FunctionTok{\_\_init\_\_}\NormalTok{(}\VariableTok{self}\NormalTok{, args):}
        \BuiltInTok{super}\NormalTok{(PositionalEncoding, }\VariableTok{self}\NormalTok{).}\FunctionTok{\_\_init\_\_}\NormalTok{()}
        \CommentTok{\# Dropout 层}
        \VariableTok{self}\NormalTok{.dropout }\OperatorTok{=}\NormalTok{ nn.Dropout(p}\OperatorTok{=}\NormalTok{args.dropout)}

        \CommentTok{\# block size 是序列的最大长度}
\NormalTok{        pe }\OperatorTok{=}\NormalTok{ torch.zeros(args.block\_size, args.n\_embd)}
\NormalTok{        position }\OperatorTok{=}\NormalTok{ torch.arange(}\DecValTok{0}\NormalTok{, args.block\_size).unsqueeze(}\DecValTok{1}\NormalTok{)}
        \CommentTok{\# 计算 theta}
\NormalTok{        div\_term }\OperatorTok{=}\NormalTok{ torch.exp(}
\NormalTok{            torch.arange(}\DecValTok{0}\NormalTok{, args.n\_embd, }\DecValTok{2}\NormalTok{) }\OperatorTok{*} \OperatorTok{{-}}\NormalTok{(math.log(}\FloatTok{10000.0}\NormalTok{) }\OperatorTok{/}\NormalTok{ args.n\_embd)}
\NormalTok{        )}
        \CommentTok{\# 分别计算 sin、cos 结果}
\NormalTok{        pe[:, }\DecValTok{0}\NormalTok{::}\DecValTok{2}\NormalTok{] }\OperatorTok{=}\NormalTok{ torch.sin(position }\OperatorTok{*}\NormalTok{ div\_term)}
\NormalTok{        pe[:, }\DecValTok{1}\NormalTok{::}\DecValTok{2}\NormalTok{] }\OperatorTok{=}\NormalTok{ torch.cos(position }\OperatorTok{*}\NormalTok{ div\_term)}
\NormalTok{        pe }\OperatorTok{=}\NormalTok{ pe.unsqueeze(}\DecValTok{0}\NormalTok{)}
        \VariableTok{self}\NormalTok{.register\_buffer(}\StringTok{"pe"}\NormalTok{, pe)}

    \KeywordTok{def}\NormalTok{ forward(}\VariableTok{self}\NormalTok{, x):}
        \CommentTok{\# 将位置编码加到 Embedding 结果上}
\NormalTok{        x }\OperatorTok{=}\NormalTok{ x }\OperatorTok{+} \VariableTok{self}\NormalTok{.pe[:, : x.size(}\DecValTok{1}\NormalTok{)].requires\_grad\_(}\VariableTok{False}\NormalTok{)}
        \ControlFlowTok{return} \VariableTok{self}\NormalTok{.dropout(x)}
\end{Highlighting}
\end{Shaded}

\subsubsection{2.3.3 一个完整的
Transformer}\label{ux4e00ux4e2aux5b8cux6574ux7684-transformer}

上述所有组件,再按照下图的 Tranfromer 结构拼接起来就是一个完整的
Transformer 模型了,如图2.7所示:

图2.7 Transformer 模型结构

但需要注意的是,上图是原论文《Attention is all you need》配图,LayerNorm
层放在了 Attention
层后面,也就是``Post-Norm''结构,但在其发布的源代码中,LayerNorm
层是放在 Attention 层前面的,也就是``Pre Norm''结构。考虑到目前 LLM
一般采用``Pre-Norm''结构(可以使 loss
更稳定),本文在实现时采用``Pre-Norm''结构。

如图,经过 tokenizer 映射后的输出先经过 Embedding 层和 Positional
Embedding 层编码,然后进入上一节讲过的 N 个 Encoder 和 N 个 Decoder(在
Transformer 原模型中,N 取为6),最后经过一个线性层和一个 Softmax
层就得到了最终输出。

基于之前所实现过的组件,我们实现完整的 Transformer 模型:

\begin{Shaded}
\begin{Highlighting}[]
\KeywordTok{class}\NormalTok{ Transformer(nn.Module):}
   \CommentTok{\textquotesingle{}\textquotesingle{}\textquotesingle{}整体模型\textquotesingle{}\textquotesingle{}\textquotesingle{}}
    \KeywordTok{def} \FunctionTok{\_\_init\_\_}\NormalTok{(}\VariableTok{self}\NormalTok{, args):}
        \BuiltInTok{super}\NormalTok{().}\FunctionTok{\_\_init\_\_}\NormalTok{()}
        \CommentTok{\# 必须输入词表大小和 block size}
        \ControlFlowTok{assert}\NormalTok{ args.vocab\_size }\KeywordTok{is} \KeywordTok{not} \VariableTok{None}
        \ControlFlowTok{assert}\NormalTok{ args.block\_size }\KeywordTok{is} \KeywordTok{not} \VariableTok{None}
        \VariableTok{self}\NormalTok{.args }\OperatorTok{=}\NormalTok{ args}
        \VariableTok{self}\NormalTok{.transformer }\OperatorTok{=}\NormalTok{ nn.ModuleDict(}\BuiltInTok{dict}\NormalTok{(}
\NormalTok{            wte }\OperatorTok{=}\NormalTok{ nn.Embedding(args.vocab\_size, args.n\_embd),}
\NormalTok{            wpe }\OperatorTok{=}\NormalTok{ PositionalEncoding(args),}
\NormalTok{            drop }\OperatorTok{=}\NormalTok{ nn.Dropout(args.dropout),}
\NormalTok{            encoder }\OperatorTok{=}\NormalTok{ Encoder(args),}
\NormalTok{            decoder }\OperatorTok{=}\NormalTok{ Decoder(args),}
\NormalTok{        ))}
        \CommentTok{\# 最后的线性层,输入是 n\_embd,输出是词表大小}
        \VariableTok{self}\NormalTok{.lm\_head }\OperatorTok{=}\NormalTok{ nn.Linear(args.n\_embd, args.vocab\_size, bias}\OperatorTok{=}\VariableTok{False}\NormalTok{)}

        \CommentTok{\# 初始化所有的权重}
        \VariableTok{self}\NormalTok{.}\BuiltInTok{apply}\NormalTok{(}\VariableTok{self}\NormalTok{.\_init\_weights)}

        \CommentTok{\# 查看所有参数的数量}
        \BuiltInTok{print}\NormalTok{(}\StringTok{"number of parameters: }\SpecialCharTok{\%.2f}\StringTok{M"} \OperatorTok{\%}\NormalTok{ (}\VariableTok{self}\NormalTok{.get\_num\_params()}\OperatorTok{/}\FloatTok{1e6}\NormalTok{,))}

    \CommentTok{\textquotesingle{}\textquotesingle{}\textquotesingle{}统计所有参数的数量\textquotesingle{}\textquotesingle{}\textquotesingle{}}
    \KeywordTok{def}\NormalTok{ get\_num\_params(}\VariableTok{self}\NormalTok{, non\_embedding}\OperatorTok{=}\VariableTok{False}\NormalTok{):}
        \CommentTok{\# non\_embedding: 是否统计 embedding 的参数}
\NormalTok{        n\_params }\OperatorTok{=} \BuiltInTok{sum}\NormalTok{(p.numel() }\ControlFlowTok{for}\NormalTok{ p }\KeywordTok{in} \VariableTok{self}\NormalTok{.parameters())}
        \CommentTok{\# 如果不统计 embedding 的参数,就减去}
        \ControlFlowTok{if}\NormalTok{ non\_embedding:}
\NormalTok{            n\_params }\OperatorTok{{-}=} \VariableTok{self}\NormalTok{.transformer.wpe.weight.numel()}
        \ControlFlowTok{return}\NormalTok{ n\_params}

    \CommentTok{\textquotesingle{}\textquotesingle{}\textquotesingle{}初始化权重\textquotesingle{}\textquotesingle{}\textquotesingle{}}
    \KeywordTok{def}\NormalTok{ \_init\_weights(}\VariableTok{self}\NormalTok{, module):}
        \CommentTok{\# 线性层和 Embedding 层初始化为正则分布}
        \ControlFlowTok{if} \BuiltInTok{isinstance}\NormalTok{(module, nn.Linear):}
\NormalTok{            torch.nn.init.normal\_(module.weight, mean}\OperatorTok{=}\FloatTok{0.0}\NormalTok{, std}\OperatorTok{=}\FloatTok{0.02}\NormalTok{)}
            \ControlFlowTok{if}\NormalTok{ module.bias }\KeywordTok{is} \KeywordTok{not} \VariableTok{None}\NormalTok{:}
\NormalTok{                torch.nn.init.zeros\_(module.bias)}
        \ControlFlowTok{elif} \BuiltInTok{isinstance}\NormalTok{(module, nn.Embedding):}
\NormalTok{            torch.nn.init.normal\_(module.weight, mean}\OperatorTok{=}\FloatTok{0.0}\NormalTok{, std}\OperatorTok{=}\FloatTok{0.02}\NormalTok{)}
    
    \CommentTok{\textquotesingle{}\textquotesingle{}\textquotesingle{}前向计算函数\textquotesingle{}\textquotesingle{}\textquotesingle{}}
    \KeywordTok{def}\NormalTok{ forward(}\VariableTok{self}\NormalTok{, idx, targets}\OperatorTok{=}\VariableTok{None}\NormalTok{):}
        \CommentTok{\# 输入为 idx,维度为 (batch size, sequence length, 1);targets 为目标序列,用于计算 loss}
\NormalTok{        device }\OperatorTok{=}\NormalTok{ idx.device}
\NormalTok{        b, t }\OperatorTok{=}\NormalTok{ idx.size()}
        \ControlFlowTok{assert}\NormalTok{ t }\OperatorTok{\textless{}=} \VariableTok{self}\NormalTok{.args.block\_size, }\SpecialStringTok{f"不能计算该序列,该序列长度为 }\SpecialCharTok{\{}\NormalTok{t}\SpecialCharTok{\}}\SpecialStringTok{, 最大序列长度只有 }\SpecialCharTok{\{}\VariableTok{self}\SpecialCharTok{.}\NormalTok{args}\SpecialCharTok{.}\NormalTok{block\_size}\SpecialCharTok{\}}\SpecialStringTok{"}

        \CommentTok{\# 通过 self.transformer}
        \CommentTok{\# 首先将输入 idx 通过 Embedding 层,得到维度为 (batch size, sequence length, n\_embd)}
        \BuiltInTok{print}\NormalTok{(}\StringTok{"idx"}\NormalTok{,idx.size())}
        \CommentTok{\# 通过 Embedding 层}
\NormalTok{        tok\_emb }\OperatorTok{=} \VariableTok{self}\NormalTok{.transformer.wte(idx)}
        \BuiltInTok{print}\NormalTok{(}\StringTok{"tok\_emb"}\NormalTok{,tok\_emb.size())}
        \CommentTok{\# 然后通过位置编码}
\NormalTok{        pos\_emb }\OperatorTok{=} \VariableTok{self}\NormalTok{.transformer.wpe(tok\_emb) }
        \CommentTok{\# 再进行 Dropout}
\NormalTok{        x }\OperatorTok{=} \VariableTok{self}\NormalTok{.transformer.drop(pos\_emb)}
        \CommentTok{\# 然后通过 Encoder}
        \BuiltInTok{print}\NormalTok{(}\StringTok{"x after wpe:"}\NormalTok{,x.size())}
\NormalTok{        enc\_out }\OperatorTok{=} \VariableTok{self}\NormalTok{.transformer.encoder(x)}
        \BuiltInTok{print}\NormalTok{(}\StringTok{"enc\_out:"}\NormalTok{,enc\_out.size())}
        \CommentTok{\# 再通过 Decoder}
\NormalTok{        x }\OperatorTok{=} \VariableTok{self}\NormalTok{.transformer.decoder(x, enc\_out)}
        \BuiltInTok{print}\NormalTok{(}\StringTok{"x after decoder:"}\NormalTok{,x.size())}

        \ControlFlowTok{if}\NormalTok{ targets }\KeywordTok{is} \KeywordTok{not} \VariableTok{None}\NormalTok{:}
            \CommentTok{\# 训练阶段,如果我们给了 targets,就计算 loss}
            \CommentTok{\# 先通过最后的 Linear 层,得到维度为 (batch size, sequence length, vocab size)}
\NormalTok{            logits }\OperatorTok{=} \VariableTok{self}\NormalTok{.lm\_head(x)}
            \CommentTok{\# 再跟 targets 计算交叉熵}
\NormalTok{            loss }\OperatorTok{=}\NormalTok{ F.cross\_entropy(logits.view(}\OperatorTok{{-}}\DecValTok{1}\NormalTok{, logits.size(}\OperatorTok{{-}}\DecValTok{1}\NormalTok{)), targets.view(}\OperatorTok{{-}}\DecValTok{1}\NormalTok{), ignore\_index}\OperatorTok{={-}}\DecValTok{1}\NormalTok{)}
        \ControlFlowTok{else}\NormalTok{:}
            \CommentTok{\# 推理阶段,我们只需要 logits,loss 为 None}
            \CommentTok{\# 取 {-}1 是只取序列中的最后一个作为输出}
\NormalTok{            logits }\OperatorTok{=} \VariableTok{self}\NormalTok{.lm\_head(x[:, [}\OperatorTok{{-}}\DecValTok{1}\NormalTok{], :]) }\CommentTok{\# note: using list [{-}1] to preserve the time dim}
\NormalTok{            loss }\OperatorTok{=} \VariableTok{None}

        \ControlFlowTok{return}\NormalTok{ logits, loss}
\end{Highlighting}
\end{Shaded}

注意,上述代码除去搭建了整个 Transformer
结构外,我们还额外实现了三个函数:

\begin{itemize}
\tightlist
\item
  get\_num\_params:用于统计模型的参数量
\item
  \_init\_weights:用于对模型所有参数进行随机初始化
\item
  forward:前向计算函数
\end{itemize}

另外,在前向计算函数中,我们对模型使用 pytorch
的交叉熵函数来计算损失,对于不同的损失函数,读者可以查阅 Pytorch
的官方文档,此处就不再赘述了。

经过上述步骤,我们就可以从零``手搓''一个完整的、可计算的 Transformer
模型。限于本书主要聚焦在 LLM,在本章,我们就不再详细讲述如何训练
Transformer 模型了;在后文中,我们将类似地从零``手搓''一个 LLaMA
模型,并手把手带大家训练一个属于自己的 Tiny LLaMA。

\textbf{参考文献}

{[}1{]} Ashish Vaswani, Noam Shazeer, Niki Parmar, Jakob Uszkoreit,
Llion Jones, Aidan N. Gomez, Lukasz Kaiser, Illia Polosukhin. (2023).
\emph{Attention Is All You Need.} arXiv preprint arXiv:1706.03762.

{[}2{]} Jay Mody 的文章 ``An Intuition for Attention''.
来源:https://jaykmody.com/blog/attention-intuition/

\end{document}


% 第三章
\chapter{预训练语言模型}
\input{chapter3/第三章 预训练语言模型}

% 第四章
\chapter{大语言模型}
\input{chapter4/第四章 大语言模型}

% 第五章
\chapter{动手搭建大模型}
\input{chapter5/第五章 动手搭建大模型}

% 第六章
\chapter{大模型训练流程实践}
% Options for packages loaded elsewhere
\PassOptionsToPackage{unicode}{hyperref}
\PassOptionsToPackage{hyphens}{url}
\documentclass[
]{article}
\usepackage{xcolor}
\usepackage{amsmath,amssymb}
\setcounter{secnumdepth}{5}
\usepackage{iftex}
\ifPDFTeX
  \usepackage[T1]{fontenc}
  \usepackage[utf8]{inputenc}
  \usepackage{textcomp} % provide euro and other symbols
\else % if luatex or xetex
  \usepackage{unicode-math} % this also loads fontspec
  \defaultfontfeatures{Scale=MatchLowercase}
  \defaultfontfeatures[\rmfamily]{Ligatures=TeX,Scale=1}
\fi
\usepackage{lmodern}
\ifPDFTeX\else
  % xetex/luatex font selection
\fi
% Use upquote if available, for straight quotes in verbatim environments
\IfFileExists{upquote.sty}{\usepackage{upquote}}{}
\IfFileExists{microtype.sty}{% use microtype if available
  \usepackage[]{microtype}
  \UseMicrotypeSet[protrusion]{basicmath} % disable protrusion for tt fonts
}{}
\makeatletter
\@ifundefined{KOMAClassName}{% if non-KOMA class
  \IfFileExists{parskip.sty}{%
    \usepackage{parskip}
  }{% else
    \setlength{\parindent}{0pt}
    \setlength{\parskip}{6pt plus 2pt minus 1pt}}
}{% if KOMA class
  \KOMAoptions{parskip=half}}
\makeatother
\usepackage{color}
\usepackage{fancyvrb}
\newcommand{\VerbBar}{|}
\newcommand{\VERB}{\Verb[commandchars=\\\{\}]}
\DefineVerbatimEnvironment{Highlighting}{Verbatim}{commandchars=\\\{\}}
% Add ',fontsize=\small' for more characters per line
\newenvironment{Shaded}{}{}
\newcommand{\AlertTok}[1]{\textcolor[rgb]{1.00,0.00,0.00}{\textbf{#1}}}
\newcommand{\AnnotationTok}[1]{\textcolor[rgb]{0.38,0.63,0.69}{\textbf{\textit{#1}}}}
\newcommand{\AttributeTok}[1]{\textcolor[rgb]{0.49,0.56,0.16}{#1}}
\newcommand{\BaseNTok}[1]{\textcolor[rgb]{0.25,0.63,0.44}{#1}}
\newcommand{\BuiltInTok}[1]{\textcolor[rgb]{0.00,0.50,0.00}{#1}}
\newcommand{\CharTok}[1]{\textcolor[rgb]{0.25,0.44,0.63}{#1}}
\newcommand{\CommentTok}[1]{\textcolor[rgb]{0.38,0.63,0.69}{\textit{#1}}}
\newcommand{\CommentVarTok}[1]{\textcolor[rgb]{0.38,0.63,0.69}{\textbf{\textit{#1}}}}
\newcommand{\ConstantTok}[1]{\textcolor[rgb]{0.53,0.00,0.00}{#1}}
\newcommand{\ControlFlowTok}[1]{\textcolor[rgb]{0.00,0.44,0.13}{\textbf{#1}}}
\newcommand{\DataTypeTok}[1]{\textcolor[rgb]{0.56,0.13,0.00}{#1}}
\newcommand{\DecValTok}[1]{\textcolor[rgb]{0.25,0.63,0.44}{#1}}
\newcommand{\DocumentationTok}[1]{\textcolor[rgb]{0.73,0.13,0.13}{\textit{#1}}}
\newcommand{\ErrorTok}[1]{\textcolor[rgb]{1.00,0.00,0.00}{\textbf{#1}}}
\newcommand{\ExtensionTok}[1]{#1}
\newcommand{\FloatTok}[1]{\textcolor[rgb]{0.25,0.63,0.44}{#1}}
\newcommand{\FunctionTok}[1]{\textcolor[rgb]{0.02,0.16,0.49}{#1}}
\newcommand{\ImportTok}[1]{\textcolor[rgb]{0.00,0.50,0.00}{\textbf{#1}}}
\newcommand{\InformationTok}[1]{\textcolor[rgb]{0.38,0.63,0.69}{\textbf{\textit{#1}}}}
\newcommand{\KeywordTok}[1]{\textcolor[rgb]{0.00,0.44,0.13}{\textbf{#1}}}
\newcommand{\NormalTok}[1]{#1}
\newcommand{\OperatorTok}[1]{\textcolor[rgb]{0.40,0.40,0.40}{#1}}
\newcommand{\OtherTok}[1]{\textcolor[rgb]{0.00,0.44,0.13}{#1}}
\newcommand{\PreprocessorTok}[1]{\textcolor[rgb]{0.74,0.48,0.00}{#1}}
\newcommand{\RegionMarkerTok}[1]{#1}
\newcommand{\SpecialCharTok}[1]{\textcolor[rgb]{0.25,0.44,0.63}{#1}}
\newcommand{\SpecialStringTok}[1]{\textcolor[rgb]{0.73,0.40,0.53}{#1}}
\newcommand{\StringTok}[1]{\textcolor[rgb]{0.25,0.44,0.63}{#1}}
\newcommand{\VariableTok}[1]{\textcolor[rgb]{0.10,0.09,0.49}{#1}}
\newcommand{\VerbatimStringTok}[1]{\textcolor[rgb]{0.25,0.44,0.63}{#1}}
\newcommand{\WarningTok}[1]{\textcolor[rgb]{0.38,0.63,0.69}{\textbf{\textit{#1}}}}
\setlength{\emergencystretch}{3em} % prevent overfull lines
\providecommand{\tightlist}{%
  \setlength{\itemsep}{0pt}\setlength{\parskip}{0pt}}
\usepackage{bookmark}
\IfFileExists{xurl.sty}{\usepackage{xurl}}{} % add URL line breaks if available
\urlstyle{same}
\hypersetup{
  hidelinks,
  pdfcreator={LaTeX via pandoc}}

\author{}
\date{}

\begin{document}

{
\setcounter{tocdepth}{3}
\tableofcontents
}
\section{第六章
大模型训练流程实践}\label{ux7b2cux516dux7ae0-ux5927ux6a21ux578bux8badux7ec3ux6d41ux7a0bux5b9eux8df5}

\subsection{6.1 模型预训练}\label{ux6a21ux578bux9884ux8badux7ec3}

在上一章,我们逐步拆解了 LLM 的模型结构及训练过程,从零手写实现了 LLaMA
模型结构及 Pretrain、SFT 全流程,更深入地理解了 LLM
的模型原理及训练细节。但是,在实际应用中,手写实现的 LLM
训练存在以下问题:

\begin{itemize}
\tightlist
\item
  手写实现 LLM 结构工作量大,难以实时跟进最新模型的结构创新;
\item
  从零实现的 LLM 训练无法较好地实现多卡分布式训练,训练效率较低;
\item
  和现有预训练 LLM 不兼容,无法使用预训练好的模型参数
\end{itemize}

因此,在本章中,我们将介绍目前 LLM 领域的主流训练框架
Transformers,并结合分布式框架 deepspeed、高效微调框架 peft
等主流框架,实践使用 transformers 进行模型 Pretrain、SFT
全流程,更好地对接业界的主流 LLM 技术方案。

\subsubsection{6.1.1 框架介绍}\label{ux6846ux67b6ux4ecbux7ecd}

Transformers 是由 Hugging Face 开发的 NLP 框架,通过模块化设计实现了对
BERT、GPT、LLaMA、T5、ViT 等上百种主流模型架构的统一支持。通过使用
Transformers,开发者无需重复实现基础网络结构,通过 AutoModel
类即可一键加载任意预训练,图6.1 为 Hugging Face Transformers 课程首页:

\begin{verbatim}
<img src="https://raw.githubusercontent.com/datawhalechina/happy-llm/main/docs/images/6-images/1-1.png" alt="alt text" width="90%" />
<p>图6.1 Hugging Face Transformers</p>
\end{verbatim}

同时,框架内置的 Trainer 类封装了分布式训练的核心逻辑,支持 PyTorch 原生
DDP、DeepSpeed、Megatron-LM
等多种分布式训练策略。通过简单配置训练参数,即可实现数据并行、模型并行、流水线并行的混合并行训练,在
8 卡 A100 集群上可轻松支持百亿参数模型的高效训练。配合 SavingPolicy 和
LoggingCallback 等组件,实现了训练过程的自动化管理。其还支持与
Deepspeed、peft、wandb、Swanlab
等框架进行集成,直接通过参数设置即可无缝对接,从而快速、高效实现 LLM
训练。

对 LLM 时代的 NLP 研究者更为重要的是,HuggingFace 基于 Transformers
框架搭建了其庞大的 AI
社区,开放了数亿个预训练模型参数、25万+不同类型数据集,通过
Transformers、Dataset、Evaluate
等多个框架实现对预训练模型、数据集及评估函数的集成,从而帮助开发者可以便捷地使用任一预训练模型,在开源模型及数据集的基础上便捷地实现个人模型的开发与应用。

\begin{verbatim}
<img src="https://raw.githubusercontent.com/datawhalechina/happy-llm/main/docs/images/6-images/1-2.png" alt="alt text" width="90%" />
<p>图6.2 Hugging Face Transformers 模型社区</p>
\end{verbatim}

在 LLM
时代,模型结构的调整和重新预训练越来越少,开发者更多的业务应用在于使用预训练好的
LLM 进行 Post Train 和
SFT,来支持自己的下游业务应用。且由于预训练模型体量大,便捷集成
deepspeed 等分布式训练框架逐渐成为 LLM 时代 NLP
模型训练的必备技能。因此,Transformers 已逐步成为学界、业界 NLP
技术的主流框架,不管是企业业务开发还是科研研究,都逐渐首选 Transformers
进行模型实现。同时,新发布的开源 LLM 如 DeepSeek、Qwen 也都会第一时间在
Transformers 社区开放其预训练权重与模型调用 Demo。通过使用 Transformers
框架,可以高效、便捷地完成 LLM
训练及开发,实现工业级的产出交付。接下来,我们就会以 Transformers
框架为基础,介绍如何通过 Transformers 框架实现 LLM 的 Pretrain 及 SFT。

\subsubsection{6.1.2 初始化 LLM}\label{ux521dux59cbux5316-llm}

我们可以使用 transformers 的 AutoModel
类来直接初始化已经实现好的模型。对于任意预训练模型,其参数中都包含有模型的配置信息。如果是想要从头训练一个
LLM,可以使用一个已有的模型架构来直接初始化。这里,我们以
\href{https://huggingface.co/Qwen/Qwen2.5-1.5B/tree/main}{Qwen-2.5-1.5B}的模型架构为例:

\begin{verbatim}
<img src="https://raw.githubusercontent.com/datawhalechina/happy-llm/main/docs/images/6-images/1-3.png" alt="alt text" width="90%" />
<p>图6.3 Qwen-2.5-1.5B</p>
\end{verbatim}

该界面即为 HuggingFace 社区中的 Qwen-2.5-1.5B 模型参数,其中的
\texttt{config.json}
文件即是模型的配置信息,包括了模型的架构、隐藏层大小、模型层数等,如图6.4所示:

\begin{verbatim}
<img src="https://raw.githubusercontent.com/datawhalechina/happy-llm/main/docs/images/6-images/1-4.png" alt="alt text" width="90%" />
<p>图6.4 Qwen-2.5-1.5B config.json 文件</p>
\end{verbatim}

我们可以沿用该模型的配置信息,初始化一个 Qwen-2.5-1.5B
模型来进行训练,也可以在该配置信息的基础上进行更改,如修改隐藏层大小、注意力头数等,来定制一个模型结构。HuggingFace
提供了 Python 工具来便捷下载想使用的模型参数:

\begin{Shaded}
\begin{Highlighting}[]
\ImportTok{import}\NormalTok{ os}
\CommentTok{\# 设置环境变量,此处使用 HuggingFace 镜像网站}
\NormalTok{os.environ[}\StringTok{\textquotesingle{}HF\_ENDPOINT\textquotesingle{}}\NormalTok{] }\OperatorTok{=} \StringTok{\textquotesingle{}https://hf{-}mirror.com\textquotesingle{}}
\CommentTok{\# 下载模型}
\NormalTok{os.system(}\StringTok{\textquotesingle{}huggingface{-}cli download {-}{-}resume{-}download Qwen/Qwen2.5{-}1.5B {-}{-}local{-}dir your\_local\_dir\textquotesingle{}}\NormalTok{)}
\end{Highlighting}
\end{Shaded}

如图6.5,此处的
``Qwen/Qwen2.5-1.5B''即为要下载模型的标识符,对于其他模型,可以直接复制
HuggingFace 上的模型名即可:

\begin{verbatim}
<img src="https://raw.githubusercontent.com/datawhalechina/happy-llm/main/docs/images/6-images/1-5.png" alt="alt text" width="90%" />
<p>图6.5 模型下载标识</p>
\end{verbatim}

下载完成后,可以使用 AutoConfig 类直接加载下载好的配置文件:

\begin{Shaded}
\begin{Highlighting}[]
\CommentTok{\# 加载定义好的模型参数{-}此处以 Qwen{-}2.5{-}1.5B 为例}
\CommentTok{\# 使用 transforemrs 的 Config 类进行加载}
\ImportTok{from}\NormalTok{ transformers }\ImportTok{import}\NormalTok{ AutoConfig}

\CommentTok{\# 下载参数的本地路径}
\NormalTok{model\_path }\OperatorTok{=} \StringTok{"qwen{-}1.5b"}
\NormalTok{config }\OperatorTok{=}\NormalTok{ AutoConfig.from\_pretrained(model\_name\_or\_path)}
\end{Highlighting}
\end{Shaded}

也可以对配置文件进行自定义,然后以同样的方式加载即可。可以使用 AutoModel
类基于加载好的配置对象生成对应的模型:

\begin{Shaded}
\begin{Highlighting}[]
\CommentTok{\# 使用该配置生成一个定义好的模型}
\ImportTok{from}\NormalTok{ transformers }\ImportTok{import}\NormalTok{ AutoModelForCausalLM}

\NormalTok{model }\OperatorTok{=}\NormalTok{ AutoModelForCausalLM.from\_config(config,trust\_remote\_code}\OperatorTok{=}\VariableTok{True}\NormalTok{)}
\end{Highlighting}
\end{Shaded}

由于 LLM 一般都是 CausalLM 架构,此处使用了 AutoModelForCausalLM
类进行加载。如果是用于分类任务训练,可使用
AutoModelForSequenceClassification 类来加载。查看该
model,图6.6可以看到其架构和定义的配置文件相同:

\begin{verbatim}
<img src="https://raw.githubusercontent.com/datawhalechina/happy-llm/main/docs/images/6-images/1-6.png" alt="alt text" width="70%" />
<p>图6.6 模型结构输出结果</p>
\end{verbatim}

该 model 就是一个从零初始化的 Qwen-2.5-1.5B
模型了。一般情况下,我们很少从零初始化 LLM
进行预训练,较多的做法是加载一个预训练好的 LLM
权重,在自己的语料上进行后训练。这里,我们也介绍如何从下载好的模型参数中初始化一个预训练好的模型。

\begin{Shaded}
\begin{Highlighting}[]
\ImportTok{from}\NormalTok{ transformers }\ImportTok{import}\NormalTok{ AutoModelForCausalLM}

\NormalTok{model }\OperatorTok{=}\NormalTok{ AutoModelForCausalLM.from\_pretrained(model\_name\_or\_path,trust\_remote\_code}\OperatorTok{=}\VariableTok{True}\NormalTok{)}
\end{Highlighting}
\end{Shaded}

类似的,直接使用 from\_pretrained 方法加载即可,此处的
model\_name\_or\_path 即为下载好的参数的本地路径。

我们还需要初始化一个 tokenizer。此处,我们直接使用 Qwen-2.5-1.5B 对应的
tokenzier 参数即可:

\begin{Shaded}
\begin{Highlighting}[]
\CommentTok{\# 加载一个预训练好的 tokenizer}
\ImportTok{from}\NormalTok{ transformers }\ImportTok{import}\NormalTok{ AutoTokenizer}

\NormalTok{tokenizer }\OperatorTok{=}\NormalTok{ AutoTokenizer.from\_pretrained(model\_name\_or\_path)}
\end{Highlighting}
\end{Shaded}

加载好的 tokenizer 即可直接使用,对任意文本进行分词处理。

\subsubsection{6.1.3
预训练数据处理}\label{ux9884ux8badux7ec3ux6570ux636eux5904ux7406}

与第五章类似,我们使用出门问问序列猴子开源数据集作为预训练数据集,可以用与第五章一致的方式进行数据集的下载和解压。HuggingFace
的 datasets 库是和 transformers
框架配套的、用于数据下载和处理的第三方库。我们可以直接使用 datasets 的
load\_dataset 函数来加载预训练数据:

\begin{Shaded}
\begin{Highlighting}[]
\CommentTok{\# 加载预训练数据}
\ImportTok{from}\NormalTok{ datasets }\ImportTok{import}\NormalTok{ load\_dataset}

\NormalTok{ds }\OperatorTok{=}\NormalTok{ load\_dataset(}\StringTok{\textquotesingle{}json\textquotesingle{}}\NormalTok{, data\_files}\OperatorTok{=}\StringTok{\textquotesingle{}/mobvoi\_seq\_monkey\_general\_open\_corpus.jsonl\textquotesingle{}}\NormalTok{)}
\end{Highlighting}
\end{Shaded}

注意,由于数据集较大,加载可能会出现时间较长或内存不够的情况,建议前期测试时将预训练数据集拆分一部分出来进行测试。加载出来的
ds 是一个 DatasetDict 对象,加载的数据会默认保存在 \texttt{train}
键对应的值中,可以通过以下代码查看:

\begin{Shaded}
\begin{Highlighting}[]
\NormalTok{ds[}\StringTok{"train"}\NormalTok{][}\DecValTok{0}\NormalTok{]}
\end{Highlighting}
\end{Shaded}

\begin{verbatim}
<img src="https://raw.githubusercontent.com/datawhalechina/happy-llm/main/docs/images/6-images/1-7.png" alt="alt text" width="100%" />
<p>图6.7 数据集展示</p>
\end{verbatim}

可以通过 feature
属性查看数据集的特征(也就是列),这里需要保存一下数据集的列名,因为后续数据处理时,再将文本
tokenize 之后,需要移除原先的文本:

\begin{Shaded}
\begin{Highlighting}[]
\CommentTok{\# 查看特征}
\NormalTok{column\_names }\OperatorTok{=} \BuiltInTok{list}\NormalTok{(ds[}\StringTok{"train"}\NormalTok{].features)}
\CommentTok{\# columnes\_name:["text"]}
\end{Highlighting}
\end{Shaded}

接着使用加载好的 tokenizer 对数据集进行处理,此处使用 map
函数来进行批量处理:

\begin{Shaded}
\begin{Highlighting}[]
\CommentTok{\# 对数据集进行 tokenize}
\KeywordTok{def}\NormalTok{ tokenize\_function(examples):}
    \CommentTok{\# 使用预先加载的 tokenizer 进行分词}
\NormalTok{    output }\OperatorTok{=}\NormalTok{ tokenizer([item }\ControlFlowTok{for}\NormalTok{ item }\KeywordTok{in}\NormalTok{ examples[}\StringTok{"text"}\NormalTok{]])}
    \ControlFlowTok{return}\NormalTok{ output}

\CommentTok{\# 批量处理}
\NormalTok{tokenized\_datasets }\OperatorTok{=}\NormalTok{ ds.}\BuiltInTok{map}\NormalTok{(}
\NormalTok{    tokenize\_function,}
\NormalTok{    batched}\OperatorTok{=}\VariableTok{True}\NormalTok{,}
\NormalTok{    num\_proc}\OperatorTok{=}\DecValTok{10}\NormalTok{,}
\NormalTok{    remove\_columns}\OperatorTok{=}\NormalTok{column\_names,}
\NormalTok{    load\_from\_cache\_file}\OperatorTok{=}\VariableTok{True}\NormalTok{,}
\NormalTok{    desc}\OperatorTok{=}\StringTok{"Running tokenizer on dataset"}\NormalTok{,}
\NormalTok{)}
\end{Highlighting}
\end{Shaded}

处理完成后的数据集会包括'input\_ids', 'attention\_mask'两列,分别是文本
tokenize 之后的数值序列和注意力掩码(标识是否 padding)。map 方法会通过
remove\_columns 参数将原先的`text'移除,训练中不再使用。

由于预训练一般为 CLM
任务,一次性学习多个样本的序列语义不影响模型性能,且训练数据量大、训练时间长,对训练效率要求比较高。在预训练过程中,一般会把多个文本段拼接在一起,处理成统一长度的文本块,再对每个文本块进行训练。在这里,我们实现一个拼接函数将文本块拼接到
2048个 token 长度,再通过 map 方法来进行批量处理:

\begin{Shaded}
\begin{Highlighting}[]
\CommentTok{\# 预训练一般将文本拼接成固定长度的文本段}
\ImportTok{from}\NormalTok{ itertools }\ImportTok{import}\NormalTok{ chain}

\CommentTok{\# 这里我们取块长为 2048}
\NormalTok{block\_size }\OperatorTok{=} \DecValTok{2048}

\KeywordTok{def}\NormalTok{ group\_texts(examples):}
    \CommentTok{\# 将文本段拼接起来}
\NormalTok{    concatenated\_examples }\OperatorTok{=}\NormalTok{ \{k: }\BuiltInTok{list}\NormalTok{(chain(}\OperatorTok{*}\NormalTok{examples[k])) }\ControlFlowTok{for}\NormalTok{ k }\KeywordTok{in}\NormalTok{ examples.keys()\}}
    \CommentTok{\# 计算拼起来的整体长度}
\NormalTok{    total\_length }\OperatorTok{=} \BuiltInTok{len}\NormalTok{(concatenated\_examples[}\BuiltInTok{list}\NormalTok{(examples.keys())[}\DecValTok{0}\NormalTok{]])}
    \CommentTok{\# 如果长度太长,进行分块}
    \ControlFlowTok{if}\NormalTok{ total\_length }\OperatorTok{\textgreater{}=}\NormalTok{ block\_size:}
\NormalTok{        total\_length }\OperatorTok{=}\NormalTok{ (total\_length }\OperatorTok{//}\NormalTok{ block\_size) }\OperatorTok{*}\NormalTok{ block\_size}
    \CommentTok{\# 按 block\_size 进行切分}
\NormalTok{    result }\OperatorTok{=}\NormalTok{ \{}
\NormalTok{        k: [t[i : i }\OperatorTok{+}\NormalTok{ block\_size] }\ControlFlowTok{for}\NormalTok{ i }\KeywordTok{in} \BuiltInTok{range}\NormalTok{(}\DecValTok{0}\NormalTok{, total\_length, block\_size)]}
        \ControlFlowTok{for}\NormalTok{ k, t }\KeywordTok{in}\NormalTok{ concatenated\_examples.items()}
\NormalTok{    \}}
    \CommentTok{\# CLM 任务,labels 和 input 是相同的}
\NormalTok{    result[}\StringTok{"labels"}\NormalTok{] }\OperatorTok{=}\NormalTok{ result[}\StringTok{"input\_ids"}\NormalTok{].copy()}
    \ControlFlowTok{return}\NormalTok{ result}

\CommentTok{\# 批量处理}
\NormalTok{lm\_datasets }\OperatorTok{=}\NormalTok{ tokenized\_datasets.}\BuiltInTok{map}\NormalTok{(}
\NormalTok{    group\_texts,}
\NormalTok{    batched}\OperatorTok{=}\VariableTok{True}\NormalTok{,}
\NormalTok{    num\_proc}\OperatorTok{=}\DecValTok{10}\NormalTok{,}
\NormalTok{    load\_from\_cache\_file}\OperatorTok{=}\VariableTok{True}\NormalTok{,}
\NormalTok{    desc}\OperatorTok{=}\SpecialStringTok{f"Grouping texts in chunks of }\SpecialCharTok{\{}\NormalTok{block\_size}\SpecialCharTok{\}}\SpecialStringTok{"}\NormalTok{,}
\NormalTok{    batch\_size }\OperatorTok{=} \DecValTok{40000}\NormalTok{,}
\NormalTok{)}
\NormalTok{train\_dataset }\OperatorTok{=}\NormalTok{ lm\_datasets[}\StringTok{"train"}\NormalTok{]}
\end{Highlighting}
\end{Shaded}

处理得到的 train\_dataset 就是一个可直接用于 CLM Pretrain
的预训练数据集了,其每个样本长度为 2048个 token。

\subsubsection{6.1.4 使用 Trainer
进行训练}\label{ux4f7fux7528-trainer-ux8fdbux884cux8badux7ec3}

接下来,我们使用 transformers 提供的 Trainer 类进行训练。Trainer
封装了模型的训练逻辑,且做了较好的效率优化、可视化等工作,可以高效、便捷地完成
LLM 的训练。

首先我们需要配置训练的超参数,使用 TrainingArguments
类来实例化一个参数对象:

\begin{Shaded}
\begin{Highlighting}[]
\ImportTok{from}\NormalTok{ transformers }\ImportTok{import}\NormalTok{ TrainingArguments}
\CommentTok{\# 配置训练参数}

\NormalTok{training\_args }\OperatorTok{=}\NormalTok{ TrainingArguments(}
\NormalTok{    output\_dir}\OperatorTok{=}\StringTok{"output"}\NormalTok{,}\CommentTok{\# 训练参数输出路径}
\NormalTok{    per\_device\_train\_batch\_size}\OperatorTok{=}\DecValTok{4}\NormalTok{,}\CommentTok{\# 训练的 batch\_size}
\NormalTok{    gradient\_accumulation\_steps}\OperatorTok{=}\DecValTok{4}\NormalTok{,}\CommentTok{\# 梯度累计步数,实际 bs = 设置的 bs * 累计步数}
\NormalTok{    logging\_steps}\OperatorTok{=}\DecValTok{10}\NormalTok{,}\CommentTok{\# 打印 loss 的步数间隔}
\NormalTok{    num\_train\_epochs}\OperatorTok{=}\DecValTok{1}\NormalTok{,}\CommentTok{\# 训练的 epoch 数}
\NormalTok{    save\_steps}\OperatorTok{=}\DecValTok{100}\NormalTok{, }\CommentTok{\# 保存模型参数的步数间隔}
\NormalTok{    learning\_rate}\OperatorTok{=}\FloatTok{1e{-}4}\NormalTok{,}\CommentTok{\# 学习率}
\NormalTok{    gradient\_checkpointing}\OperatorTok{=}\VariableTok{True}\CommentTok{\# 开启梯度检查点}
\NormalTok{)}
\end{Highlighting}
\end{Shaded}

然后基于初始化的 model、tokenzier 和
training\_args,并传入处理好的训练数据集,实例化一个 trainer 对象:

\begin{Shaded}
\begin{Highlighting}[]
\ImportTok{from}\NormalTok{ transformers }\ImportTok{import}\NormalTok{ Trainer, default\_data\_collator}
\ImportTok{from}\NormalTok{ torchdata.datapipes.}\BuiltInTok{iter} \ImportTok{import}\NormalTok{ IterableWrapper}

\CommentTok{\# 训练器}
\NormalTok{trainer }\OperatorTok{=}\NormalTok{ Trainer(}
\NormalTok{    model}\OperatorTok{=}\NormalTok{model,}
\NormalTok{    args}\OperatorTok{=}\NormalTok{training\_args,}
\NormalTok{    train\_dataset}\OperatorTok{=}\NormalTok{ IterableWrapper(train\_dataset),}
\NormalTok{    eval\_dataset}\OperatorTok{=} \VariableTok{None}\NormalTok{,}
\NormalTok{    tokenizer}\OperatorTok{=}\NormalTok{tokenizer,}
    \CommentTok{\# 默认为 MLM 的 collator,使用 CLM 的 collater}
\NormalTok{    data\_collator}\OperatorTok{=}\NormalTok{default\_data\_collator}
\NormalTok{)}
\end{Highlighting}
\end{Shaded}

再使用 train 方法,即会按照配置好的训练超参进行训练和保存:

\begin{Shaded}
\begin{Highlighting}[]
\NormalTok{trainer.train()}
\end{Highlighting}
\end{Shaded}

\begin{quote}
注:上述代码存放于 \texttt{./code/pretrian.ipynb} 文件中。
\end{quote}

\subsubsection{6.1.5 使用 DeepSpeed
实现分布式训练}\label{ux4f7fux7528-deepspeed-ux5b9eux73b0ux5206ux5e03ux5f0fux8badux7ec3}

由于预训练规模大、时间长,一般不推荐使用 Jupyter Notebook
来运行,容易发生中断。且由于预训练规模大,一般需要使用多卡进行分布式训练,否则训练时间太长。在这里,我们介绍如何基于上述代码,使用
DeepSpeed 框架实现分布式训练,从而完成业界可用的 LLM Pretrain。

长时间训练一般使用 bash 脚本设定超参,再启动写好的 python
脚本实现训练。我们使用一个 Python
脚本(\texttt{./code/pretrain.py})来实现训练全流程。

先导入所需第三方库:

\begin{Shaded}
\begin{Highlighting}[]
\ImportTok{import}\NormalTok{ logging}
\ImportTok{import}\NormalTok{ math}
\ImportTok{import}\NormalTok{ os}
\ImportTok{import}\NormalTok{ sys}
\ImportTok{from}\NormalTok{ dataclasses }\ImportTok{import}\NormalTok{ dataclass, field}
\ImportTok{from}\NormalTok{ torchdata.datapipes.}\BuiltInTok{iter} \ImportTok{import}\NormalTok{ IterableWrapper}
\ImportTok{from}\NormalTok{ itertools }\ImportTok{import}\NormalTok{ chain}
\ImportTok{import}\NormalTok{ deepspeed}
\ImportTok{from}\NormalTok{ typing }\ImportTok{import}\NormalTok{ Optional,List}

\ImportTok{import}\NormalTok{ datasets}
\ImportTok{import}\NormalTok{ pandas }\ImportTok{as}\NormalTok{ pd}
\ImportTok{import}\NormalTok{ torch}
\ImportTok{from}\NormalTok{ datasets }\ImportTok{import}\NormalTok{ load\_dataset}
\ImportTok{import}\NormalTok{ transformers}
\ImportTok{from}\NormalTok{ transformers }\ImportTok{import}\NormalTok{ (}
\NormalTok{    AutoConfig,}
\NormalTok{    AutoModelForCausalLM,}
\NormalTok{    AutoTokenizer,}
\NormalTok{    HfArgumentParser,}
\NormalTok{    Trainer,}
\NormalTok{    TrainingArguments,}
\NormalTok{    default\_data\_collator,}
\NormalTok{    set\_seed,}
\NormalTok{)}
\ImportTok{import}\NormalTok{ datetime}
\ImportTok{from}\NormalTok{ transformers.testing\_utils }\ImportTok{import}\NormalTok{ CaptureLogger}
\ImportTok{from}\NormalTok{ transformers.trainer\_utils }\ImportTok{import}\NormalTok{ get\_last\_checkpoint}
\ImportTok{import}\NormalTok{ swanlab}
\end{Highlighting}
\end{Shaded}

首先需要定义几个超参的类型,用于处理 sh 脚本中设定的超参值。由于
transformers 本身有 TraingingArguments
类,其中包括了训练的一些必备超参数。我们这里只需定义 TrainingArguments
中未包含的超参即可,主要包括模型相关的超参(定义在
ModelArguments)和数据相关的超参(定义在 DataTrainingArguments):

\begin{Shaded}
\begin{Highlighting}[]
\CommentTok{\# 超参类}
\AttributeTok{@dataclass}
\KeywordTok{class}\NormalTok{ ModelArguments:}
    \CommentTok{"""}
\CommentTok{    关于模型的参数}
\CommentTok{    """}

\NormalTok{    model\_name\_or\_path: Optional[}\BuiltInTok{str}\NormalTok{] }\OperatorTok{=}\NormalTok{ field(}
\NormalTok{        default}\OperatorTok{=}\VariableTok{None}\NormalTok{,}
\NormalTok{        metadata}\OperatorTok{=}\NormalTok{\{}
            \StringTok{"help"}\NormalTok{: (}
                \StringTok{"后训练使用,为预训练模型参数地址"}
\NormalTok{            )}
\NormalTok{        \},}
\NormalTok{    )}
\NormalTok{    config\_name: Optional[}\BuiltInTok{str}\NormalTok{] }\OperatorTok{=}\NormalTok{ field(}
\NormalTok{        default}\OperatorTok{=}\VariableTok{None}\NormalTok{, metadata}\OperatorTok{=}\NormalTok{\{}\StringTok{"help"}\NormalTok{: }\StringTok{"预训练使用,Config 文件地址"}\NormalTok{\}}
\NormalTok{    )}
\NormalTok{    tokenizer\_name: Optional[}\BuiltInTok{str}\NormalTok{] }\OperatorTok{=}\NormalTok{ field(}
\NormalTok{        default}\OperatorTok{=}\VariableTok{None}\NormalTok{, metadata}\OperatorTok{=}\NormalTok{\{}\StringTok{"help"}\NormalTok{: }\StringTok{"预训练 Tokenizer 地址"}\NormalTok{\}}
\NormalTok{    )}
\NormalTok{    torch\_dtype: Optional[}\BuiltInTok{str}\NormalTok{] }\OperatorTok{=}\NormalTok{ field(}
\NormalTok{        default}\OperatorTok{=}\VariableTok{None}\NormalTok{,}
\NormalTok{        metadata}\OperatorTok{=}\NormalTok{\{}
            \StringTok{"help"}\NormalTok{: (}
                \StringTok{"模型训练使用的数据类型,推荐 bfloat16"}
\NormalTok{            ),}
            \StringTok{"choices"}\NormalTok{: [}\StringTok{"auto"}\NormalTok{, }\StringTok{"bfloat16"}\NormalTok{, }\StringTok{"float16"}\NormalTok{, }\StringTok{"float32"}\NormalTok{],}
\NormalTok{        \},}
\NormalTok{    )}


\AttributeTok{@dataclass}
\KeywordTok{class}\NormalTok{ DataTrainingArguments:}
    \CommentTok{"""}
\CommentTok{    关于训练的参数}
\CommentTok{    """}

\NormalTok{    train\_files: Optional[List[}\BuiltInTok{str}\NormalTok{]]  }\OperatorTok{=}\NormalTok{ field(default}\OperatorTok{=}\VariableTok{None}\NormalTok{, metadata}\OperatorTok{=}\NormalTok{\{}\StringTok{"help"}\NormalTok{: }\StringTok{"训练数据路径"}\NormalTok{\})}
\NormalTok{    block\_size: Optional[}\BuiltInTok{int}\NormalTok{] }\OperatorTok{=}\NormalTok{ field(}
\NormalTok{        default}\OperatorTok{=}\VariableTok{None}\NormalTok{,}
\NormalTok{        metadata}\OperatorTok{=}\NormalTok{\{}
            \StringTok{"help"}\NormalTok{: (}
                \StringTok{"设置的文本块长度"}
\NormalTok{            )}
\NormalTok{        \},}
\NormalTok{    )}
\NormalTok{    preprocessing\_num\_workers: Optional[}\BuiltInTok{int}\NormalTok{] }\OperatorTok{=}\NormalTok{ field(}
\NormalTok{        default}\OperatorTok{=}\VariableTok{None}\NormalTok{,}
\NormalTok{        metadata}\OperatorTok{=}\NormalTok{\{}\StringTok{"help"}\NormalTok{: }\StringTok{"预处理使用线程数."}\NormalTok{\},}
\NormalTok{    )}
\end{Highlighting}
\end{Shaded}

然后即可定义一个主函数实现上述训练过程的封装。首先通过 transformers
提供的 HfArgumentParser 工具来加载 sh 脚本中设定的超参:

\begin{Shaded}
\begin{Highlighting}[]
\CommentTok{\# 加载脚本参数}
\NormalTok{parser }\OperatorTok{=}\NormalTok{ HfArgumentParser((ModelArguments, DataTrainingArguments, TrainingArguments))}
\NormalTok{model\_args, data\_args, training\_args }\OperatorTok{=}\NormalTok{ parser.parse\_args\_into\_dataclasses()}
\end{Highlighting}
\end{Shaded}

在大规模的训练中,一般使用 log 来保存训练过程的信息,一般不推荐使用
print 直接打印,容易发生关键训练信息的丢失。这里,我们直接使用 python
自带的 logging 库来实现日志记录。首先需要进行 log 的设置:

\begin{Shaded}
\begin{Highlighting}[]
\CommentTok{\# 设置日志}
\NormalTok{logging.basicConfig(}
    \BuiltInTok{format}\OperatorTok{=}\StringTok{"}\SpecialCharTok{\%(asctime)s}\StringTok{ {-} }\SpecialCharTok{\%(levelname)s}\StringTok{ {-} }\SpecialCharTok{\%(name)s}\StringTok{ {-} }\SpecialCharTok{\%(message)s}\StringTok{"}\NormalTok{,}
\NormalTok{    datefmt}\OperatorTok{=}\StringTok{"\%m/}\SpecialCharTok{\%d}\StringTok{/\%Y \%H:\%M:\%S"}\NormalTok{,}
\NormalTok{    handlers}\OperatorTok{=}\NormalTok{[logging.StreamHandler(sys.stdout)],}
\NormalTok{)}

\CommentTok{\# 将日志级别设置为 INFO}
\NormalTok{transformers.utils.logging.set\_verbosity\_info()}
\NormalTok{log\_level }\OperatorTok{=}\NormalTok{ training\_args.get\_process\_log\_level()}
\NormalTok{logger.setLevel(log\_level)}
\NormalTok{datasets.utils.logging.set\_verbosity(log\_level)}
\NormalTok{transformers.utils.logging.set\_verbosity(log\_level)}
\NormalTok{transformers.utils.logging.enable\_default\_handler()}
\NormalTok{transformers.utils.logging.enable\_explicit\_format()}
\end{Highlighting}
\end{Shaded}

这里将日志的级别设置为 INFO。logging 的日志共有
DEBUG、INFO、WARNING、ERROR 以及 CRITICAL
五个级别,将日志设置为哪个级别,就会只输出该级别及该级别之上的信息。设置完成后,在需要记录日志的地方,直接使用
logger 即可,记录时会指定记录日志的级别,例如:

\begin{Shaded}
\begin{Highlighting}[]
\CommentTok{\# 训练整体情况记录}
\NormalTok{logger.warning(}
    \SpecialStringTok{f"Process rank: }\SpecialCharTok{\{}\NormalTok{training\_args}\SpecialCharTok{.}\NormalTok{local\_rank}\SpecialCharTok{\}}\SpecialStringTok{, device: }\SpecialCharTok{\{}\NormalTok{training\_args}\SpecialCharTok{.}\NormalTok{device}\SpecialCharTok{\}}\SpecialStringTok{, n\_gpu: }\SpecialCharTok{\{}\NormalTok{training\_args}\SpecialCharTok{.}\NormalTok{n\_gpu}\SpecialCharTok{\}}\SpecialStringTok{"}
    \OperatorTok{+} \SpecialStringTok{f"distributed training: }\SpecialCharTok{\{}\BuiltInTok{bool}\NormalTok{(training\_args.local\_rank }\OperatorTok{!=} \OperatorTok{{-}}\DecValTok{1}\NormalTok{)}\SpecialCharTok{\}}\SpecialStringTok{, 16{-}bits training: }\SpecialCharTok{\{}\NormalTok{training\_args}\SpecialCharTok{.}\NormalTok{fp16}\SpecialCharTok{\}}\SpecialStringTok{"}
\NormalTok{)}
\NormalTok{logger.info(}\SpecialStringTok{f"Training/evaluation parameters }\SpecialCharTok{\{}\NormalTok{training\_args}\SpecialCharTok{\}}\SpecialStringTok{"}\NormalTok{)}
\end{Highlighting}
\end{Shaded}

后续就不再赘述脚本中的日志记录。

在大规模训练中,发生中断是往往难以避免的,训练一般会固定间隔保存
checkpoint,中断之后基于最近的 checkpoint
恢复训练即可。因此,我们需要首先检测是否存在旧的 checkpoint 并从
checkpoint 恢复训练:

\begin{Shaded}
\begin{Highlighting}[]
\CommentTok{\# 检查 checkpoint}
\NormalTok{last\_checkpoint }\OperatorTok{=} \VariableTok{None}
\ControlFlowTok{if}\NormalTok{ os.path.isdir(training\_args.output\_dir):}
    \CommentTok{\# 使用 transformers 自带的 get\_last\_checkpoint 自动检测}
\NormalTok{    last\_checkpoint }\OperatorTok{=}\NormalTok{ get\_last\_checkpoint(training\_args.output\_dir)}
    \ControlFlowTok{if}\NormalTok{ last\_checkpoint }\KeywordTok{is} \VariableTok{None} \KeywordTok{and} \BuiltInTok{len}\NormalTok{(os.listdir(training\_args.output\_dir)) }\OperatorTok{\textgreater{}} \DecValTok{0}\NormalTok{:}
        \ControlFlowTok{raise} \PreprocessorTok{ValueError}\NormalTok{(}
            \SpecialStringTok{f"输出路径 (}\SpecialCharTok{\{}\NormalTok{training\_args}\SpecialCharTok{.}\NormalTok{output\_dir}\SpecialCharTok{\}}\SpecialStringTok{) 非空 "}
\NormalTok{        )}
    \ControlFlowTok{elif}\NormalTok{ last\_checkpoint }\KeywordTok{is} \KeywordTok{not} \VariableTok{None} \KeywordTok{and}\NormalTok{ training\_args.resume\_from\_checkpoint }\KeywordTok{is} \VariableTok{None}\NormalTok{:}
\NormalTok{        logger.info(}
            \SpecialStringTok{f"从 }\SpecialCharTok{\{}\NormalTok{last\_checkpoint}\SpecialCharTok{\}}\SpecialStringTok{恢复训练"}
\NormalTok{        )}
\end{Highlighting}
\end{Shaded}

接着以上文介绍过的方式初始化模型,此处将从零初始化和基于已有预训练模型初始化包装在一起:

\begin{Shaded}
\begin{Highlighting}[]
\CommentTok{\# 初始化模型}
\ControlFlowTok{if}\NormalTok{ model\_args.config\_name }\KeywordTok{is} \KeywordTok{not} \VariableTok{None}\NormalTok{:}
    \CommentTok{\# from scrach}
\NormalTok{    config }\OperatorTok{=}\NormalTok{ AutoConfig.from\_pretrained(model\_args.config\_name)}
\NormalTok{    logger.warning(}\StringTok{"你正在从零初始化一个模型"}\NormalTok{)}
\NormalTok{    logger.info(}\SpecialStringTok{f"模型参数配置地址:}\SpecialCharTok{\{}\NormalTok{model\_args}\SpecialCharTok{.}\NormalTok{config\_name}\SpecialCharTok{\}}\SpecialStringTok{"}\NormalTok{)}
\NormalTok{    logger.info(}\SpecialStringTok{f"模型参数:}\SpecialCharTok{\{}\NormalTok{config}\SpecialCharTok{\}}\SpecialStringTok{"}\NormalTok{)}
\NormalTok{    model }\OperatorTok{=}\NormalTok{ AutoModelForCausalLM.from\_config(config,trust\_remote\_code}\OperatorTok{=}\VariableTok{True}\NormalTok{)}
\NormalTok{    n\_params }\OperatorTok{=} \BuiltInTok{sum}\NormalTok{(\{p.data\_ptr(): p.numel() }\ControlFlowTok{for}\NormalTok{ p }\KeywordTok{in}\NormalTok{ model.parameters()\}.values())}
\NormalTok{    logger.info(}\SpecialStringTok{f"预训练一个新模型 {-} Total size=}\SpecialCharTok{\{}\NormalTok{n\_params}\OperatorTok{/}\DecValTok{2}\OperatorTok{**}\DecValTok{20}\SpecialCharTok{:.2f\}}\SpecialStringTok{M params"}\NormalTok{)}
\ControlFlowTok{elif}\NormalTok{ model\_args.model\_name\_or\_path }\KeywordTok{is} \KeywordTok{not} \VariableTok{None}\NormalTok{:}
\NormalTok{    logger.warning(}\StringTok{"你正在初始化一个预训练模型"}\NormalTok{)}
\NormalTok{    logger.info(}\SpecialStringTok{f"模型参数地址:}\SpecialCharTok{\{}\NormalTok{model\_args}\SpecialCharTok{.}\NormalTok{model\_name\_or\_path}\SpecialCharTok{\}}\SpecialStringTok{"}\NormalTok{)}
\NormalTok{    model }\OperatorTok{=}\NormalTok{ AutoModelForCausalLM.from\_pretrained(model\_args.model\_name\_or\_path,trust\_remote\_code}\OperatorTok{=}\VariableTok{True}\NormalTok{)}
\NormalTok{    n\_params }\OperatorTok{=} \BuiltInTok{sum}\NormalTok{(\{p.data\_ptr(): p.numel() }\ControlFlowTok{for}\NormalTok{ p }\KeywordTok{in}\NormalTok{ model.parameters()\}.values())}
\NormalTok{    logger.info(}\SpecialStringTok{f"继承一个预训练模型 {-} Total size=}\SpecialCharTok{\{}\NormalTok{n\_params}\OperatorTok{/}\DecValTok{2}\OperatorTok{**}\DecValTok{20}\SpecialCharTok{:.2f\}}\SpecialStringTok{M params"}\NormalTok{)}
\ControlFlowTok{else}\NormalTok{:}
\NormalTok{    logger.error(}\StringTok{"config\_name 和 model\_name\_or\_path 不能均为空"}\NormalTok{)}
    \ControlFlowTok{raise} \PreprocessorTok{ValueError}\NormalTok{(}\StringTok{"config\_name 和 model\_name\_or\_path 不能均为空"}\NormalTok{)}
\end{Highlighting}
\end{Shaded}

再类似的进行 tokenizer
的加载和预训练数据的处理。该部分和上文完全一致,此处不再赘述,读者可以在代码中详细查看细节。类似的,使用
Trainer 进行训练:

\begin{Shaded}
\begin{Highlighting}[]
\NormalTok{logger.info(}\StringTok{"初始化 Trainer"}\NormalTok{)}
\NormalTok{trainer }\OperatorTok{=}\NormalTok{ Trainer(}
\NormalTok{    model}\OperatorTok{=}\NormalTok{model,}
\NormalTok{    args}\OperatorTok{=}\NormalTok{training\_args,}
\NormalTok{    train\_dataset}\OperatorTok{=}\NormalTok{ IterableWrapper(train\_dataset),}
\NormalTok{    tokenizer}\OperatorTok{=}\NormalTok{tokenizer,}
\NormalTok{    data\_collator}\OperatorTok{=}\NormalTok{default\_data\_collator}
\NormalTok{)}

\CommentTok{\# 从 checkpoint 加载}
\NormalTok{checkpoint }\OperatorTok{=} \VariableTok{None}
\ControlFlowTok{if}\NormalTok{ training\_args.resume\_from\_checkpoint }\KeywordTok{is} \KeywordTok{not} \VariableTok{None}\NormalTok{:}
\NormalTok{    checkpoint }\OperatorTok{=}\NormalTok{ training\_args.resume\_from\_checkpoint}
\ControlFlowTok{elif}\NormalTok{ last\_checkpoint }\KeywordTok{is} \KeywordTok{not} \VariableTok{None}\NormalTok{:}
\NormalTok{        checkpoint }\OperatorTok{=}\NormalTok{ last\_checkpoint}

\NormalTok{logger.info(}\StringTok{"开始训练"}\NormalTok{)}
\NormalTok{train\_result }\OperatorTok{=}\NormalTok{ trainer.train(resume\_from\_checkpoint}\OperatorTok{=}\NormalTok{checkpoint)}
\NormalTok{trainer.save\_model() }
\end{Highlighting}
\end{Shaded}

注意,由于上文检测了是否存在 checkpoint,此处使用
resume\_from\_checkpoint 来实现从 checkpoint 恢复训练的功能。

由于在大规模训练中监测训练进度、loss
下降趋势尤为重要,在脚本中,我们使用了 swanlab
作为训练检测的工具。在脚本开始进行了 swanlab 的初始化:

\begin{Shaded}
\begin{Highlighting}[]
\CommentTok{\# 初始化 SwanLab}
\NormalTok{swanlab.init(project}\OperatorTok{=}\StringTok{"pretrain"}\NormalTok{, experiment\_name}\OperatorTok{=}\StringTok{"from\_scrach"}\NormalTok{)}
\end{Highlighting}
\end{Shaded}

在启动训练后,终端会输出 swanlab 监测的
url,点击即可观察训练进度。此处不再赘述 swanlab
的使用细节,欢迎读者查阅相关的资料说明。

完成上述代码后,我们使用一个 sh
脚本(\texttt{./code/pretrain.sh})定义超参数的值,并通过 Deepspeed
启动训练,从而实现高效的多卡分布式训练:

\begin{Shaded}
\begin{Highlighting}[]
\CommentTok{\# 设置可见显卡}
\VariableTok{CUDA\_VISIBLE\_DEVICES}\OperatorTok{=}\NormalTok{0,1}

\ExtensionTok{deepspeed}\NormalTok{ pretrain.py }\DataTypeTok{\textbackslash{}}
    \AttributeTok{{-}{-}config\_name}\NormalTok{ autodl{-}tmp/qwen{-}1.5b }\DataTypeTok{\textbackslash{}}
    \AttributeTok{{-}{-}tokenizer\_name}\NormalTok{ autodl{-}tmp/qwen{-}1.5b }\DataTypeTok{\textbackslash{}}
    \AttributeTok{{-}{-}train\_files}\NormalTok{ autodl{-}tmp/dataset/pretrain\_data/mobvoi\_seq\_monkey\_general\_open\_corpus\_small.jsonl }\DataTypeTok{\textbackslash{}}
    \AttributeTok{{-}{-}per\_device\_train\_batch\_size}\NormalTok{ 16 }\DataTypeTok{\textbackslash{}}
    \AttributeTok{{-}{-}gradient\_accumulation\_steps}\NormalTok{ 4 }\DataTypeTok{\textbackslash{}}
    \AttributeTok{{-}{-}do\_train} \DataTypeTok{\textbackslash{}}
    \AttributeTok{{-}{-}output\_dir}\NormalTok{ autodl{-}tmp/output/pretrain }\DataTypeTok{\textbackslash{}}
    \AttributeTok{{-}{-}evaluation\_strategy}\NormalTok{  no }\DataTypeTok{\textbackslash{}}
    \AttributeTok{{-}{-}learning\_rate}\NormalTok{ 1e{-}4 }\DataTypeTok{\textbackslash{}}
    \AttributeTok{{-}{-}num\_train\_epochs}\NormalTok{ 1 }\DataTypeTok{\textbackslash{}}
    \AttributeTok{{-}{-}warmup\_steps}\NormalTok{ 200 }\DataTypeTok{\textbackslash{}}
    \AttributeTok{{-}{-}logging\_dir}\NormalTok{ autodl{-}tmp/output/pretrain/logs }\DataTypeTok{\textbackslash{}}
    \AttributeTok{{-}{-}logging\_strategy}\NormalTok{ steps }\DataTypeTok{\textbackslash{}}
    \AttributeTok{{-}{-}logging\_steps}\NormalTok{ 5 }\DataTypeTok{\textbackslash{}}
    \AttributeTok{{-}{-}save\_strategy}\NormalTok{ steps }\DataTypeTok{\textbackslash{}}
    \AttributeTok{{-}{-}save\_steps}\NormalTok{ 100 }\DataTypeTok{\textbackslash{}}
    \AttributeTok{{-}{-}preprocessing\_num\_workers}\NormalTok{ 10 }\DataTypeTok{\textbackslash{}}
    \AttributeTok{{-}{-}save\_total\_limit}\NormalTok{ 1 }\DataTypeTok{\textbackslash{}}
    \AttributeTok{{-}{-}seed}\NormalTok{ 12 }\DataTypeTok{\textbackslash{}}
    \AttributeTok{{-}{-}block\_size}\NormalTok{ 2048 }\DataTypeTok{\textbackslash{}}
    \AttributeTok{{-}{-}bf16} \DataTypeTok{\textbackslash{}}
    \AttributeTok{{-}{-}gradient\_checkpointing} \DataTypeTok{\textbackslash{}}
    \AttributeTok{{-}{-}deepspeed}\NormalTok{ ./ds\_config\_zero2.json }\DataTypeTok{\textbackslash{}}
    \AttributeTok{{-}{-}report\_to}\NormalTok{ swanlab}
    \CommentTok{\# {-}{-}resume\_from\_checkpoint $\{output\_model\}/checkpoint{-}20400 \textbackslash{}}
\end{Highlighting}
\end{Shaded}

在安装了 Deepspeed 第三方库后,可以直接通过 Deepspeed
命令来启动多卡训练。上述脚本命令主要是定义了各种超参数的值,可参考使用。在第四章中,我们介绍了
DeepSpeed 分布式训练的原理和 ZeRO 阶段设置,在这里,我们使用 ZeRO-2
进行训练。此处加载了 \texttt{ds\_config\_zero.json} 作为 DeepSpeed
的配置参数:

\begin{Shaded}
\begin{Highlighting}[]
\FunctionTok{\{}
    \DataTypeTok{"fp16"}\FunctionTok{:} \FunctionTok{\{}
        \DataTypeTok{"enabled"}\FunctionTok{:} \StringTok{"auto"}\FunctionTok{,}
        \DataTypeTok{"loss\_scale"}\FunctionTok{:} \DecValTok{0}\FunctionTok{,}
        \DataTypeTok{"loss\_scale\_window"}\FunctionTok{:} \DecValTok{1000}\FunctionTok{,}
        \DataTypeTok{"initial\_scale\_power"}\FunctionTok{:} \DecValTok{16}\FunctionTok{,}
        \DataTypeTok{"hysteresis"}\FunctionTok{:} \DecValTok{2}\FunctionTok{,}
        \DataTypeTok{"min\_loss\_scale"}\FunctionTok{:} \DecValTok{1}
    \FunctionTok{\},}
    \DataTypeTok{"bf16"}\FunctionTok{:} \FunctionTok{\{}
        \DataTypeTok{"enabled"}\FunctionTok{:} \StringTok{"auto"}
    \FunctionTok{\},}
    \DataTypeTok{"optimizer"}\FunctionTok{:} \FunctionTok{\{}
        \DataTypeTok{"type"}\FunctionTok{:} \StringTok{"AdamW"}\FunctionTok{,}
        \DataTypeTok{"params"}\FunctionTok{:} \FunctionTok{\{}
            \DataTypeTok{"lr"}\FunctionTok{:} \StringTok{"auto"}\FunctionTok{,}
            \DataTypeTok{"betas"}\FunctionTok{:} \StringTok{"auto"}\FunctionTok{,}
            \DataTypeTok{"eps"}\FunctionTok{:} \StringTok{"auto"}\FunctionTok{,}
            \DataTypeTok{"weight\_decay"}\FunctionTok{:} \StringTok{"auto"}
        \FunctionTok{\}}
    \FunctionTok{\},}

    \DataTypeTok{"scheduler"}\FunctionTok{:} \FunctionTok{\{}
        \DataTypeTok{"type"}\FunctionTok{:} \StringTok{"WarmupLR"}\FunctionTok{,}
        \DataTypeTok{"params"}\FunctionTok{:} \FunctionTok{\{}
            \DataTypeTok{"warmup\_min\_lr"}\FunctionTok{:} \StringTok{"auto"}\FunctionTok{,}
            \DataTypeTok{"warmup\_max\_lr"}\FunctionTok{:} \StringTok{"auto"}\FunctionTok{,}
            \DataTypeTok{"warmup\_num\_steps"}\FunctionTok{:} \StringTok{"auto"}
        \FunctionTok{\}}
    \FunctionTok{\},}

    \DataTypeTok{"zero\_optimization"}\FunctionTok{:} \FunctionTok{\{}
        \DataTypeTok{"stage"}\FunctionTok{:} \DecValTok{2}\FunctionTok{,}
        \DataTypeTok{"offload\_optimizer"}\FunctionTok{:} \FunctionTok{\{}
            \DataTypeTok{"device"}\FunctionTok{:} \StringTok{"none"}\FunctionTok{,}
            \DataTypeTok{"pin\_memory"}\FunctionTok{:} \KeywordTok{true}
        \FunctionTok{\},}
        \DataTypeTok{"allgather\_partitions"}\FunctionTok{:} \KeywordTok{true}\FunctionTok{,}
        \DataTypeTok{"allgather\_bucket\_size"}\FunctionTok{:} \DecValTok{2e8}\FunctionTok{,}
        \DataTypeTok{"overlap\_comm"}\FunctionTok{:} \KeywordTok{true}\FunctionTok{,}
        \DataTypeTok{"reduce\_scatter"}\FunctionTok{:} \KeywordTok{true}\FunctionTok{,}
        \DataTypeTok{"reduce\_bucket\_size"}\FunctionTok{:} \DecValTok{2e8}\FunctionTok{,}
        \DataTypeTok{"contiguous\_gradients"}\FunctionTok{:} \KeywordTok{true}
    \FunctionTok{\},}

    \DataTypeTok{"gradient\_accumulation\_steps"}\FunctionTok{:} \StringTok{"auto"}\FunctionTok{,}
    \DataTypeTok{"gradient\_clipping"}\FunctionTok{:} \StringTok{"auto"}\FunctionTok{,}
    \DataTypeTok{"steps\_per\_print"}\FunctionTok{:} \DecValTok{100}\FunctionTok{,}
    \DataTypeTok{"train\_batch\_size"}\FunctionTok{:} \StringTok{"auto"}\FunctionTok{,}
    \DataTypeTok{"train\_micro\_batch\_size\_per\_gpu"}\FunctionTok{:} \StringTok{"auto"}\FunctionTok{,}
    \DataTypeTok{"wall\_clock\_breakdown"}\FunctionTok{:} \KeywordTok{false}
\FunctionTok{\}}
\end{Highlighting}
\end{Shaded}

最后,在终端 bash 运行该 \texttt{pretrain.sh} 脚本即可开始训练。

\subsection{6.2
模型有监督微调}\label{ux6a21ux578bux6709ux76d1ux7763ux5faeux8c03}

在上一节,我们介绍了如何使用 Transformers
框架快速、高效地进行模型预训练。在本部分,我们将基于上部分内容,介绍如何使用
Transformers 框架对预训练好的模型进行有监督微调。

\subsubsection{6.2.1 Pretrain VS SFT}\label{pretrain-vs-sft}

首先需要回顾一下,对 LLM
进行预训练和进行有监督微调的核心差异在于什么。在第四章中提到过,目前成型的
LLM 一般通过 Pretrain-SFT-RLHF 三个阶段来训练,在 Pretrain
阶段,会对海量无监督文本进行自监督建模,来学习文本语义规则和文本中的世界知识;在
SFT 阶段,一般通过对 Pretrain
好的模型进行指令微调,即训练模型根据用户指令完成对应任务,从而使模型能够遵循用户指令,根据用户指令进行规划、行动和输出。因此,Pretrain
和 SFT 均使用 CLM 建模,其核心差异在于,Pretrain
使用海量无监督文本进行训练,模型直接对文本执行``预测下一个
token''的任务;而 SFT
使用构建成对的指令对数据,模型根据输入的指令,建模后续的输出。反映到具体的训练实现上,Pretrain
会对全部 text 进行 loss 计算,要求模型对整个文本实现建模预测;而 SFT
仅对输出进行 loss 计算,不计算指令部分的 loss。

因此,相较于上一节完成的 Pretrain 代码,SFT
部分仅需要修改数据处理环节,实现对指令对数据转化为训练样本的构建,其余部分和
Pretrain
是完全一致的实现逻辑。本部分代码脚本为\texttt{./code/finetune.py}。

\subsubsection{6.2.2
微调数据处理}\label{ux5faeux8c03ux6570ux636eux5904ux7406}

同样与第五章类似,我们此处使用贝壳开源的 BelleGroup 数据集进行 SFT。

在 SFT 过程中,我们会定义一个 Chat Template,这个 Template
即表示了如何将对话数据转化为一个模型可以建模拟合的文本序列。当我们使用做过
SFT 的模型进行下游任务微调时,一般需要查看该模型的 Chat Template
并进行适配,即是为了不损伤其在 SFT
中学到的指令遵循能力。由于我们此处使用 Pretrain 模型进行
SFT,可以自定义一个 Chat Template。由于我们使用了 Qwen-2.5-1.5B
模型结构进行 Pretrain,此处我们沿承使用 Qwen-2.5 的 Chat
Template。如果读者没有足够的资源进行上一部分模型的 Pretrain
的话,此处也可以使用官方的 Qwen-2.5-1.5B 模型作为 SFT 的基座模型。

我们首先定义几个特殊 token,特殊 token
在模型进行拟合中有特殊的作用,包括文本序列开始(BOS)、文本序列结束(EOS)、换行符等。定义特殊
token,有助于避免模型在拟合过程中的语义混淆:

\begin{Shaded}
\begin{Highlighting}[]

\CommentTok{\# 不同的 tokenizer 需要特别定义}
\CommentTok{\# BOS}
\NormalTok{im\_start }\OperatorTok{=}\NormalTok{ tokenizer(}\StringTok{"\textless{}|im\_start|\textgreater{}"}\NormalTok{).input\_ids}
\CommentTok{\# EOS}
\NormalTok{im\_end }\OperatorTok{=}\NormalTok{ tokenizer(}\StringTok{"\textless{}|im\_end|\textgreater{}"}\NormalTok{).input\_ids}
\CommentTok{\# PAD}
\NormalTok{IGNORE\_TOKEN\_ID }\OperatorTok{=}\NormalTok{ tokenizer.pad\_token\_id}
\CommentTok{\# 换行符}
\NormalTok{nl\_tokens }\OperatorTok{=}\NormalTok{ tokenizer(}\StringTok{\textquotesingle{}}\CharTok{\textbackslash{}n}\StringTok{\textquotesingle{}}\NormalTok{).input\_ids}
\CommentTok{\# 角色标识符}
\NormalTok{\_system }\OperatorTok{=}\NormalTok{ tokenizer(}\StringTok{\textquotesingle{}system\textquotesingle{}}\NormalTok{).input\_ids }\OperatorTok{+}\NormalTok{ nl\_tokens}
\NormalTok{\_user }\OperatorTok{=}\NormalTok{ tokenizer(}\StringTok{\textquotesingle{}human\textquotesingle{}}\NormalTok{).input\_ids }\OperatorTok{+}\NormalTok{ nl\_tokens}
\NormalTok{\_assistant }\OperatorTok{=}\NormalTok{ tokenizer(}\StringTok{\textquotesingle{}assistant\textquotesingle{}}\NormalTok{).input\_ids }\OperatorTok{+}\NormalTok{ nl\_tokens}
\end{Highlighting}
\end{Shaded}

Qwen 系列的 Chat Template 一般有三个对话角色:System、User 和
Assistant。System 是系统提示词,负责激活模型的能力,默认为``You are a
helpful assistant.'',一般不会在 SFT 过程中更改使用。User
即为用户给出的提示词,此处由于数据集中的对话角色为 ``human'',我们将
``user'' 修改为了``human''。Assistant 即为 LLM 给出的回复,也就是模型在
SFT 过程中需要拟合的文本。

接着,由于该数据集是一个多轮对话数据集,我们需要对多轮对话进行拼接处理,将多轮对话拼接到一个文本序列中:

\begin{Shaded}
\begin{Highlighting}[]
\CommentTok{\# 拼接多轮对话}
\NormalTok{input\_ids, targets }\OperatorTok{=}\NormalTok{ [], []}
\CommentTok{\# 多个样本}
\ControlFlowTok{for}\NormalTok{ i }\KeywordTok{in}\NormalTok{ tqdm(}\BuiltInTok{range}\NormalTok{(}\BuiltInTok{len}\NormalTok{(sources))):}
    \CommentTok{\# source 为一个多轮对话样本}
\NormalTok{    source }\OperatorTok{=}\NormalTok{ sources[i]}
    \CommentTok{\# 从 user 开始}
    \ControlFlowTok{if}\NormalTok{ source[}\DecValTok{0}\NormalTok{][}\StringTok{"from"}\NormalTok{] }\OperatorTok{!=} \StringTok{"human"}\NormalTok{:}
\NormalTok{        source }\OperatorTok{=}\NormalTok{ source[}\DecValTok{1}\NormalTok{:]}
    \CommentTok{\# 分别是输入和输出}
\NormalTok{    input\_id, target }\OperatorTok{=}\NormalTok{ [], []}
    \CommentTok{\# system: 【BOS】system\textbackslash{}nYou are a helpful assistant.【EOS】\textbackslash{}n}
\NormalTok{    system }\OperatorTok{=}\NormalTok{ im\_start }\OperatorTok{+}\NormalTok{ \_system }\OperatorTok{+}\NormalTok{ tokenizer(system\_message).input\_ids }\OperatorTok{+}\NormalTok{ im\_end }\OperatorTok{+}\NormalTok{ nl\_tokens}
\NormalTok{    input\_id }\OperatorTok{+=}\NormalTok{ system}
    \CommentTok{\# system 不需要拟合}
\NormalTok{    target }\OperatorTok{+=}\NormalTok{ im\_start }\OperatorTok{+}\NormalTok{ [IGNORE\_TOKEN\_ID] }\OperatorTok{*}\NormalTok{ (}\BuiltInTok{len}\NormalTok{(system)}\OperatorTok{{-}}\DecValTok{3}\NormalTok{) }\OperatorTok{+}\NormalTok{ im\_end }\OperatorTok{+}\NormalTok{ nl\_tokens}
    \ControlFlowTok{assert} \BuiltInTok{len}\NormalTok{(input\_id) }\OperatorTok{==} \BuiltInTok{len}\NormalTok{(target)}
    \CommentTok{\# 依次拼接}
    \ControlFlowTok{for}\NormalTok{ j, sentence }\KeywordTok{in} \BuiltInTok{enumerate}\NormalTok{(source):}
        \CommentTok{\# sentence 为一轮对话}
\NormalTok{        role }\OperatorTok{=}\NormalTok{ roles[sentence[}\StringTok{"from"}\NormalTok{]]}
        \CommentTok{\# user:\textless{}|im\_start|\textgreater{}human\textbackslash{}ninstruction【EOS】\textbackslash{}n}
        \CommentTok{\# assistant:\textless{}|im\_start|\textgreater{}assistant\textbackslash{}nresponse【EOS】\textbackslash{}n}
\NormalTok{        \_input\_id }\OperatorTok{=}\NormalTok{ tokenizer(role).input\_ids }\OperatorTok{+}\NormalTok{ nl\_tokens }\OperatorTok{+} \OperatorTok{\textbackslash{}}
\NormalTok{            tokenizer(sentence[}\StringTok{"value"}\NormalTok{]).input\_ids }\OperatorTok{+}\NormalTok{ im\_end }\OperatorTok{+}\NormalTok{ nl\_tokens}
\NormalTok{        input\_id }\OperatorTok{+=}\NormalTok{ \_input\_id}
        \ControlFlowTok{if}\NormalTok{ role }\OperatorTok{==} \StringTok{\textquotesingle{}\textless{}|im\_start|\textgreater{}human\textquotesingle{}}\NormalTok{:}
            \CommentTok{\# user 不需要拟合}
\NormalTok{            \_target }\OperatorTok{=}\NormalTok{ im\_start }\OperatorTok{+}\NormalTok{ [IGNORE\_TOKEN\_ID] }\OperatorTok{*}\NormalTok{ (}\BuiltInTok{len}\NormalTok{(\_input\_id)}\OperatorTok{{-}}\DecValTok{3}\NormalTok{) }\OperatorTok{+}\NormalTok{ im\_end }\OperatorTok{+}\NormalTok{ nl\_tokens}
        \ControlFlowTok{elif}\NormalTok{ role }\OperatorTok{==} \StringTok{\textquotesingle{}\textless{}|im\_start|\textgreater{}assistant\textquotesingle{}}\NormalTok{:}
            \CommentTok{\# assistant 需要拟合}
\NormalTok{            \_target }\OperatorTok{=}\NormalTok{ im\_start }\OperatorTok{+}\NormalTok{ [IGNORE\_TOKEN\_ID] }\OperatorTok{*} \BuiltInTok{len}\NormalTok{(tokenizer(role).input\_ids) }\OperatorTok{+} \OperatorTok{\textbackslash{}}
\NormalTok{                \_input\_id[}\BuiltInTok{len}\NormalTok{(tokenizer(role).input\_ids)}\OperatorTok{+}\DecValTok{1}\NormalTok{:}\OperatorTok{{-}}\DecValTok{2}\NormalTok{] }\OperatorTok{+}\NormalTok{ im\_end }\OperatorTok{+}\NormalTok{ nl\_tokens}
        \ControlFlowTok{else}\NormalTok{:}
            \BuiltInTok{print}\NormalTok{(role)}
            \ControlFlowTok{raise} \PreprocessorTok{NotImplementedError}
\NormalTok{        target }\OperatorTok{+=}\NormalTok{ \_target}
    \ControlFlowTok{assert} \BuiltInTok{len}\NormalTok{(input\_id) }\OperatorTok{==} \BuiltInTok{len}\NormalTok{(target)}
    \CommentTok{\# 最后进行 PAD}
\NormalTok{    input\_id }\OperatorTok{+=}\NormalTok{ [tokenizer.pad\_token\_id] }\OperatorTok{*}\NormalTok{ (max\_len }\OperatorTok{{-}} \BuiltInTok{len}\NormalTok{(input\_id))}
\NormalTok{    target }\OperatorTok{+=}\NormalTok{ [IGNORE\_TOKEN\_ID] }\OperatorTok{*}\NormalTok{ (max\_len }\OperatorTok{{-}} \BuiltInTok{len}\NormalTok{(target))}
\NormalTok{    input\_ids.append(input\_id[:max\_len])}
\NormalTok{    targets.append(target[:max\_len])}
\end{Highlighting}
\end{Shaded}

上述代码沿承了 Qwen 的 Chat Template
逻辑,读者也可以根据自己的偏好进行修改,其核心点在于 User
的文本不需要拟合,因此 targets 中 User 对应的文本内容是使用的
IGNORE\_TOKEN\_ID 进行遮蔽,而 Assistant
对应的文本内容则是文本原文,是需要计算 loss 的。目前主流 LLM
IGNORE\_TOKEN\_ID 一般设置为 -100。

完成拼接后,将 tokenize 后的数值序列转化为
\texttt{Torch.tensor},再拼接成 Dataset 所需的字典返回即可:

\begin{Shaded}
\begin{Highlighting}[]
\NormalTok{input\_ids }\OperatorTok{=}\NormalTok{ torch.tensor(input\_ids)}
\NormalTok{targets }\OperatorTok{=}\NormalTok{ torch.tensor(targets)}

\ControlFlowTok{return} \BuiltInTok{dict}\NormalTok{(}
\NormalTok{    input\_ids}\OperatorTok{=}\NormalTok{input\_ids,}
\NormalTok{    labels}\OperatorTok{=}\NormalTok{targets,}
\NormalTok{    attention\_mask}\OperatorTok{=}\NormalTok{input\_ids.ne(tokenizer.pad\_token\_id),}
\NormalTok{)}
\end{Highlighting}
\end{Shaded}

完成上述处理逻辑后,需要自定义一个 Dataset
类,在该类中调用该逻辑进行数据的处理:

\begin{Shaded}
\begin{Highlighting}[]
\KeywordTok{class}\NormalTok{ SupervisedDataset(Dataset):}

    \KeywordTok{def} \FunctionTok{\_\_init\_\_}\NormalTok{(}\VariableTok{self}\NormalTok{, raw\_data, tokenizer, max\_len: }\BuiltInTok{int}\NormalTok{):}
        \BuiltInTok{super}\NormalTok{(SupervisedDataset, }\VariableTok{self}\NormalTok{).}\FunctionTok{\_\_init\_\_}\NormalTok{()}
        \CommentTok{\# 加载并预处理数据}
\NormalTok{        sources }\OperatorTok{=}\NormalTok{ [example[}\StringTok{"conversations"}\NormalTok{] }\ControlFlowTok{for}\NormalTok{ example }\KeywordTok{in}\NormalTok{ raw\_data]}
        \CommentTok{\# preprocess 即上文定义的数据预处理逻辑}
\NormalTok{        data\_dict }\OperatorTok{=}\NormalTok{ preprocess(sources, tokenizer, max\_len)}

        \VariableTok{self}\NormalTok{.input\_ids }\OperatorTok{=}\NormalTok{ data\_dict[}\StringTok{"input\_ids"}\NormalTok{]}
        \VariableTok{self}\NormalTok{.labels }\OperatorTok{=}\NormalTok{ data\_dict[}\StringTok{"labels"}\NormalTok{]}
        \VariableTok{self}\NormalTok{.attention\_mask }\OperatorTok{=}\NormalTok{ data\_dict[}\StringTok{"attention\_mask"}\NormalTok{]}

    \KeywordTok{def} \FunctionTok{\_\_len\_\_}\NormalTok{(}\VariableTok{self}\NormalTok{):}
        \ControlFlowTok{return} \BuiltInTok{len}\NormalTok{(}\VariableTok{self}\NormalTok{.input\_ids)}

    \KeywordTok{def} \FunctionTok{\_\_getitem\_\_}\NormalTok{(}\VariableTok{self}\NormalTok{, i) }\OperatorTok{{-}\textgreater{}}\NormalTok{ Dict[}\BuiltInTok{str}\NormalTok{, torch.Tensor]:}
        \ControlFlowTok{return} \BuiltInTok{dict}\NormalTok{(}
\NormalTok{            input\_ids}\OperatorTok{=}\VariableTok{self}\NormalTok{.input\_ids[i],}
\NormalTok{            labels}\OperatorTok{=}\VariableTok{self}\NormalTok{.labels[i],}
\NormalTok{            attention\_mask}\OperatorTok{=}\VariableTok{self}\NormalTok{.attention\_mask[i],}
\NormalTok{        )}
\end{Highlighting}
\end{Shaded}

该类继承自 Torch 的 Dataset 类,可以直接在 Trainer
中使用。完成数据处理后,基于上一节脚本,修改数据处理逻辑即可,后续模型训练等几乎完全一致,此处附上主函数逻辑:

\begin{Shaded}
\begin{Highlighting}[]
\CommentTok{\# 加载脚本参数}
\NormalTok{parser }\OperatorTok{=}\NormalTok{ HfArgumentParser((ModelArguments, DataTrainingArguments, TrainingArguments))}
\NormalTok{model\_args, data\_args, training\_args }\OperatorTok{=}\NormalTok{ parser.parse\_args\_into\_dataclasses()}

\CommentTok{\# 初始化 SwanLab}
\NormalTok{swanlab.init(project}\OperatorTok{=}\StringTok{"sft"}\NormalTok{, experiment\_name}\OperatorTok{=}\StringTok{"qwen{-}1.5b"}\NormalTok{)}

\CommentTok{\# 设置日志}
\NormalTok{logging.basicConfig(}
    \BuiltInTok{format}\OperatorTok{=}\StringTok{"}\SpecialCharTok{\%(asctime)s}\StringTok{ {-} }\SpecialCharTok{\%(levelname)s}\StringTok{ {-} }\SpecialCharTok{\%(name)s}\StringTok{ {-} }\SpecialCharTok{\%(message)s}\StringTok{"}\NormalTok{,}
\NormalTok{    datefmt}\OperatorTok{=}\StringTok{"\%m/}\SpecialCharTok{\%d}\StringTok{/\%Y \%H:\%M:\%S"}\NormalTok{,}
\NormalTok{    handlers}\OperatorTok{=}\NormalTok{[logging.StreamHandler(sys.stdout)],}
\NormalTok{)}

\CommentTok{\# 将日志级别设置为 INFO}
\NormalTok{transformers.utils.logging.set\_verbosity\_info()}
\NormalTok{log\_level }\OperatorTok{=}\NormalTok{ training\_args.get\_process\_log\_level()}
\NormalTok{logger.setLevel(log\_level)}
\NormalTok{datasets.utils.logging.set\_verbosity(log\_level)}
\NormalTok{transformers.utils.logging.set\_verbosity(log\_level)}
\NormalTok{transformers.utils.logging.enable\_default\_handler()}
\NormalTok{transformers.utils.logging.enable\_explicit\_format()}

\CommentTok{\# 训练整体情况记录}
\NormalTok{logger.warning(}
    \SpecialStringTok{f"Process rank: }\SpecialCharTok{\{}\NormalTok{training\_args}\SpecialCharTok{.}\NormalTok{local\_rank}\SpecialCharTok{\}}\SpecialStringTok{, device: }\SpecialCharTok{\{}\NormalTok{training\_args}\SpecialCharTok{.}\NormalTok{device}\SpecialCharTok{\}}\SpecialStringTok{, n\_gpu: }\SpecialCharTok{\{}\NormalTok{training\_args}\SpecialCharTok{.}\NormalTok{n\_gpu}\SpecialCharTok{\}}\SpecialStringTok{"}
    \OperatorTok{+} \SpecialStringTok{f"distributed training: }\SpecialCharTok{\{}\BuiltInTok{bool}\NormalTok{(training\_args.local\_rank }\OperatorTok{!=} \OperatorTok{{-}}\DecValTok{1}\NormalTok{)}\SpecialCharTok{\}}\SpecialStringTok{, 16{-}bits training: }\SpecialCharTok{\{}\NormalTok{training\_args}\SpecialCharTok{.}\NormalTok{fp16}\SpecialCharTok{\}}\SpecialStringTok{"}
\NormalTok{)}
\NormalTok{logger.info(}\SpecialStringTok{f"Training/evaluation parameters }\SpecialCharTok{\{}\NormalTok{training\_args}\SpecialCharTok{\}}\SpecialStringTok{"}\NormalTok{)}

\CommentTok{\# 检查 checkpoint}
\NormalTok{last\_checkpoint }\OperatorTok{=} \VariableTok{None}
\ControlFlowTok{if}\NormalTok{ os.path.isdir(training\_args.output\_dir):}
\NormalTok{    last\_checkpoint }\OperatorTok{=}\NormalTok{ get\_last\_checkpoint(training\_args.output\_dir)}
    \ControlFlowTok{if}\NormalTok{ last\_checkpoint }\KeywordTok{is} \VariableTok{None} \KeywordTok{and} \BuiltInTok{len}\NormalTok{(os.listdir(training\_args.output\_dir)) }\OperatorTok{\textgreater{}} \DecValTok{0}\NormalTok{:}
        \ControlFlowTok{raise} \PreprocessorTok{ValueError}\NormalTok{(}
            \SpecialStringTok{f"输出路径 (}\SpecialCharTok{\{}\NormalTok{training\_args}\SpecialCharTok{.}\NormalTok{output\_dir}\SpecialCharTok{\}}\SpecialStringTok{) 非空 "}
\NormalTok{        )}
    \ControlFlowTok{elif}\NormalTok{ last\_checkpoint }\KeywordTok{is} \KeywordTok{not} \VariableTok{None} \KeywordTok{and}\NormalTok{ training\_args.resume\_from\_checkpoint }\KeywordTok{is} \VariableTok{None}\NormalTok{:}
\NormalTok{        logger.info(}
            \SpecialStringTok{f"从 }\SpecialCharTok{\{}\NormalTok{last\_checkpoint}\SpecialCharTok{\}}\SpecialStringTok{恢复训练"}
\NormalTok{        )}

\CommentTok{\# 设置随机数种子.}
\NormalTok{set\_seed(training\_args.seed)}

\CommentTok{\# 初始化模型}
\NormalTok{logger.warning(}\StringTok{"加载预训练模型"}\NormalTok{)}
\NormalTok{logger.info(}\SpecialStringTok{f"模型参数地址:}\SpecialCharTok{\{}\NormalTok{model\_args}\SpecialCharTok{.}\NormalTok{model\_name\_or\_path}\SpecialCharTok{\}}\SpecialStringTok{"}\NormalTok{)}
\NormalTok{model }\OperatorTok{=}\NormalTok{ AutoModelForCausalLM.from\_pretrained(model\_args.model\_name\_or\_path,trust\_remote\_code}\OperatorTok{=}\VariableTok{True}\NormalTok{)}
\NormalTok{n\_params }\OperatorTok{=} \BuiltInTok{sum}\NormalTok{(\{p.data\_ptr(): p.numel() }\ControlFlowTok{for}\NormalTok{ p }\KeywordTok{in}\NormalTok{ model.parameters()\}.values())}
\NormalTok{logger.info(}\SpecialStringTok{f"继承一个预训练模型 {-} Total size=}\SpecialCharTok{\{}\NormalTok{n\_params}\OperatorTok{/}\DecValTok{2}\OperatorTok{**}\DecValTok{20}\SpecialCharTok{:.2f\}}\SpecialStringTok{M params"}\NormalTok{)}

\CommentTok{\# 初始化 Tokenizer}
\NormalTok{tokenizer }\OperatorTok{=}\NormalTok{ AutoTokenizer.from\_pretrained(model\_args.model\_name\_or\_path)}
\NormalTok{logger.info(}\StringTok{"完成 tokenzier 加载"}\NormalTok{)}

\CommentTok{\# 加载微调数据}
\ControlFlowTok{with} \BuiltInTok{open}\NormalTok{(data\_args.train\_files) }\ImportTok{as}\NormalTok{ f:}
\NormalTok{    lst }\OperatorTok{=}\NormalTok{ [json.loads(line) }\ControlFlowTok{for}\NormalTok{ line }\KeywordTok{in}\NormalTok{ f.readlines()[:}\DecValTok{10000}\NormalTok{]]}
\NormalTok{logger.info(}\StringTok{"完成训练集加载"}\NormalTok{)}
\NormalTok{logger.info(}\SpecialStringTok{f"训练集地址:}\SpecialCharTok{\{}\NormalTok{data\_args}\SpecialCharTok{.}\NormalTok{train\_files}\SpecialCharTok{\}}\SpecialStringTok{"}\NormalTok{)}
\NormalTok{logger.info(}\SpecialStringTok{f\textquotesingle{}训练样本总数:}\SpecialCharTok{\{}\BuiltInTok{len}\NormalTok{(lst)}\SpecialCharTok{\}}\SpecialStringTok{\textquotesingle{}}\NormalTok{)}
\CommentTok{\# logger.info(f"训练集采样:\{ds["train"][0]\}")}

\NormalTok{train\_dataset }\OperatorTok{=}\NormalTok{ SupervisedDataset(lst, tokenizer}\OperatorTok{=}\NormalTok{tokenizer, max\_len}\OperatorTok{=}\DecValTok{2048}\NormalTok{)}

\NormalTok{logger.info(}\StringTok{"初始化 Trainer"}\NormalTok{)}
\NormalTok{trainer }\OperatorTok{=}\NormalTok{ Trainer(}
\NormalTok{    model}\OperatorTok{=}\NormalTok{model,}
\NormalTok{    args}\OperatorTok{=}\NormalTok{training\_args,}
\NormalTok{    train\_dataset}\OperatorTok{=}\NormalTok{ IterableWrapper(train\_dataset),}
\NormalTok{    tokenizer}\OperatorTok{=}\NormalTok{tokenizer}
\NormalTok{)}

\CommentTok{\# 从 checkpoint 加载}
\NormalTok{checkpoint }\OperatorTok{=} \VariableTok{None}
\ControlFlowTok{if}\NormalTok{ training\_args.resume\_from\_checkpoint }\KeywordTok{is} \KeywordTok{not} \VariableTok{None}\NormalTok{:}
\NormalTok{    checkpoint }\OperatorTok{=}\NormalTok{ training\_args.resume\_from\_checkpoint}
\ControlFlowTok{elif}\NormalTok{ last\_checkpoint }\KeywordTok{is} \KeywordTok{not} \VariableTok{None}\NormalTok{:}
\NormalTok{        checkpoint }\OperatorTok{=}\NormalTok{ last\_checkpoint}

\NormalTok{logger.info(}\StringTok{"开始训练"}\NormalTok{)}
\NormalTok{train\_result }\OperatorTok{=}\NormalTok{ trainer.train(resume\_from\_checkpoint}\OperatorTok{=}\NormalTok{checkpoint)}
\NormalTok{trainer.save\_model() }
\end{Highlighting}
\end{Shaded}

启动方式也同样在 sh 脚本中使用 deepspeed 启动即可,此处不再赘述,源码见
./code/finetune.sh。

\subsection{6.3 高效微调}\label{ux9ad8ux6548ux5faeux8c03}

在前面几节,我们详细介绍了基于 Transformers 框架对模型进行 Pretrain、SFT
以及 RLHF 的原理和实践细节。但是,由于 LLM
参数量大,训练数据多,通过上述方式对模型进行训练(主要指 SFT 及
RLHF)需要调整模型全部参数,资源压力非常大。对资源有限的企业或课题组来说,如何高效、快速对模型进行领域或任务的微调,以低成本地使用
LLM 完成目标任务,是非常重要的。

\subsubsection{6.3.1
高效微调方案}\label{ux9ad8ux6548ux5faeux8c03ux65b9ux6848}

针对全量微调的昂贵问题,目前主要有两种解决方案:

\textbf{Adapt Tuning}。即在模型中添加 Adapter
层,在微调时冻结原参数,仅更新 Adapter 层。

具体而言,其在预训练模型每层中插入用于下游任务的参数,即 Adapter
模块,在微调时冻结模型主体,仅训练特定于任务的参数,如图6.8所示。

\begin{verbatim}
<img src="https://raw.githubusercontent.com/datawhalechina/happy-llm/main/docs/images/6-images/3-1.png" alt="alt text" width="90%" />
<p>图6.8 Adapt Tuning</p>
\end{verbatim}

每个 Adapter 模块由两个前馈子层组成,第一个前馈子层将 Transformer
块的输出作为输入,将原始输入维度 \(d\) 投影到 \(m\),通过控制 \(m\)
的大小来限制 Adapter 模块的参数量,通常情况下
\(m << d\)。在输出阶段,通过第二个前馈子层还原输入维度,将 \(m\)
重新投影到 \(d\),作为 Adapter 模块的输出(如上图右侧结构)。

LoRA 事实上就是一种改进的 Adapt Tuning 方法。但 Adapt Tuning
方法存在推理延迟问题,由于增加了额外参数和额外计算量,导致微调之后的模型计算速度相较原预训练模型更慢。

\textbf{Prefix Tuning}。该种方法固定预训练 LM,为 LM
添加可训练,任务特定的前缀,这样就可以为不同任务保存不同的前缀,微调成本也小。具体而言,在每一个输入
token 前构造一段与下游任务相关的 virtual tokens 作为
prefix,在微调时只更新 prefix 部分的参数,而其他参数冻结不变。

也是目前常用的微量微调方法的 Ptuning,其实就是 Prefix Tuning
的一种改进。但 Prefix Tuning
也存在固定的缺陷:模型可用序列长度减少。由于加入了 virtual
tokens,占用了可用序列长度,因此越高的微调质量,模型可用序列长度就越低。

\subsubsection{6.3.2 LoRA 微调}\label{lora-ux5faeux8c03}

如果一个大模型是将数据映射到高维空间进行处理,这里假定在处理一个细分的小任务时,是不需要那么复杂的大模型的,可能只需要在某个子空间范围内就可以解决,那么也就不需要对全量参数进行优化了,我们可以定义当对某个子空间参数进行优化时,能够达到全量参数优化的性能的一定水平(如90\%精度)时,那么这个子空间参数矩阵的秩就可以称为对应当前待解决问题的本征秩(intrinsic
rank)。

预训练模型本身就隐式地降低了本征秩,当针对特定任务进行微调后,模型中权重矩阵其实具有更低的本征秩(intrinsic
rank)。同时,越简单的下游任务,对应的本征秩越低。(\href{https://arxiv.org/abs/2012.13255}{Intrinsic
Dimensionality Explains the Effectiveness of Language Model
Fine-Tuning})因此,权重更新的那部分参数矩阵尽管随机投影到较小的子空间,仍然可以有效的学习,可以理解为针对特定的下游任务这些权重矩阵就不要求满秩。我们可以通过优化密集层在适应过程中变化的秩分解矩阵来间接训练神经网络中的一些密集层,从而实现仅优化密集层的秩分解矩阵来达到微调效果。

例如,假设预训练参数为
\(\theta^D_0\),在特定下游任务上密集层权重参数矩阵对应的本征秩为
\(\theta^d\),对应特定下游任务微调参数为 \(\theta^D\),那么有:

\[\theta^D = \theta^D_0 + \theta^d M\]

这个 \(M\) 即为 LoRA 优化的秩分解矩阵。

想对于其他高效微调方法,LoRA 存在以下优势:

\begin{enumerate}
\def\labelenumi{\arabic{enumi}.}
\tightlist
\item
  可以针对不同的下游任务构建小型 LoRA
  模块,从而在共享预训练模型参数基础上有效地切换下游任务。
\item
  LoRA 使用自适应优化器(Adaptive
  Optimizer),不需要计算梯度或维护大多数参数的优化器状态,训练更有效、硬件门槛更低。
\item
  LoRA
  使用简单的线性设计,在部署时将可训练矩阵与冻结权重合并,不存在推理延迟。
\item
  LoRA 与其他方法正交,可以组合。
\end{enumerate}

因此,LoRA 成为目前高效微调 LLM
的主流方法,尤其是对于资源受限、有监督训练数据受限的情况下,LoRA
微调往往会成为 LLM 微调的首选方法。

\subsubsection{6.3.3 LoRA
微调的原理}\label{lora-ux5faeux8c03ux7684ux539fux7406}

\paragraph{(1)低秩参数化更新矩阵}\label{ux4f4eux79e9ux53c2ux6570ux5316ux66f4ux65b0ux77e9ux9635}

LoRA 假设权重更新的过程中也有一个较低的本征秩,对于预训练的权重参数矩阵
\(W0 \in R^{d \times k}\) (\(d\) 为上一层输出维度,\(k\)
为下一层输入维度),使用低秩分解来表示其更新:

\[W_0 + {\Delta}W = W_0 + BA \space\space  where \space B \in R^{d \times r}, A \in R^{r \times k}\]

在训练过程中,\(W_0\) 冻结不更新,\(A\)、\(B\) 包含可训练参数。

因此,LoRA 的前向传递函数为:

\[h = W_0 x + \Delta W x = W_0 x + B A x\]

在开始训练时,对 \(A\) 使用随机高斯初始化,对 \(B\)
使用零初始化,然后使用 Adam 进行优化。

训练思路如图6.9所示:

\begin{verbatim}
<img src="https://raw.githubusercontent.com/datawhalechina/happy-llm/main/docs/images/6-images/3-2.jpg" alt="alt text" width="90%" />
<p>图6.9 LoRA</p>
\end{verbatim}

\paragraph{(2)应用于
Transformer}\label{ux5e94ux7528ux4e8e-transformer}

在 Transformer 结构中,LoRA
技术主要应用在注意力模块的四个权重矩阵:\(W_q\)、\(W_k\)、\(W_v\)、\(W_0\),而冻结
MLP 的权重矩阵。

通过消融实验发现同时调整 \(W_q\) 和 \(W_v\) 会产生最佳结果。

在上述条件下,可训练参数个数为:

\[\Theta = 2 \times L_{LoRA} \times d_{model} \times r\]

其中,\(L_{LoRA}\) 为应用 LoRA 的权重矩阵的个数,\(d_{model}\) 为
Transformer 的输入输出维度,\(r\) 为设定的 LoRA 秩。

一般情况下,r 取到 4、8、16。

\subsubsection{6.3.4 LoRA
的代码实现}\label{lora-ux7684ux4ee3ux7801ux5b9eux73b0}

目前一般通过 peft 库来实现模型的 LoRA 微调。peft 库是 huggingface
开发的第三方库,其中封装了包括 LoRA、Adapt Tuning、P-tuning
等多种高效微调方法,可以基于此便捷地实现模型的 LoRA 微调。

本文简单解析 peft 库中的 LoRA 微调代码,简单分析 LoRA 微调的代码实现。

\paragraph{(1)实现流程}\label{ux5b9eux73b0ux6d41ux7a0b}

LoRA 微调的内部实现流程主要包括以下几个步骤:

\begin{enumerate}
\def\labelenumi{\arabic{enumi}.}
\item
  确定要使用 LoRA 的层。peft 库目前支持调用 LoRA
  的层包括:nn.Linear、nn.Embedding、nn.Conv2d 三种。
\item
  对每一个要使用 LoRA 的层,替换为 LoRA 层。所谓 LoRA
  层,实则是在该层原结果基础上增加了一个旁路,通过低秩分解(即矩阵 \(A\)
  和矩阵 \(B\))来模拟参数更新。
\item
  冻结原参数,进行微调,更新 LoRA 层参数。
\end{enumerate}

\paragraph{(2)确定 LoRA 层}\label{ux786eux5b9a-lora-ux5c42}

在进行 LoRA 微调时,首先需要确定 LoRA 微调参数,其中一个重要参数即是
target\_modules。target\_modules
一般是一个字符串列表,每一个字符串是需要进行 LoRA 的层名称,例如:

\begin{Shaded}
\begin{Highlighting}[]
\NormalTok{target\_modules }\OperatorTok{=}\NormalTok{ [}\StringTok{"q\_proj"}\NormalTok{,}\StringTok{"v\_proj"}\NormalTok{]}
\end{Highlighting}
\end{Shaded}

这里的 q\_proj 即为注意力机制中的 \(W_q\), v\_proj 即为注意力机制中的
\(W_v\)。我们可以根据模型架构和任务要求自定义需要进行 LoRA 操作的层。

在创建 LoRA
模型时,会获取该参数,然后在原模型中找到对应的层,该操作主要通过使用 re
对层名进行正则匹配实现:

\begin{Shaded}
\begin{Highlighting}[]
\CommentTok{\# 找到模型的各个组件中,名字里带"q\_proj","v\_proj"的}
\NormalTok{target\_module\_found }\OperatorTok{=}\NormalTok{ re.fullmatch(}\VariableTok{self}\NormalTok{.peft\_config.target\_modules, key)}
\CommentTok{\# 这里的 key,是模型的组件名}
\end{Highlighting}
\end{Shaded}

\paragraph{(3)替换 LoRA 层}\label{ux66ffux6362-lora-ux5c42}

对于找到的每一个目标层,会创建一个新的 LoRA 层进行替换。

LoRA 层在具体实现上,是定义了一个基于 Lora 基类的 Linear
类,该类同时继承了 nn.Linear 和 LoraLayer。LoraLayer 即是 Lora
基类,其主要构造了 LoRA 的各种超参:

\begin{Shaded}
\begin{Highlighting}[]
\KeywordTok{class}\NormalTok{ LoraLayer:}
    \KeywordTok{def} \FunctionTok{\_\_init\_\_}\NormalTok{(}
        \VariableTok{self}\NormalTok{,}
\NormalTok{        r: }\BuiltInTok{int}\NormalTok{, }\CommentTok{\# LoRA 的秩}
\NormalTok{        lora\_alpha: }\BuiltInTok{int}\NormalTok{, }\CommentTok{\# 归一化参数}
\NormalTok{        lora\_dropout: }\BuiltInTok{float}\NormalTok{, }\CommentTok{\# LoRA 层的 dropout 比例}
\NormalTok{        merge\_weights: }\BuiltInTok{bool}\NormalTok{, }\CommentTok{\# eval 模式中,是否将 LoRA 矩阵的值加到原权重矩阵上}
\NormalTok{    ):}
        \VariableTok{self}\NormalTok{.r }\OperatorTok{=}\NormalTok{ r}
        \VariableTok{self}\NormalTok{.lora\_alpha }\OperatorTok{=}\NormalTok{ lora\_alpha}
        \CommentTok{\# Optional dropout}
        \ControlFlowTok{if}\NormalTok{ lora\_dropout }\OperatorTok{\textgreater{}} \FloatTok{0.0}\NormalTok{:}
            \VariableTok{self}\NormalTok{.lora\_dropout }\OperatorTok{=}\NormalTok{ nn.Dropout(p}\OperatorTok{=}\NormalTok{lora\_dropout)}
        \ControlFlowTok{else}\NormalTok{:}
            \VariableTok{self}\NormalTok{.lora\_dropout }\OperatorTok{=} \KeywordTok{lambda}\NormalTok{ x: x}
        \CommentTok{\# Mark the weight as unmerged}
        \VariableTok{self}\NormalTok{.merged }\OperatorTok{=} \VariableTok{False}
        \VariableTok{self}\NormalTok{.merge\_weights }\OperatorTok{=}\NormalTok{ merge\_weights}
        \VariableTok{self}\NormalTok{.disable\_adapters }\OperatorTok{=} \VariableTok{False}
\end{Highlighting}
\end{Shaded}

nn.Linear 就是 Pytorch 的线性层实现。Linear 类就是具体的 LoRA
层,其主要实现如下:

\begin{Shaded}
\begin{Highlighting}[]
\KeywordTok{class}\NormalTok{ Linear(nn.Linear, LoraLayer):}
    \CommentTok{\# LoRA 层}
    \KeywordTok{def} \FunctionTok{\_\_init\_\_}\NormalTok{(}
        \VariableTok{self}\NormalTok{,}
\NormalTok{        in\_features: }\BuiltInTok{int}\NormalTok{,}
\NormalTok{        out\_features: }\BuiltInTok{int}\NormalTok{,}
\NormalTok{        r: }\BuiltInTok{int} \OperatorTok{=} \DecValTok{0}\NormalTok{,}
\NormalTok{        lora\_alpha: }\BuiltInTok{int} \OperatorTok{=} \DecValTok{1}\NormalTok{,}
\NormalTok{        lora\_dropout: }\BuiltInTok{float} \OperatorTok{=} \FloatTok{0.0}\NormalTok{,}
\NormalTok{        fan\_in\_fan\_out: }\BuiltInTok{bool} \OperatorTok{=} \VariableTok{False}\NormalTok{, }
\NormalTok{        merge\_weights: }\BuiltInTok{bool} \OperatorTok{=} \VariableTok{True}\NormalTok{,}
        \OperatorTok{**}\NormalTok{kwargs,}
\NormalTok{    ):}
        \CommentTok{\# 继承两个基类的构造函数}
\NormalTok{        nn.Linear.}\FunctionTok{\_\_init\_\_}\NormalTok{(}\VariableTok{self}\NormalTok{, in\_features, out\_features, }\OperatorTok{**}\NormalTok{kwargs)}
\NormalTok{        LoraLayer.}\FunctionTok{\_\_init\_\_}\NormalTok{(}\VariableTok{self}\NormalTok{, r}\OperatorTok{=}\NormalTok{r, lora\_alpha}\OperatorTok{=}\NormalTok{lora\_alpha, lora\_dropout}\OperatorTok{=}\NormalTok{lora\_dropout, merge\_weights}\OperatorTok{=}\NormalTok{merge\_weights)}

        \VariableTok{self}\NormalTok{.fan\_in\_fan\_out }\OperatorTok{=}\NormalTok{ fan\_in\_fan\_out}
        \CommentTok{\# Actual trainable parameters}
        \ControlFlowTok{if}\NormalTok{ r }\OperatorTok{\textgreater{}} \DecValTok{0}\NormalTok{:}
            \CommentTok{\# 参数矩阵 A}
            \VariableTok{self}\NormalTok{.lora\_A }\OperatorTok{=}\NormalTok{ nn.Linear(in\_features, r, bias}\OperatorTok{=}\VariableTok{False}\NormalTok{)}
            \CommentTok{\# 参数矩阵 B}
            \VariableTok{self}\NormalTok{.lora\_B }\OperatorTok{=}\NormalTok{ nn.Linear(r, out\_features, bias}\OperatorTok{=}\VariableTok{False}\NormalTok{)}
            \CommentTok{\# 归一化系数}
            \VariableTok{self}\NormalTok{.scaling }\OperatorTok{=} \VariableTok{self}\NormalTok{.lora\_alpha }\OperatorTok{/} \VariableTok{self}\NormalTok{.r}
            \CommentTok{\# 冻结原参数,仅更新 A 和 B}
            \VariableTok{self}\NormalTok{.weight.requires\_grad }\OperatorTok{=} \VariableTok{False}
        \CommentTok{\# 初始化 A 和 B}
        \VariableTok{self}\NormalTok{.reset\_parameters()}
        \ControlFlowTok{if}\NormalTok{ fan\_in\_fan\_out:}
            \VariableTok{self}\NormalTok{.weight.data }\OperatorTok{=} \VariableTok{self}\NormalTok{.weight.data.T}
\end{Highlighting}
\end{Shaded}

替换时,直接将原层的 weight 和 bias 复制给新的 LoRA 层,再将新的 LoRA
层分配到指定设备即可。

\paragraph{(4)训练}\label{ux8badux7ec3}

实现了 LoRA 层的替换后,进行微调训练即可。由于在 LoRA
层中已冻结原参数,在训练中只有 A 和 B
的参数会被更新,从而实现了高效微调。训练的整体过程与原 Fine-tune
类似,此处不再赘述。由于采用了 LoRA 方式,forward 函数也会对应调整:

\begin{Shaded}
\begin{Highlighting}[]
    \KeywordTok{def}\NormalTok{ forward(}\VariableTok{self}\NormalTok{, x: torch.Tensor):}
        \ControlFlowTok{if} \VariableTok{self}\NormalTok{.disable\_adapters:}
            \ControlFlowTok{if} \VariableTok{self}\NormalTok{.r }\OperatorTok{\textgreater{}} \DecValTok{0} \KeywordTok{and} \VariableTok{self}\NormalTok{.merged:}
                \VariableTok{self}\NormalTok{.weight.data }\OperatorTok{{-}=}\NormalTok{ (}
\NormalTok{                    transpose(}\VariableTok{self}\NormalTok{.lora\_B.weight }\OperatorTok{@} \VariableTok{self}\NormalTok{.lora\_A.weight, }\VariableTok{self}\NormalTok{.fan\_in\_fan\_out) }\OperatorTok{*} \VariableTok{self}\NormalTok{.scaling}
\NormalTok{                )}
                \VariableTok{self}\NormalTok{.merged }\OperatorTok{=} \VariableTok{False}

            \ControlFlowTok{return}\NormalTok{ F.linear(x, transpose(}\VariableTok{self}\NormalTok{.weight, }\VariableTok{self}\NormalTok{.fan\_in\_fan\_out), bias}\OperatorTok{=}\VariableTok{self}\NormalTok{.bias)}
        \CommentTok{\textquotesingle{}\textquotesingle{}\textquotesingle{}主要分支\textquotesingle{}\textquotesingle{}\textquotesingle{}}
        \ControlFlowTok{elif} \VariableTok{self}\NormalTok{.r }\OperatorTok{\textgreater{}} \DecValTok{0} \KeywordTok{and} \KeywordTok{not} \VariableTok{self}\NormalTok{.merged:}
\NormalTok{            result }\OperatorTok{=}\NormalTok{ F.linear(x, transpose(}\VariableTok{self}\NormalTok{.weight, }\VariableTok{self}\NormalTok{.fan\_in\_fan\_out), bias}\OperatorTok{=}\VariableTok{self}\NormalTok{.bias)}
            \ControlFlowTok{if} \VariableTok{self}\NormalTok{.r }\OperatorTok{\textgreater{}} \DecValTok{0}\NormalTok{:}
\NormalTok{                result }\OperatorTok{+=} \VariableTok{self}\NormalTok{.lora\_B(}\VariableTok{self}\NormalTok{.lora\_A(}\VariableTok{self}\NormalTok{.lora\_dropout(x))) }\OperatorTok{*} \VariableTok{self}\NormalTok{.scaling}
            \ControlFlowTok{return}\NormalTok{ result}
        \ControlFlowTok{else}\NormalTok{:}
            \ControlFlowTok{return}\NormalTok{ F.linear(x, transpose(}\VariableTok{self}\NormalTok{.weight, }\VariableTok{self}\NormalTok{.fan\_in\_fan\_out), bias}\OperatorTok{=}\VariableTok{self}\NormalTok{.bias)}
\end{Highlighting}
\end{Shaded}

上述代码由于考虑到参数合并问题,有几个分支,此处我们仅阅读第二个分支即
elif 分支即可。基于 LoRA
的前向计算过程如前文公式所示,首先计算原参数与输入的乘积,再加上 A、B
分别与输入的乘积即可。

\subsubsection{6.3.5 使用 peft 实现 LoRA
微调}\label{ux4f7fux7528-peft-ux5b9eux73b0-lora-ux5faeux8c03}

peft
进行了很好的封装,支持我们便捷、高效地对大模型进行微调。此处以第二节的
LLM SFT 为例,简要介绍如何使用 peft 对大模型进行微调。如果是应用在 RLHF
上,整体思路是一致的。

首先加载所需使用库:

\begin{Shaded}
\begin{Highlighting}[]
\ImportTok{import}\NormalTok{ torch.nn }\ImportTok{as}\NormalTok{ nn}
\ImportTok{from}\NormalTok{ transformers }\ImportTok{import}\NormalTok{ AutoTokenizer, AutoModel}
\ImportTok{from}\NormalTok{ peft }\ImportTok{import}\NormalTok{ get\_peft\_model, LoraConfig, TaskType, PeftModel}
\ImportTok{from}\NormalTok{ transformers }\ImportTok{import}\NormalTok{ Trainer}
\end{Highlighting}
\end{Shaded}

其次加载原模型与原 tokenizer,此处和第二节一致:

\begin{Shaded}
\begin{Highlighting}[]
\CommentTok{\# 加载基座模型}
\NormalTok{tokenizer }\OperatorTok{=}\NormalTok{ AutoTokenizer.from\_pretrained(MODEL\_PATH, trust\_remote\_code}\OperatorTok{=}\VariableTok{True}\NormalTok{)}
\NormalTok{model }\OperatorTok{=}\NormalTok{ AutoModel.from\_pretrained(}
\NormalTok{    MODEL\_PATH, trust\_remote\_code}\OperatorTok{=}\VariableTok{True}
\NormalTok{)}
\end{Highlighting}
\end{Shaded}

接着,设定 peft 参数:

\begin{Shaded}
\begin{Highlighting}[]
\NormalTok{peft\_config }\OperatorTok{=}\NormalTok{ LoraConfig(}
\NormalTok{            task\_type}\OperatorTok{=}\NormalTok{TaskType.CAUSAL\_LM,}
\NormalTok{            inference\_mode}\OperatorTok{=}\VariableTok{False}\NormalTok{,}
\NormalTok{            r}\OperatorTok{=}\DecValTok{8}\NormalTok{,}
\NormalTok{            lora\_alpha}\OperatorTok{=}\DecValTok{32}\NormalTok{,}
\NormalTok{            lora\_dropout}\OperatorTok{=}\FloatTok{0.1}\NormalTok{,}
\NormalTok{        )}
\end{Highlighting}
\end{Shaded}

注意,对不同的模型,LoRA 参数可能有所区别。例如,对于 ChatGLM,无需指定
target\_modeules,peft 可以自行找到;对于
BaiChuan,就需要手动指定。task\_type 是模型的任务类型,大模型一般都是
CAUSAL\_LM 即传统语言模型。

然后获取 LoRA 模型:

\begin{Shaded}
\begin{Highlighting}[]
\NormalTok{model }\OperatorTok{=}\NormalTok{ get\_peft\_model(model, peft\_config)}
\end{Highlighting}
\end{Shaded}

此处的 get\_peft\_model 的底层操作,即为上文分析的具体实现。

最后使用 transformers 提供的 Trainer
进行训练即可,训练占用的显存就会有大幅度的降低:

\begin{Shaded}
\begin{Highlighting}[]
\NormalTok{trainer }\OperatorTok{=}\NormalTok{ Trainer(}
\NormalTok{    model}\OperatorTok{=}\NormalTok{model,}
\NormalTok{    args}\OperatorTok{=}\NormalTok{training\_args,}
\NormalTok{    train\_dataset}\OperatorTok{=}\NormalTok{ IterableWrapper(train\_dataset),}
\NormalTok{    tokenizer}\OperatorTok{=}\NormalTok{tokenizer}
\NormalTok{)}
\NormalTok{trainer.train()}
\end{Highlighting}
\end{Shaded}

如果是应用在 DPO、KTO 上,则也相同的加入 LoRA 参数并通过
\texttt{get\_peft\_model} 获取一个 LoRA
模型即可,其他的不需要进行任何修改。但要注意的是,LoRA
微调能够大幅度降低显卡占用,且在下游任务适配上能够取得较好的效果,但如果是需要学习对应知识的任务,LoRA
由于只调整低秩矩阵,难以实现知识的注入,一般效果不佳,因此不推荐使用
LoRA 进行模型预训练或后训练。

\textbf{参考资料}

{[}1{]} Neil Houlsby, Andrei Giurgiu, Stanislaw Jastrzebski, Bruna
Morrone, Quentin de Laroussilhe, Andrea Gesmundo, Mona Attariyan, and
Sylvain Gelly. (2019). \emph{Parameter-Efficient Transfer Learning for
NLP.} arXiv preprint arXiv:1902.00751.

{[}2{]} Edward J. Hu, Yelong Shen, Phillip Wallis, Zeyuan Allen-Zhu,
Yuanzhi Li, Shean Wang, Lu Wang, and Weizhu Chen. (2021). \emph{LoRA:
Low-Rank Adaptation of Large Language Models.} arXiv preprint
arXiv:2106.09685.

{[}3{]} Armen Aghajanyan, Luke Zettlemoyer, and Sonal Gupta. (2020).
\emph{Intrinsic Dimensionality Explains the Effectiveness of Language
Model Fine-Tuning.} arXiv preprint arXiv:2012.13255.

{[}4{]} Xiang Lisa Li 和 Percy Liang. (2021). \emph{Prefix-Tuning:
Optimizing Continuous Prompts for Generation.} arXiv preprint
arXiv:2101.00190.

\end{document}


% 第七章
\chapter{大模型应用}
% Options for packages loaded elsewhere
\PassOptionsToPackage{unicode}{hyperref}
\PassOptionsToPackage{hyphens}{url}
\documentclass[
]{article}
\usepackage{xcolor}
\usepackage{amsmath,amssymb}
\setcounter{secnumdepth}{5}
\usepackage{iftex}
\ifPDFTeX
  \usepackage[T1]{fontenc}
  \usepackage[utf8]{inputenc}
  \usepackage{textcomp} % provide euro and other symbols
\else % if luatex or xetex
  \usepackage{unicode-math} % this also loads fontspec
  \defaultfontfeatures{Scale=MatchLowercase}
  \defaultfontfeatures[\rmfamily]{Ligatures=TeX,Scale=1}
\fi
\usepackage{lmodern}
\ifPDFTeX\else
  % xetex/luatex font selection
\fi
% Use upquote if available, for straight quotes in verbatim environments
\IfFileExists{upquote.sty}{\usepackage{upquote}}{}
\IfFileExists{microtype.sty}{% use microtype if available
  \usepackage[]{microtype}
  \UseMicrotypeSet[protrusion]{basicmath} % disable protrusion for tt fonts
}{}
\makeatletter
\@ifundefined{KOMAClassName}{% if non-KOMA class
  \IfFileExists{parskip.sty}{%
    \usepackage{parskip}
  }{% else
    \setlength{\parindent}{0pt}
    \setlength{\parskip}{6pt plus 2pt minus 1pt}}
}{% if KOMA class
  \KOMAoptions{parskip=half}}
\makeatother
\usepackage{color}
\usepackage{fancyvrb}
\newcommand{\VerbBar}{|}
\newcommand{\VERB}{\Verb[commandchars=\\\{\}]}
\DefineVerbatimEnvironment{Highlighting}{Verbatim}{commandchars=\\\{\}}
% Add ',fontsize=\small' for more characters per line
\newenvironment{Shaded}{}{}
\newcommand{\AlertTok}[1]{\textcolor[rgb]{1.00,0.00,0.00}{\textbf{#1}}}
\newcommand{\AnnotationTok}[1]{\textcolor[rgb]{0.38,0.63,0.69}{\textbf{\textit{#1}}}}
\newcommand{\AttributeTok}[1]{\textcolor[rgb]{0.49,0.56,0.16}{#1}}
\newcommand{\BaseNTok}[1]{\textcolor[rgb]{0.25,0.63,0.44}{#1}}
\newcommand{\BuiltInTok}[1]{\textcolor[rgb]{0.00,0.50,0.00}{#1}}
\newcommand{\CharTok}[1]{\textcolor[rgb]{0.25,0.44,0.63}{#1}}
\newcommand{\CommentTok}[1]{\textcolor[rgb]{0.38,0.63,0.69}{\textit{#1}}}
\newcommand{\CommentVarTok}[1]{\textcolor[rgb]{0.38,0.63,0.69}{\textbf{\textit{#1}}}}
\newcommand{\ConstantTok}[1]{\textcolor[rgb]{0.53,0.00,0.00}{#1}}
\newcommand{\ControlFlowTok}[1]{\textcolor[rgb]{0.00,0.44,0.13}{\textbf{#1}}}
\newcommand{\DataTypeTok}[1]{\textcolor[rgb]{0.56,0.13,0.00}{#1}}
\newcommand{\DecValTok}[1]{\textcolor[rgb]{0.25,0.63,0.44}{#1}}
\newcommand{\DocumentationTok}[1]{\textcolor[rgb]{0.73,0.13,0.13}{\textit{#1}}}
\newcommand{\ErrorTok}[1]{\textcolor[rgb]{1.00,0.00,0.00}{\textbf{#1}}}
\newcommand{\ExtensionTok}[1]{#1}
\newcommand{\FloatTok}[1]{\textcolor[rgb]{0.25,0.63,0.44}{#1}}
\newcommand{\FunctionTok}[1]{\textcolor[rgb]{0.02,0.16,0.49}{#1}}
\newcommand{\ImportTok}[1]{\textcolor[rgb]{0.00,0.50,0.00}{\textbf{#1}}}
\newcommand{\InformationTok}[1]{\textcolor[rgb]{0.38,0.63,0.69}{\textbf{\textit{#1}}}}
\newcommand{\KeywordTok}[1]{\textcolor[rgb]{0.00,0.44,0.13}{\textbf{#1}}}
\newcommand{\NormalTok}[1]{#1}
\newcommand{\OperatorTok}[1]{\textcolor[rgb]{0.40,0.40,0.40}{#1}}
\newcommand{\OtherTok}[1]{\textcolor[rgb]{0.00,0.44,0.13}{#1}}
\newcommand{\PreprocessorTok}[1]{\textcolor[rgb]{0.74,0.48,0.00}{#1}}
\newcommand{\RegionMarkerTok}[1]{#1}
\newcommand{\SpecialCharTok}[1]{\textcolor[rgb]{0.25,0.44,0.63}{#1}}
\newcommand{\SpecialStringTok}[1]{\textcolor[rgb]{0.73,0.40,0.53}{#1}}
\newcommand{\StringTok}[1]{\textcolor[rgb]{0.25,0.44,0.63}{#1}}
\newcommand{\VariableTok}[1]{\textcolor[rgb]{0.10,0.09,0.49}{#1}}
\newcommand{\VerbatimStringTok}[1]{\textcolor[rgb]{0.25,0.44,0.63}{#1}}
\newcommand{\WarningTok}[1]{\textcolor[rgb]{0.38,0.63,0.69}{\textbf{\textit{#1}}}}
\setlength{\emergencystretch}{3em} % prevent overfull lines
\providecommand{\tightlist}{%
  \setlength{\itemsep}{0pt}\setlength{\parskip}{0pt}}
\usepackage{bookmark}
\IfFileExists{xurl.sty}{\usepackage{xurl}}{} % add URL line breaks if available
\urlstyle{same}
\hypersetup{
  hidelinks,
  pdfcreator={LaTeX via pandoc}}

\author{}
\date{}

\begin{document}

{
\setcounter{tocdepth}{3}
\tableofcontents
}
\section{大模型应用}\label{ux5927ux6a21ux578bux5e94ux7528}

\subsection{7.1 LLM 的评测}\label{llm-ux7684ux8bc4ux6d4b}

近年来,随着人工智能领域的迅猛发展,大规模预训练语言模型(简称大模型)成为了推动技术进步的核心力量。这些大模型在自然语言处理等任务中展现出了令人惊叹的能力。然而,要准确衡量一个大模型的性能,必须依靠科学而合理的评测。

什么是大模型评测?大模型评测就是通过各种标准化的方法和数据集,对大模型在不同任务上的表现进行量化和比较。这些评测不仅包括模型在特定任务上的准确性,还涉及模型的泛化能力、推理速度、资源消耗等多个方面。通过评测,我们能够更全面地了解大模型的实际表现,以及它们在现实世界中的应用潜力。

大模型的开发成本高昂,涉及大量的计算资源和数据,因此评测对于确保模型的实际价值至关重要。首先,评测能够揭示模型在各种任务中的表现,帮助研究人员和企业判断模型的适用性和可靠性。其次,评测可以暴露模型的潜在弱点,例如偏见、鲁棒性问题等,从而为进一步优化和改进提供依据。此外,公平、公开的评测还为学术界和工业界提供了一个共同的标准,促进了技术的交流与进步。

\subsubsection{7.1.1 LLM
的评测数据集}\label{llm-ux7684ux8bc4ux6d4bux6570ux636eux96c6}

在大模型的评测过程中,使用标准化的评测集至关重要。目前,主流的大模型评测集主要从以下几个方面进行评估,每个评测集都有其独特的用途和典型应用场景:

\begin{enumerate}
\def\labelenumi{\arabic{enumi}.}
\tightlist
\item
  \textbf{通用评测集}:

  \begin{itemize}
  \tightlist
  \item
    \textbf{MMLU(Massive Multitask Language
    Understanding)}:MMLU评测模型在多种任务中的理解能力,包括各类学科和知识领域。具体包含了历史、数学、物理、生物、法律等任务类型,全面考察模型在不同学科的知识储备和语言理解能力。
  \end{itemize}
\item
  \textbf{工具使用评测集}:

  \begin{itemize}
  \tightlist
  \item
    \textbf{BFCL
    V2}:用于评测模型在复杂工具使用任务中的表现,特别是在执行多步骤操作时的正确性和效率。这些任务通常涉及与数据库交互或执行特定指令,以模拟实际工具使用场景。
  \end{itemize}
\item
  \textbf{数学评测集}:

  \begin{itemize}
  \tightlist
  \item
    \textbf{GSM8K}:GSM8K是一个包含小学数学问题的数据集,用于测试模型的数学推理和逻辑分析能力。具体任务包括算术运算、简单方程求解、数字推理等。GSM8K中的问题虽然看似简单,但模型需要理解问题语义并进行正确的数学运算,体现了逻辑推理和语言理解的双重挑战。
  \item
    \textbf{MATH}:MATH数据集用于测试模型在更复杂的数学问题上的表现,包括代数和几何。
  \end{itemize}
\item
  \textbf{推理评测集}:

  \begin{itemize}
  \tightlist
  \item
    \textbf{ARC Challenge}:ARC
    Challenge评测模型在科学推理任务中的表现,尤其是常识性和科学性问题的解答,典型应用场景包括科学考试题解答和百科问答系统的开发。
  \item
    \textbf{GPQA}:用于评测模型在零样本条件下对开放性问题的回答能力,通常应用于客服聊天机器人和知识问答系统中,帮助模型在缺乏特定领域数据的情况下给出合理的回答。
  \item
    \textbf{HellaSwag}:评测模型在复杂语境下选择最符合逻辑的答案的能力,适用于故事续写、对话生成等需要高水平理解和推理的场景。
  \end{itemize}
\item
  \textbf{长文本理解评测集}:

  \begin{itemize}
  \tightlist
  \item
    \textbf{InfiniteBench/En.MC}:评测模型在处理长文本阅读理解方面的能力,尤其是对科学文献的理解,适用于学术文献自动摘要、长篇报道分析等应用场景。
  \item
    \textbf{NIH/Multi-needle}:用于测试模型在多样本长文档环境中的理解和总结能力,应用于政府报告解读、企业内部长文档分析等需要处理海量信息的场景。
  \end{itemize}
\item
  \textbf{多语言评测集}:

  \begin{itemize}
  \tightlist
  \item
    \textbf{MGSM}:用于评估模型在不同语言下的数学问题解决能力,考察模型的多语言适应性,尤其适用于国际化环境中的数学教育和跨语言技术支持场景。
  \end{itemize}
\end{enumerate}

这些评测集的多样性帮助我们全面评估大模型在不同任务和应用场景中的表现,确保模型在处理多样化任务时能够保持高效和精准的表现。例如,在MMLU评测中,某些大模型在历史、物理等学科任务中表现优异,展现出对多领域知识的深度理解;在GSM8K数学评测中,最新的大模型在算术和方程求解方面表现接近甚至超越了一些人类基准,显示出在复杂数学推理任务中的潜力。这些实际评测结果展示了模型在各类复杂任务中的进步和应用潜力。

\subsubsection{7.1.2
主流的评测榜单}\label{ux4e3bux6d41ux7684ux8bc4ux6d4bux699cux5355}

大模型的评测不仅限于使用特定的数据集,许多机构还会根据评测结果发布模型排行榜,这些榜单为学术界和工业界提供了重要的参考,帮助他们了解当前最前沿的技术和模型。以下是一些主流的评测榜单:

\paragraph{Open LLM Leaderboard}\label{open-llm-leaderboard}

Open LLM Leaderboard 为由 Hugging Face
提供的开放式榜单,汇集了多个开源大模型的评测结果,帮助用户了解不同模型在各种任务上的表现。该榜单通过多个标准化测试集来评估模型的性能,并通过持续更新的方式反映最新的技术进展,为研究者和开发者提供了高价值的对比参考,如图7.1所示。

\begin{figure}[htbp]\centering
\includegraphics[width=0.9\textwidth]{https://raw.githubusercontent.com/datawhalechina/happy-llm/main/docs/images/7-images/7-1-Open%20LLM%20Leaderboard.png}
\caption{图 7.1 Open LLM Leaderboard}
\end{figure}

\paragraph{Lmsys Chatbot Arena
Leaderboard}\label{lmsys-chatbot-arena-leaderboard}

由lmsys提供的聊天机器人评测榜单,通过多维度的评估,展示各类大模型在对话任务中的能力。该榜单采用真实用户与模型交互的方式来评测对话质量,重点考察模型的自然语言生成能力、上下文理解能力以及用户满意度,是当前评估聊天机器人性能的重要工具,如图7.2所示。

\begin{figure}[htbp]\centering
\includegraphics[width=0.9\textwidth]{https://raw.githubusercontent.com/datawhalechina/happy-llm/main/docs/images/7-images/7-1-lmsys%20Chatbot%20Arena%20Leaderboard.png}
\caption{图7.2 Lmsys Chatbot Arena Leaderboard}
\end{figure}

\paragraph{OpenCompass}\label{opencompass}

OpenCompass
是国内的评测榜单,针对大模型在多种语言和任务上的表现进行评估,提供了中国市场特定应用的参考。该榜单结合了中文语言理解和多语言能力的测试,以适应本地化需求,并特别关注大模型在中文语境下的准确性、鲁棒性和适应性,为国内企业和研究者选择合适的模型提供了重要参考。

\begin{figure}[htbp]\centering
\includegraphics[width=0.9\textwidth]{https://raw.githubusercontent.com/datawhalechina/happy-llm/main/docs/images/7-images/7-1-opencompass.png}
\caption{图7.3 OpenCompass}
\end{figure}

\subsubsection{7.1.3
特定的评测榜单}\label{ux7279ux5b9aux7684ux8bc4ux6d4bux699cux5355}

另外,还有针对不同领域特定任务的大模型评测榜单,如图7.4所示。这些榜单专注于特定应用领域,帮助用户了解大模型在某一垂直领域的能力:

\begin{itemize}
\item
  金融榜:基于CFBenchmark评测集,评估大模型在金融自然语言处理、金融预测计算、金融分析与安全检查等多项基础任务中的能力。由同济大学与上海人工智能实验室及东方财经提供。
\item
  安全榜:基于Flames评测集,评估大模型在公平、安全、数据保护以及合法五大维度的抗性,帮助深入了解模型在安全性上的表现。由上海人工智能实验室与复旦大学提供。
\item
  通识榜:基于BotChat评测集,评估大语言模型生成日常多轮对话能力的综合程度,判断模型在对话中是否具备类人水平。由上海人工智能实验室提供。
\item
  法律榜:基于LawBench评测集,评估模型在法律领域的理解、推理和应用能力,涵盖法律问题回答、文本生成、法律判例分析等任务。由南京大学提供。
\item
  医疗榜:基于MedBench评测集,评估大语言模型在医学知识问答、安全伦理理解等方面的表现。由上海人工智能实验室提供。
\end{itemize}

\begin{figure}[htbp]\centering
\includegraphics[width=0.9\textwidth]{https://raw.githubusercontent.com/datawhalechina/happy-llm/main/docs/images/7-images/7-1-垂直领域榜单.png}
\caption{图7.4 垂直领域榜单}
\end{figure}

\subsection{7.2 RAG}\label{rag}

\subsubsection{7.2.1 RAG
的基本原理}\label{rag-ux7684ux57faux672cux539fux7406}

大语言模型(LLM)在生成内容时,虽然具备强大的语言理解和生成能力,但也面临着一些挑战。例如,LLM有时会生成不准确或误导性的内容,这被称为大模型``幻觉''。此外,模型所依赖的训练数据可能过时,尤其在面对最新的信息时,生成结果的准确性和时效性难以保证。对于特定领域的专业知识,LLM
的处理效率也较低,无法深入理解复杂的领域知识。因此,如何提升大模型的生成质量和效率,成为了当前研究的重要方向。

在这样的背景下,检索增强生成(Retrieval-Augmented
Generation,RAG)技术应运而生,成为AI领域中的一大创新趋势。RAG
在生成答案之前,首先从外部的大规模文档数据库中检索出相关信息,并将这些信息融入到生成过程之中,从而指导和优化语言模型的输出。这一流程不仅极大地提升了内容生成的准确性和相关性,还使得生成的内容更加符合实时性要求。

RAG
的核心原理在于将``检索''与``生成''结合:当用户提出查询时,系统首先通过检索模块找到与问题相关的文本片段,然后将这些片段作为附加信息传递给语言模型,模型据此生成更为精准和可靠的回答。通过这种方式,RAG
有效缓解了大语言模型的``幻觉''问题,因为生成的内容建立在真实文档的基础上,使得答案更具可追溯性和可信度。同时,由于引入了最新的信息源,RAG
技术大大加快了知识更新速度,使得系统可以及时吸收和反映最新的领域动态。

\subsubsection{7.2.2 搭建一个 RAG
框架}\label{ux642dux5efaux4e00ux4e2a-rag-ux6846ux67b6}

接下来我会带领大家一步一步实现一个简单的RAG模型,这个模型是基于RAG的一个简化版本,我们称之为
Tiny-RAG 。Tiny-RAG只保留了 RAG
的核心功能,即检索和生成,其目的是帮助大家更好地理解 RAG
模型的原理和实现。

\paragraph{Step 1:
RAG流程介绍}\label{step-1-ragux6d41ux7a0bux4ecbux7ecd}

RAG通过在语言模型生成答案之前,先从广泛的文档数据库中检索相关信息,然后利用这些信息来引导生成过程,从而极大地提升了内容的准确性和相关性。RAG有效地缓解了幻觉问题,提高了知识更新的速度,并增强了内容生成的可追溯性,使得大型语言模型在实际应用中变得更加实用和可信。

RAG的基本结构有哪些呢?

\begin{itemize}
\tightlist
\item
  向量化模块:用来将文档片段向量化。
\item
  文档加载和切分模块:用来加载文档并切分成文档片段。
\item
  数据库:存放文档片段及其对应的向量表示。
\item
  检索模块:根据 Query(问题)检索相关的文档片段。
\item
  大模型模块:根据检索到的文档回答用户的问题。
\end{itemize}

上述也就是 TinyRAG 的所有模块内容,如图7.5所示。

\begin{figure}[htbp]\centering
\includegraphics[width=0.9\textwidth]{https://raw.githubusercontent.com/datawhalechina/happy-llm/main/docs/images/7-images/7-2-tinyrag.png}
\caption{图7.5 TinyRAG 项目结构}
\end{figure}

接下来,让我们梳理一下RAG的流程是什么样的呢?

\begin{itemize}
\tightlist
\item
  \textbf{索引}:将文档库分割成较短的片段,并通过编码器构建向量索引。
\item
  \textbf{检索}:根据问题和片段的相似度检索相关文档片段。
\item
  \textbf{生成}:以检索到的上下文为条件,生成问题的回答。
\end{itemize}

如下图7.6所示的流程图,图片出处
\textbf{\emph{\href{https://arxiv.org/pdf/2312.10997.pdf}{Retrieval-Augmented
Generation for Large Language Models: A Survey}}}

\begin{figure}[htbp]\centering
\includegraphics[width=0.9\textwidth]{https://raw.githubusercontent.com/datawhalechina/happy-llm/main/docs/images/7-images/7-2-rag.png}
\caption{图7.6 RAG 流程图}
\end{figure}

\paragraph{Step 2: 向量化}\label{step-2-ux5411ux91cfux5316}

首先我们来动手实现一个向量化的类,这是RAG架构的基础。向量化类主要用来将文档片段向量化,将一段文本映射为一个向量。

首先我们要设置一个 \texttt{BaseEmbeddings}
基类,这样我们在使用其他模型时,只需要继承这个基类,然后在此基础上进行修改即可,方便代码扩展。

\begin{Shaded}
\begin{Highlighting}[]
\KeywordTok{class}\NormalTok{ BaseEmbeddings:}
    \CommentTok{"""}
\CommentTok{    Base class for embeddings}
\CommentTok{    """}
    \KeywordTok{def} \FunctionTok{\_\_init\_\_}\NormalTok{(}\VariableTok{self}\NormalTok{, path: }\BuiltInTok{str}\NormalTok{, is\_api: }\BuiltInTok{bool}\NormalTok{) }\OperatorTok{{-}\textgreater{}} \VariableTok{None}\NormalTok{:}
        \CommentTok{"""}
\CommentTok{        初始化嵌入基类}
\CommentTok{        Args:}
\CommentTok{            path (str): 模型或数据的路径}
\CommentTok{            is\_api (bool): 是否使用API方式。True表示使用在线API服务,False表示使用本地模型}
\CommentTok{        """}
        \VariableTok{self}\NormalTok{.path }\OperatorTok{=}\NormalTok{ path}
        \VariableTok{self}\NormalTok{.is\_api }\OperatorTok{=}\NormalTok{ is\_api}
    
    \KeywordTok{def}\NormalTok{ get\_embedding(}\VariableTok{self}\NormalTok{, text: }\BuiltInTok{str}\NormalTok{, model: }\BuiltInTok{str}\NormalTok{) }\OperatorTok{{-}\textgreater{}}\NormalTok{ List[}\BuiltInTok{float}\NormalTok{]:}
        \CommentTok{"""}
\CommentTok{        获取文本的嵌入向量表示}
\CommentTok{        Args:}
\CommentTok{            text (str): 输入文本}
\CommentTok{            model (str): 使用的模型名称}
\CommentTok{        Returns:}
\CommentTok{            List[float]: 文本的嵌入向量}
\CommentTok{        Raises:}
\CommentTok{            NotImplementedError: 该方法需要在子类中实现}
\CommentTok{        """}
        \ControlFlowTok{raise} \PreprocessorTok{NotImplementedError}
    
    \AttributeTok{@classmethod}
    \KeywordTok{def}\NormalTok{ cosine\_similarity(cls, vector1: List[}\BuiltInTok{float}\NormalTok{], vector2: List[}\BuiltInTok{float}\NormalTok{]) }\OperatorTok{{-}\textgreater{}} \BuiltInTok{float}\NormalTok{:}
        \CommentTok{"""}
\CommentTok{        计算两个向量之间的余弦相似度}
\CommentTok{        Args:}
\CommentTok{            vector1 (List[float]): 第一个向量}
\CommentTok{            vector2 (List[float]): 第二个向量}
\CommentTok{        Returns:}
\CommentTok{            float: 两个向量的余弦相似度,范围在[{-}1,1]之间}
\CommentTok{        """}
        \CommentTok{\# 将输入列表转换为numpy数组,并指定数据类型为float32}
\NormalTok{        v1 }\OperatorTok{=}\NormalTok{ np.array(vector1, dtype}\OperatorTok{=}\NormalTok{np.float32)}
\NormalTok{        v2 }\OperatorTok{=}\NormalTok{ np.array(vector2, dtype}\OperatorTok{=}\NormalTok{np.float32)}

        \CommentTok{\# 检查向量中是否包含无穷大或NaN值}
        \ControlFlowTok{if} \KeywordTok{not}\NormalTok{ np.}\BuiltInTok{all}\NormalTok{(np.isfinite(v1)) }\KeywordTok{or} \KeywordTok{not}\NormalTok{ np.}\BuiltInTok{all}\NormalTok{(np.isfinite(v2)):}
            \ControlFlowTok{return} \FloatTok{0.0}

        \CommentTok{\# 计算向量的点积}
\NormalTok{        dot\_product }\OperatorTok{=}\NormalTok{ np.dot(v1, v2)}
        \CommentTok{\# 计算向量的范数(长度)}
\NormalTok{        norm\_v1 }\OperatorTok{=}\NormalTok{ np.linalg.norm(v1)}
\NormalTok{        norm\_v2 }\OperatorTok{=}\NormalTok{ np.linalg.norm(v2)}
        
        \CommentTok{\# 计算分母(两个向量范数的乘积)}
\NormalTok{        magnitude }\OperatorTok{=}\NormalTok{ norm\_v1 }\OperatorTok{*}\NormalTok{ norm\_v2}
        \CommentTok{\# 处理分母为0的特殊情况}
        \ControlFlowTok{if}\NormalTok{ magnitude }\OperatorTok{==} \DecValTok{0}\NormalTok{:}
            \ControlFlowTok{return} \FloatTok{0.0}
            
        \CommentTok{\# 返回余弦相似度}
        \ControlFlowTok{return}\NormalTok{ dot\_product }\OperatorTok{/}\NormalTok{ magnitude}
\end{Highlighting}
\end{Shaded}

\texttt{BaseEmbeddings}基类有两个主要方法:\texttt{get\_embedding}和\texttt{cosine\_similarity}。\texttt{get\_embedding}用于获取文本的向量表示,\texttt{cosine\_similarity}用于计算两个向量之间的余弦相似度。在初始化类时设置了模型的路径和是否是API模型,例如使用OpenAI的Embedding
API需要设置\texttt{self.is\_api=True}。

继承\texttt{BaseEmbeddings}类只需要实现\texttt{get\_embedding}方法,\texttt{cosine\_similarity}方法会被继承下来。这就是编写基类的好处。

\begin{Shaded}
\begin{Highlighting}[]
\KeywordTok{class}\NormalTok{ OpenAIEmbedding(BaseEmbeddings):}
    \CommentTok{"""}
\CommentTok{    class for OpenAI embeddings}
\CommentTok{    """}
    \KeywordTok{def} \FunctionTok{\_\_init\_\_}\NormalTok{(}\VariableTok{self}\NormalTok{, path: }\BuiltInTok{str} \OperatorTok{=} \StringTok{\textquotesingle{}\textquotesingle{}}\NormalTok{, is\_api: }\BuiltInTok{bool} \OperatorTok{=} \VariableTok{True}\NormalTok{) }\OperatorTok{{-}\textgreater{}} \VariableTok{None}\NormalTok{:}
        \BuiltInTok{super}\NormalTok{().}\FunctionTok{\_\_init\_\_}\NormalTok{(path, is\_api)}
        \ControlFlowTok{if} \VariableTok{self}\NormalTok{.is\_api:}
            \VariableTok{self}\NormalTok{.client }\OperatorTok{=}\NormalTok{ OpenAI()}
            \CommentTok{\# 从环境变量中获取 硅基流动 密钥}
            \VariableTok{self}\NormalTok{.client.api\_key }\OperatorTok{=}\NormalTok{ os.getenv(}\StringTok{"OPENAI\_API\_KEY"}\NormalTok{)}
            \CommentTok{\# 从环境变量中获取 硅基流动 的基础URL}
            \VariableTok{self}\NormalTok{.client.base\_url }\OperatorTok{=}\NormalTok{ os.getenv(}\StringTok{"OPENAI\_BASE\_URL"}\NormalTok{)}
    
    \KeywordTok{def}\NormalTok{ get\_embedding(}\VariableTok{self}\NormalTok{, text: }\BuiltInTok{str}\NormalTok{, model: }\BuiltInTok{str} \OperatorTok{=} \StringTok{"BAAI/bge{-}m3"}\NormalTok{) }\OperatorTok{{-}\textgreater{}}\NormalTok{ List[}\BuiltInTok{float}\NormalTok{]:}
        \CommentTok{"""}
\CommentTok{        此处默认使用轨迹流动的免费嵌入模型 BAAI/bge{-}m3}
\CommentTok{        """}
        \ControlFlowTok{if} \VariableTok{self}\NormalTok{.is\_api:}
\NormalTok{            text }\OperatorTok{=}\NormalTok{ text.replace(}\StringTok{"}\CharTok{\textbackslash{}n}\StringTok{"}\NormalTok{, }\StringTok{" "}\NormalTok{)}
            \ControlFlowTok{return} \VariableTok{self}\NormalTok{.client.embeddings.create(}\BuiltInTok{input}\OperatorTok{=}\NormalTok{[text], model}\OperatorTok{=}\NormalTok{model).data[}\DecValTok{0}\NormalTok{].embedding}
        \ControlFlowTok{else}\NormalTok{:}
            \ControlFlowTok{raise} \PreprocessorTok{NotImplementedError}
\end{Highlighting}
\end{Shaded}

\begin{quote}
注:此处我们默认使用国内用户可访问的\href{https://cloud.siliconflow.cn/i/ybUFvmqK}{硅基流动大模型API服务平台}。
\end{quote}

\paragraph{Step 3:
文档加载和切分}\label{step-3-ux6587ux6863ux52a0ux8f7dux548cux5207ux5206}

接下来我们来实现一个文档加载和切分的类,这个类主要用于加载文档并将其切分成文档片段。

文档可以是文章、书籍、对话、代码等文本内容,例如pdf文件、md文件、txt文件等。完整代码可以在
\textbf{\emph{\href{./RAG/utils.py}{RAG/utils.py}}}
文件中找到。该代码支持加载pdf、md、txt等类型的文件,只需编写相应的函数即可。

\begin{Shaded}
\begin{Highlighting}[]
\KeywordTok{def}\NormalTok{ read\_file\_content(cls, file\_path: }\BuiltInTok{str}\NormalTok{):}
    \CommentTok{\# 根据文件扩展名选择读取方法}
    \ControlFlowTok{if}\NormalTok{ file\_path.endswith(}\StringTok{\textquotesingle{}.pdf\textquotesingle{}}\NormalTok{):}
        \ControlFlowTok{return}\NormalTok{ cls.read\_pdf(file\_path)}
    \ControlFlowTok{elif}\NormalTok{ file\_path.endswith(}\StringTok{\textquotesingle{}.md\textquotesingle{}}\NormalTok{):}
        \ControlFlowTok{return}\NormalTok{ cls.read\_markdown(file\_path)}
    \ControlFlowTok{elif}\NormalTok{ file\_path.endswith(}\StringTok{\textquotesingle{}.txt\textquotesingle{}}\NormalTok{):}
        \ControlFlowTok{return}\NormalTok{ cls.read\_text(file\_path)}
    \ControlFlowTok{else}\NormalTok{:}
        \ControlFlowTok{raise} \PreprocessorTok{ValueError}\NormalTok{(}\StringTok{"Unsupported file type"}\NormalTok{)}
\end{Highlighting}
\end{Shaded}

文档读取后需要进行切分。我们可以设置一个最大的Token长度,然后根据这个最大长度来切分文档。切分文档时最好以句子为单位(按\texttt{\textbackslash{}n}粗切分),并保证片段之间有一些重叠内容,以提高检索的准确性。

\begin{Shaded}
\begin{Highlighting}[]
\KeywordTok{def}\NormalTok{ get\_chunk(cls, text: }\BuiltInTok{str}\NormalTok{, max\_token\_len: }\BuiltInTok{int} \OperatorTok{=} \DecValTok{600}\NormalTok{, cover\_content: }\BuiltInTok{int} \OperatorTok{=} \DecValTok{150}\NormalTok{):}
\NormalTok{    chunk\_text }\OperatorTok{=}\NormalTok{ []}

\NormalTok{    curr\_len }\OperatorTok{=} \DecValTok{0}
\NormalTok{    curr\_chunk }\OperatorTok{=} \StringTok{\textquotesingle{}\textquotesingle{}}

\NormalTok{    token\_len }\OperatorTok{=}\NormalTok{ max\_token\_len }\OperatorTok{{-}}\NormalTok{ cover\_content}
\NormalTok{    lines }\OperatorTok{=}\NormalTok{ text.splitlines()  }\CommentTok{\# 假设以换行符分割文本为行}

    \ControlFlowTok{for}\NormalTok{ line }\KeywordTok{in}\NormalTok{ lines:}
        \CommentTok{\# 保留空格,只移除行首行尾空格}
\NormalTok{        line }\OperatorTok{=}\NormalTok{ line.strip()}
\NormalTok{        line\_len }\OperatorTok{=} \BuiltInTok{len}\NormalTok{(enc.encode(line))}
        
        \ControlFlowTok{if}\NormalTok{ line\_len }\OperatorTok{\textgreater{}}\NormalTok{ max\_token\_len:}
            \CommentTok{\# 如果单行长度就超过限制,则将其分割成多个块}
            \CommentTok{\# 先保存当前块(如果有内容)}
            \ControlFlowTok{if}\NormalTok{ curr\_chunk:}
\NormalTok{                chunk\_text.append(curr\_chunk)}
\NormalTok{                curr\_chunk }\OperatorTok{=} \StringTok{\textquotesingle{}\textquotesingle{}}
\NormalTok{                curr\_len }\OperatorTok{=} \DecValTok{0}
            
            \CommentTok{\# 将长行按token长度分割}
\NormalTok{            line\_tokens }\OperatorTok{=}\NormalTok{ enc.encode(line)}
\NormalTok{            num\_chunks }\OperatorTok{=}\NormalTok{ (}\BuiltInTok{len}\NormalTok{(line\_tokens) }\OperatorTok{+}\NormalTok{ token\_len }\OperatorTok{{-}} \DecValTok{1}\NormalTok{) }\OperatorTok{//}\NormalTok{ token\_len}
            
            \ControlFlowTok{for}\NormalTok{ i }\KeywordTok{in} \BuiltInTok{range}\NormalTok{(num\_chunks):}
\NormalTok{                start\_token }\OperatorTok{=}\NormalTok{ i }\OperatorTok{*}\NormalTok{ token\_len}
\NormalTok{                end\_token }\OperatorTok{=} \BuiltInTok{min}\NormalTok{(start\_token }\OperatorTok{+}\NormalTok{ token\_len, }\BuiltInTok{len}\NormalTok{(line\_tokens))}
                
                \CommentTok{\# 解码token片段回文本}
\NormalTok{                chunk\_tokens }\OperatorTok{=}\NormalTok{ line\_tokens[start\_token:end\_token]}
\NormalTok{                chunk\_part }\OperatorTok{=}\NormalTok{ enc.decode(chunk\_tokens)}
                
                \CommentTok{\# 添加覆盖内容(除了第一个块)}
                \ControlFlowTok{if}\NormalTok{ i }\OperatorTok{\textgreater{}} \DecValTok{0} \KeywordTok{and}\NormalTok{ chunk\_text:}
\NormalTok{                    prev\_chunk }\OperatorTok{=}\NormalTok{ chunk\_text[}\OperatorTok{{-}}\DecValTok{1}\NormalTok{]}
\NormalTok{                    cover\_part }\OperatorTok{=}\NormalTok{ prev\_chunk[}\OperatorTok{{-}}\NormalTok{cover\_content:] }\ControlFlowTok{if} \BuiltInTok{len}\NormalTok{(prev\_chunk) }\OperatorTok{\textgreater{}}\NormalTok{ cover\_content }\ControlFlowTok{else}\NormalTok{ prev\_chunk}
\NormalTok{                    chunk\_part }\OperatorTok{=}\NormalTok{ cover\_part }\OperatorTok{+}\NormalTok{ chunk\_part}
                
\NormalTok{                chunk\_text.append(chunk\_part)}
            
            \CommentTok{\# 重置当前块状态}
\NormalTok{            curr\_chunk }\OperatorTok{=} \StringTok{\textquotesingle{}\textquotesingle{}}
\NormalTok{            curr\_len }\OperatorTok{=} \DecValTok{0}
            
        \ControlFlowTok{elif}\NormalTok{ curr\_len }\OperatorTok{+}\NormalTok{ line\_len }\OperatorTok{+} \DecValTok{1} \OperatorTok{\textless{}=}\NormalTok{ token\_len:  }\CommentTok{\# +1 for newline}
            \CommentTok{\# 当前行可以加入当前块}
            \ControlFlowTok{if}\NormalTok{ curr\_chunk:}
\NormalTok{                curr\_chunk }\OperatorTok{+=} \StringTok{\textquotesingle{}}\CharTok{\textbackslash{}n}\StringTok{\textquotesingle{}}
\NormalTok{                curr\_len }\OperatorTok{+=} \DecValTok{1}
\NormalTok{            curr\_chunk }\OperatorTok{+=}\NormalTok{ line}
\NormalTok{            curr\_len }\OperatorTok{+=}\NormalTok{ line\_len}
        \ControlFlowTok{else}\NormalTok{:}
            \CommentTok{\# 当前行无法加入当前块,开始新块}
            \ControlFlowTok{if}\NormalTok{ curr\_chunk:}
\NormalTok{                chunk\_text.append(curr\_chunk)}
            
            \CommentTok{\# 开始新块,添加覆盖内容}
            \ControlFlowTok{if}\NormalTok{ chunk\_text:}
\NormalTok{                prev\_chunk }\OperatorTok{=}\NormalTok{ chunk\_text[}\OperatorTok{{-}}\DecValTok{1}\NormalTok{]}
\NormalTok{                cover\_part }\OperatorTok{=}\NormalTok{ prev\_chunk[}\OperatorTok{{-}}\NormalTok{cover\_content:] }\ControlFlowTok{if} \BuiltInTok{len}\NormalTok{(prev\_chunk) }\OperatorTok{\textgreater{}}\NormalTok{ cover\_content }\ControlFlowTok{else}\NormalTok{ prev\_chunk}
\NormalTok{                curr\_chunk }\OperatorTok{=}\NormalTok{ cover\_part }\OperatorTok{+} \StringTok{\textquotesingle{}}\CharTok{\textbackslash{}n}\StringTok{\textquotesingle{}} \OperatorTok{+}\NormalTok{ line}
\NormalTok{                curr\_len }\OperatorTok{=} \BuiltInTok{len}\NormalTok{(enc.encode(cover\_part)) }\OperatorTok{+} \DecValTok{1} \OperatorTok{+}\NormalTok{ line\_len}
            \ControlFlowTok{else}\NormalTok{:}
\NormalTok{                curr\_chunk }\OperatorTok{=}\NormalTok{ line}
\NormalTok{                curr\_len }\OperatorTok{=}\NormalTok{ line\_len}

    \CommentTok{\# 添加最后一个块(如果有内容)}
    \ControlFlowTok{if}\NormalTok{ curr\_chunk:}
\NormalTok{        chunk\_text.append(curr\_chunk)}

    \ControlFlowTok{return}\NormalTok{ chunk\_text}
\end{Highlighting}
\end{Shaded}

\paragraph{Step 4:
数据库与向量检索}\label{step-4-ux6570ux636eux5e93ux4e0eux5411ux91cfux68c0ux7d22}

完成文档切分和Embedding模型加载后,需要设计一个向量数据库来存放文档片段和对应的向量表示,以及设计一个检索模块用于根据Query检索相关文档片段。

向量数据库的功能包括:

\begin{itemize}
\tightlist
\item
  \texttt{persist}:数据库持久化保存。
\item
  \texttt{load\_vector}:从本地加载数据库。
\item
  \texttt{get\_vector}:获取文档的向量表示。
\item
  \texttt{query}:根据问题检索相关文档片段。
\end{itemize}

完整代码可以在
\textbf{\emph{\href{./RAG/VectorBase.py}{/VectorBase.py}}} 文件中找到。

\begin{Shaded}
\begin{Highlighting}[]
\KeywordTok{class}\NormalTok{ VectorStore:}
    \KeywordTok{def} \FunctionTok{\_\_init\_\_}\NormalTok{(}\VariableTok{self}\NormalTok{, document: List[}\BuiltInTok{str}\NormalTok{] }\OperatorTok{=}\NormalTok{ [}\StringTok{\textquotesingle{}\textquotesingle{}}\NormalTok{]) }\OperatorTok{{-}\textgreater{}} \VariableTok{None}\NormalTok{:}
        \VariableTok{self}\NormalTok{.document }\OperatorTok{=}\NormalTok{ document}

    \KeywordTok{def}\NormalTok{ get\_vector(}\VariableTok{self}\NormalTok{, EmbeddingModel: BaseEmbeddings) }\OperatorTok{{-}\textgreater{}}\NormalTok{ List[List[}\BuiltInTok{float}\NormalTok{]]:}
        \CommentTok{\# 获得文档的向量表示}
        \ControlFlowTok{pass}

    \KeywordTok{def}\NormalTok{ persist(}\VariableTok{self}\NormalTok{, path: }\BuiltInTok{str} \OperatorTok{=} \StringTok{\textquotesingle{}storage\textquotesingle{}}\NormalTok{):}
        \CommentTok{\# 数据库持久化保存}
        \ControlFlowTok{pass}

    \KeywordTok{def}\NormalTok{ load\_vector(}\VariableTok{self}\NormalTok{, path: }\BuiltInTok{str} \OperatorTok{=} \StringTok{\textquotesingle{}storage\textquotesingle{}}\NormalTok{):}
        \CommentTok{\# 从本地加载数据库}
        \ControlFlowTok{pass}

    \KeywordTok{def}\NormalTok{ query(}\VariableTok{self}\NormalTok{, query: }\BuiltInTok{str}\NormalTok{, EmbeddingModel: BaseEmbeddings, k: }\BuiltInTok{int} \OperatorTok{=} \DecValTok{1}\NormalTok{) }\OperatorTok{{-}\textgreater{}}\NormalTok{ List[}\BuiltInTok{str}\NormalTok{]:}
        \CommentTok{\# 根据问题检索相关文档片段}
        \ControlFlowTok{pass}
\end{Highlighting}
\end{Shaded}

\texttt{query}
方法用于将用户提出的问题向量化,然后在数据库中检索相关文档片段并返回结果。

\begin{Shaded}
\begin{Highlighting}[]
\KeywordTok{def}\NormalTok{ query(}\VariableTok{self}\NormalTok{, query: }\BuiltInTok{str}\NormalTok{, EmbeddingModel: BaseEmbeddings, k: }\BuiltInTok{int} \OperatorTok{=} \DecValTok{1}\NormalTok{) }\OperatorTok{{-}\textgreater{}}\NormalTok{ List[}\BuiltInTok{str}\NormalTok{]:}
\NormalTok{    query\_vector }\OperatorTok{=}\NormalTok{ EmbeddingModel.get\_embedding(query)}
\NormalTok{    result }\OperatorTok{=}\NormalTok{ np.array([}\VariableTok{self}\NormalTok{.get\_similarity(query\_vector, vector) }\ControlFlowTok{for}\NormalTok{ vector }\KeywordTok{in} \VariableTok{self}\NormalTok{.vectors])}
    \ControlFlowTok{return}\NormalTok{ np.array(}\VariableTok{self}\NormalTok{.document)[result.argsort()[}\OperatorTok{{-}}\NormalTok{k:][::}\OperatorTok{{-}}\DecValTok{1}\NormalTok{]].tolist()}
\end{Highlighting}
\end{Shaded}

\paragraph{Step 5:
大模型模块}\label{step-5-ux5927ux6a21ux578bux6a21ux5757}

接下来是大模型模块,用于根据检索到的文档回答用户的问题。

首先实现一个基类,这样可以方便扩展其他模型。

\begin{Shaded}
\begin{Highlighting}[]
\KeywordTok{class}\NormalTok{ BaseModel:}
    \KeywordTok{def} \FunctionTok{\_\_init\_\_}\NormalTok{(}\VariableTok{self}\NormalTok{, path: }\BuiltInTok{str} \OperatorTok{=} \StringTok{\textquotesingle{}\textquotesingle{}}\NormalTok{) }\OperatorTok{{-}\textgreater{}} \VariableTok{None}\NormalTok{:}
        \VariableTok{self}\NormalTok{.path }\OperatorTok{=}\NormalTok{ path}

    \KeywordTok{def}\NormalTok{ chat(}\VariableTok{self}\NormalTok{, prompt: }\BuiltInTok{str}\NormalTok{, history: List[}\BuiltInTok{dict}\NormalTok{], content: }\BuiltInTok{str}\NormalTok{) }\OperatorTok{{-}\textgreater{}} \BuiltInTok{str}\NormalTok{:}
        \ControlFlowTok{pass}

    \KeywordTok{def}\NormalTok{ load\_model(}\VariableTok{self}\NormalTok{):}
        \ControlFlowTok{pass}
\end{Highlighting}
\end{Shaded}

\texttt{BaseModel}
包含两个方法:\texttt{chat}和\texttt{load\_model}。对于本地化运行的开源模型需要实现\texttt{load\_model},而API模型则不需要。在此处我们还是使用国内用户可访问的硅基流动大模型API服务平台,使用API服务的好处就是用户不需要本地的计算资源,可以大大降低学习者的学习门槛。

\begin{Shaded}
\begin{Highlighting}[]
\ImportTok{from}\NormalTok{ openai }\ImportTok{import}\NormalTok{ OpenAI}

\KeywordTok{class}\NormalTok{ OpenAIChat(BaseModel):}
    \KeywordTok{def} \FunctionTok{\_\_init\_\_}\NormalTok{(}\VariableTok{self}\NormalTok{, model: }\BuiltInTok{str} \OperatorTok{=} \StringTok{"Qwen/Qwen2.5{-}32B{-}Instruct"}\NormalTok{) }\OperatorTok{{-}\textgreater{}} \VariableTok{None}\NormalTok{:}
        \VariableTok{self}\NormalTok{.model }\OperatorTok{=}\NormalTok{ model}

    \KeywordTok{def}\NormalTok{ chat(}\VariableTok{self}\NormalTok{, prompt: }\BuiltInTok{str}\NormalTok{, history: List[}\BuiltInTok{dict}\NormalTok{], content: }\BuiltInTok{str}\NormalTok{) }\OperatorTok{{-}\textgreater{}} \BuiltInTok{str}\NormalTok{:}
\NormalTok{        client }\OperatorTok{=}\NormalTok{ OpenAI()}
\NormalTok{        client.api\_key }\OperatorTok{=}\NormalTok{ os.getenv(}\StringTok{"OPENAI\_API\_KEY"}\NormalTok{)   }
\NormalTok{        client.base\_url }\OperatorTok{=}\NormalTok{ os.getenv(}\StringTok{"OPENAI\_BASE\_URL"}\NormalTok{)}
\NormalTok{        history.append(\{}\StringTok{\textquotesingle{}role\textquotesingle{}}\NormalTok{: }\StringTok{\textquotesingle{}user\textquotesingle{}}\NormalTok{, }\StringTok{\textquotesingle{}content\textquotesingle{}}\NormalTok{: RAG\_PROMPT\_TEMPLATE.}\BuiltInTok{format}\NormalTok{(question}\OperatorTok{=}\NormalTok{prompt, context}\OperatorTok{=}\NormalTok{content)\})}
\NormalTok{        response }\OperatorTok{=}\NormalTok{ client.chat.completions.create(}
\NormalTok{                model}\OperatorTok{=}\VariableTok{self}\NormalTok{.model,}
\NormalTok{                messages}\OperatorTok{=}\NormalTok{history,}
\NormalTok{                max\_tokens}\OperatorTok{=}\DecValTok{2048}\NormalTok{,}
\NormalTok{                temperature}\OperatorTok{=}\FloatTok{0.1}
\NormalTok{            )}
        \ControlFlowTok{return}\NormalTok{ response.choices[}\DecValTok{0}\NormalTok{].message.content}
\end{Highlighting}
\end{Shaded}

设计一个专用于RAG的大模型提示词,如下:

\begin{Shaded}
\begin{Highlighting}[]
\NormalTok{RAG\_PROMPT\_TEMPLATE}\OperatorTok{=}\StringTok{"""}
\StringTok{使用以上下文来回答用户的问题。如果你不知道答案,就说你不知道。总是使用中文回答。}
\StringTok{问题: }\SpecialCharTok{\{question\}}
\StringTok{可参考的上下文:}
\StringTok{···}
\SpecialCharTok{\{context\}}
\StringTok{···}
\StringTok{如果给定的上下文无法让你做出回答,请回答数据库中没有这个内容,你不知道。}
\StringTok{有用的回答:}
\StringTok{"""}
\end{Highlighting}
\end{Shaded}

这样我们就可以利用InternLM2模型来做RAG啦!

\paragraph{Step 6: Tiny-RAG Demo}\label{step-6-tiny-rag-demo}

接下来,我们来看看Tiny-RAG的Demo吧!

\begin{Shaded}
\begin{Highlighting}[]
\ImportTok{from}\NormalTok{ VectorBase }\ImportTok{import}\NormalTok{ VectorStore}
\ImportTok{from}\NormalTok{ utils }\ImportTok{import}\NormalTok{ ReadFiles}
\ImportTok{from}\NormalTok{ LLM }\ImportTok{import}\NormalTok{ OpenAIChat}
\ImportTok{from}\NormalTok{ Embeddings }\ImportTok{import}\NormalTok{ OpenAIEmbedding}

\CommentTok{\# 没有保存数据库}
\NormalTok{docs }\OperatorTok{=}\NormalTok{ ReadFiles(}\StringTok{\textquotesingle{}./data\textquotesingle{}}\NormalTok{).get\_content(max\_token\_len}\OperatorTok{=}\DecValTok{600}\NormalTok{, cover\_content}\OperatorTok{=}\DecValTok{150}\NormalTok{) }\CommentTok{\# 获得data目录下的所有文件内容并分割}
\NormalTok{vector }\OperatorTok{=}\NormalTok{ VectorStore(docs)}
\NormalTok{embedding }\OperatorTok{=}\NormalTok{ OpenAIEmbedding() }\CommentTok{\# 创建EmbeddingModel}
\NormalTok{vector.get\_vector(EmbeddingModel}\OperatorTok{=}\NormalTok{embedding)}
\NormalTok{vector.persist(path}\OperatorTok{=}\StringTok{\textquotesingle{}storage\textquotesingle{}}\NormalTok{) }\CommentTok{\# 将向量和文档内容保存到storage目录下,下次再用就可以直接加载本地的数据库}

\CommentTok{\# vector.load\_vector(\textquotesingle{}./storage\textquotesingle{}) \# 加载本地的数据库}

\NormalTok{question }\OperatorTok{=} \StringTok{\textquotesingle{}RAG的原理是什么?\textquotesingle{}}

\NormalTok{content }\OperatorTok{=}\NormalTok{ vector.query(question, EmbeddingModel}\OperatorTok{=}\NormalTok{embedding, k}\OperatorTok{=}\DecValTok{1}\NormalTok{)[}\DecValTok{0}\NormalTok{]}
\NormalTok{chat }\OperatorTok{=}\NormalTok{ OpenAIChat(model}\OperatorTok{=}\StringTok{\textquotesingle{}Qwen/Qwen2.5{-}32B{-}Instruct\textquotesingle{}}\NormalTok{)}
\BuiltInTok{print}\NormalTok{(chat.chat(question, [], content))}
\end{Highlighting}
\end{Shaded}

也可以从本地加载已处理好的数据库:

\begin{Shaded}
\begin{Highlighting}[]
\ImportTok{from}\NormalTok{ VectorBase }\ImportTok{import}\NormalTok{ VectorStore}
\ImportTok{from}\NormalTok{ utils }\ImportTok{import}\NormalTok{ ReadFiles}
\ImportTok{from}\NormalTok{ LLM }\ImportTok{import}\NormalTok{ OpenAIChat}
\ImportTok{from}\NormalTok{ Embeddings }\ImportTok{import}\NormalTok{ OpenAIEmbedding}

\CommentTok{\# 保存数据库之后}
\NormalTok{vector }\OperatorTok{=}\NormalTok{ VectorStore()}

\NormalTok{vector.load\_vector(}\StringTok{\textquotesingle{}./storage\textquotesingle{}}\NormalTok{) }\CommentTok{\# 加载本地的数据库}

\NormalTok{question }\OperatorTok{=} \StringTok{\textquotesingle{}RAG的原理是什么?\textquotesingle{}}

\NormalTok{embedding }\OperatorTok{=}\NormalTok{ ZhipuEmbedding() }\CommentTok{\# 创建EmbeddingModel}

\NormalTok{content }\OperatorTok{=}\NormalTok{ vector.query(question, EmbeddingModel}\OperatorTok{=}\NormalTok{embedding, k}\OperatorTok{=}\DecValTok{1}\NormalTok{)[}\DecValTok{0}\NormalTok{]}
\NormalTok{chat }\OperatorTok{=}\NormalTok{ OpenAIChat(model}\OperatorTok{=}\StringTok{\textquotesingle{}Qwen/Qwen2.5{-}32B{-}Instruct\textquotesingle{}}\NormalTok{)}
\BuiltInTok{print}\NormalTok{(chat.chat(question, [], content))}
\end{Highlighting}
\end{Shaded}

\begin{quote}
注:7.2 章节的所有代码均可在
\href{https://github.com/datawhalechina/happy-llm/tree/main/docs/chapter7/RAG}{Happy-LLM
Chapter7 RAG} 中找到。
\end{quote}

\subsection{7.3 Agent}\label{agent}

\subsubsection{7.3.1 什么是 LLM
Agent?}\label{ux4ec0ux4e48ux662f-llm-agent}

简单来说,大模型Agent是一个以LLM为核心``大脑'',并赋予其自主规划、记忆和使用工具能力的系统。
它不再仅仅是被动地响应用户的提示(Prompt),而是能够:

\begin{enumerate}
\def\labelenumi{\arabic{enumi}.}
\tightlist
\item
  理解目标(Goal Understanding):
  接收一个相对复杂或高层次的目标(例如,``帮我规划一个周末去北京的旅游行程并预订机票酒店'')。
\item
  自主规划(Planning):
  将大目标分解成一系列可执行的小步骤(例如,``搜索北京景点''、``查询天气''、``比较机票价格''、``查找合适的酒店''、``调用预订API''等)。
\item
  记忆(Memory):
  拥有短期记忆(记住当前任务的上下文)和长期记忆(从过去的交互或外部知识库中学习和检索信息)。
\item
  工具使用(Tool Use):
  调用外部API、插件或代码执行环境来获取信息(如搜索引擎、数据库)、执行操作(如发送邮件、预订服务)或进行计算。
\item
  反思与迭代(Reflection \& Iteration):
  (在更高级的Agent中)能够评估自己的行为和结果,从中学习并调整后续计划。
\end{enumerate}

传统的LLM像一个知识渊博但只能纸上谈兵的图书馆员,而 LLM Agent
则更像一个全能的私人助理,不仅懂得多,还能跑腿办事,甚至能主动思考最优方案。

\begin{figure}[htbp]\centering
\includegraphics[width=0.9\textwidth]{https://raw.githubusercontent.com/datawhalechina/happy-llm/main/docs/images/7-images/7-3-Agent工作原理.png}
\caption{图7.7 Agent 工作原理}
\end{figure}

LLM Agent
通过将大型语言模型的强大语言理解和生成能力与规划、记忆和工具使用等关键模块相结合,实现了超越传统大模型的自主性和复杂任务处理能力,这种能力使得
LLM Agent
在许多垂直领域(如法律、医疗、金融等)都具有广泛的应用潜力,如图7.7所示
Agent 工作原理。

\subsubsection{7.3.2 LLM Agent
的类型}\label{llm-agent-ux7684ux7c7bux578b}

虽然LLM
Agent的概念还在快速发展中,但根据其设计理念和能力侧重,我们可以大致将其分为几类:

任务导向型Agent(Task-Oriented Agents): - 特点:
专注于完成特定领域的、定义明确的任务,例如客户服务、代码生成、数据分析等。
- 工作方式:
通常有预设的流程和可调用的特定工具集。LLM主要负责理解用户意图、填充任务槽位、生成回应或调用合适-
的工具。 - 例子:
专门用于预订餐厅的聊天机器人、辅助编程的代码助手(如GitHub
Copilot在某些高级功能上体现了Agent特性)。

规划与推理型Agent(Planning \& Reasoning Agents): - 特点:
强调自主分解复杂任务、制定多步计划,并根据环境反馈进行调整的能力。它们通常需要更强的推理能力。
- 工作方式: 常采用特定的思维框架,如ReAct
(Reason+Act),让模型先进行``思考''(Reasoning)分析当前情况和所需行动,然后执行``行动''(Action)调用工具,再根据工具返回结果进行下一轮思考。Chain-of-Thought
(CoT) 等提示工程技术也是其推理的基础。 - 例子:
需要整合网络搜索、计算器、数据库查询等多种工具来回答复杂问题的研究型Agent,或者能够自主完成``写一篇关于XX主题的报告,并配上相关数据图表''这类任务的Agent。

多Agent系统(Multi-Agent Systems): - 特点:
由多个具有不同角色或能力的Agent协同工作,共同完成一个更宏大的目标。 -
工作方式:
Agent之间可以进行通信、协作、辩论甚至竞争。例如,一个Agent负责规划,一个负责执行,一个负责审查。
- 例子:
模拟软件开发团队(产品经理Agent、程序员Agent、测试员Agent)来自动生成和测试代码;模拟一个公司组织结构来完成商业策划。AutoGen、ChatDev等框架支持这类系统的构建。

探索与学习型Agent(Exploration \& Learning Agents): - 特点:
这类Agent不仅执行任务,还能在与环境的交互中主动学习新知识、新技能或优化自身策略,类似于强化学习中的Agent概念。
- 工作方式:
可能包含更复杂的记忆和反思机制,能够根据成功或失败的经验调整未来的规划和行动。
- 例子:
能在未知软件环境中自主探索学习如何操作的Agent,或者在玩游戏时不断提升策略的Agent。

\subsubsection{7.3.3 动手构造一个
Tiny-Agent}\label{ux52a8ux624bux6784ux9020ux4e00ux4e2a-tiny-agent}

我们来基于 \texttt{openai} 库和其 \texttt{tool\_calls}
功能,动手构造一个 Tiny-Agent,这个 Agent 是一个简单的任务导向型
Agent,它能够根据用户的输入,回答一些简单的问题。

最终的实现效果如图7.8所示:

\begin{figure}[htbp]\centering
\includegraphics[width=1.0\textwidth]{https://raw.githubusercontent.com/datawhalechina/happy-llm/main/docs/images/7-images/7-3-tinyagent-example.png}
\caption{图7.8 效果示意图}
\end{figure}

\paragraph{Step 1 :
初始化客户端和模型}\label{step-1-ux521dux59cbux5316ux5ba2ux6237ux7aefux548cux6a21ux578b}

首先,我们需要一个能够调用大模型的客户端。这里我们使用 \texttt{openai}
库,并配置其指向一个兼容 OpenAI API 的服务终端,例如
\href{https://cloud.siliconflow.cn/i/ybUFvmqK}{SiliconFlow}。同时,指定要使用的模型,如
\texttt{Qwen/Qwen2.5-32B-Instruct}。

\begin{Shaded}
\begin{Highlighting}[]
\ImportTok{from}\NormalTok{ openai }\ImportTok{import}\NormalTok{ OpenAI}

\CommentTok{\# 初始化 OpenAI 客户端}
\NormalTok{client }\OperatorTok{=}\NormalTok{ OpenAI(}
\NormalTok{    api\_key}\OperatorTok{=}\StringTok{"YOUR\_API\_KEY"}\NormalTok{,  }\CommentTok{\# 替换为你的 API Key}
\NormalTok{    base\_url}\OperatorTok{=}\StringTok{"https://api.siliconflow.cn/v1"}\NormalTok{, }\CommentTok{\# 使用 SiliconFlow 的 API 地址}
\NormalTok{)}

\CommentTok{\# 指定模型名称}
\NormalTok{model\_name }\OperatorTok{=} \StringTok{"Qwen/Qwen2.5{-}32B{-}Instruct"}
\end{Highlighting}
\end{Shaded}

\begin{quote}
\textbf{注意:} 你需要将 \texttt{YOUR\_API\_KEY} 替换为你从
\href{https://cloud.siliconflow.cn/i/ybUFvmqK}{SiliconFlow}
或其他服务商获取的有效 API Key。
\end{quote}

\paragraph{Step 2:
定义工具函数}\label{step-2-ux5b9aux4e49ux5de5ux5177ux51fdux6570}

我们在 \texttt{src/tools.py} 文件中定义 Agent
可以使用的工具函数。每个函数都需要有清晰的文档字符串(docstring),描述其功能和参数,因为这将用于自动生成工具的
JSON Schema。

\begin{Shaded}
\begin{Highlighting}[]
\CommentTok{\# src/tools.py}
\ImportTok{from}\NormalTok{ datetime }\ImportTok{import}\NormalTok{ datetime}

\CommentTok{\# 获取当前日期和时间}
\KeywordTok{def}\NormalTok{ get\_current\_datetime() }\OperatorTok{{-}\textgreater{}} \BuiltInTok{str}\NormalTok{:}
    \CommentTok{"""}
\CommentTok{    获取当前日期和时间。}
\CommentTok{    :return: 当前日期和时间的字符串表示。}
\CommentTok{    """}
\NormalTok{    current\_datetime }\OperatorTok{=}\NormalTok{ datetime.now()}
\NormalTok{    formatted\_datetime }\OperatorTok{=}\NormalTok{ current\_datetime.strftime(}\StringTok{"\%Y{-}\%m{-}}\SpecialCharTok{\%d}\StringTok{ \%H:\%M:\%S"}\NormalTok{)}
    \ControlFlowTok{return}\NormalTok{ formatted\_datetime}

\KeywordTok{def}\NormalTok{ count\_letter\_in\_string(a: }\BuiltInTok{str}\NormalTok{, b: }\BuiltInTok{str}\NormalTok{):}
    \CommentTok{"""}
\CommentTok{    统计字符串中某个字母的出现次数。}
\CommentTok{    :param a: 要搜索的字符串。}
\CommentTok{    :param b: 要统计的字母。}
\CommentTok{    :return: 字母在字符串中出现的次数。}
\CommentTok{    """}
    \ControlFlowTok{return} \BuiltInTok{str}\NormalTok{(a.count(b))}

\KeywordTok{def}\NormalTok{ search\_wikipedia(query: }\BuiltInTok{str}\NormalTok{) }\OperatorTok{{-}\textgreater{}} \BuiltInTok{str}\NormalTok{:}
    \CommentTok{"""}
\CommentTok{    在维基百科中搜索指定查询的前三个页面摘要。}
\CommentTok{    :param query: 要搜索的查询字符串。}
\CommentTok{    :return: 包含前三个页面摘要的字符串。}
\CommentTok{    """}
\NormalTok{    page\_titles }\OperatorTok{=}\NormalTok{ wikipedia.search(query)}
\NormalTok{    summaries }\OperatorTok{=}\NormalTok{ []}
    \ControlFlowTok{for}\NormalTok{ page\_title }\KeywordTok{in}\NormalTok{ page\_titles[: }\DecValTok{3}\NormalTok{]:  }\CommentTok{\# 取前三个页面标题}
        \ControlFlowTok{try}\NormalTok{:}
            \CommentTok{\# 使用 wikipedia 模块的 page 函数,获取指定标题的维基百科页面对象。}
\NormalTok{            wiki\_page }\OperatorTok{=}\NormalTok{ wikipedia.page(title}\OperatorTok{=}\NormalTok{page\_title, auto\_suggest}\OperatorTok{=}\VariableTok{False}\NormalTok{)}
            \CommentTok{\# 获取页面摘要}
\NormalTok{            summaries.append(}\SpecialStringTok{f"页面: }\SpecialCharTok{\{}\NormalTok{page\_title}\SpecialCharTok{\}}\CharTok{\textbackslash{}n}\SpecialStringTok{摘要: }\SpecialCharTok{\{}\NormalTok{wiki\_page}\SpecialCharTok{.}\NormalTok{summary}\SpecialCharTok{\}}\SpecialStringTok{"}\NormalTok{)}
        \ControlFlowTok{except}\NormalTok{ (}
\NormalTok{                wikipedia.exceptions.PageError,}
\NormalTok{                wikipedia.exceptions.DisambiguationError,}
\NormalTok{        ):}
            \ControlFlowTok{pass}
    \ControlFlowTok{if} \KeywordTok{not}\NormalTok{ summaries:}
        \ControlFlowTok{return} \StringTok{"维基百科没有搜索到合适的结果"}
    \ControlFlowTok{return} \StringTok{"}\CharTok{\textbackslash{}n\textbackslash{}n}\StringTok{"}\NormalTok{.join(summaries)}
\CommentTok{\# ... (可能还有其他工具函数)}
\end{Highlighting}
\end{Shaded}

为了让 OpenAI API 理解这些工具,我们需要将它们转换成特定的 JSON Schema
格式。这可以通过 \texttt{src/utils.py} 中的 \texttt{function\_to\_json}
辅助函数完成。

\begin{Shaded}
\begin{Highlighting}[]
\CommentTok{\# src/utils.py (部分)}
\ImportTok{import}\NormalTok{ inspect}

\KeywordTok{def}\NormalTok{ function\_to\_json(func) }\OperatorTok{{-}\textgreater{}} \BuiltInTok{dict}\NormalTok{:}
    \CommentTok{\# ... (函数实现细节)}
    \CommentTok{\# 返回符合 OpenAI tool schema 的字典}
    \ControlFlowTok{return}\NormalTok{ \{}
        \StringTok{"type"}\NormalTok{: }\StringTok{"function"}\NormalTok{,}
        \StringTok{"function"}\NormalTok{: \{}
            \StringTok{"name"}\NormalTok{: func.}\VariableTok{\_\_name\_\_}\NormalTok{,}
            \StringTok{"description"}\NormalTok{: inspect.getdoc(func),}
            \StringTok{"parameters"}\NormalTok{: \{}
                \StringTok{"type"}\NormalTok{: }\StringTok{"object"}\NormalTok{,}
                \StringTok{"properties"}\NormalTok{: parameters,}
                \StringTok{"required"}\NormalTok{: required,}
\NormalTok{            \},}
\NormalTok{        \},}
\NormalTok{    \}}
\end{Highlighting}
\end{Shaded}

\paragraph{Step 3: 构造 Agent
类}\label{step-3-ux6784ux9020-agent-ux7c7b}

我们在 \texttt{src/core.py} 文件中定义 \texttt{Agent}
类。这个类负责管理对话历史、调用 OpenAI
API、处理工具调用请求以及执行工具函数。

\begin{Shaded}
\begin{Highlighting}[]
\CommentTok{\# src/core.py (部分)}
\ImportTok{from}\NormalTok{ openai }\ImportTok{import}\NormalTok{ OpenAI}
\ImportTok{import}\NormalTok{ json}
\ImportTok{from}\NormalTok{ typing }\ImportTok{import}\NormalTok{ List, Dict, Any}
\ImportTok{from}\NormalTok{ utils }\ImportTok{import}\NormalTok{ function\_to\_json}
\CommentTok{\# 导入定义好的工具函数}
\ImportTok{from}\NormalTok{ tools }\ImportTok{import}\NormalTok{ get\_current\_datetime, add, compare, count\_letter\_in\_string}

\NormalTok{SYSTEM\_PROMPT }\OperatorTok{=} \StringTok{"""}
\StringTok{你是一个叫不要葱姜蒜的人工智能助手。你的输出应该与用户的语言保持一致。}
\StringTok{当用户的问题需要调用工具时,你可以从提供的工具列表中调用适当的工具函数。}
\StringTok{"""}

\KeywordTok{class}\NormalTok{ Agent:}
    \KeywordTok{def} \FunctionTok{\_\_init\_\_}\NormalTok{(}\VariableTok{self}\NormalTok{, client: OpenAI, model: }\BuiltInTok{str} \OperatorTok{=} \StringTok{"Qwen/Qwen2.5{-}32B{-}Instruct"}\NormalTok{, tools: List}\OperatorTok{=}\NormalTok{[], verbose : }\BuiltInTok{bool} \OperatorTok{=} \VariableTok{True}\NormalTok{):}
        \VariableTok{self}\NormalTok{.client }\OperatorTok{=}\NormalTok{ client}
        \VariableTok{self}\NormalTok{.tools }\OperatorTok{=}\NormalTok{ tools}
        \VariableTok{self}\NormalTok{.model }\OperatorTok{=}\NormalTok{ model}
        \VariableTok{self}\NormalTok{.messages }\OperatorTok{=}\NormalTok{ [}
\NormalTok{            \{}\StringTok{"role"}\NormalTok{: }\StringTok{"system"}\NormalTok{, }\StringTok{"content"}\NormalTok{: SYSREM\_PROMPT\},}
\NormalTok{        ]}
        \VariableTok{self}\NormalTok{.verbose }\OperatorTok{=}\NormalTok{ verbose}

    \KeywordTok{def}\NormalTok{ get\_tool\_schema(}\VariableTok{self}\NormalTok{) }\OperatorTok{{-}\textgreater{}}\NormalTok{ List[Dict[}\BuiltInTok{str}\NormalTok{, Any]]:}
        \CommentTok{\# 获取所有工具的 JSON 模式}
        \ControlFlowTok{return}\NormalTok{ [function\_to\_json(tool) }\ControlFlowTok{for}\NormalTok{ tool }\KeywordTok{in} \VariableTok{self}\NormalTok{.tools]}

    \KeywordTok{def}\NormalTok{ handle\_tool\_call(}\VariableTok{self}\NormalTok{, tool\_call):}
        \CommentTok{\# 处理工具调用}
\NormalTok{        function\_name }\OperatorTok{=}\NormalTok{ tool\_call.function.name}
\NormalTok{        function\_args }\OperatorTok{=}\NormalTok{ tool\_call.function.arguments}
\NormalTok{        function\_id }\OperatorTok{=}\NormalTok{ tool\_call.}\BuiltInTok{id}

\NormalTok{        function\_call\_content }\OperatorTok{=} \BuiltInTok{eval}\NormalTok{(}\SpecialStringTok{f"}\SpecialCharTok{\{}\NormalTok{function\_name}\SpecialCharTok{\}}\SpecialStringTok{(**}\SpecialCharTok{\{}\NormalTok{function\_args}\SpecialCharTok{\}}\SpecialStringTok{)"}\NormalTok{)}

        \ControlFlowTok{return}\NormalTok{ \{}
            \StringTok{"role"}\NormalTok{: }\StringTok{"tool"}\NormalTok{,}
            \StringTok{"content"}\NormalTok{: function\_call\_content,}
            \StringTok{"tool\_call\_id"}\NormalTok{: function\_id,}
\NormalTok{        \}}

    \KeywordTok{def}\NormalTok{ get\_completion(}\VariableTok{self}\NormalTok{, prompt) }\OperatorTok{{-}\textgreater{}} \BuiltInTok{str}\NormalTok{:}

        \VariableTok{self}\NormalTok{.messages.append(\{}\StringTok{"role"}\NormalTok{: }\StringTok{"user"}\NormalTok{, }\StringTok{"content"}\NormalTok{: prompt\})}

        \CommentTok{\# 获取模型的完成响应}
\NormalTok{        response }\OperatorTok{=} \VariableTok{self}\NormalTok{.client.chat.completions.create(}
\NormalTok{            model}\OperatorTok{=}\VariableTok{self}\NormalTok{.model,}
\NormalTok{            messages}\OperatorTok{=}\VariableTok{self}\NormalTok{.messages,}
\NormalTok{            tools}\OperatorTok{=}\VariableTok{self}\NormalTok{.get\_tool\_schema(),}
\NormalTok{            stream}\OperatorTok{=}\VariableTok{False}\NormalTok{,}
\NormalTok{        )}
        
        \CommentTok{\# 检查模型是否调用了工具        }
        \ControlFlowTok{if}\NormalTok{ response.choices[}\DecValTok{0}\NormalTok{].message.tool\_calls:}
            \VariableTok{self}\NormalTok{.messages.append(\{}\StringTok{"role"}\NormalTok{: }\StringTok{"assistant"}\NormalTok{, }\StringTok{"content"}\NormalTok{: response.choices[}\DecValTok{0}\NormalTok{].message.content\})}
            \CommentTok{\# 处理工具调用}
\NormalTok{            tool\_list }\OperatorTok{=}\NormalTok{ []}
            \ControlFlowTok{for}\NormalTok{ tool\_call }\KeywordTok{in}\NormalTok{ response.choices[}\DecValTok{0}\NormalTok{].message.tool\_calls:}
                \CommentTok{\# 处理工具调用并将结果添加到消息列表中}
                \VariableTok{self}\NormalTok{.messages.append(}\VariableTok{self}\NormalTok{.handle\_tool\_call(tool\_call))}
\NormalTok{                tool\_list.append([tool\_call.function.name, tool\_call.function.arguments])}
            \ControlFlowTok{if} \VariableTok{self}\NormalTok{.verbose:}
                \BuiltInTok{print}\NormalTok{(}\StringTok{"调用工具:"}\NormalTok{, response.choices[}\DecValTok{0}\NormalTok{].message.content, tool\_list)}
            \CommentTok{\# 再次获取模型的完成响应,这次包含工具调用的结果}
\NormalTok{            response }\OperatorTok{=} \VariableTok{self}\NormalTok{.client.chat.completions.create(}
\NormalTok{                model}\OperatorTok{=}\VariableTok{self}\NormalTok{.model,}
\NormalTok{                messages}\OperatorTok{=}\VariableTok{self}\NormalTok{.messages,}
\NormalTok{                tools}\OperatorTok{=}\VariableTok{self}\NormalTok{.get\_tool\_schema(),}
\NormalTok{                stream}\OperatorTok{=}\VariableTok{False}\NormalTok{,}
\NormalTok{            )}

        \CommentTok{\# 将模型的完成响应添加到消息列表中}
        \VariableTok{self}\NormalTok{.messages.append(\{}\StringTok{"role"}\NormalTok{: }\StringTok{"assistant"}\NormalTok{, }\StringTok{"content"}\NormalTok{: response.choices[}\DecValTok{0}\NormalTok{].message.content\})}
        \ControlFlowTok{return}\NormalTok{ response.choices[}\DecValTok{0}\NormalTok{].message.content}
\end{Highlighting}
\end{Shaded}

Agent 的工作流程如下:

\begin{enumerate}
\def\labelenumi{\arabic{enumi}.}
\tightlist
\item
  接收用户输入。
\item
  调用大模型(如 Qwen),并告知其可用的工具及其 Schema。
\item
  如果模型决定调用工具,Agent 会解析请求,执行相应的 Python 函数。
\item
  Agent 将工具的执行结果返回给模型。
\item
  模型根据工具结果生成最终回复。
\item
  Agent 将最终回复返回给用户。
\end{enumerate}

如图7.9所示,Agent 调用工具流程:

\begin{figure}[htbp]\centering
\includegraphics[width=0.8\textwidth]{https://raw.githubusercontent.com/datawhalechina/happy-llm/main/docs/images/7-images/7-3-Tiny_Agent.jpg}
\caption{图7.9 Agent 工作流程}
\end{figure}

\paragraph{Step 4: 运行 Agent}\label{step-4-ux8fd0ux884c-agent}

现在我们可以实例化并运行 Agent。在 \texttt{demo.py} 的
\texttt{if\ \_\_name\_\_\ ==\ "\_\_main\_\_":}
部分提供了一个简单的命令行交互示例。

\begin{Shaded}
\begin{Highlighting}[]
\CommentTok{\# demo.py (部分)}
\ControlFlowTok{if} \VariableTok{\_\_name\_\_} \OperatorTok{==} \StringTok{"\_\_main\_\_"}\NormalTok{:}
\NormalTok{    client }\OperatorTok{=}\NormalTok{ OpenAI(}
\NormalTok{        api\_key}\OperatorTok{=}\StringTok{"YOUR\_API\_KEY"}\NormalTok{, }\CommentTok{\# 替换为你的 API Key}
\NormalTok{        base\_url}\OperatorTok{=}\StringTok{"https://api.siliconflow.cn/v1"}\NormalTok{,}
\NormalTok{    )}

    \CommentTok{\# 创建 Agent 实例,传入 client、模型名称和工具函数列表}
\NormalTok{    agent }\OperatorTok{=}\NormalTok{ Agent(}
\NormalTok{        client}\OperatorTok{=}\NormalTok{client,}
\NormalTok{        model}\OperatorTok{=}\StringTok{"Qwen/Qwen2.5{-}32B{-}Instruct"}\NormalTok{,}
\NormalTok{        tools}\OperatorTok{=}\NormalTok{[get\_current\_datetime, add, compare, count\_letter\_in\_string],}
\NormalTok{        verbose}\OperatorTok{=}\VariableTok{True} \CommentTok{\# 设置为 True 可以看到工具调用信息}
\NormalTok{    )}

    \CommentTok{\# 开始交互式对话循环}
    \ControlFlowTok{while} \VariableTok{True}\NormalTok{:}
        \CommentTok{\# 使用彩色输出区分用户输入和AI回答}
\NormalTok{        prompt }\OperatorTok{=} \BuiltInTok{input}\NormalTok{(}\StringTok{"}\CharTok{\textbackslash{}033}\StringTok{[94mUser: }\CharTok{\textbackslash{}033}\StringTok{[0m"}\NormalTok{)  }\CommentTok{\# 蓝色显示用户输入提示}
        \ControlFlowTok{if}\NormalTok{ prompt.lower() }\OperatorTok{==} \StringTok{"exit"}\NormalTok{:}
            \ControlFlowTok{break}
\NormalTok{        response }\OperatorTok{=}\NormalTok{ agent.get\_completion(prompt)}
        \BuiltInTok{print}\NormalTok{(}\StringTok{"}\CharTok{\textbackslash{}033}\StringTok{[92mAssistant: }\CharTok{\textbackslash{}033}\StringTok{[0m"}\NormalTok{, response)  }\CommentTok{\# 绿色显示AI助手回答}
\end{Highlighting}
\end{Shaded}

\textbf{示例交互:}

\begin{Shaded}
\begin{Highlighting}[]
\ExtensionTok{User:}\NormalTok{ 你好}
\ExtensionTok{Assistant:}\NormalTok{  你好!有什么可以帮助你的吗?}
\ExtensionTok{User:}\NormalTok{ 9.12和9 .2哪个更大?}
\ExtensionTok{调用工具:} \PreprocessorTok{[}\StringTok{\textquotesingle{}compare\textquotesingle{}}\PreprocessorTok{]}
\ExtensionTok{Assistant:}\NormalTok{  9.2 比 9.12 更大。}
\ExtensionTok{User:}\NormalTok{ 为什么?}
\ExtensionTok{Assistant:}\NormalTok{  当我们比较9.12和9.2时,可以将它们看作是9.12和9.20。由于9.20在小数点后第二位是0,而9.12在小数点后第二位是2,所以在小数点后第一位相等的情况下,9.20(即9.2)大于9.12。因此,9.2 比 9.12 更大。}
\ExtensionTok{User:}\NormalTok{ strawberry中有几个r?}
\ExtensionTok{调用工具:} \PreprocessorTok{[}\StringTok{\textquotesingle{}count\_letter\_in\_string\textquotesingle{}}\PreprocessorTok{]}
\ExtensionTok{Assistant:}\NormalTok{  单词 }\StringTok{"strawberry"}\NormalTok{ 中有3个字母 }\StringTok{\textquotesingle{}r\textquotesingle{}}\NormalTok{。}
\ExtensionTok{User:}\NormalTok{ 你确信嘛?}
\ExtensionTok{调用工具:} \PreprocessorTok{[}\StringTok{\textquotesingle{}count\_letter\_in\_string\textquotesingle{}}\PreprocessorTok{]}
\ExtensionTok{Assistant:}\NormalTok{  是的,我确定。单词 }\StringTok{"strawberry"}\NormalTok{ 中确实有3个字母 }\StringTok{\textquotesingle{}r\textquotesingle{}}\NormalTok{。让我们再次确认一下,}\StringTok{"strawberry"}\NormalTok{ 中的 }\StringTok{\textquotesingle{}r\textquotesingle{}}\NormalTok{ 确实出现了3次。}
\ExtensionTok{User:}\NormalTok{ 好的 你很薄,现在几点 了?}
\ExtensionTok{调用工具:} \PreprocessorTok{[}\StringTok{\textquotesingle{}get\_current\_datetime\textquotesingle{}}\PreprocessorTok{]}
\ExtensionTok{Assistant:}\NormalTok{  当前的时间是2025年4月26日17:01:33。不过,我注意到您提到“你很薄”,这似乎是一个打字错误,如果您有任何其他问题或者需要进一步的帮助,请告诉我!}
\ExtensionTok{User:}\NormalTok{ exit}
\end{Highlighting}
\end{Shaded}

另外,我们也准备了一份可以展示的 Streamlit 应用,可以运行在本地,展示
Agent 的功能。\texttt{streamlit\ run\ web\_demo.py} 来运行,以下为 Agent
运行效果。

\begin{figure}[htbp]\centering
\includegraphics[width=0.8\textwidth]{https://raw.githubusercontent.com/datawhalechina/happy-llm/main/docs/images/7-images/7-3-streamlit-demo.png}
\caption{图 7.10 Streamlit Demo}
\end{figure}

\textbf{参考文献}

{[}1{]} Hugging Face. (2023). \emph{Open LLM Leaderboard:
开源大语言模型基准测试平台}.
https://huggingface.co/spaces/open-llm-leaderboard/open\_llm\_leaderboard

{[}2{]} awacke1. (2023). \emph{LMSYS Chatbot Arena Leaderboard:
大型语言模型竞技场评估平台}.
https://huggingface.co/spaces/awacke1/lmsys-chatbot-arena-leaderboard

{[}3{]} OpenCompass 团队. (2023). \emph{OpenCompass:
大模型统一评测平台}. https://rank.opencompass.org.cn/home

{[}4{]} OpenCompass 金融榜团队. (2024). \emph{CFBENCHMARK:
金融领域大模型评测榜单}.
https://specialist.opencompass.org.cn/CFBenchmark

{[}5{]} OpenCompass 安全榜团队. (2024). \emph{Flames:
大模型安全评测榜单}. https://flames.opencompass.org.cn/leaderboard

{[}6{]} OpenCompass 通识榜团队. (2024). \emph{BotChat:
大模型通用对话能力评测}. https://botchat.opencompass.org.cn/

{[}7{]} OpenCompass 法律榜团队. (2024). \emph{LawBench:
法律领域大模型评测}. https://lawbench.opencompass.org.cn/leaderboard

{[}8{]} OpenCompass 医疗榜团队. (2024). \emph{MedBench:
医疗领域大模型评测}. https://medbench.opencompass.org.cn/leaderboard

{[}9{]} Zhi Jing, Yongye Su, and Yikun Han. (2024). \emph{When Large
Language Models Meet Vector Databases: A Survey.} arXiv preprint
arXiv:2402.01763.

{[}10{]} Yunfan Gao, Yun Xiong, Xinyu Gao, Kangxiang Jia, Jinliu Pan,
Yuxi Bi, Yi Dai, Jiawei Sun, Meng Wang, and Haofen Wang. (2024).
\emph{Retrieval-Augmented Generation for Large Language Models: A
Survey.} arXiv preprint arXiv:2312.10997.

{[}11{]} Zhiruo Wang, Jun Araki, Zhengbao Jiang, Md Rizwan Parvez, 和
Graham Neubig. (2023). \emph{Learning to Filter Context for
Retrieval-Augmented Generation.} arXiv preprint arXiv:2311.08377.

{[}12{]} Ori Ram, Yoav Levine, Itay Dalmedigos, Dor Muhlgay, Amnon
Shashua, Kevin Leyton-Brown 和 Yoav Shoham. (2023). \emph{In-Context
Retrieval-Augmented Language Models.} arXiv preprint arXiv:2302.00083.

\end{document}


% 额外章节(如果存在)
% \IfFileExists{Extra-Chapter/Readme.tex}{
%     \chapter{额外章节}
%     % Options for packages loaded elsewhere
\PassOptionsToPackage{unicode}{hyperref}
\PassOptionsToPackage{hyphens}{url}
\documentclass[
]{article}
\usepackage{xcolor}
\usepackage{amsmath,amssymb}
\setcounter{secnumdepth}{5}
\usepackage{iftex}
\ifPDFTeX
  \usepackage[T1]{fontenc}
  \usepackage[utf8]{inputenc}
  \usepackage{textcomp} % provide euro and other symbols
\else % if luatex or xetex
  \usepackage{unicode-math} % this also loads fontspec
  \defaultfontfeatures{Scale=MatchLowercase}
  \defaultfontfeatures[\rmfamily]{Ligatures=TeX,Scale=1}
\fi
\usepackage{lmodern}
\ifPDFTeX\else
  % xetex/luatex font selection
\fi
% Use upquote if available, for straight quotes in verbatim environments
\IfFileExists{upquote.sty}{\usepackage{upquote}}{}
\IfFileExists{microtype.sty}{% use microtype if available
  \usepackage[]{microtype}
  \UseMicrotypeSet[protrusion]{basicmath} % disable protrusion for tt fonts
}{}
\makeatletter
\@ifundefined{KOMAClassName}{% if non-KOMA class
  \IfFileExists{parskip.sty}{%
    \usepackage{parskip}
  }{% else
    \setlength{\parindent}{0pt}
    \setlength{\parskip}{6pt plus 2pt minus 1pt}}
}{% if KOMA class
  \KOMAoptions{parskip=half}}
\makeatother
\setlength{\emergencystretch}{3em} % prevent overfull lines
\providecommand{\tightlist}{%
  \setlength{\itemsep}{0pt}\setlength{\parskip}{0pt}}
\usepackage{bookmark}
\IfFileExists{xurl.sty}{\usepackage{xurl}}{} % add URL line breaks if available
\urlstyle{same}
\hypersetup{
  hidelinks,
  pdfcreator={LaTeX via pandoc}}

\author{}
\date{}

\begin{document}

{
\setcounter{tocdepth}{3}
\tableofcontents
}
🚀 Happy-LLM 扩展内容

社区驱动的大语言模型学习资源

\begin{center}\rule{0.5\linewidth}{0.5pt}\end{center}

\subsection{📖 为什么会有 Extra
Chapter?}\label{ux4e3aux4ec0ux4e48ux4f1aux6709-extra-chapter}

  在 Happy-LLM
主教程的基础上,我们发现社区中有许多优秀的学习者和实践者,他们在学习和使用大语言模型的过程中积累了宝贵的经验、独到的见解和实用的技巧。这些内容虽然不属于主教程的核心知识体系,但对于深入理解和应用大语言模型具有重要价值。

\textbf{Extra Chapter 的设立目的:}

\begin{itemize}
\tightlist
\item
  🌟 \textbf{汇聚智慧}:收集社区成员的优秀学习笔记、实践经验和技术博客
\item
  🔄 \textbf{持续更新}:保持内容的时效性,跟上大语言模型领域的快速发展
\item
  🤝 \textbf{促进交流}:为社区成员提供分享和交流的平台
\item
  📚 \textbf{补充完善}:对主教程内容进行有益的补充和扩展
\item
  💡 \textbf{启发思考}:通过不同视角和实践案例,启发读者的深度思考
\end{itemize}

\textbf{Extra Chapter 包含的内容类型:}

\begin{itemize}
\tightlist
\item
  📝 \textbf{学习笔记}:深度学习心得和知识总结
\item
  🛠️ \textbf{实践案例}:真实项目中的应用经验
\item
  🔬 \textbf{技术探索}:前沿技术的研究和实验
\item
  💭 \textbf{思考感悟}:对大语言模型发展的独特见解
\item
  🎯 \textbf{专题研究}:特定领域或问题的深入分析
\end{itemize}

\begin{center}\rule{0.5\linewidth}{0.5pt}\end{center}

\subsection{📋 PR 贡献规范}\label{pr-ux8d21ux732eux89c4ux8303}

  我们热烈欢迎社区成员为 Extra Chapter
贡献优质内容!为了保证内容质量和项目的整体性,请遵循以下规范:

\subsubsection{🗂️
目录结构规范}\label{ux76eeux5f55ux7ed3ux6784ux89c4ux8303}

每个贡献的内容应按照以下目录结构组织:

\begin{verbatim}
Extra-Chapter/
├── your-topic-name/                    # 你的主题文件夹
│   ├── readme.md                       # 主要内容文件(必需)
│   ├── images/                         # 图片资源文件夹(可选)
│   │   ├── figure1.png
│   │   └── figure2.jpg
│   ├── code/                           # 代码文件夹(可选)
│   │   ├── example.py
│   │   └── requirements.txt
│   ├── data/                           # 数据文件夹(可选)
│   │   └── sample_data.json
│   └── notebook.ipynb                  # Jupyter Notebook(如涉及代码必选)
└── Readme.md                           # 本文件
\end{verbatim}

\subsubsection{📝
文件命名规范}\label{ux6587ux4ef6ux547dux540dux89c4ux8303}

\begin{enumerate}
\def\labelenumi{\arabic{enumi}.}
\tightlist
\item
  \textbf{主题文件夹命名}:

  \begin{itemize}
  \tightlist
  \item
    使用小写字母和连字符
  \item
    名称要简洁明了,能够概括主题内容
  \item
    例如:\texttt{why-fine-tune-small-large-language-models}、\texttt{rag-optimization-techniques}
  \end{itemize}
\item
  \textbf{主要内容文件}:

  \begin{itemize}
  \tightlist
  \item
    必须命名为 \texttt{readme.md}
  \item
    使用 Markdown 格式编写
  \end{itemize}
\item
  \textbf{图片文件}:

  \begin{itemize}
  \tightlist
  \item
    统一放在 \texttt{images/} 文件夹下
  \item
    使用描述性的文件名
  \item
    支持格式:\texttt{.png}、\texttt{.jpg}、\texttt{.jpeg}、\texttt{.gif}、\texttt{.svg}
  \end{itemize}
\item
  \textbf{代码文件}:

  \begin{itemize}
  \tightlist
  \item
    如涉及代码,请尽量提供可直接运行的 Jupyter Notebook 文件
  \item
    统一放在 \texttt{code/} 文件夹下
  \item
    使用标准的文件扩展名
  \item
    如有依赖,请提供 \texttt{requirements.txt}
  \item
    如有 Jupyter Notebook 文件,请放在主文件夹下
  \end{itemize}
\end{enumerate}

\subsubsection{✍️
内容质量要求}\label{ux5185ux5bb9ux8d28ux91cfux8981ux6c42}

\begin{enumerate}
\def\labelenumi{\arabic{enumi}.}
\tightlist
\item
  \textbf{原创性}:

  \begin{itemize}
  \tightlist
  \item
    内容必须是原创或经过授权的
  \item
    如引用他人内容,请注明出处
  \end{itemize}
\item
  \textbf{技术准确性}:

  \begin{itemize}
  \tightlist
  \item
    确保技术内容的准确性
  \item
    代码示例应能正常运行
  \item
    提供必要的环境说明
  \end{itemize}
\item
  \textbf{结构清晰}:

  \begin{itemize}
  \tightlist
  \item
    使用清晰的标题层次
  \item
    合理使用列表、表格等格式
  \item
    重要内容使用适当的强调
  \end{itemize}
\item
  \textbf{语言规范}:

  \begin{itemize}
  \tightlist
  \item
    使用规范的中文表达
  \item
    技术术语使用准确
  \item
    避免错别字和语法错误
  \end{itemize}
\end{enumerate}

\subsubsection{PR commit messgae
内容}\label{pr-commit-messgae-ux5185ux5bb9}

请在 PR commit message 中 包含以下内容:

\begin{itemize}
\tightlist
\item
  新增的主题文件夹名称
\item
  贡献内容的概述
\item
  贡献内容的详细描述
\item
  你的 Github 个人主页链接,及你的个人介绍
\item
  个人 title 及工作经历 or 学校 or 研究方向
\end{itemize}

如以下所示:

\begin{verbatim}
Extra Chapter: 你的主题名称

详细描述你的贡献内容,包括新增的主题文件夹、文件内容和目录结构。

- 新增的主题文件夹名称:your-topic-name
- 贡献内容的概述:详细介绍你的贡献内容
- 贡献内容的详细描述:详细描述你的贡献内容,包括新增的主题文件夹、文件内容和目录结构。
- 你的 Github 个人主页链接及个人介绍:[你的个人主页链接](https://example.com),介绍你的研究方向、技术专长等。
- 个人 title 及工作经历 or 学校 or 研究方向:内容贡献者-xxxx学校,研究方向为自然语言处理。
\end{verbatim}

\end{document}

\end{document}
