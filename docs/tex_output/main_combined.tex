\documentclass[12pt,a4paper]{book}
\usepackage[UTF8]{ctex}
\usepackage{geometry}
\usepackage{graphicx}
\usepackage{hyperref}
\usepackage{bookmark}
\usepackage{amsmath,amssymb}
\usepackage{listings}
\usepackage{xcolor}
\usepackage{fancyhdr}
\usepackage{titlesec}
\usepackage{tocloft}

% 页面设置
\geometry{left=2.5cm,right=2.5cm,top=2.5cm,bottom=2.5cm}

% 代码块设置
\lstset{
    basicstyle=\ttfamily\small,
    breaklines=true,
    frame=single,
    numbers=left,
    numberstyle=\tiny,
    keywordstyle=\color{blue},
    commentstyle=\color{green!60!black},
    stringstyle=\color{red},
    backgroundcolor=\color{gray!10},
    showspaces=false,
    showstringspaces=false,
    showtabs=false,
    tabsize=2
}

% 超链接设置
\hypersetup{
    colorlinks=true,
    linkcolor=blue,
    filecolor=magenta,
    urlcolor=cyan,
    citecolor=green,
    bookmarks=true,
    bookmarksopen=true,
    pdfstartview=FitH
}

% 页眉页脚设置
\pagestyle{fancy}
\fancyhf{}
\fancyhead[LE,RO]{\thepage}
\fancyhead[RE]{\leftmark}
\fancyhead[LO]{\rightmark}
\renewcommand{\headrulewidth}{0.4pt}

% 章节标题格式
\titleformat{\chapter}{\Large\bfseries}{\thechapter}{1em}{}
\titleformat{\section}{\large\bfseries}{\thesection}{1em}{}
\titleformat{\subsection}{\normalsize\bfseries}{\thesubsection}{1em}{}

% 目录格式
\renewcommand{\cftchapfont}{\bfseries}
\renewcommand{\cftsecfont}{\normalsize}

\begin{document}

% 标题页
\begin{titlepage}
    \centering
    \vspace*{2cm}
    
    {\Huge\bfseries Happy-LLM 大语言模型教程}\\[2cm]
    
    {\Large 从零开始的大语言模型原理与实践教程}\\[1cm]
    
    {\large Datawhale 开源学习社区}\\[2cm]
    
    \vfill
    
    {\large \today}
\end{titlepage}

% 版权页
\newpage
\thispagestyle{empty}
\vspace*{2cm}
\begin{center}
    \textbf{版权声明}
    
    \vspace{1cm}
    
    本书由 Datawhale 开源学习社区编写,采用开源协议发布。
    
    欢迎读者在遵守开源协议的前提下自由使用、修改和分发本书内容。
    
    \vspace{1cm}
    
    GitHub: https://github.com/datawhalechina/happy-llm
\end{center}

% 目录
\tableofcontents
\newpage

{
\setcounter{tocdepth}{3}
\tableofcontents
}
Happy-LLM

📚 从零开始的大语言模型原理与实践教程

深入理解 LLM 核心原理,动手实现你的第一个大模型

\begin{center}\rule{0.5\linewidth}{0.5pt}\end{center}

\subsection{🎯 项目介绍}\label{ux9879ux76eeux4ecbux7ecd}

\begin{quote}
  \emph{很多小伙伴在看完 Datawhale开源项目:
\href{https://github.com/datawhalechina/self-llm}{self-llm
开源大模型食用指南}
后,感觉意犹未尽,想要深入了解大语言模型的原理和训练过程。于是我们(Datawhale)决定推出《Happy-LLM》项目,旨在帮助大家深入理解大语言模型的原理和训练过程。}
\end{quote}

  本项目是一个\textbf{系统性的 LLM 学习教程},将从 NLP
的基本研究方法出发,根据 LLM 的思路及原理逐层深入,依次为读者剖析 LLM
的架构基础和训练过程。同时,我们会结合目前 LLM
领域最主流的代码框架,演练如何亲手搭建、训练一个
LLM,期以实现授之以鱼,更授之以渔。希望大家能从这本书开始走入 LLM
的浩瀚世界,探索 LLM 的无尽可能。

\subsubsection{✨
你将收获什么?}\label{ux4f60ux5c06ux6536ux83b7ux4ec0ux4e48}

\begin{itemize}
\tightlist
\item
  📚 \textbf{Datawhale 开源免费} 完全免费的学习本项目所有内容
\item
  🔍 \textbf{深入理解} Transformer 架构和注意力机制
\item
  📚 \textbf{掌握} 预训练语言模型的基本原理
\item
  🧠 \textbf{了解} 现有大模型的基本结构
\item
  🏗️ \textbf{动手实现} 一个完整的 LLaMA2 模型
\item
  ⚙️ \textbf{掌握训练} 从预训练到微调的全流程
\item
  🚀 \textbf{实战应用} RAG、Agent 等前沿技术
\end{itemize}

\subsection{📖 内容导航}\label{ux5185ux5bb9ux5bfcux822a}

\begin{longtable}[]{@{}
  >{\raggedright\arraybackslash}p{(\linewidth - 4\tabcolsep) * \real{0.3333}}
  >{\raggedright\arraybackslash}p{(\linewidth - 4\tabcolsep) * \real{0.3333}}
  >{\raggedright\arraybackslash}p{(\linewidth - 4\tabcolsep) * \real{0.3333}}@{}}
\toprule\noalign{}
\begin{minipage}[b]{\linewidth}\raggedright
章节
\end{minipage} & \begin{minipage}[b]{\linewidth}\raggedright
关键内容
\end{minipage} & \begin{minipage}[b]{\linewidth}\raggedright
状态
\end{minipage} \\
\midrule\noalign{}
\endhead
\bottomrule\noalign{}
\endlastfoot
\href{./前言.md}{前言} & 本项目的缘起、背景及读者建议 & ✅ \\
\href{./chapter1/第一章\%20NLP基础概念.md}{第一章 NLP 基础概念} & 什么是
NLP、发展历程、任务分类、文本表示演进 & ✅ \\
\href{./chapter2/第二章\%20Transformer架构.md}{第二章 Transformer 架构}
& 注意力机制、Encoder-Decoder、手把手搭建 Transformer & ✅ \\
\href{./chapter3/第三章\%20预训练语言模型.md}{第三章 预训练语言模型} &
Encoder-only、Encoder-Decoder、Decoder-Only 模型对比 & ✅ \\
\href{./chapter4/第四章\%20大语言模型.md}{第四章 大语言模型} & LLM
定义、训练策略、涌现能力分析 & ✅ \\
\href{./chapter5/第五章\%20动手搭建大模型.md}{第五章 动手搭建大模型} &
实现 LLaMA2、训练 Tokenizer、预训练小型 LLM & ✅ \\
\href{./chapter6/第六章\%20大模型训练流程实践.md}{第六章 大模型训练实践}
& 预训练、有监督微调、LoRA/QLoRA 高效微调 & 🚧 \\
\href{./chapter7/第七章\%20大模型应用.md}{第七章 大模型应用} &
模型评测、RAG 检索增强、Agent 智能体 & ✅ \\
\end{longtable}

\subsubsection{模型下载}\label{ux6a21ux578bux4e0bux8f7d}

\begin{longtable}[]{@{}
  >{\raggedright\arraybackslash}p{(\linewidth - 2\tabcolsep) * \real{0.5000}}
  >{\raggedright\arraybackslash}p{(\linewidth - 2\tabcolsep) * \real{0.5000}}@{}}
\toprule\noalign{}
\begin{minipage}[b]{\linewidth}\raggedright
模型名称
\end{minipage} & \begin{minipage}[b]{\linewidth}\raggedright
下载地址
\end{minipage} \\
\midrule\noalign{}
\endhead
\bottomrule\noalign{}
\endlastfoot
Happy-LLM-Chapter5-Base-215M &
\href{https://www.modelscope.cn/models/kmno4zx/happy-llm-215M-base}{🤖
ModelScope} \\
Happy-LLM-Chapter5-SFT-215M &
\href{https://www.modelscope.cn/models/kmno4zx/happy-llm-215M-sft}{🤖
ModelScope} \\
\end{longtable}

\begin{quote}
\emph{ModelScope
创空间体验地址:\href{https://www.modelscope.cn/studios/kmno4zx/happy_llm_215M_sft}{🤖
创空间}}
\end{quote}

\subsubsection{PDF 版本下载}\label{pdf-ux7248ux672cux4e0bux8f7d}

  \textbf{\emph{本 Happy-LLM PDF
教程完全开源免费。为防止各类营销号加水印后贩卖给大模型初学者,我们特地在
PDF 文件中预先添加了不影响阅读的 Datawhale 开源标志水印,敬请谅解~}}

\begin{quote}
\emph{Happy-LLM PDF :
https://github.com/datawhalechina/happy-llm/releases/tag/PDF}\\
\emph{Happy-LLM PDF 国内下载地址 :
https://www.datawhale.cn/learn/summary/179}
\end{quote}

\subsection{💡 如何学习}\label{ux5982ux4f55ux5b66ux4e60}

  本项目适合大学生、研究人员、LLM
爱好者。在学习本项目之前,建议具备一定的编程经验,尤其是要对 Python
编程语言有一定的了解。最好具备深度学习的相关知识,并了解 NLP
领域的相关概念和术语,以便更轻松地学习本项目。

  本项目分为两部分------基础知识与实战应用。第1章~第4章是基础知识部分,从浅入深介绍
LLM 的基本原理。其中,第1章简单介绍 NLP 的基本任务和发展,为非 NLP
领域研究者提供参考;第2章介绍 LLM
的基本架构------Transformer,包括原理介绍及代码实现,作为 LLM
最重要的理论基础;第3章整体介绍经典的 PLM,包括
Encoder-Only、Encoder-Decoder 和 Decoder-Only
三种架构,也同时介绍了当前一些主流 LLM 的架构和思想;第4章则正式进入 LLM
部分,详细介绍 LLM
的特点、能力和整体训练过程。第5章~第7章是实战应用部分,将逐步带领大家深入
LLM 的底层细节。其中,第5章将带领大家者基于 PyTorch 层亲手搭建一个
LLM,并实现预训练、有监督微调的全流程;第6章将引入目前业界主流的 LLM
训练框架 Transformers,带领学习者基于该框架快速、高效地实现 LLM
训练过程;第7章则将介绍 基于 LLM 的各种应用,补全学习者对 LLM
体系的认知,包括 LLM 的评测、检索增强生产(Retrieval-Augmented
Generation,RAG)、智能体(Agent)的思想和简单实现。你可以根据个人兴趣和需求,选择性地阅读相关章节。

  在阅读本书的过程中,建议你将理论和实际相结合。LLM
是一个快速发展、注重实践的领域,我们建议你多投入实战,复现本书提供的各种代码,同时积极参加
LLM 相关的项目与比赛,真正投入到 LLM 开发的浪潮中。我们鼓励你关注
Datawhale 及其他 LLM 相关开源社区,当遇到问题时,你可以随时在本项目的
issue 区提问。

  最后,欢迎每一位读者在学习完本项目后加入到 LLM 开发者的行列。作为国内
AI 开源社区,我们希望充分聚集共创者,一起丰富这个开源 LLM
的世界,打造更多、更全面特色 LLM
的教程。星火点点,汇聚成海。我们希望成为 LLM
与普罗大众的阶梯,以自由、平等的开源精神,拥抱更恢弘而辽阔的 LLM 世界。

\subsection{🤝 如何贡献}\label{ux5982ux4f55ux8d21ux732e}

我们欢迎任何形式的贡献!

\begin{itemize}
\tightlist
\item
  🐛 \textbf{报告 Bug} - 发现问题请提交 Issue
\item
  💡 \textbf{功能建议} - 有好想法就告诉我们
\item
  📝 \textbf{内容完善} - 帮助改进教程内容
\item
  🔧 \textbf{代码优化} - 提交 Pull Request
\end{itemize}

\subsection{🙏 致谢}\label{ux81f4ux8c22}

\subsubsection{核心贡献者}\label{ux6838ux5fc3ux8d21ux732eux8005}

\begin{itemize}
\tightlist
\item
  \href{https://github.com/KMnO4-zx}{宋志学-项目负责人}
  (Datawhale成员-中国矿业大学(北京))
\item
  \href{https://github.com/logan-zou}{邹雨衡-项目负责人}
  (Datawhale成员-对外经济贸易大学)
\item
  \href{https://xinzhongzhu.github.io/}{朱信忠-指导专家}(Datawhale首席科学家-浙江师范大学杭州人工智能研究院教授)
\end{itemize}

\subsubsection{特别感谢}\label{ux7279ux522bux611fux8c22}

\begin{itemize}
\tightlist
\item
  感谢 \href{https://github.com/Sm1les}{@Sm1les} 对本项目的帮助与支持
\item
  感谢所有为本项目做出贡献的开发者们 ❤️
\end{itemize}

\subsection{Star History}\label{star-history}

⭐ 如果这个项目对你有帮助,请给我们一个 Star!

\subsection{关于 Datawhale}\label{ux5173ux4e8e-datawhale}

扫描二维码关注 Datawhale 公众号,获取更多优质开源内容

\begin{center}\rule{0.5\linewidth}{0.5pt}\end{center}

\subsection{📜 开源协议}\label{ux5f00ux6e90ux534fux8bae}

本作品采用\href{http://creativecommons.org/licenses/by-nc-sa/4.0/}{知识共享署名-非商业性使用-相同方式共享
4.0 国际许可协议}进行许可。



\end{document}
