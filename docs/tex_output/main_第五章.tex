\documentclass[12pt,a4paper]{book}
\usepackage[UTF8]{ctex}
\usepackage{geometry}
\usepackage{graphicx}
\usepackage{hyperref}
\usepackage{bookmark}
\usepackage{amsmath,amssymb}
\usepackage{listings}
\usepackage{xcolor}
\usepackage{fancyhdr}
\usepackage{titlesec}
\usepackage{tocloft}

% 页面设置
\geometry{left=2.5cm,right=2.5cm,top=2.5cm,bottom=2.5cm}

% 代码块设置
\lstset{
    basicstyle=\ttfamily\small,
    breaklines=true,
    frame=single,
    numbers=left,
    numberstyle=\tiny,
    keywordstyle=\color{blue},
    commentstyle=\color{green!60!black},
    stringstyle=\color{red},
    backgroundcolor=\color{gray!10},
    showspaces=false,
    showstringspaces=false,
    showtabs=false,
    tabsize=2
}

% 超链接设置
\hypersetup{
    colorlinks=true,
    linkcolor=blue,
    filecolor=magenta,
    urlcolor=cyan,
    citecolor=green,
    bookmarks=true,
    bookmarksopen=true,
    pdfstartview=FitH
}

% 页眉页脚设置
\pagestyle{fancy}
\fancyhf{}
\fancyhead[LE,RO]{\thepage}
\fancyhead[RE]{\leftmark}
\fancyhead[LO]{\rightmark}
\renewcommand{\headrulewidth}{0.4pt}

% 章节标题格式
\titleformat{\chapter}{\Large\bfseries}{\thechapter}{1em}{}
\titleformat{\section}{\large\bfseries}{\thesection}{1em}{}
\titleformat{\subsection}{\normalsize\bfseries}{\thesubsection}{1em}{}

% 目录格式
\renewcommand{\cftchapfont}{\bfseries}
\renewcommand{\cftsecfont}{\normalsize}

\begin{document}

% 标题页
\begin{titlepage}
    \centering
    \vspace*{2cm}
    
    {\Huge\bfseries 动手学大语言模型}\\[2cm]
    
    {\Large 从理论到实践}\\[1cm]
    
    {\large Datawhale 开源学习社区}\\[2cm]
    
    \vfill
    
    {\large \today}
\end{titlepage}

% 版权页
\newpage
\thispagestyle{empty}
\vspace*{2cm}
\begin{center}
    \textbf{版权声明}
    
    \vspace{1cm}
    
    本书由 Datawhale 开源学习社区编写,采用开源协议发布。
    
    欢迎读者在遵守开源协议的前提下自由使用、修改和分发本书内容。
    
    \vspace{1cm}
    
    GitHub: https://github.com/datawhalechina/happy-llm
\end{center}

% 目录
\tableofcontents
\newpage

% 动手搭建大模型
\chapter{动手搭建大模型}
% Options for packages loaded elsewhere
\PassOptionsToPackage{unicode}{hyperref}
\PassOptionsToPackage{hyphens}{url}
\documentclass[
]{article}
\usepackage{xcolor}
\usepackage{amsmath,amssymb}
\setcounter{secnumdepth}{5}
\usepackage{iftex}
\ifPDFTeX
  \usepackage[T1]{fontenc}
  \usepackage[utf8]{inputenc}
  \usepackage{textcomp} % provide euro and other symbols
\else % if luatex or xetex
  \usepackage{unicode-math} % this also loads fontspec
  \defaultfontfeatures{Scale=MatchLowercase}
  \defaultfontfeatures[\rmfamily]{Ligatures=TeX,Scale=1}
\fi
\usepackage{lmodern}
\ifPDFTeX\else
  % xetex/luatex font selection
\fi
% Use upquote if available, for straight quotes in verbatim environments
\IfFileExists{upquote.sty}{\usepackage{upquote}}{}
\IfFileExists{microtype.sty}{% use microtype if available
  \usepackage[]{microtype}
  \UseMicrotypeSet[protrusion]{basicmath} % disable protrusion for tt fonts
}{}
\makeatletter
\@ifundefined{KOMAClassName}{% if non-KOMA class
  \IfFileExists{parskip.sty}{%
    \usepackage{parskip}
  }{% else
    \setlength{\parindent}{0pt}
    \setlength{\parskip}{6pt plus 2pt minus 1pt}}
}{% if KOMA class
  \KOMAoptions{parskip=half}}
\makeatother
\usepackage{color}
\usepackage{fancyvrb}
\newcommand{\VerbBar}{|}
\newcommand{\VERB}{\Verb[commandchars=\\\{\}]}
\DefineVerbatimEnvironment{Highlighting}{Verbatim}{commandchars=\\\{\}}
% Add ',fontsize=\small' for more characters per line
\newenvironment{Shaded}{}{}
\newcommand{\AlertTok}[1]{\textcolor[rgb]{1.00,0.00,0.00}{\textbf{#1}}}
\newcommand{\AnnotationTok}[1]{\textcolor[rgb]{0.38,0.63,0.69}{\textbf{\textit{#1}}}}
\newcommand{\AttributeTok}[1]{\textcolor[rgb]{0.49,0.56,0.16}{#1}}
\newcommand{\BaseNTok}[1]{\textcolor[rgb]{0.25,0.63,0.44}{#1}}
\newcommand{\BuiltInTok}[1]{\textcolor[rgb]{0.00,0.50,0.00}{#1}}
\newcommand{\CharTok}[1]{\textcolor[rgb]{0.25,0.44,0.63}{#1}}
\newcommand{\CommentTok}[1]{\textcolor[rgb]{0.38,0.63,0.69}{\textit{#1}}}
\newcommand{\CommentVarTok}[1]{\textcolor[rgb]{0.38,0.63,0.69}{\textbf{\textit{#1}}}}
\newcommand{\ConstantTok}[1]{\textcolor[rgb]{0.53,0.00,0.00}{#1}}
\newcommand{\ControlFlowTok}[1]{\textcolor[rgb]{0.00,0.44,0.13}{\textbf{#1}}}
\newcommand{\DataTypeTok}[1]{\textcolor[rgb]{0.56,0.13,0.00}{#1}}
\newcommand{\DecValTok}[1]{\textcolor[rgb]{0.25,0.63,0.44}{#1}}
\newcommand{\DocumentationTok}[1]{\textcolor[rgb]{0.73,0.13,0.13}{\textit{#1}}}
\newcommand{\ErrorTok}[1]{\textcolor[rgb]{1.00,0.00,0.00}{\textbf{#1}}}
\newcommand{\ExtensionTok}[1]{#1}
\newcommand{\FloatTok}[1]{\textcolor[rgb]{0.25,0.63,0.44}{#1}}
\newcommand{\FunctionTok}[1]{\textcolor[rgb]{0.02,0.16,0.49}{#1}}
\newcommand{\ImportTok}[1]{\textcolor[rgb]{0.00,0.50,0.00}{\textbf{#1}}}
\newcommand{\InformationTok}[1]{\textcolor[rgb]{0.38,0.63,0.69}{\textbf{\textit{#1}}}}
\newcommand{\KeywordTok}[1]{\textcolor[rgb]{0.00,0.44,0.13}{\textbf{#1}}}
\newcommand{\NormalTok}[1]{#1}
\newcommand{\OperatorTok}[1]{\textcolor[rgb]{0.40,0.40,0.40}{#1}}
\newcommand{\OtherTok}[1]{\textcolor[rgb]{0.00,0.44,0.13}{#1}}
\newcommand{\PreprocessorTok}[1]{\textcolor[rgb]{0.74,0.48,0.00}{#1}}
\newcommand{\RegionMarkerTok}[1]{#1}
\newcommand{\SpecialCharTok}[1]{\textcolor[rgb]{0.25,0.44,0.63}{#1}}
\newcommand{\SpecialStringTok}[1]{\textcolor[rgb]{0.73,0.40,0.53}{#1}}
\newcommand{\StringTok}[1]{\textcolor[rgb]{0.25,0.44,0.63}{#1}}
\newcommand{\VariableTok}[1]{\textcolor[rgb]{0.10,0.09,0.49}{#1}}
\newcommand{\VerbatimStringTok}[1]{\textcolor[rgb]{0.25,0.44,0.63}{#1}}
\newcommand{\WarningTok}[1]{\textcolor[rgb]{0.38,0.63,0.69}{\textbf{\textit{#1}}}}
\setlength{\emergencystretch}{3em} % prevent overfull lines
\providecommand{\tightlist}{%
  \setlength{\itemsep}{0pt}\setlength{\parskip}{0pt}}
\usepackage{bookmark}
\IfFileExists{xurl.sty}{\usepackage{xurl}}{} % add URL line breaks if available
\urlstyle{same}
\hypersetup{
  hidelinks,
  pdfcreator={LaTeX via pandoc}}

\author{}
\date{}

\begin{document}

{
\setcounter{tocdepth}{3}
\tableofcontents
}
\section{第五章
动手搭建大模型}\label{ux7b2cux4e94ux7ae0-ux52a8ux624bux642dux5efaux5927ux6a21ux578b}

\subsection{5.1 动手实现一个 LLaMA2
大模型}\label{ux52a8ux624bux5b9eux73b0ux4e00ux4e2a-llama2-ux5927ux6a21ux578b}

Meta(原Facebook)于2023年2月发布第一款基于Transformer结构的大型语言模型LLaMA,并于同年7月发布同系列模型LLaMA2。我们在第四章已经学习和了解了LLM,以及如何训练LLM等内容。本小节我们就来学习如何动手实现一个LLaMA2模型。

LLaMA2 模型结构如下图5.1所示:

\begin{figure}[htbp]\centering
\includegraphics[width=1.0\textwidth]{https://raw.githubusercontent.com/datawhalechina/happy-llm/main/docs/images/5-images/LLama2.png}
\caption{图 5.1 LLaMA2结构}
\end{figure}

\subsubsection{5.1.1 定义超参数}\label{ux5b9aux4e49ux8d85ux53c2ux6570}

首先我们需要定义一些超参数,这些超参数包括模型的大小、层数、头数、词嵌入维度、隐藏层维度等等。这些超参数可以根据实际情况进行调整。

这里我们自定义一个\texttt{ModelConfig}类,来存储和记录我们的超参数,这里我们继承了\texttt{PretrainedConfig}类,这是\texttt{transformers}库中的参数类,我们可以通过继承这个类来方便的使用\texttt{transformers}库中的一些功能,也方便在后续导出Hugging
Face模型。

\begin{Shaded}
\begin{Highlighting}[]
\ImportTok{from}\NormalTok{ transformers }\ImportTok{import}\NormalTok{ PretrainedConfig}

\KeywordTok{class}\NormalTok{ ModelConfig(PretrainedConfig):}
\NormalTok{    model\_type }\OperatorTok{=} \StringTok{"Tiny{-}K"}
    \KeywordTok{def} \FunctionTok{\_\_init\_\_}\NormalTok{(}
            \VariableTok{self}\NormalTok{,}
\NormalTok{            dim: }\BuiltInTok{int} \OperatorTok{=} \DecValTok{768}\NormalTok{, }\CommentTok{\# 模型维度}
\NormalTok{            n\_layers: }\BuiltInTok{int} \OperatorTok{=} \DecValTok{12}\NormalTok{, }\CommentTok{\# Transformer的层数}
\NormalTok{            n\_heads: }\BuiltInTok{int} \OperatorTok{=} \DecValTok{16}\NormalTok{, }\CommentTok{\# 注意力机制的头数}
\NormalTok{            n\_kv\_heads: }\BuiltInTok{int} \OperatorTok{=} \DecValTok{8}\NormalTok{, }\CommentTok{\# 键值头的数量}
\NormalTok{            vocab\_size: }\BuiltInTok{int} \OperatorTok{=} \DecValTok{6144}\NormalTok{, }\CommentTok{\# 词汇表大小}
\NormalTok{            hidden\_dim: }\BuiltInTok{int} \OperatorTok{=} \VariableTok{None}\NormalTok{, }\CommentTok{\# 隐藏层维度}
\NormalTok{            multiple\_of: }\BuiltInTok{int} \OperatorTok{=} \DecValTok{64}\NormalTok{, }
\NormalTok{            norm\_eps: }\BuiltInTok{float} \OperatorTok{=} \FloatTok{1e{-}5}\NormalTok{, }\CommentTok{\# 归一化层的eps}
\NormalTok{            max\_seq\_len: }\BuiltInTok{int} \OperatorTok{=} \DecValTok{512}\NormalTok{, }\CommentTok{\# 最大序列长度}
\NormalTok{            dropout: }\BuiltInTok{float} \OperatorTok{=} \FloatTok{0.0}\NormalTok{, }\CommentTok{\# dropout概率}
\NormalTok{            flash\_attn: }\BuiltInTok{bool} \OperatorTok{=} \VariableTok{True}\NormalTok{, }\CommentTok{\# 是否使用Flash Attention}
            \OperatorTok{**}\NormalTok{kwargs,}
\NormalTok{    ):}
        \VariableTok{self}\NormalTok{.dim }\OperatorTok{=}\NormalTok{ dim}
        \VariableTok{self}\NormalTok{.n\_layers }\OperatorTok{=}\NormalTok{ n\_layers}
        \VariableTok{self}\NormalTok{.n\_heads }\OperatorTok{=}\NormalTok{ n\_heads}
        \VariableTok{self}\NormalTok{.n\_kv\_heads }\OperatorTok{=}\NormalTok{ n\_kv\_heads}
        \VariableTok{self}\NormalTok{.vocab\_size }\OperatorTok{=}\NormalTok{ vocab\_size}
        \VariableTok{self}\NormalTok{.hidden\_dim }\OperatorTok{=}\NormalTok{ hidden\_dim}
        \VariableTok{self}\NormalTok{.multiple\_of }\OperatorTok{=}\NormalTok{ multiple\_of}
        \VariableTok{self}\NormalTok{.norm\_eps }\OperatorTok{=}\NormalTok{ norm\_eps}
        \VariableTok{self}\NormalTok{.max\_seq\_len }\OperatorTok{=}\NormalTok{ max\_seq\_len}
        \VariableTok{self}\NormalTok{.dropout }\OperatorTok{=}\NormalTok{ dropout}
        \VariableTok{self}\NormalTok{.flash\_attn }\OperatorTok{=}\NormalTok{ flash\_attn}
        \BuiltInTok{super}\NormalTok{().}\FunctionTok{\_\_init\_\_}\NormalTok{(}\OperatorTok{**}\NormalTok{kwargs)}
\end{Highlighting}
\end{Shaded}

\begin{quote}
在以下代码中出现 \texttt{args} 时,即默认为以上 \texttt{ModelConfig}
参数配置。
\end{quote}

我们来看一下其中的一些超参数的含义,比如\texttt{dim}是模型维度,\texttt{n\_layers}是Transformer的层数,\texttt{n\_heads}是注意力机制的头数,\texttt{vocab\_size}是词汇表大小,\texttt{max\_seq\_len}是输入的最大序列长度等等。上面的代码中也对每一个参数做了详细的注释,在后面的代码中我们会根据这些超参数来构建我们的模型。

\subsubsection{5.1.2 构建 RMSNorm}\label{ux6784ux5efa-rmsnorm}

\texttt{RMSNorm}可以用如下的数学公式表示:

\[
\text{RMSNorm}(x) = \frac{x}{\sqrt{\frac{1}{n}\sum_{i=1}^{n}x_i^2 + \epsilon}} \cdot \gamma
\]

其中: - \(x_i\) 是输入向量的第 \(i\) 个元素 - \(\gamma\)
是可学习的缩放参数(对应代码中的 \texttt{self.weight}) - \(n\)
是输入向量的维度数量 - \(\epsilon\)
是一个小常数,用于数值稳定性(以避免除以零的情况)

这种归一化有助于通过确保权重的规模不会变得过大或过小来稳定学习过程,这在具有许多层的深度学习模型中特别有用。

我们可以通过如下代码实现\texttt{RMSNorm}:

\begin{Shaded}
\begin{Highlighting}[]
\KeywordTok{class}\NormalTok{ RMSNorm(nn.Module):}
    \KeywordTok{def} \FunctionTok{\_\_init\_\_}\NormalTok{(}\VariableTok{self}\NormalTok{, dim: }\BuiltInTok{int}\NormalTok{, eps: }\BuiltInTok{float}\NormalTok{):}
        \BuiltInTok{super}\NormalTok{().}\FunctionTok{\_\_init\_\_}\NormalTok{()}
        \CommentTok{\# eps是为了防止除以0的情况}
        \VariableTok{self}\NormalTok{.eps }\OperatorTok{=}\NormalTok{ eps}
        \CommentTok{\# weight是一个可学习的参数,全部初始化为1}
        \VariableTok{self}\NormalTok{.weight }\OperatorTok{=}\NormalTok{ nn.Parameter(torch.ones(dim))}

    \KeywordTok{def}\NormalTok{ \_norm(}\VariableTok{self}\NormalTok{, x):}
        \CommentTok{\# 计算RMSNorm的核心部分}
        \CommentTok{\# x.pow(2).mean({-}1, keepdim=True)计算了输入x的平方的均值}
        \CommentTok{\# torch.rsqrt是平方根的倒数,这样就得到了RMSNorm的分母部分,再加上eps防止分母为0}
        \CommentTok{\# 最后乘以x,得到RMSNorm的结果}
        \ControlFlowTok{return}\NormalTok{ x }\OperatorTok{*}\NormalTok{ torch.rsqrt(x.}\BuiltInTok{pow}\NormalTok{(}\DecValTok{2}\NormalTok{).mean(}\OperatorTok{{-}}\DecValTok{1}\NormalTok{, keepdim}\OperatorTok{=}\VariableTok{True}\NormalTok{) }\OperatorTok{+} \VariableTok{self}\NormalTok{.eps)}

    \KeywordTok{def}\NormalTok{ forward(}\VariableTok{self}\NormalTok{, x):}
        \CommentTok{\# forward函数是模型的前向传播}
        \CommentTok{\# 首先将输入x转为float类型,然后进行RMSNorm,最后再转回原来的数据类型}
        \CommentTok{\# 最后乘以weight,这是RMSNorm的一个可学习的缩放因子}
\NormalTok{        output }\OperatorTok{=} \VariableTok{self}\NormalTok{.\_norm(x.}\BuiltInTok{float}\NormalTok{()).type\_as(x)}
        \ControlFlowTok{return}\NormalTok{ output }\OperatorTok{*} \VariableTok{self}\NormalTok{.weight}
\end{Highlighting}
\end{Shaded}

并且,我们可以用下面的代码来对\texttt{RMSNorm}模块进行测试,可以看到代码最终输出的形状为\texttt{torch.Size({[}1,\ 50,\ 288{]})},与我们输入的形状一致,说明模块的实现是正确的,归一化并不会改变输入的形状。

\begin{Shaded}
\begin{Highlighting}[]
\NormalTok{norm }\OperatorTok{=}\NormalTok{ RMSNorm(args.dim, args.norm\_eps)}
\NormalTok{x }\OperatorTok{=}\NormalTok{ torch.randn(}\DecValTok{1}\NormalTok{, }\DecValTok{50}\NormalTok{, args.dim)}
\NormalTok{output }\OperatorTok{=}\NormalTok{ norm(x)}
\BuiltInTok{print}\NormalTok{(output.shape)}

\NormalTok{out:}
\NormalTok{torch.Size([}\DecValTok{1}\NormalTok{, }\DecValTok{50}\NormalTok{, }\DecValTok{768}\NormalTok{])}
\end{Highlighting}
\end{Shaded}

\subsubsection{5.1.3 构建 LLaMA2
Attention}\label{ux6784ux5efa-llama2-attention}

在 LLaMA2 模型中,虽然只有
LLaMA2-70B模型使用了分组查询注意力机制(Grouped-Query
Attention,GQA),但我们依然选择使用 GQA 来构建我们的 LLaMA Attention
模块,它可以提高模型的效率,并节省一些显存占用。

\begin{figure}[htbp]\centering
\includegraphics[width=0.5\textwidth]{https://raw.githubusercontent.com/datawhalechina/happy-llm/main/docs/images/5-images/llama2-attention.png}
\caption{图 5.2 LLaMA2 Attention 结构}
\end{figure}

\paragraph{5.1.3.1 repeat\_kv}\label{repeat_kv}

在 LLaMA2
模型中,我们需要将键和值的维度扩展到和查询的维度一样,这样才能进行注意力计算。我们可以通过如下代码实现\texttt{repeat\_kv}:

\begin{Shaded}
\begin{Highlighting}[]
\KeywordTok{def}\NormalTok{ repeat\_kv(x: torch.Tensor, n\_rep: }\BuiltInTok{int}\NormalTok{) }\OperatorTok{{-}\textgreater{}}\NormalTok{ torch.Tensor:}
    \CommentTok{\# 获取输入张量的形状:批量大小、序列长度、键/值对头的数量、每个头的维度大小}
\NormalTok{    bs, slen, n\_kv\_heads, head\_dim }\OperatorTok{=}\NormalTok{ x.shape}
    
    \CommentTok{\# 如果重复次数为1,则不需要重复,直接返回原始张量}
    \ControlFlowTok{if}\NormalTok{ n\_rep }\OperatorTok{==} \DecValTok{1}\NormalTok{:}
        \ControlFlowTok{return}\NormalTok{ x}
    
    \CommentTok{\# 对张量进行扩展和重塑操作以重复键值对}
    \ControlFlowTok{return}\NormalTok{ (}
\NormalTok{        x[:, :, :, }\VariableTok{None}\NormalTok{, :]  }\CommentTok{\# 在第四个维度(头的维度前)添加一个新的维度}
\NormalTok{        .expand(bs, slen, n\_kv\_heads, n\_rep, head\_dim)  }\CommentTok{\# 将新添加的维度扩展到n\_rep大小,实现重复的效果}
\NormalTok{        .reshape(bs, slen, n\_kv\_heads }\OperatorTok{*}\NormalTok{ n\_rep, head\_dim)  }\CommentTok{\# 重新塑形,合并键/值对头的数量和重复次数的维度}
\NormalTok{    )}
\end{Highlighting}
\end{Shaded}

在上述代码中:

\begin{itemize}
\item
  首先,获取输入张量的形状:首先,代码通过 x.shape
  获取输入张量的形状,包括批量大小(bs)、序列长度(slen)、键/值对头的数量(n\_kv\_heads)以及每个头的维度大小(head\_dim)。
\item
  然后,检查重复次数:接着,代码检查重复次数 n\_rep
  是否为1。如果是1,则说明不需要对键和值进行重复,直接返回原始张量 x。
\item
  最后,扩展和重塑张量:

  \begin{itemize}
  \tightlist
  \item
    在第三个维度(即键/值对头的维度)之后添加一个新的维度,形成
    \texttt{x{[}:,\ :,\ :,\ None,\ :{]}}。
  \item
    使用 \texttt{expand} 方法将新添加的维度扩展到 \texttt{n\_rep}
    大小,实现键/值对的重复效果。
  \item
    最后,通过 reshape
    方法重新塑形,将扩展后的维度合并回键/值对头的数量中,即
    \texttt{x.reshape(bs,\ slen,\ n\_kv\_heads\ *\ n\_rep,\ head\_dim)},这样最终的张量形状就达到了与查询维度一致的效果。
  \end{itemize}
\end{itemize}

\paragraph{5.1.3.2 旋转嵌入}\label{ux65cbux8f6cux5d4cux5165}

接着我们来实现旋转嵌入,旋转嵌入是 LLaMA2
模型中的一个重要组件,它可以为注意力机制提供更强的上下文信息,从而提高模型的性能。

首先,我们要构造获得旋转嵌入的实部和虚部的函数:

\begin{Shaded}
\begin{Highlighting}[]
\CommentTok{\# 注意:此处的dim应为 dim//n\_head,因为我们是对每个head进行旋转嵌入}
\KeywordTok{def}\NormalTok{ precompute\_freqs\_cis(dim: }\BuiltInTok{int}\NormalTok{, end: }\BuiltInTok{int}\NormalTok{, theta: }\BuiltInTok{float} \OperatorTok{=} \FloatTok{10000.0}\NormalTok{):}
    \CommentTok{\# torch.arange(0, dim, 2)[: (dim // 2)].float()生成了一个从0开始,步长为2的序列,长度为dim的一半}
    \CommentTok{\# 然后每个元素除以dim,再取theta的倒数,得到频率}
\NormalTok{    freqs }\OperatorTok{=} \FloatTok{1.0} \OperatorTok{/}\NormalTok{ (theta }\OperatorTok{**}\NormalTok{ (torch.arange(}\DecValTok{0}\NormalTok{, dim, }\DecValTok{2}\NormalTok{)[: (dim }\OperatorTok{//} \DecValTok{2}\NormalTok{)].}\BuiltInTok{float}\NormalTok{() }\OperatorTok{/}\NormalTok{ dim))}
    \CommentTok{\# 生成一个从0到end的序列,长度为end}
\NormalTok{    t }\OperatorTok{=}\NormalTok{ torch.arange(end, device}\OperatorTok{=}\NormalTok{freqs.device)}
    \CommentTok{\# 计算外积,得到一个二维矩阵,每一行是t的元素乘以freqs的元素}
\NormalTok{    freqs }\OperatorTok{=}\NormalTok{ torch.outer(t, freqs).}\BuiltInTok{float}\NormalTok{()}
    \CommentTok{\# 计算频率的余弦值,得到实部}
\NormalTok{    freqs\_cos }\OperatorTok{=}\NormalTok{ torch.cos(freqs)}
    \CommentTok{\# 计算频率的正弦值,得到虚部}
\NormalTok{    freqs\_sin }\OperatorTok{=}\NormalTok{ torch.sin(freqs)}
    \ControlFlowTok{return}\NormalTok{ freqs\_cos, freqs\_sin}
\end{Highlighting}
\end{Shaded}

\begin{itemize}
\tightlist
\item
  计算频率序列:

  \begin{itemize}
  \tightlist
  \item
    \texttt{torch.arange(0,\ dim,\ 2){[}:\ (dim\ //\ 2){]}.float()}
    生成了一个从0开始,步长为2的序列,其长度为\texttt{dim}的一半。
  \item
    每个元素除以\texttt{dim}后取\texttt{theta}的倒数,得到一个频率序列
    \texttt{freqs}。这一步是为了生成适合旋转嵌入的频率。
  \end{itemize}
\item
  生成时间序列:

  \begin{itemize}
  \tightlist
  \item
    \texttt{t\ =\ torch.arange(end,\ device=freqs.device)}
    生成一个从\texttt{0}到\texttt{end}的序列,长度为\texttt{end}。\texttt{end}通常是序列的最大长度。
  \end{itemize}
\item
  计算频率的外积

  \begin{itemize}
  \tightlist
  \item
    \texttt{freqs\ =\ torch.outer(t,\ freqs).float()} 计算时间序列
    \texttt{t} 和频率序列 \texttt{freqs} 的外积,得到一个二维矩阵
    \texttt{freqs}。每一行是时间序列 \texttt{t} 的元素乘以频率序列
    \texttt{freqs} 的元素。
  \end{itemize}
\item
  计算实部和虚部

  \begin{itemize}
  \tightlist
  \item
    \texttt{freqs\_cos\ =\ torch.cos(freqs)} 计算频率矩阵 \texttt{freqs}
    的余弦值,得到旋转嵌入的实部。
  \item
    \texttt{freqs\_sin\ =\ torch.sin(freqs)} 计算频率矩阵 \texttt{freqs}
    的正弦值,得到旋转嵌入的虚部。
  \end{itemize}
\end{itemize}

最终,该函数返回两个矩阵 \texttt{freqs\_cos} 和
\texttt{freqs\_sin},分别表示旋转嵌入的实部和虚部,用于后续的计算。

接着,我们来构造调整张量形状的\texttt{reshape\_for\_broadcast}函数,这个函数的主要目的是调整
\texttt{freqs\_cis} 的形状,使其在进行广播操作时与 \texttt{x}
的维度对齐,从而能够进行正确的张量运算。

\begin{Shaded}
\begin{Highlighting}[]
\KeywordTok{def}\NormalTok{ reshape\_for\_broadcast(freqs\_cis: torch.Tensor, x: torch.Tensor):}
    \CommentTok{\# 获取x的维度数}
\NormalTok{    ndim }\OperatorTok{=}\NormalTok{ x.ndim}
    
    \CommentTok{\# 断言,确保1在x的维度范围内}
    \ControlFlowTok{assert} \DecValTok{0} \OperatorTok{\textless{}=} \DecValTok{1} \OperatorTok{\textless{}}\NormalTok{ ndim}
    
    \CommentTok{\# 断言,确保freqs\_cis的形状与x的第二维和最后一维相同}
    \ControlFlowTok{assert}\NormalTok{ freqs\_cis.shape }\OperatorTok{==}\NormalTok{ (x.shape[}\DecValTok{1}\NormalTok{], x.shape[}\OperatorTok{{-}}\DecValTok{1}\NormalTok{])}
    
    \CommentTok{\# 构造一个新的形状,除了第二维和最后一维,其他维度都为1,这样做是为了能够将freqs\_cis与x进行广播操作}
\NormalTok{    shape }\OperatorTok{=}\NormalTok{ [d }\ControlFlowTok{if}\NormalTok{ i }\OperatorTok{==} \DecValTok{1} \KeywordTok{or}\NormalTok{ i }\OperatorTok{==}\NormalTok{ ndim }\OperatorTok{{-}} \DecValTok{1} \ControlFlowTok{else} \DecValTok{1} \ControlFlowTok{for}\NormalTok{ i, d }\KeywordTok{in} \BuiltInTok{enumerate}\NormalTok{(x.shape)]}
    
    \CommentTok{\# 将freqs\_cis调整为新的形状,并返回}
    \ControlFlowTok{return}\NormalTok{ freqs\_cis.view(shape)}
\end{Highlighting}
\end{Shaded}

最后,我们可以通过如下代码实现旋转嵌入:

\begin{Shaded}
\begin{Highlighting}[]
\KeywordTok{def}\NormalTok{ apply\_rotary\_emb(}
\NormalTok{    xq: torch.Tensor,}
\NormalTok{    xk: torch.Tensor,}
\NormalTok{    freqs\_cos: torch.Tensor,}
\NormalTok{    freqs\_sin: torch.Tensor}
\NormalTok{) }\OperatorTok{{-}\textgreater{}}\NormalTok{ Tuple[torch.Tensor, torch.Tensor]:}

    \CommentTok{\# 将查询和键张量转换为浮点数,并重塑形状以分离实部和虚部}
\NormalTok{    xq\_r, xq\_i }\OperatorTok{=}\NormalTok{ xq.}\BuiltInTok{float}\NormalTok{().reshape(xq.shape[:}\OperatorTok{{-}}\DecValTok{1}\NormalTok{] }\OperatorTok{+}\NormalTok{ (}\OperatorTok{{-}}\DecValTok{1}\NormalTok{, }\DecValTok{2}\NormalTok{)).unbind(}\OperatorTok{{-}}\DecValTok{1}\NormalTok{)}
\NormalTok{    xk\_r, xk\_i }\OperatorTok{=}\NormalTok{ xk.}\BuiltInTok{float}\NormalTok{().reshape(xk.shape[:}\OperatorTok{{-}}\DecValTok{1}\NormalTok{] }\OperatorTok{+}\NormalTok{ (}\OperatorTok{{-}}\DecValTok{1}\NormalTok{, }\DecValTok{2}\NormalTok{)).unbind(}\OperatorTok{{-}}\DecValTok{1}\NormalTok{)}

    \CommentTok{\# 重新塑形频率张量以进行广播}
\NormalTok{    freqs\_cos }\OperatorTok{=}\NormalTok{ reshape\_for\_broadcast(freqs\_cos, xq\_r)}
\NormalTok{    freqs\_sin }\OperatorTok{=}\NormalTok{ reshape\_for\_broadcast(freqs\_sin, xq\_r)}

    \CommentTok{\# 应用旋转,分别计算旋转后的实部和虚部}
\NormalTok{    xq\_out\_r }\OperatorTok{=}\NormalTok{ xq\_r }\OperatorTok{*}\NormalTok{ freqs\_cos }\OperatorTok{{-}}\NormalTok{ xq\_i }\OperatorTok{*}\NormalTok{ freqs\_sin}
\NormalTok{    xq\_out\_i }\OperatorTok{=}\NormalTok{ xq\_r }\OperatorTok{*}\NormalTok{ freqs\_sin }\OperatorTok{+}\NormalTok{ xq\_i }\OperatorTok{*}\NormalTok{ freqs\_cos}
\NormalTok{    xk\_out\_r }\OperatorTok{=}\NormalTok{ xk\_r }\OperatorTok{*}\NormalTok{ freqs\_cos }\OperatorTok{{-}}\NormalTok{ xk\_i }\OperatorTok{*}\NormalTok{ freqs\_sin}
\NormalTok{    xk\_out\_i }\OperatorTok{=}\NormalTok{ xk\_r }\OperatorTok{*}\NormalTok{ freqs\_sin }\OperatorTok{+}\NormalTok{ xk\_i }\OperatorTok{*}\NormalTok{ freqs\_cos}

    \CommentTok{\# 将最后两个维度合并,并还原为原始张量的形状}
\NormalTok{    xq\_out }\OperatorTok{=}\NormalTok{ torch.stack([xq\_out\_r, xq\_out\_i], dim}\OperatorTok{={-}}\DecValTok{1}\NormalTok{).flatten(}\DecValTok{3}\NormalTok{)}
\NormalTok{    xk\_out }\OperatorTok{=}\NormalTok{ torch.stack([xk\_out\_r, xk\_out\_i], dim}\OperatorTok{={-}}\DecValTok{1}\NormalTok{).flatten(}\DecValTok{3}\NormalTok{)}

    \ControlFlowTok{return}\NormalTok{ xq\_out.type\_as(xq), xk\_out.type\_as(xk)}
\end{Highlighting}
\end{Shaded}

这里我们给出可以测试\texttt{apply\_rotary\_emb}函数的代码,大家也可以尝试在代码中添加断点,来查看每一步的计算结果。

\begin{Shaded}
\begin{Highlighting}[]
\NormalTok{xq }\OperatorTok{=}\NormalTok{ torch.randn(}\DecValTok{1}\NormalTok{, }\DecValTok{50}\NormalTok{, }\DecValTok{6}\NormalTok{, }\DecValTok{48}\NormalTok{) }\CommentTok{\# bs, seq\_len, dim//n\_head, n\_head\_dim}
\NormalTok{xk }\OperatorTok{=}\NormalTok{ torch.randn(}\DecValTok{1}\NormalTok{, }\DecValTok{50}\NormalTok{, }\DecValTok{6}\NormalTok{, }\DecValTok{48}\NormalTok{) }\CommentTok{\# bs, seq\_len, dim//n\_head, n\_head\_dim}

\CommentTok{\# 使用 precompute\_freqs\_cis 函数获取 sin和cos}
\NormalTok{cos, sin }\OperatorTok{=}\NormalTok{ precompute\_freqs\_cis(}\DecValTok{288}\OperatorTok{//}\DecValTok{6}\NormalTok{, }\DecValTok{50}\NormalTok{)}
\BuiltInTok{print}\NormalTok{(cos.shape, sin.shape)}
\NormalTok{xq\_out, xk\_out }\OperatorTok{=}\NormalTok{ apply\_rotary\_emb(xq, xk, cos, sin)}

\NormalTok{xq\_out.shape, xk\_out.shape}
\end{Highlighting}
\end{Shaded}

OUT:

\begin{verbatim}
torch.Size([50, 24]) torch.Size([50, 24])

(torch.Size([1, 50, 6, 48]), torch.Size([1, 50, 6, 48]))
\end{verbatim}

\paragraph{5.1.3.3 组装 LLaMA2
Attention}\label{ux7ec4ux88c5-llama2-attention}

在上面我们已经完成了旋转嵌入的实现,接下来我们就可以构建 LLaMA2
Attention 模块了。

\begin{Shaded}
\begin{Highlighting}[]
\KeywordTok{class}\NormalTok{ Attention(nn.Module):}
    \KeywordTok{def} \FunctionTok{\_\_init\_\_}\NormalTok{(}\VariableTok{self}\NormalTok{, args: ModelConfig):}
        \BuiltInTok{super}\NormalTok{().}\FunctionTok{\_\_init\_\_}\NormalTok{()}
        \CommentTok{\# 根据是否指定n\_kv\_heads,确定用于键(key)和值(value)的头的数量。}
        \VariableTok{self}\NormalTok{.n\_kv\_heads }\OperatorTok{=}\NormalTok{ args.n\_heads }\ControlFlowTok{if}\NormalTok{ args.n\_kv\_heads }\KeywordTok{is} \VariableTok{None} \ControlFlowTok{else}\NormalTok{ args.n\_kv\_heads}
        \CommentTok{\# 确保总头数可以被键值头数整除。}
        \ControlFlowTok{assert}\NormalTok{ args.n\_heads }\OperatorTok{\%} \VariableTok{self}\NormalTok{.n\_kv\_heads }\OperatorTok{==} \DecValTok{0}

        \CommentTok{\# 模型并行处理大小,默认为1。}
\NormalTok{        model\_parallel\_size }\OperatorTok{=} \DecValTok{1}
        \CommentTok{\# 本地计算头数,等于总头数除以模型并行处理大小。}
        \VariableTok{self}\NormalTok{.n\_local\_heads }\OperatorTok{=}\NormalTok{ args.n\_heads }\OperatorTok{//}\NormalTok{ model\_parallel\_size}
        \CommentTok{\# 本地键值头数,等于键值头数除以模型并行处理大小。}
        \VariableTok{self}\NormalTok{.n\_local\_kv\_heads }\OperatorTok{=} \VariableTok{self}\NormalTok{.n\_kv\_heads }\OperatorTok{//}\NormalTok{ model\_parallel\_size}
        \CommentTok{\# 重复次数,用于扩展键和值的尺寸。}
        \VariableTok{self}\NormalTok{.n\_rep }\OperatorTok{=} \VariableTok{self}\NormalTok{.n\_local\_heads }\OperatorTok{//} \VariableTok{self}\NormalTok{.n\_local\_kv\_heads}
        \CommentTok{\# 每个头的维度,等于模型维度除以头的总数。}
        \VariableTok{self}\NormalTok{.head\_dim }\OperatorTok{=}\NormalTok{ args.dim }\OperatorTok{//}\NormalTok{ args.n\_heads}

        \CommentTok{\# 定义权重矩阵。}
        \VariableTok{self}\NormalTok{.wq }\OperatorTok{=}\NormalTok{ nn.Linear(args.dim, args.n\_heads }\OperatorTok{*} \VariableTok{self}\NormalTok{.head\_dim, bias}\OperatorTok{=}\VariableTok{False}\NormalTok{)}
        \VariableTok{self}\NormalTok{.wk }\OperatorTok{=}\NormalTok{ nn.Linear(args.dim, }\VariableTok{self}\NormalTok{.n\_kv\_heads }\OperatorTok{*} \VariableTok{self}\NormalTok{.head\_dim, bias}\OperatorTok{=}\VariableTok{False}\NormalTok{)}
        \VariableTok{self}\NormalTok{.wv }\OperatorTok{=}\NormalTok{ nn.Linear(args.dim, }\VariableTok{self}\NormalTok{.n\_kv\_heads }\OperatorTok{*} \VariableTok{self}\NormalTok{.head\_dim, bias}\OperatorTok{=}\VariableTok{False}\NormalTok{)}
        \CommentTok{\# 输出权重矩阵。}
        \VariableTok{self}\NormalTok{.wo }\OperatorTok{=}\NormalTok{ nn.Linear(args.n\_heads }\OperatorTok{*} \VariableTok{self}\NormalTok{.head\_dim, args.dim, bias}\OperatorTok{=}\VariableTok{False}\NormalTok{)}

        \CommentTok{\# 定义dropout。}
        \VariableTok{self}\NormalTok{.attn\_dropout }\OperatorTok{=}\NormalTok{ nn.Dropout(args.dropout)}
        \VariableTok{self}\NormalTok{.resid\_dropout }\OperatorTok{=}\NormalTok{ nn.Dropout(args.dropout)}
        \CommentTok{\# 保存dropout概率。}
        \VariableTok{self}\NormalTok{.dropout }\OperatorTok{=}\NormalTok{ args.dropout}

        \CommentTok{\# 检查是否使用Flash Attention(需要PyTorch \textgreater{}= 2.0)。}
        \VariableTok{self}\NormalTok{.flash }\OperatorTok{=} \BuiltInTok{hasattr}\NormalTok{(torch.nn.functional, }\StringTok{\textquotesingle{}scaled\_dot\_product\_attention\textquotesingle{}}\NormalTok{)}
        \ControlFlowTok{if} \KeywordTok{not} \VariableTok{self}\NormalTok{.flash:}
            \CommentTok{\# 若不支持Flash Attention,则使用手动实现的注意力机制,并设置mask。}
            \BuiltInTok{print}\NormalTok{(}\StringTok{"WARNING: using slow attention. Flash Attention requires PyTorch \textgreater{}= 2.0"}\NormalTok{)}
            \CommentTok{\# 创建一个上三角矩阵,用于遮蔽未来信息。}
\NormalTok{            mask }\OperatorTok{=}\NormalTok{ torch.full((}\DecValTok{1}\NormalTok{, }\DecValTok{1}\NormalTok{, args.max\_seq\_len, args.max\_seq\_len), }\BuiltInTok{float}\NormalTok{(}\StringTok{"{-}inf"}\NormalTok{))}
\NormalTok{            mask }\OperatorTok{=}\NormalTok{ torch.triu(mask, diagonal}\OperatorTok{=}\DecValTok{1}\NormalTok{)}
            \CommentTok{\# 注册为模型的缓冲区}
            \VariableTok{self}\NormalTok{.register\_buffer(}\StringTok{"mask"}\NormalTok{, mask)}

    \KeywordTok{def}\NormalTok{ forward(}\VariableTok{self}\NormalTok{, x: torch.Tensor, freqs\_cos: torch.Tensor, freqs\_sin: torch.Tensor):}
        \CommentTok{\# 获取批次大小和序列长度,[batch\_size, seq\_len, dim]}
\NormalTok{        bsz, seqlen, \_ }\OperatorTok{=}\NormalTok{ x.shape}

        \CommentTok{\# 计算查询(Q)、键(K)、值(V)。}
\NormalTok{        xq, xk, xv }\OperatorTok{=} \VariableTok{self}\NormalTok{.wq(x), }\VariableTok{self}\NormalTok{.wk(x), }\VariableTok{self}\NormalTok{.wv(x)}
        \CommentTok{\# 调整形状以适应头的维度。}
\NormalTok{        xq }\OperatorTok{=}\NormalTok{ xq.view(bsz, seqlen, }\VariableTok{self}\NormalTok{.n\_local\_heads, }\VariableTok{self}\NormalTok{.head\_dim)}
\NormalTok{        xk }\OperatorTok{=}\NormalTok{ xk.view(bsz, seqlen, }\VariableTok{self}\NormalTok{.n\_local\_kv\_heads, }\VariableTok{self}\NormalTok{.head\_dim)}
\NormalTok{        xv }\OperatorTok{=}\NormalTok{ xv.view(bsz, seqlen, }\VariableTok{self}\NormalTok{.n\_local\_kv\_heads, }\VariableTok{self}\NormalTok{.head\_dim)}

        \CommentTok{\# 应用旋转位置嵌入(RoPE)。}
\NormalTok{        xq, xk }\OperatorTok{=}\NormalTok{ apply\_rotary\_emb(xq, xk, freqs\_cos, freqs\_sin)}

        \CommentTok{\# 对键和值进行扩展以适应重复次数。}
\NormalTok{        xk }\OperatorTok{=}\NormalTok{ repeat\_kv(xk, }\VariableTok{self}\NormalTok{.n\_rep)}
\NormalTok{        xv }\OperatorTok{=}\NormalTok{ repeat\_kv(xv, }\VariableTok{self}\NormalTok{.n\_rep)}

        \CommentTok{\# 将头作为批次维度处理。}
\NormalTok{        xq }\OperatorTok{=}\NormalTok{ xq.transpose(}\DecValTok{1}\NormalTok{, }\DecValTok{2}\NormalTok{)}
\NormalTok{        xk }\OperatorTok{=}\NormalTok{ xk.transpose(}\DecValTok{1}\NormalTok{, }\DecValTok{2}\NormalTok{)}
\NormalTok{        xv }\OperatorTok{=}\NormalTok{ xv.transpose(}\DecValTok{1}\NormalTok{, }\DecValTok{2}\NormalTok{)}

        \CommentTok{\# 根据是否支持Flash Attention,选择实现方式。}
        \ControlFlowTok{if} \VariableTok{self}\NormalTok{.flash:}
            \CommentTok{\# 使用Flash Attention。}
\NormalTok{            output }\OperatorTok{=}\NormalTok{ torch.nn.functional.scaled\_dot\_product\_attention(xq, xk, xv, attn\_mask}\OperatorTok{=}\VariableTok{None}\NormalTok{, dropout\_p}\OperatorTok{=}\VariableTok{self}\NormalTok{.dropout }\ControlFlowTok{if} \VariableTok{self}\NormalTok{.training }\ControlFlowTok{else} \FloatTok{0.0}\NormalTok{, is\_causal}\OperatorTok{=}\VariableTok{True}\NormalTok{)}
        \ControlFlowTok{else}\NormalTok{:}
            \CommentTok{\# 使用手动实现的注意力机制。}
\NormalTok{            scores }\OperatorTok{=}\NormalTok{ torch.matmul(xq, xk.transpose(}\DecValTok{2}\NormalTok{, }\DecValTok{3}\NormalTok{)) }\OperatorTok{/}\NormalTok{ math.sqrt(}\VariableTok{self}\NormalTok{.head\_dim)}
            \ControlFlowTok{assert} \BuiltInTok{hasattr}\NormalTok{(}\VariableTok{self}\NormalTok{, }\StringTok{\textquotesingle{}mask\textquotesingle{}}\NormalTok{)}
\NormalTok{            scores }\OperatorTok{=}\NormalTok{ scores }\OperatorTok{+} \VariableTok{self}\NormalTok{.mask[:, :, :seqlen, :seqlen]}
\NormalTok{            scores }\OperatorTok{=}\NormalTok{ F.softmax(scores.}\BuiltInTok{float}\NormalTok{(), dim}\OperatorTok{={-}}\DecValTok{1}\NormalTok{).type\_as(xq)}
\NormalTok{            scores }\OperatorTok{=} \VariableTok{self}\NormalTok{.attn\_dropout(scores)}
\NormalTok{            output }\OperatorTok{=}\NormalTok{ torch.matmul(scores, xv)}

        \CommentTok{\# 恢复时间维度并合并头。}
\NormalTok{        output }\OperatorTok{=}\NormalTok{ output.transpose(}\DecValTok{1}\NormalTok{, }\DecValTok{2}\NormalTok{).contiguous().view(bsz, seqlen, }\OperatorTok{{-}}\DecValTok{1}\NormalTok{)}

        \CommentTok{\# 最终投影回残差流。}
\NormalTok{        output }\OperatorTok{=} \VariableTok{self}\NormalTok{.wo(output)}
\NormalTok{        output }\OperatorTok{=} \VariableTok{self}\NormalTok{.resid\_dropout(output)}
        \ControlFlowTok{return}\NormalTok{ output}
\end{Highlighting}
\end{Shaded}

同样大家可以使用下面的代码来对注意力模块进行测试,可以看到代码最终输出的形状为\texttt{torch.Size({[}1,\ 50,\ 768{]})},与我们输入的形状一致,说明模块的实现是正确的。

\begin{Shaded}
\begin{Highlighting}[]
\CommentTok{\# 创建Attention实例}
\NormalTok{attention\_model }\OperatorTok{=}\NormalTok{ Attention(args)}

\CommentTok{\# 模拟输入数据}
\NormalTok{batch\_size }\OperatorTok{=} \DecValTok{1}
\NormalTok{seq\_len }\OperatorTok{=} \DecValTok{50}  \CommentTok{\# 假设实际使用的序列长度为50}
\NormalTok{dim }\OperatorTok{=}\NormalTok{ args.dim}
\NormalTok{x }\OperatorTok{=}\NormalTok{ torch.rand(batch\_size, seq\_len, dim)  }\CommentTok{\# 随机生成输入张量}
\CommentTok{\# freqs\_cos = torch.rand(seq\_len, dim // 2)  \# 模拟cos频率,用于RoPE}
\CommentTok{\# freqs\_sin = torch.rand(seq\_len, dim // 2)  \# 模拟sin频率,用于RoPE}

\NormalTok{freqs\_cos, freqs\_sin }\OperatorTok{=}\NormalTok{ precompute\_freqs\_cis(dim}\OperatorTok{//}\NormalTok{args.n\_heads, seq\_len)}

\CommentTok{\# 运行Attention模型}
\NormalTok{output }\OperatorTok{=}\NormalTok{ attention\_model(x, freqs\_cos, freqs\_sin)}

\CommentTok{\# attention出来之后的形状 依然是[batch\_size, seq\_len, dim]}
\BuiltInTok{print}\NormalTok{(}\StringTok{"Output shape:"}\NormalTok{, output.shape)}
\end{Highlighting}
\end{Shaded}

OUT:

\begin{verbatim}
Output shape: torch.Size([1, 50, 768])
\end{verbatim}

\subsubsection{5.1.4 构建 LLaMA2
MLP模块}\label{ux6784ux5efa-llama2-mlpux6a21ux5757}

相对于前面我们实现的LLaMA2 Attention模块,LLaMA2
MLP模块的实现要简单一些。我们可以通过如下代码实现\texttt{MLP}:

\begin{Shaded}
\begin{Highlighting}[]
\KeywordTok{class}\NormalTok{ MLP(nn.Module):}
    \KeywordTok{def} \FunctionTok{\_\_init\_\_}\NormalTok{(}\VariableTok{self}\NormalTok{, dim: }\BuiltInTok{int}\NormalTok{, hidden\_dim: }\BuiltInTok{int}\NormalTok{, multiple\_of: }\BuiltInTok{int}\NormalTok{, dropout: }\BuiltInTok{float}\NormalTok{):}
        \BuiltInTok{super}\NormalTok{().}\FunctionTok{\_\_init\_\_}\NormalTok{()}
        \CommentTok{\# 如果没有指定隐藏层的维度,我们将其设置为输入维度的4倍}
        \CommentTok{\# 然后将其减少到2/3,最后确保它是multiple\_of的倍数}
        \ControlFlowTok{if}\NormalTok{ hidden\_dim }\KeywordTok{is} \VariableTok{None}\NormalTok{:}
\NormalTok{            hidden\_dim }\OperatorTok{=} \DecValTok{4} \OperatorTok{*}\NormalTok{ dim}
\NormalTok{            hidden\_dim }\OperatorTok{=} \BuiltInTok{int}\NormalTok{(}\DecValTok{2} \OperatorTok{*}\NormalTok{ hidden\_dim }\OperatorTok{/} \DecValTok{3}\NormalTok{)}
\NormalTok{            hidden\_dim }\OperatorTok{=}\NormalTok{ multiple\_of }\OperatorTok{*}\NormalTok{ ((hidden\_dim }\OperatorTok{+}\NormalTok{ multiple\_of }\OperatorTok{{-}} \DecValTok{1}\NormalTok{) }\OperatorTok{//}\NormalTok{ multiple\_of)}
        \CommentTok{\# 定义第一层线性变换,从输入维度到隐藏维度}
        \VariableTok{self}\NormalTok{.w1 }\OperatorTok{=}\NormalTok{ nn.Linear(dim, hidden\_dim, bias}\OperatorTok{=}\VariableTok{False}\NormalTok{)}
        \CommentTok{\# 定义第二层线性变换,从隐藏维度到输入维度}
        \VariableTok{self}\NormalTok{.w2 }\OperatorTok{=}\NormalTok{ nn.Linear(hidden\_dim, dim, bias}\OperatorTok{=}\VariableTok{False}\NormalTok{)}
        \CommentTok{\# 定义第三层线性变换,从输入维度到隐藏维度}
        \VariableTok{self}\NormalTok{.w3 }\OperatorTok{=}\NormalTok{ nn.Linear(dim, hidden\_dim, bias}\OperatorTok{=}\VariableTok{False}\NormalTok{)}
        \CommentTok{\# 定义dropout层,用于防止过拟合}
        \VariableTok{self}\NormalTok{.dropout }\OperatorTok{=}\NormalTok{ nn.Dropout(dropout)}

    \KeywordTok{def}\NormalTok{ forward(}\VariableTok{self}\NormalTok{, x):}
        \CommentTok{\# 前向传播函数}
        \CommentTok{\# 首先,输入x通过第一层线性变换和SILU激活函数}
        \CommentTok{\# 然后,结果乘以输入x通过第三层线性变换的结果}
        \CommentTok{\# 最后,通过第二层线性变换和dropout层}
        \ControlFlowTok{return} \VariableTok{self}\NormalTok{.dropout(}\VariableTok{self}\NormalTok{.w2(F.silu(}\VariableTok{self}\NormalTok{.w1(x)) }\OperatorTok{*} \VariableTok{self}\NormalTok{.w3(x)))}
\end{Highlighting}
\end{Shaded}

我们着重观察一下\texttt{forward}函数的实现,首先,输入 \texttt{x}
通过第一层线性变换 \texttt{self.w1} 和 \texttt{SILU}
激活函数,然后,结果乘以输入 \texttt{x} 通过第三层线性变换
\texttt{self.w3} 的结果,最后,通过第二层线性变换 \texttt{self.w2} 和
\texttt{dropout} 层,得到最终输出。

同样大家可以使用下面的代码来对\texttt{LLaMAMLP}模块进行测试,可以看到代码最终输出的形状为\texttt{torch.Size({[}1,\ 50,\ 768{]})},与我们输入的形状一致,说明模块的实现是正确的。

\begin{Shaded}
\begin{Highlighting}[]
\CommentTok{\# 创建MLP实例}
\NormalTok{mlp }\OperatorTok{=}\NormalTok{ MLP(args.dim, args.hidden\_dim, args.multiple\_of, args.dropout)}
\CommentTok{\# 随机生成数据}
\NormalTok{x }\OperatorTok{=}\NormalTok{ torch.randn(}\DecValTok{1}\NormalTok{, }\DecValTok{50}\NormalTok{, args.dim)}
\CommentTok{\# 运行MLP模型}
\NormalTok{output }\OperatorTok{=}\NormalTok{ mlp(x)}
\BuiltInTok{print}\NormalTok{(output.shape)}
\end{Highlighting}
\end{Shaded}

OUT:

\begin{verbatim}
torch.Size([1, 50, 768])
\end{verbatim}

\subsubsection{5.1.5 LLaMA2 Decoder Layer}\label{llama2-decoder-layer}

到这里,我们已经实现了\texttt{LLaMA2}模型的\texttt{Attention}模块和\texttt{MLP}模块,接下来我们就可以构建\texttt{LLaMA2}的\texttt{Decoder\ Layer}了。

\begin{Shaded}
\begin{Highlighting}[]
\KeywordTok{class}\NormalTok{ DecoderLayer(nn.Module):}
    \KeywordTok{def} \FunctionTok{\_\_init\_\_}\NormalTok{(}\VariableTok{self}\NormalTok{, layer\_id: }\BuiltInTok{int}\NormalTok{, args: ModelConfig):}
        \BuiltInTok{super}\NormalTok{().}\FunctionTok{\_\_init\_\_}\NormalTok{()}
        \CommentTok{\# 定义多头注意力的头数}
        \VariableTok{self}\NormalTok{.n\_heads }\OperatorTok{=}\NormalTok{ args.n\_heads}
        \CommentTok{\# 定义输入维度}
        \VariableTok{self}\NormalTok{.dim }\OperatorTok{=}\NormalTok{ args.dim}
        \CommentTok{\# 定义每个头的维度,等于输入维度除以头数}
        \VariableTok{self}\NormalTok{.head\_dim }\OperatorTok{=}\NormalTok{ args.dim }\OperatorTok{//}\NormalTok{ args.n\_heads}
        \CommentTok{\# 定义LLaMA2Attention对象,用于进行多头注意力计算}
        \VariableTok{self}\NormalTok{.attention }\OperatorTok{=}\NormalTok{ Attention(args)}
        \CommentTok{\# 定义LLaMAMLP对象,用于进行前馈神经网络计算}
        \VariableTok{self}\NormalTok{.feed\_forward }\OperatorTok{=}\NormalTok{ MLP(}
\NormalTok{            dim}\OperatorTok{=}\NormalTok{args.dim,}
\NormalTok{            hidden\_dim}\OperatorTok{=}\NormalTok{args.hidden\_dim,}
\NormalTok{            multiple\_of}\OperatorTok{=}\NormalTok{args.multiple\_of,}
\NormalTok{            dropout}\OperatorTok{=}\NormalTok{args.dropout,}
\NormalTok{        )}
        \CommentTok{\# 定义层的ID}
        \VariableTok{self}\NormalTok{.layer\_id }\OperatorTok{=}\NormalTok{ layer\_id}
        \CommentTok{\# 定义注意力计算的归一化层}
        \VariableTok{self}\NormalTok{.attention\_norm }\OperatorTok{=}\NormalTok{ RMSNorm(args.dim, eps}\OperatorTok{=}\NormalTok{args.norm\_eps)}
        \CommentTok{\# 定义前馈神经网络计算的归一化层}
        \VariableTok{self}\NormalTok{.ffn\_norm }\OperatorTok{=}\NormalTok{ RMSNorm(args.dim, eps}\OperatorTok{=}\NormalTok{args.norm\_eps)}

    \KeywordTok{def}\NormalTok{ forward(}\VariableTok{self}\NormalTok{, x, freqs\_cos, freqs\_sin):}
        \CommentTok{\# 前向传播函数}
        \CommentTok{\# 首先,输入x经过注意力归一化层,然后进行注意力计算,结果与输入x相加得到h}
        \CommentTok{\# 然后,h经过前馈神经网络归一化层,然后进行前馈神经网络计算,结果与h相加得到输出}
\NormalTok{        h }\OperatorTok{=}\NormalTok{ x }\OperatorTok{+} \VariableTok{self}\NormalTok{.attention.forward(}\VariableTok{self}\NormalTok{.attention\_norm(x), freqs\_cos, freqs\_sin)}
\NormalTok{        out }\OperatorTok{=}\NormalTok{ h }\OperatorTok{+} \VariableTok{self}\NormalTok{.feed\_forward.forward(}\VariableTok{self}\NormalTok{.ffn\_norm(h))}
        \ControlFlowTok{return}\NormalTok{ out}
\end{Highlighting}
\end{Shaded}

\texttt{DecoderLayer}就是把我们上面完成的\texttt{Attention}模块和\texttt{MLP}模块组合在一起,实现了一个完整的\texttt{Transformer}模块。

同样大家可以使用下面的代码来对\texttt{DecoderLayer}模块进行测试,可以看到代码最终输出的形状为\texttt{torch.Size({[}1,\ 50,\ 768{]})},与我们输入的形状一致,说明模块的实现是正确的。

\begin{Shaded}
\begin{Highlighting}[]
\CommentTok{\# 创建LLaMADecoderLayer实例}
\NormalTok{decoderlayer }\OperatorTok{=}\NormalTok{ DecoderLayer(}\DecValTok{0}\NormalTok{, args)}

\CommentTok{\# 模拟输入数据}
\NormalTok{dim }\OperatorTok{=}\NormalTok{ args.dim}
\NormalTok{seq\_len }\OperatorTok{=} \DecValTok{50}

\NormalTok{x }\OperatorTok{=}\NormalTok{ torch.randn(}\DecValTok{1}\NormalTok{, seq\_len, dim) }\CommentTok{\# [bs, seq\_len, dim]}

\NormalTok{freqs\_cos, freqs\_sin }\OperatorTok{=}\NormalTok{ precompute\_freqs\_cis(dim}\OperatorTok{//}\NormalTok{args.n\_heads, seq\_len)}

\NormalTok{out }\OperatorTok{=}\NormalTok{ decoderlayer(x, freqs\_cos, freqs\_sin)}

\BuiltInTok{print}\NormalTok{(out.shape) }\CommentTok{\# 形状和输入的x一样 [batch\_size, seq\_len, dim]}
\end{Highlighting}
\end{Shaded}

OUT:

\begin{verbatim}
torch.Size([1, 50, 768])
\end{verbatim}

\subsubsection{5.1.6 构建 LLaMA2
模型}\label{ux6784ux5efa-llama2-ux6a21ux578b}

好了,我们已经完了上述所有的模块的实现,接下来就是激动人心的时刻,我们可以构建\texttt{LLaMA2}模型了。,\texttt{LLaMA2}模型就是将\texttt{DecoderLayer}模块堆叠起来,构成一个完整的\texttt{Transformer}模型。

\begin{Shaded}
\begin{Highlighting}[]
\KeywordTok{class}\NormalTok{ Transformer(PreTrainedModel):}
\NormalTok{    config\_class }\OperatorTok{=}\NormalTok{ ModelConfig  }\CommentTok{\# 配置类}
\NormalTok{    last\_loss: Optional[torch.Tensor] }\CommentTok{\# 记录最后一次计算的损失}

    \KeywordTok{def} \FunctionTok{\_\_init\_\_}\NormalTok{(}\VariableTok{self}\NormalTok{, args: ModelConfig }\OperatorTok{=} \VariableTok{None}\NormalTok{):}
        \BuiltInTok{super}\NormalTok{().}\FunctionTok{\_\_init\_\_}\NormalTok{(args)}
        \CommentTok{\# 初始化模型参数}
        \VariableTok{self}\NormalTok{.args }\OperatorTok{=}\NormalTok{ args}
        \CommentTok{\# 词汇表大小}
        \VariableTok{self}\NormalTok{.vocab\_size }\OperatorTok{=}\NormalTok{ args.vocab\_size}
        \CommentTok{\# 层数}
        \VariableTok{self}\NormalTok{.n\_layers }\OperatorTok{=}\NormalTok{ args.n\_layers}

        \CommentTok{\# 词嵌入层}
        \VariableTok{self}\NormalTok{.tok\_embeddings }\OperatorTok{=}\NormalTok{ nn.Embedding(args.vocab\_size, args.dim)}
        \CommentTok{\# Dropout层}
        \VariableTok{self}\NormalTok{.dropout }\OperatorTok{=}\NormalTok{ nn.Dropout(args.dropout)}
        \CommentTok{\# Decoder层}
        \VariableTok{self}\NormalTok{.layers }\OperatorTok{=}\NormalTok{ torch.nn.ModuleList()}
        \ControlFlowTok{for}\NormalTok{ layer\_id }\KeywordTok{in} \BuiltInTok{range}\NormalTok{(args.n\_layers):}
            \VariableTok{self}\NormalTok{.layers.append(DecoderLayer(layer\_id, args))}
        \CommentTok{\# 归一化层}
        \VariableTok{self}\NormalTok{.norm }\OperatorTok{=}\NormalTok{ RMSNorm(args.dim, eps}\OperatorTok{=}\NormalTok{args.norm\_eps)}
        \CommentTok{\# 输出层}
        \VariableTok{self}\NormalTok{.output }\OperatorTok{=}\NormalTok{ nn.Linear(args.dim, args.vocab\_size, bias}\OperatorTok{=}\VariableTok{False}\NormalTok{)}

        \CommentTok{\# 将词嵌入层的权重与输出层的权重共享}
        \VariableTok{self}\NormalTok{.tok\_embeddings.weight }\OperatorTok{=} \VariableTok{self}\NormalTok{.output.weight }

        \CommentTok{\# 预计算相对位置嵌入的频率}
\NormalTok{        freqs\_cos, freqs\_sin }\OperatorTok{=}\NormalTok{ precompute\_freqs\_cis(}\VariableTok{self}\NormalTok{.args.dim }\OperatorTok{//} \VariableTok{self}\NormalTok{.args.n\_heads, }\VariableTok{self}\NormalTok{.args.max\_seq\_len)}
        \VariableTok{self}\NormalTok{.register\_buffer(}\StringTok{"freqs\_cos"}\NormalTok{, freqs\_cos, persistent}\OperatorTok{=}\VariableTok{False}\NormalTok{)}
        \VariableTok{self}\NormalTok{.register\_buffer(}\StringTok{"freqs\_sin"}\NormalTok{, freqs\_sin, persistent}\OperatorTok{=}\VariableTok{False}\NormalTok{)}

        \CommentTok{\# 初始化所有权重}
        \VariableTok{self}\NormalTok{.}\BuiltInTok{apply}\NormalTok{(}\VariableTok{self}\NormalTok{.\_init\_weights)}
        \CommentTok{\# 对残差投影进行特殊的缩放初始化}
        \ControlFlowTok{for}\NormalTok{ pn, p }\KeywordTok{in} \VariableTok{self}\NormalTok{.named\_parameters():}
            \ControlFlowTok{if}\NormalTok{ pn.endswith(}\StringTok{\textquotesingle{}w3.weight\textquotesingle{}}\NormalTok{) }\KeywordTok{or}\NormalTok{ pn.endswith(}\StringTok{\textquotesingle{}wo.weight\textquotesingle{}}\NormalTok{):}
\NormalTok{                torch.nn.init.normal\_(p, mean}\OperatorTok{=}\FloatTok{0.0}\NormalTok{, std}\OperatorTok{=}\FloatTok{0.02}\OperatorTok{/}\NormalTok{math.sqrt(}\DecValTok{2} \OperatorTok{*}\NormalTok{ args.n\_layers))}

        \CommentTok{\# 初始化最后一次前向传播的损失属性}
        \VariableTok{self}\NormalTok{.last\_loss }\OperatorTok{=} \VariableTok{None}
        \VariableTok{self}\NormalTok{.OUT }\OperatorTok{=}\NormalTok{ CausalLMOutputWithPast()  }\CommentTok{\# 输出容器}
        \VariableTok{self}\NormalTok{.\_no\_split\_modules }\OperatorTok{=}\NormalTok{ [name }\ControlFlowTok{for}\NormalTok{ name, \_ }\KeywordTok{in} \VariableTok{self}\NormalTok{.named\_modules()]  }\CommentTok{\# 不分割的模块列表}

    \KeywordTok{def}\NormalTok{ \_init\_weights(}\VariableTok{self}\NormalTok{, module):}
        \CommentTok{\# 初始化权重的函数}
        \ControlFlowTok{if} \BuiltInTok{isinstance}\NormalTok{(module, nn.Linear):}
\NormalTok{            torch.nn.init.normal\_(module.weight, mean}\OperatorTok{=}\FloatTok{0.0}\NormalTok{, std}\OperatorTok{=}\FloatTok{0.02}\NormalTok{)}
            \ControlFlowTok{if}\NormalTok{ module.bias }\KeywordTok{is} \KeywordTok{not} \VariableTok{None}\NormalTok{:}
\NormalTok{                torch.nn.init.zeros\_(module.bias)}
        \ControlFlowTok{elif} \BuiltInTok{isinstance}\NormalTok{(module, nn.Embedding):}
\NormalTok{            torch.nn.init.normal\_(module.weight, mean}\OperatorTok{=}\FloatTok{0.0}\NormalTok{, std}\OperatorTok{=}\FloatTok{0.02}\NormalTok{)}
    
    \KeywordTok{def}\NormalTok{ forward(}\VariableTok{self}\NormalTok{, tokens: torch.Tensor, targets: Optional[torch.Tensor] }\OperatorTok{=} \VariableTok{None}\NormalTok{, }\OperatorTok{**}\NormalTok{keyargs) }\OperatorTok{{-}\textgreater{}}\NormalTok{ torch.Tensor:}
        \CommentTok{"""}
\CommentTok{        {-} tokens: Optional[torch.Tensor], 输入 token 张量。}
\CommentTok{        {-} targets: Optional[torch.Tensor], 目标 token 张量。}
\CommentTok{        {-} kv\_cache: bool, 是否使用键值缓存。}
\CommentTok{        {-} keyargs: 其他关键字参数。}

\CommentTok{        {-} self.OUT: CausalLMOutputWithPast, 包含 logits 和损失。}
\CommentTok{        """}

        \ControlFlowTok{if} \StringTok{\textquotesingle{}input\_ids\textquotesingle{}} \KeywordTok{in}\NormalTok{ keyargs:}
\NormalTok{            tokens }\OperatorTok{=}\NormalTok{ keyargs[}\StringTok{\textquotesingle{}input\_ids\textquotesingle{}}\NormalTok{]}
        \ControlFlowTok{if} \StringTok{\textquotesingle{}attention\_mask\textquotesingle{}} \KeywordTok{in}\NormalTok{ keyargs:}
\NormalTok{            targets }\OperatorTok{=}\NormalTok{ keyargs[}\StringTok{\textquotesingle{}attention\_mask\textquotesingle{}}\NormalTok{]}

        \CommentTok{\# 前向传播函数}
\NormalTok{        \_bsz, seqlen }\OperatorTok{=}\NormalTok{ tokens.shape}
        \CommentTok{\# 通过词嵌入层和Dropout层}
\NormalTok{        h }\OperatorTok{=} \VariableTok{self}\NormalTok{.tok\_embeddings(tokens)}
\NormalTok{        h }\OperatorTok{=} \VariableTok{self}\NormalTok{.dropout(h)}
        \CommentTok{\# 获取相对位置嵌入的频率}
\NormalTok{        freqs\_cos }\OperatorTok{=} \VariableTok{self}\NormalTok{.freqs\_cos[:seqlen]}
\NormalTok{        freqs\_sin }\OperatorTok{=} \VariableTok{self}\NormalTok{.freqs\_sin[:seqlen]}

        \CommentTok{\# 通过Decoder层}
        \ControlFlowTok{for}\NormalTok{ layer }\KeywordTok{in} \VariableTok{self}\NormalTok{.layers:}
\NormalTok{            h }\OperatorTok{=}\NormalTok{ layer(h, freqs\_cos, freqs\_sin)}
        \CommentTok{\# 通过归一化层}
\NormalTok{        h }\OperatorTok{=} \VariableTok{self}\NormalTok{.norm(h)}

        \ControlFlowTok{if}\NormalTok{ targets }\KeywordTok{is} \KeywordTok{not} \VariableTok{None}\NormalTok{:}
            \CommentTok{\# 如果给定了目标,计算损失}
\NormalTok{            logits }\OperatorTok{=} \VariableTok{self}\NormalTok{.output(h)}
            \VariableTok{self}\NormalTok{.last\_loss }\OperatorTok{=}\NormalTok{ F.cross\_entropy(logits.view(}\OperatorTok{{-}}\DecValTok{1}\NormalTok{, logits.size(}\OperatorTok{{-}}\DecValTok{1}\NormalTok{)), targets.view(}\OperatorTok{{-}}\DecValTok{1}\NormalTok{), ignore\_index}\OperatorTok{=}\DecValTok{0}\NormalTok{, reduction}\OperatorTok{=}\StringTok{\textquotesingle{}none\textquotesingle{}}\NormalTok{)}
        \ControlFlowTok{else}\NormalTok{:}
            \CommentTok{\# 推理时的小优化:只对最后一个位置的输出进行前向传播}
\NormalTok{            logits }\OperatorTok{=} \VariableTok{self}\NormalTok{.output(h[:, [}\OperatorTok{{-}}\DecValTok{1}\NormalTok{], :]) }
            \VariableTok{self}\NormalTok{.last\_loss }\OperatorTok{=} \VariableTok{None}

        \CommentTok{\# 设置输出}
        \VariableTok{self}\NormalTok{.OUT.}\FunctionTok{\_\_setitem\_\_}\NormalTok{(}\StringTok{\textquotesingle{}logits\textquotesingle{}}\NormalTok{, logits)}
        \VariableTok{self}\NormalTok{.OUT.}\FunctionTok{\_\_setitem\_\_}\NormalTok{(}\StringTok{\textquotesingle{}last\_loss\textquotesingle{}}\NormalTok{, }\VariableTok{self}\NormalTok{.last\_loss)}
        \ControlFlowTok{return} \VariableTok{self}\NormalTok{.OUT}

    
    \AttributeTok{@torch.inference\_mode}\NormalTok{()}
    \KeywordTok{def}\NormalTok{ generate(}\VariableTok{self}\NormalTok{, idx, stop\_id}\OperatorTok{=}\VariableTok{None}\NormalTok{, max\_new\_tokens}\OperatorTok{=}\DecValTok{256}\NormalTok{, temperature}\OperatorTok{=}\FloatTok{1.0}\NormalTok{, top\_k}\OperatorTok{=}\VariableTok{None}\NormalTok{):}
        \CommentTok{"""}
\CommentTok{        给定输入序列 idx(形状为 (bz,seq\_len) 的长整型张量),通过多次生成新 token 来完成序列。}
\CommentTok{        在 model.eval() 模式下运行。效率较低的采样版本,没有使用键k/v cache。}
\CommentTok{        """}
\NormalTok{        index }\OperatorTok{=}\NormalTok{ idx.shape[}\DecValTok{1}\NormalTok{]}
        \ControlFlowTok{for}\NormalTok{ \_ }\KeywordTok{in} \BuiltInTok{range}\NormalTok{(max\_new\_tokens):}
            \CommentTok{\# 如果序列上下文过长,截断它到最大长度}
\NormalTok{            idx\_cond }\OperatorTok{=}\NormalTok{ idx }\ControlFlowTok{if}\NormalTok{ idx.size(}\DecValTok{1}\NormalTok{) }\OperatorTok{\textless{}=} \VariableTok{self}\NormalTok{.args.max\_seq\_len }\ControlFlowTok{else}\NormalTok{ idx[:, }\OperatorTok{{-}}\VariableTok{self}\NormalTok{.args.max\_seq\_len:]}
            
            \CommentTok{\# 前向传播获取序列中最后一个位置的 logits}
\NormalTok{            logits }\OperatorTok{=} \VariableTok{self}\NormalTok{(idx\_cond).logits}
\NormalTok{            logits }\OperatorTok{=}\NormalTok{ logits[:, }\OperatorTok{{-}}\DecValTok{1}\NormalTok{, :] }\CommentTok{\# 只保留最后一个时间步的输出}
            
            \ControlFlowTok{if}\NormalTok{ temperature }\OperatorTok{==} \FloatTok{0.0}\NormalTok{:}
                \CommentTok{\# 选择最有可能的索引}
\NormalTok{                \_, idx\_next }\OperatorTok{=}\NormalTok{ torch.topk(logits, k}\OperatorTok{=}\DecValTok{1}\NormalTok{, dim}\OperatorTok{={-}}\DecValTok{1}\NormalTok{)}
            \ControlFlowTok{else}\NormalTok{:}
                \CommentTok{\# 缩放 logits 并应用 softmax}
\NormalTok{                logits }\OperatorTok{=}\NormalTok{ logits }\OperatorTok{/}\NormalTok{ temperature}
                \ControlFlowTok{if}\NormalTok{ top\_k }\KeywordTok{is} \KeywordTok{not} \VariableTok{None}\NormalTok{:}
\NormalTok{                    v, \_ }\OperatorTok{=}\NormalTok{ torch.topk(logits, }\BuiltInTok{min}\NormalTok{(top\_k, logits.size(}\OperatorTok{{-}}\DecValTok{1}\NormalTok{)))}
\NormalTok{                    logits[logits }\OperatorTok{\textless{}}\NormalTok{ v[:, [}\OperatorTok{{-}}\DecValTok{1}\NormalTok{]]] }\OperatorTok{=} \OperatorTok{{-}}\BuiltInTok{float}\NormalTok{(}\StringTok{\textquotesingle{}Inf\textquotesingle{}}\NormalTok{)}
\NormalTok{                probs }\OperatorTok{=}\NormalTok{ F.softmax(logits, dim}\OperatorTok{={-}}\DecValTok{1}\NormalTok{)}
\NormalTok{                idx\_next }\OperatorTok{=}\NormalTok{ torch.multinomial(probs, num\_samples}\OperatorTok{=}\DecValTok{1}\NormalTok{)}
            

            \ControlFlowTok{if}\NormalTok{ idx\_next }\OperatorTok{==}\NormalTok{ stop\_id:}
                \ControlFlowTok{break}

            \CommentTok{\# 将采样的索引添加到序列中并继续}
\NormalTok{            idx }\OperatorTok{=}\NormalTok{ torch.cat((idx, idx\_next), dim}\OperatorTok{=}\DecValTok{1}\NormalTok{)}

        \ControlFlowTok{return}\NormalTok{ idx[:, index:] }\CommentTok{\# 只返回生成的token}
\end{Highlighting}
\end{Shaded}

同样大家可以使用下面的代码来对\texttt{Transformer}模块进行测试,可以看到代码最终输出的形状为\texttt{torch.Size({[}1,\ 1,\ 6144{]})},与我们输入的形状一致,说明模块的实现是正确的。

\begin{Shaded}
\begin{Highlighting}[]
\CommentTok{\# LLaMA2Model.forward 接受两个参数,tokens和targets,其中tokens是输入的张量, 应为int类型}
\NormalTok{x }\OperatorTok{=}\NormalTok{ torch.randint(}\DecValTok{0}\NormalTok{, }\DecValTok{6144}\NormalTok{, (}\DecValTok{1}\NormalTok{, }\DecValTok{50}\NormalTok{)) }\CommentTok{\# [bs, seq\_len]}
\CommentTok{\# 实例化LLaMA2Model}
\NormalTok{model }\OperatorTok{=}\NormalTok{ Transformer(args}\OperatorTok{=}\NormalTok{args)}
\CommentTok{\# 计算model的全部参数}
\NormalTok{num\_params }\OperatorTok{=} \BuiltInTok{sum}\NormalTok{(p.numel() }\ControlFlowTok{for}\NormalTok{ p }\KeywordTok{in}\NormalTok{ model.parameters())}
\BuiltInTok{print}\NormalTok{(}\StringTok{\textquotesingle{}Number of parameters:\textquotesingle{}}\NormalTok{, num\_params)}

\NormalTok{out }\OperatorTok{=}\NormalTok{ model(x)}
\BuiltInTok{print}\NormalTok{(out.logits.shape) }\CommentTok{\# [batch\_size, 1, vocab\_size]}
\end{Highlighting}
\end{Shaded}

OUT:

\begin{verbatim}
Number of parameters: 82594560
torch.Size([1, 1, 6144])
\end{verbatim}

\subsection{5.2 训练 Tokenizer}\label{ux8badux7ec3-tokenizer}

在自然语言处理 (NLP) 中,Tokenizer 是一种将文本分解为较小单位(称为
token)的工具。这些 token
可以是词、子词、字符,甚至是特定的符号。Tokenization 是 NLP
中的第一步,直接影响后续处理和分析的效果。不同类型的 tokenizer
适用于不同的应用场景,以下是几种常见的 tokenizer 及其特点。

\subsubsection{5.2.1 Word-based Tokenizer}\label{word-based-tokenizer}

\textbf{Word-based Tokenizer}
是最简单和直观的一种分词方法。它将文本按空格和标点符号分割成单词。这种方法的优点在于其简单和直接,易于实现,且与人类对语言的直觉相符。然而,它也存在一些明显的缺点,如无法处理未登录词(OOV,out-of-vocabulary)和罕见词,对复合词(如``New
York'')或缩略词(如``don't'')的处理也不够精细。此外,Word-based
Tokenizer
在处理不同语言时也会遇到挑战,因为一些语言(如中文、日文)没有显式的单词分隔符。

示例:

\begin{verbatim}
Input: "Hello, world! There is Datawhale."
Output: ["Hello", ",", "world", "!", "There", "is", "Datawhale", "."]
\end{verbatim}

在这个例子中,输入的句子被分割成一系列单词和标点符号,每个单词或标点符号都作为一个独立的
token。

\subsubsection{5.2.2 Character-based
Tokenizer}\label{character-based-tokenizer}

\textbf{Character-based Tokenizer} 将文本中的每个字符视为一个独立的
token。这种方法能非常精细地处理文本,适用于处理拼写错误、未登录词或新词。由于每个字符都是一个独立的
token,因此这种方法可以捕捉到非常细微的语言特征。这对于一些特定的应用场景,如生成式任务或需要处理大量未登录词的任务,特别有用。但是,这种方法也会导致
token
序列变得非常长,增加了模型的计算复杂度和训练时间。此外,字符级的分割可能会丢失一些词级别的语义信息,使得模型难以理解上下文。

示例:

\begin{verbatim}
Input: "Hello"
Output: ["H", "e", "l", "l", "o"]
\end{verbatim}

在这个例子中,单词``Hello''被分割成单个字符,每个字符作为一个独立的
token。这种方法能够处理任何语言和字符集,具有极大的灵活性。

\subsubsection{5.2.3 Subword Tokenizer}\label{subword-tokenizer}

\textbf{Subword Tokenizer}
介于词和字符之间,能够更好地平衡分词的细粒度和处理未登录词的能力。Subword
Tokenizer
的关键思想是将文本分割成比单词更小的单位,但又比字符更大,这样既能处理未知词,又能保持一定的语义信息。常见的子词分词方法包括
BPE、WordPiece 和 Unigram。

\paragraph{(1)Byte Pair Encoding (BPE)}\label{byte-pair-encoding-bpe}

\textbf{BPE}
是一种基于统计方法,通过反复合并频率最高的字符或字符序列对来生成子词词典。这种方法的优点在于其简单和高效,能够有效地处理未知词和罕见词,同时保持较低的词典大小。BPE
的合并过程是自底向上的,逐步将频率最高的字符对合并成新的子词,直到达到预定的词典大小或不再有高频的字符对。

示例:

\begin{verbatim}
Input: "lower"
Output: ["low", "er"]

Input: "newest"
Output: ["new", "est"]
\end{verbatim}

在这个例子中,单词``lower''被分割成子词``low''和``er'',而``newest''被分割成``new''和``est''。这种方法有效地处理了词干和词缀,保持了单词的基本语义结构。

\paragraph{(2)WordPiece}\label{wordpiece}

\textbf{WordPiece} 是另一种基于子词的分词方法,最初用于谷歌的 BERT
模型。与 BPE 类似,WordPiece
通过最大化子词序列的似然函数来生成词典,但在合并子词时更注重语言模型的优化。WordPiece
会优先选择能够最大化整体句子概率的子词,使得分词结果在语言模型中具有更高的概率。

示例:

\begin{verbatim}
Input: "unhappiness"
Output: ["un", "##happiness"]
\end{verbatim}

在这个例子中,单词``unhappiness''被分割成子词``un''和``\#\#happiness'',其中``\#\#''表示这是一个后缀子词。通过这种方式,WordPiece
能够更好地处理复合词和派生词,保留更多的语义信息。

\paragraph{(3)Unigram}\label{unigram}

\textbf{Unigram}
分词方法基于概率模型,通过选择具有最高概率的子词来分割文本。Unigram
词典是通过训练语言模型生成的,可以处理多种语言和不同类型的文本。Unigram
模型会为每个子词分配一个概率,然后根据这些概率进行最优分割。

示例:

\begin{verbatim}
Input: "unhappiness"
Output: ["un", "happiness"]

Input: "newest"
Output: ["new", "est"]
\end{verbatim}

在这个例子中,单词``unhappiness''被分割成子词``un''和``happiness'',而``newest''被分割成``new''和``est''。这种方法通过概率模型有效地处理了子词分割,使得分割结果更符合语言使用习惯。

每种 Tokenizer 方法都有其特定的应用场景和优缺点,选择适合的 Tokenizer
对于自然语言处理任务的成功至关重要。

\subsubsection{5.2.4 训练一个
Tokenizer}\label{ux8badux7ec3ux4e00ux4e2a-tokenizer}

这里我们选择使用 BPE 算法来训练一个 Subword Tokenizer。BPE
是一种简单而有效的分词方法,能够处理未登录词和罕见词,同时保持较小的词典大小。我们将使用
Hugging Face 的 \texttt{tokenizers} 库来训练一个 BPE Tokenizer。

\paragraph{Step 1:
安装和导入依赖库}\label{step-1-ux5b89ux88c5ux548cux5bfcux5165ux4f9dux8d56ux5e93}

首先,我们需要安装 \texttt{tokenizers} 库,除此之外还需要安装
\texttt{datasets} 和 \texttt{transformers}
库,用于加载训练数据和加载训练完成后的 Tokenizer。

\begin{Shaded}
\begin{Highlighting}[]
\ExtensionTok{pip}\NormalTok{ install tokenizers datasets transformers}
\end{Highlighting}
\end{Shaded}

然后,导入所需的库。

\begin{Shaded}
\begin{Highlighting}[]
\ImportTok{import}\NormalTok{ random}
\ImportTok{import}\NormalTok{ json}
\ImportTok{import}\NormalTok{ os}
\ImportTok{from}\NormalTok{ transformers }\ImportTok{import}\NormalTok{ AutoTokenizer, PreTrainedTokenizerFast}
\ImportTok{from}\NormalTok{ tokenizers }\ImportTok{import}\NormalTok{ (}
\NormalTok{    decoders,}
\NormalTok{    models,}
\NormalTok{    pre\_tokenizers,}
\NormalTok{    trainers,}
\NormalTok{    Tokenizer,}
\NormalTok{)}
\ImportTok{from}\NormalTok{ tokenizers.normalizers }\ImportTok{import}\NormalTok{ NFKC}
\ImportTok{from}\NormalTok{ typing }\ImportTok{import}\NormalTok{ Generator}
\end{Highlighting}
\end{Shaded}

\paragraph{Step 2:
加载训练数据}\label{step-2-ux52a0ux8f7dux8badux7ec3ux6570ux636e}

这里我们使用与预训练相同的数据集(出门问问序列猴子开源数据集)训练tokenizer,可使用\texttt{code/download\_dataset.sh}
和 \texttt{code/deal\_dataset.py} 下载和预处理数据集。

\begin{quote}
注:由于数据集过大,可能会导致在训练过程中内存不足。因为本项目为学习目的,建议学习者手动分割小部分数据集用于训练验证,笔者也在
Github 仓库中存放了训练好的 tokenizer,可以直接使用。
\end{quote}

\begin{Shaded}
\begin{Highlighting}[]
\KeywordTok{def}\NormalTok{ read\_texts\_from\_jsonl(file\_path: }\BuiltInTok{str}\NormalTok{) }\OperatorTok{{-}\textgreater{}}\NormalTok{ Generator[}\BuiltInTok{str}\NormalTok{, }\VariableTok{None}\NormalTok{, }\VariableTok{None}\NormalTok{]:}
    \CommentTok{"""读取JSONL文件并安全提取文本数据"""}
    \ControlFlowTok{with} \BuiltInTok{open}\NormalTok{(file\_path, }\StringTok{\textquotesingle{}r\textquotesingle{}}\NormalTok{, encoding}\OperatorTok{=}\StringTok{\textquotesingle{}utf{-}8\textquotesingle{}}\NormalTok{) }\ImportTok{as}\NormalTok{ f:}
        \ControlFlowTok{for}\NormalTok{ line\_num, line }\KeywordTok{in} \BuiltInTok{enumerate}\NormalTok{(f, }\DecValTok{1}\NormalTok{):}
            \ControlFlowTok{try}\NormalTok{:}
\NormalTok{                data }\OperatorTok{=}\NormalTok{ json.loads(line)}
                \ControlFlowTok{if} \StringTok{\textquotesingle{}text\textquotesingle{}} \KeywordTok{not} \KeywordTok{in}\NormalTok{ data:}
                    \ControlFlowTok{raise} \PreprocessorTok{KeyError}\NormalTok{(}\SpecialStringTok{f"Missing \textquotesingle{}text\textquotesingle{} field in line }\SpecialCharTok{\{}\NormalTok{line\_num}\SpecialCharTok{\}}\SpecialStringTok{"}\NormalTok{)}
                \ControlFlowTok{yield}\NormalTok{ data[}\StringTok{\textquotesingle{}text\textquotesingle{}}\NormalTok{]}
            \ControlFlowTok{except}\NormalTok{ json.JSONDecodeError:}
                \BuiltInTok{print}\NormalTok{(}\SpecialStringTok{f"Error decoding JSON in line }\SpecialCharTok{\{}\NormalTok{line\_num}\SpecialCharTok{\}}\SpecialStringTok{"}\NormalTok{)}
                \ControlFlowTok{continue}
            \ControlFlowTok{except} \PreprocessorTok{KeyError} \ImportTok{as}\NormalTok{ e:}
                \BuiltInTok{print}\NormalTok{(e)}
                \ControlFlowTok{continue}
\end{Highlighting}
\end{Shaded}

\paragraph{Step 3:
创建配置文件}\label{step-3-ux521bux5efaux914dux7f6eux6587ux4ef6}

在训练 BPE Tokenizer 之前,我们需要创建一个完整的 \texttt{Tokenizer}
配置文件,包括 \texttt{tokenizer\_config.json} 和
\texttt{special\_tokens\_map.json}。这些配置文件定义了
\texttt{Tokenizer} 的参数和特殊标记,用于训练和加载
\texttt{Tokenizer}。此处的\texttt{chat\_template}我们与\texttt{Qwen2.5}模型保持一致。

\begin{Shaded}
\begin{Highlighting}[]
\KeywordTok{def}\NormalTok{ create\_tokenizer\_config(save\_dir: }\BuiltInTok{str}\NormalTok{) }\OperatorTok{{-}\textgreater{}} \VariableTok{None}\NormalTok{:}
    \CommentTok{"""创建完整的tokenizer配置文件"""}
\NormalTok{    config }\OperatorTok{=}\NormalTok{ \{}
        \StringTok{"add\_bos\_token"}\NormalTok{: }\VariableTok{False}\NormalTok{,}
        \StringTok{"add\_eos\_token"}\NormalTok{: }\VariableTok{False}\NormalTok{,}
        \StringTok{"add\_prefix\_space"}\NormalTok{: }\VariableTok{False}\NormalTok{,}
        \StringTok{"bos\_token"}\NormalTok{: }\StringTok{"\textless{}|im\_start|\textgreater{}"}\NormalTok{,}
        \StringTok{"eos\_token"}\NormalTok{: }\StringTok{"\textless{}|im\_end|\textgreater{}"}\NormalTok{,}
        \StringTok{"pad\_token"}\NormalTok{: }\StringTok{"\textless{}|im\_end|\textgreater{}"}\NormalTok{,}
        \StringTok{"unk\_token"}\NormalTok{: }\StringTok{"\textless{}unk\textgreater{}"}\NormalTok{,}
        \StringTok{"model\_max\_length"}\NormalTok{: }\DecValTok{1000000000000000019884624838656}\NormalTok{,}
        \StringTok{"clean\_up\_tokenization\_spaces"}\NormalTok{: }\VariableTok{False}\NormalTok{,}
        \StringTok{"tokenizer\_class"}\NormalTok{: }\StringTok{"PreTrainedTokenizerFast"}\NormalTok{,}
        \StringTok{"chat\_template"}\NormalTok{: (}
            \StringTok{"\{}\SpecialCharTok{\% f}\StringTok{or message in messages \%\}"}
            \StringTok{"\{}\SpecialCharTok{\% i}\StringTok{f message[\textquotesingle{}role\textquotesingle{}] == \textquotesingle{}system\textquotesingle{} \%\}"}
            \StringTok{"\textless{}|im\_start|\textgreater{}system}\CharTok{\textbackslash{}n}\SpecialCharTok{\{\{}\StringTok{ message[\textquotesingle{}content\textquotesingle{}] }\SpecialCharTok{\}\}}\StringTok{\textless{}|im\_end|\textgreater{}}\CharTok{\textbackslash{}n}\StringTok{"}
            \StringTok{"\{}\SpecialCharTok{\% e}\StringTok{lif message[\textquotesingle{}role\textquotesingle{}] == \textquotesingle{}user\textquotesingle{} \%\}"}
            \StringTok{"\textless{}|im\_start|\textgreater{}user}\CharTok{\textbackslash{}n}\SpecialCharTok{\{\{}\StringTok{ message[\textquotesingle{}content\textquotesingle{}] }\SpecialCharTok{\}\}}\StringTok{\textless{}|im\_end|\textgreater{}}\CharTok{\textbackslash{}n}\StringTok{"}
            \StringTok{"\{}\SpecialCharTok{\% e}\StringTok{lif message[\textquotesingle{}role\textquotesingle{}] == \textquotesingle{}assistant\textquotesingle{} \%\}"}
            \StringTok{"\textless{}|im\_start|\textgreater{}assistant}\CharTok{\textbackslash{}n}\SpecialCharTok{\{\{}\StringTok{ message[\textquotesingle{}content\textquotesingle{}] }\SpecialCharTok{\}\}}\StringTok{\textless{}|im\_end|\textgreater{}}\CharTok{\textbackslash{}n}\StringTok{"}
            \StringTok{"\{}\SpecialCharTok{\% e}\StringTok{ndif \%\}"}
            \StringTok{"\{}\SpecialCharTok{\% e}\StringTok{ndfor \%\}"}
            \StringTok{"\{}\SpecialCharTok{\% i}\StringTok{f add\_generation\_prompt \%\}"}
            \StringTok{"}\SpecialCharTok{\{\{}\StringTok{ \textquotesingle{}\textless{}|im\_start|\textgreater{}assistant}\CharTok{\textbackslash{}n}\StringTok{\textquotesingle{} }\SpecialCharTok{\}\}}\StringTok{"}
            \StringTok{"\{}\SpecialCharTok{\% e}\StringTok{ndif \%\}"}
\NormalTok{        )}
\NormalTok{    \}}

    \CommentTok{\# 保存主配置文件}
    \ControlFlowTok{with} \BuiltInTok{open}\NormalTok{(os.path.join(save\_dir, }\StringTok{"tokenizer\_config.json"}\NormalTok{), }\StringTok{"w"}\NormalTok{, encoding}\OperatorTok{=}\StringTok{"utf{-}8"}\NormalTok{) }\ImportTok{as}\NormalTok{ f:}
\NormalTok{        json.dump(config, f, ensure\_ascii}\OperatorTok{=}\VariableTok{False}\NormalTok{, indent}\OperatorTok{=}\DecValTok{4}\NormalTok{)}

    \CommentTok{\# 创建special\_tokens\_map.json}
\NormalTok{    special\_tokens\_map }\OperatorTok{=}\NormalTok{ \{}
        \StringTok{"bos\_token"}\NormalTok{: }\StringTok{"\textless{}|im\_start|\textgreater{}"}\NormalTok{,}
        \StringTok{"eos\_token"}\NormalTok{: }\StringTok{"\textless{}|im\_end|\textgreater{}"}\NormalTok{,}
        \StringTok{"unk\_token"}\NormalTok{: }\StringTok{"\textless{}unk\textgreater{}"}\NormalTok{,}
        \StringTok{"pad\_token"}\NormalTok{: }\StringTok{"\textless{}|im\_end|\textgreater{}"}\NormalTok{,}
        \StringTok{"additional\_special\_tokens"}\NormalTok{: [}\StringTok{"\textless{}s\textgreater{}"}\NormalTok{, }\StringTok{"\textless{}/s\textgreater{}"}\NormalTok{]}
\NormalTok{    \}}
    \ControlFlowTok{with} \BuiltInTok{open}\NormalTok{(os.path.join(save\_dir, }\StringTok{"special\_tokens\_map.json"}\NormalTok{), }\StringTok{"w"}\NormalTok{, encoding}\OperatorTok{=}\StringTok{"utf{-}8"}\NormalTok{) }\ImportTok{as}\NormalTok{ f:}
\NormalTok{        json.dump(special\_tokens\_map, f, ensure\_ascii}\OperatorTok{=}\VariableTok{False}\NormalTok{, indent}\OperatorTok{=}\DecValTok{4}\NormalTok{)}
\end{Highlighting}
\end{Shaded}

\paragraph{Step 4: 训练 BPE
Tokenizer}\label{step-4-ux8badux7ec3-bpe-tokenizer}

在训练 BPE Tokenizer 之前,我们需要定义一个训练函数,用于训练 Tokenizer
并保存训练好的 Tokenizer 文件。这里我们使用 \texttt{tokenizers} 库中的
\texttt{Tokenizer} 类来训练 BPE Tokenizer。

可以看到我们在训练 Tokenizer 时,配置了一些特殊的 token,如
\texttt{\textless{}unk\textgreater{}}、\texttt{\textless{}s\textgreater{}}、\texttt{\textless{}/s\textgreater{}}、\texttt{\textless{}\textbar{}im\_start\textbar{}\textgreater{}}
和 \texttt{\textless{}\textbar{}im\_end\textbar{}\textgreater{}}。这些
token 用于标记未知词、句子的开始和结束,以及对话的开始和结束。这些特殊
token 可以帮助模型更好地理解文本数据,提高模型的泛化能力和效果。

\begin{Shaded}
\begin{Highlighting}[]
\KeywordTok{def}\NormalTok{ train\_tokenizer(data\_path: }\BuiltInTok{str}\NormalTok{, save\_dir: }\BuiltInTok{str}\NormalTok{, vocab\_size: }\BuiltInTok{int} \OperatorTok{=} \DecValTok{8192}\NormalTok{) }\OperatorTok{{-}\textgreater{}} \VariableTok{None}\NormalTok{:}
    \CommentTok{"""训练并保存自定义tokenizer"""}
\NormalTok{    os.makedirs(save\_dir, exist\_ok}\OperatorTok{=}\VariableTok{True}\NormalTok{)}
    
    \CommentTok{\# 初始化tokenizer}
\NormalTok{    tokenizer }\OperatorTok{=}\NormalTok{ Tokenizer(models.BPE(unk\_token}\OperatorTok{=}\StringTok{"\textless{}unk\textgreater{}"}\NormalTok{))}
\NormalTok{    tokenizer.normalizer }\OperatorTok{=}\NormalTok{ NFKC()  }\CommentTok{\# 添加文本规范化}
\NormalTok{    tokenizer.pre\_tokenizer }\OperatorTok{=}\NormalTok{ pre\_tokenizers.ByteLevel(add\_prefix\_space}\OperatorTok{=}\VariableTok{False}\NormalTok{)}
\NormalTok{    tokenizer.decoder }\OperatorTok{=}\NormalTok{ decoders.ByteLevel()}

    \CommentTok{\# 配置特殊token}
\NormalTok{    special\_tokens }\OperatorTok{=}\NormalTok{ [}
        \StringTok{"\textless{}unk\textgreater{}"}\NormalTok{, }
        \StringTok{"\textless{}s\textgreater{}"}\NormalTok{, }
        \StringTok{"\textless{}/s\textgreater{}"}\NormalTok{, }
        \StringTok{"\textless{}|im\_start|\textgreater{}"}\NormalTok{, }
        \StringTok{"\textless{}|im\_end|\textgreater{}"}
\NormalTok{    ]}

    \CommentTok{\# 配置训练器}
\NormalTok{    trainer }\OperatorTok{=}\NormalTok{ trainers.BpeTrainer(}
\NormalTok{        vocab\_size}\OperatorTok{=}\NormalTok{vocab\_size,}
\NormalTok{        special\_tokens}\OperatorTok{=}\NormalTok{special\_tokens,}
\NormalTok{        min\_frequency}\OperatorTok{=}\DecValTok{2}\NormalTok{,  }\CommentTok{\# 提高低频词过滤}
\NormalTok{        show\_progress}\OperatorTok{=}\VariableTok{True}\NormalTok{,}
\NormalTok{        initial\_alphabet}\OperatorTok{=}\NormalTok{pre\_tokenizers.ByteLevel.alphabet()}
\NormalTok{    )}

    \CommentTok{\# 训练tokenizer}
    \BuiltInTok{print}\NormalTok{(}\SpecialStringTok{f"Training tokenizer with data from }\SpecialCharTok{\{}\NormalTok{data\_path}\SpecialCharTok{\}}\SpecialStringTok{"}\NormalTok{)}
\NormalTok{    texts }\OperatorTok{=}\NormalTok{ read\_texts\_from\_jsonl(data\_path)}
\NormalTok{    tokenizer.train\_from\_iterator(texts, trainer}\OperatorTok{=}\NormalTok{trainer, length}\OperatorTok{=}\NormalTok{os.path.getsize(data\_path))}

    \CommentTok{\# 验证特殊token映射}
    \ControlFlowTok{try}\NormalTok{:}
        \ControlFlowTok{assert}\NormalTok{ tokenizer.token\_to\_id(}\StringTok{"\textless{}unk\textgreater{}"}\NormalTok{) }\OperatorTok{==} \DecValTok{0}
        \ControlFlowTok{assert}\NormalTok{ tokenizer.token\_to\_id(}\StringTok{"\textless{}s\textgreater{}"}\NormalTok{) }\OperatorTok{==} \DecValTok{1}
        \ControlFlowTok{assert}\NormalTok{ tokenizer.token\_to\_id(}\StringTok{"\textless{}/s\textgreater{}"}\NormalTok{) }\OperatorTok{==} \DecValTok{2}
        \ControlFlowTok{assert}\NormalTok{ tokenizer.token\_to\_id(}\StringTok{"\textless{}|im\_start|\textgreater{}"}\NormalTok{) }\OperatorTok{==} \DecValTok{3}
        \ControlFlowTok{assert}\NormalTok{ tokenizer.token\_to\_id(}\StringTok{"\textless{}|im\_end|\textgreater{}"}\NormalTok{) }\OperatorTok{==} \DecValTok{4}
    \ControlFlowTok{except} \PreprocessorTok{AssertionError} \ImportTok{as}\NormalTok{ e:}
        \BuiltInTok{print}\NormalTok{(}\StringTok{"Special tokens mapping error:"}\NormalTok{, e)}
        \ControlFlowTok{raise}

    \CommentTok{\# 保存tokenizer文件}
\NormalTok{    tokenizer.save(os.path.join(save\_dir, }\StringTok{"tokenizer.json"}\NormalTok{))}
    
    \CommentTok{\# 创建配置文件}
\NormalTok{    create\_tokenizer\_config(save\_dir)}
    \BuiltInTok{print}\NormalTok{(}\SpecialStringTok{f"Tokenizer saved to }\SpecialCharTok{\{}\NormalTok{save\_dir}\SpecialCharTok{\}}\SpecialStringTok{"}\NormalTok{)}
\end{Highlighting}
\end{Shaded}

\paragraph{Step 5: 使用训练好的
Tokenizer}\label{step-5-ux4f7fux7528ux8badux7ec3ux597dux7684-tokenizer}

我们可以使用训练好的 Tokenizer
来处理文本数据,如编码、解码、生成对话等。下面是一个简单的示例,展示了如何使用训练好的
Tokenizer 来处理文本数据。

\begin{Shaded}
\begin{Highlighting}[]
\KeywordTok{def}\NormalTok{ eval\_tokenizer(tokenizer\_path: }\BuiltInTok{str}\NormalTok{) }\OperatorTok{{-}\textgreater{}} \VariableTok{None}\NormalTok{:}
    \CommentTok{"""评估tokenizer功能"""}
    \ControlFlowTok{try}\NormalTok{:}
\NormalTok{        tokenizer }\OperatorTok{=}\NormalTok{ AutoTokenizer.from\_pretrained(tokenizer\_path)}
    \ControlFlowTok{except} \PreprocessorTok{Exception} \ImportTok{as}\NormalTok{ e:}
        \BuiltInTok{print}\NormalTok{(}\SpecialStringTok{f"Error loading tokenizer: }\SpecialCharTok{\{}\NormalTok{e}\SpecialCharTok{\}}\SpecialStringTok{"}\NormalTok{)}
        \ControlFlowTok{return}

    \CommentTok{\# 测试基本属性}
    \BuiltInTok{print}\NormalTok{(}\StringTok{"}\CharTok{\textbackslash{}n}\StringTok{=== Tokenizer基本信息 ==="}\NormalTok{)}
    \BuiltInTok{print}\NormalTok{(}\SpecialStringTok{f"Vocab size: }\SpecialCharTok{\{}\BuiltInTok{len}\NormalTok{(tokenizer)}\SpecialCharTok{\}}\SpecialStringTok{"}\NormalTok{)}
    \BuiltInTok{print}\NormalTok{(}\SpecialStringTok{f"Special tokens: }\SpecialCharTok{\{}\NormalTok{tokenizer}\SpecialCharTok{.}\NormalTok{all\_special\_tokens}\SpecialCharTok{\}}\SpecialStringTok{"}\NormalTok{)}
    \BuiltInTok{print}\NormalTok{(}\SpecialStringTok{f"Special token IDs: }\SpecialCharTok{\{}\NormalTok{tokenizer}\SpecialCharTok{.}\NormalTok{all\_special\_ids}\SpecialCharTok{\}}\SpecialStringTok{"}\NormalTok{)}

    \CommentTok{\# 测试聊天模板}
\NormalTok{    messages }\OperatorTok{=}\NormalTok{ [}
\NormalTok{        \{}\StringTok{"role"}\NormalTok{: }\StringTok{"system"}\NormalTok{, }\StringTok{"content"}\NormalTok{: }\StringTok{"你是一个AI助手。"}\NormalTok{\},}
\NormalTok{        \{}\StringTok{"role"}\NormalTok{: }\StringTok{"user"}\NormalTok{, }\StringTok{"content"}\NormalTok{: }\StringTok{"How are you?"}\NormalTok{\},}
\NormalTok{        \{}\StringTok{"role"}\NormalTok{: }\StringTok{"assistant"}\NormalTok{, }\StringTok{"content"}\NormalTok{: }\StringTok{"I\textquotesingle{}m fine, thank you. and you?"}\NormalTok{\},}
\NormalTok{        \{}\StringTok{"role"}\NormalTok{: }\StringTok{"user"}\NormalTok{, }\StringTok{"content"}\NormalTok{: }\StringTok{"I\textquotesingle{}m good too."}\NormalTok{\},}
\NormalTok{        \{}\StringTok{"role"}\NormalTok{: }\StringTok{"assistant"}\NormalTok{, }\StringTok{"content"}\NormalTok{: }\StringTok{"That\textquotesingle{}s great to hear!"}\NormalTok{\},}
\NormalTok{    ]}
    
    \BuiltInTok{print}\NormalTok{(}\StringTok{"}\CharTok{\textbackslash{}n}\StringTok{=== 聊天模板测试 ==="}\NormalTok{)}
\NormalTok{    prompt }\OperatorTok{=}\NormalTok{ tokenizer.apply\_chat\_template(}
\NormalTok{        messages, }
\NormalTok{        tokenize}\OperatorTok{=}\VariableTok{False}\NormalTok{, }
        \CommentTok{\# add\_generation\_prompt=True}
\NormalTok{    )}
    \BuiltInTok{print}\NormalTok{(}\StringTok{"Generated prompt:}\CharTok{\textbackslash{}n}\StringTok{"}\NormalTok{, prompt, sep}\OperatorTok{=}\StringTok{""}\NormalTok{)}

    \CommentTok{\# 测试编码解码}
    \BuiltInTok{print}\NormalTok{(}\StringTok{"}\CharTok{\textbackslash{}n}\StringTok{=== 编码解码测试 ==="}\NormalTok{)}
\NormalTok{    encoded }\OperatorTok{=}\NormalTok{ tokenizer(prompt, truncation}\OperatorTok{=}\VariableTok{True}\NormalTok{, max\_length}\OperatorTok{=}\DecValTok{256}\NormalTok{)}
\NormalTok{    decoded }\OperatorTok{=}\NormalTok{ tokenizer.decode(encoded[}\StringTok{"input\_ids"}\NormalTok{], skip\_special\_tokens}\OperatorTok{=}\VariableTok{False}\NormalTok{)}
    \BuiltInTok{print}\NormalTok{(}\StringTok{"Decoded text matches original:"}\NormalTok{, decoded }\OperatorTok{==}\NormalTok{ prompt)}

    \CommentTok{\# 测试特殊token处理}
    \BuiltInTok{print}\NormalTok{(}\StringTok{"}\CharTok{\textbackslash{}n}\StringTok{=== 特殊token处理 ==="}\NormalTok{)}
\NormalTok{    test\_text }\OperatorTok{=} \StringTok{"\textless{}|im\_start|\textgreater{}user}\CharTok{\textbackslash{}n}\StringTok{Hello\textless{}|im\_end|\textgreater{}"}
\NormalTok{    encoded }\OperatorTok{=}\NormalTok{ tokenizer(test\_text).input\_ids}
\NormalTok{    decoded }\OperatorTok{=}\NormalTok{ tokenizer.decode(encoded)}
    \BuiltInTok{print}\NormalTok{(}\SpecialStringTok{f"Original: }\SpecialCharTok{\{}\NormalTok{test\_text}\SpecialCharTok{\}}\SpecialStringTok{"}\NormalTok{)}
    \BuiltInTok{print}\NormalTok{(}\SpecialStringTok{f"Decoded:  }\SpecialCharTok{\{}\NormalTok{decoded}\SpecialCharTok{\}}\SpecialStringTok{"}\NormalTok{)}
    \BuiltInTok{print}\NormalTok{(}\StringTok{"Special tokens preserved:"}\NormalTok{, decoded }\OperatorTok{==}\NormalTok{ test\_text)}
\end{Highlighting}
\end{Shaded}

\begin{Shaded}
\begin{Highlighting}[]
\NormalTok{eval\_tokenizer(}\StringTok{\textquotesingle{}your tokenizer path\textquotesingle{}}\NormalTok{)}
\end{Highlighting}
\end{Shaded}

OUT:

\begin{verbatim}
=== Tokenizer基本信息 ===
Vocab size: 6144
Special tokens: ['<|im_start|>', '<|im_end|>', '<unk>', '<s>', '</s>']
Special token IDs: [3, 4, 0, 1, 2]

=== 聊天模板测试 ===
Generated prompt:
<|im_start|>system
你是一个AI助手。<|im_end|>
<|im_start|>user
How are you?<|im_end|>
<|im_start|>assistant
I'm fine, thank you. and you?<|im_end|>
<|im_start|>user
I'm good too.<|im_end|>
<|im_start|>assistant
That's great to hear!<|im_end|>


=== 编码解码测试 ===
Decoded text matches original: False

=== 特殊token处理 ===
Original: <|im_start|>user
Hello<|im_end|>
Decoded:  <|im_start|> user
Hello<|im_end|>
Special tokens preserved: False
\end{verbatim}

\subsection{5.3
预训练一个小型LLM}\label{ux9884ux8badux7ec3ux4e00ux4e2aux5c0fux578bllm}

在前面的章节中,我们熟悉了各种大模型的模型结构,以及如如何训练Tokenizer。在本节中,我们将动手训练一个八千万参数的LLM。

\subsubsection{5.3.1 数据下载}\label{ux6570ux636eux4e0bux8f7d}

首先,我们需要下载预训练数据集。在这里,我们使用两个开源的数据集,包含了大量的中文对话数据,可以用于训练对话生成模型。

\begin{itemize}
\item
  出门问问序列猴子开源数据集:出门问问序列猴子通用文本数据集由来自网页、百科、博客、问答、开源代码、书籍、报刊、专利、教材、考题等多种公开可获取的数据进行汇总清洗之后而形成的大语言模型预训练语料。总量大概在
  10B Token。
\item
  BelleGroup:350万条中文对话数据集,包含了人机对话、人人对话、人物对话等多种对话数据,可以用于训练对话生成模型。
\end{itemize}

\begin{Shaded}
\begin{Highlighting}[]
\CommentTok{\# 下载预训练数据集}
\NormalTok{os.system(}\StringTok{"modelscope download {-}{-}dataset ddzhu123/seq{-}monkey mobvoi\_seq\_monkey\_general\_open\_corpus.jsonl.tar.bz2 {-}{-}local\_dir your\_local\_dir"}\NormalTok{)}
\CommentTok{\# 解压预训练数据集}
\NormalTok{os.system(}\StringTok{"tar {-}xvf your\_local\_dir/mobvoi\_seq\_monkey\_general\_open\_corpus.jsonl.tar.bz2"}\NormalTok{)}

\CommentTok{\# 下载SFT数据集}
\NormalTok{os.system(}\SpecialStringTok{f\textquotesingle{}huggingface{-}cli download {-}{-}repo{-}type dataset {-}{-}resume{-}download BelleGroup/train\_3.5M\_CN {-}{-}local{-}dir BelleGroup\textquotesingle{}}\NormalTok{)}



\CommentTok{\# 1 处理预训练数据}
\KeywordTok{def}\NormalTok{ split\_text(text, chunk\_size}\OperatorTok{=}\DecValTok{512}\NormalTok{):}
    \CommentTok{"""将文本按指定长度切分成块"""}
    \ControlFlowTok{return}\NormalTok{ [text[i:i}\OperatorTok{+}\NormalTok{chunk\_size] }\ControlFlowTok{for}\NormalTok{ i }\KeywordTok{in} \BuiltInTok{range}\NormalTok{(}\DecValTok{0}\NormalTok{, }\BuiltInTok{len}\NormalTok{(text), chunk\_size)]}

\NormalTok{input\_file }\OperatorTok{=} \StringTok{\textquotesingle{}mobvoi\_seq\_monkey\_general\_open\_corpus.jsonl\textquotesingle{}}

\ControlFlowTok{with} \BuiltInTok{open}\NormalTok{(}\StringTok{\textquotesingle{}seq\_monkey\_datawhale.jsonl\textquotesingle{}}\NormalTok{, }\StringTok{\textquotesingle{}a\textquotesingle{}}\NormalTok{, encoding}\OperatorTok{=}\StringTok{\textquotesingle{}utf{-}8\textquotesingle{}}\NormalTok{) }\ImportTok{as}\NormalTok{ pretrain:}
    \ControlFlowTok{with} \BuiltInTok{open}\NormalTok{(input\_file, }\StringTok{\textquotesingle{}r\textquotesingle{}}\NormalTok{, encoding}\OperatorTok{=}\StringTok{\textquotesingle{}utf{-}8\textquotesingle{}}\NormalTok{) }\ImportTok{as}\NormalTok{ f:}
\NormalTok{        data }\OperatorTok{=}\NormalTok{ f.readlines()}
        \ControlFlowTok{for}\NormalTok{ line }\KeywordTok{in}\NormalTok{ tqdm(data, desc}\OperatorTok{=}\SpecialStringTok{f"Processing lines in }\SpecialCharTok{\{}\NormalTok{input\_file}\SpecialCharTok{\}}\SpecialStringTok{"}\NormalTok{, leave}\OperatorTok{=}\VariableTok{False}\NormalTok{):  }\CommentTok{\# 添加行级别的进度条}
\NormalTok{            line }\OperatorTok{=}\NormalTok{ json.loads(line)}
\NormalTok{            text }\OperatorTok{=}\NormalTok{ line[}\StringTok{\textquotesingle{}text\textquotesingle{}}\NormalTok{]}
\NormalTok{            chunks }\OperatorTok{=}\NormalTok{ split\_text(text)}
            \ControlFlowTok{for}\NormalTok{ chunk }\KeywordTok{in}\NormalTok{ chunks:}
\NormalTok{                pretrain.write(json.dumps(\{}\StringTok{\textquotesingle{}text\textquotesingle{}}\NormalTok{: chunk\}, ensure\_ascii}\OperatorTok{=}\VariableTok{False}\NormalTok{) }\OperatorTok{+} \StringTok{\textquotesingle{}}\CharTok{\textbackslash{}n}\StringTok{\textquotesingle{}}\NormalTok{)}

\CommentTok{\# 2 处理SFT数据}

\KeywordTok{def}\NormalTok{ convert\_message(data):}
    \CommentTok{"""}
\CommentTok{    将原始数据转换为标准格式}
\CommentTok{    """}
\NormalTok{    message }\OperatorTok{=}\NormalTok{ [}
\NormalTok{        \{}\StringTok{"role"}\NormalTok{: }\StringTok{"system"}\NormalTok{, }\StringTok{"content"}\NormalTok{: }\StringTok{"你是一个AI助手"}\NormalTok{\},}
\NormalTok{    ]}
    \ControlFlowTok{for}\NormalTok{ item }\KeywordTok{in}\NormalTok{ data:}
        \ControlFlowTok{if}\NormalTok{ item[}\StringTok{\textquotesingle{}from\textquotesingle{}}\NormalTok{] }\OperatorTok{==} \StringTok{\textquotesingle{}human\textquotesingle{}}\NormalTok{:}
\NormalTok{            message.append(\{}\StringTok{\textquotesingle{}role\textquotesingle{}}\NormalTok{: }\StringTok{\textquotesingle{}user\textquotesingle{}}\NormalTok{, }\StringTok{\textquotesingle{}content\textquotesingle{}}\NormalTok{: item[}\StringTok{\textquotesingle{}value\textquotesingle{}}\NormalTok{]\})}
        \ControlFlowTok{elif}\NormalTok{ item[}\StringTok{\textquotesingle{}from\textquotesingle{}}\NormalTok{] }\OperatorTok{==} \StringTok{\textquotesingle{}assistant\textquotesingle{}}\NormalTok{:}
\NormalTok{            message.append(\{}\StringTok{\textquotesingle{}role\textquotesingle{}}\NormalTok{: }\StringTok{\textquotesingle{}assistant\textquotesingle{}}\NormalTok{, }\StringTok{\textquotesingle{}content\textquotesingle{}}\NormalTok{: item[}\StringTok{\textquotesingle{}value\textquotesingle{}}\NormalTok{]\})}
    \ControlFlowTok{return}\NormalTok{ message}

\ControlFlowTok{with} \BuiltInTok{open}\NormalTok{(}\StringTok{\textquotesingle{}BelleGroup\_sft.jsonl\textquotesingle{}}\NormalTok{, }\StringTok{\textquotesingle{}a\textquotesingle{}}\NormalTok{, encoding}\OperatorTok{=}\StringTok{\textquotesingle{}utf{-}8\textquotesingle{}}\NormalTok{) }\ImportTok{as}\NormalTok{ sft:}
    \ControlFlowTok{with} \BuiltInTok{open}\NormalTok{(}\StringTok{\textquotesingle{}BelleGroup/train\_3.5M\_CN.json\textquotesingle{}}\NormalTok{, }\StringTok{\textquotesingle{}r\textquotesingle{}}\NormalTok{) }\ImportTok{as}\NormalTok{ f:}
\NormalTok{        data }\OperatorTok{=}\NormalTok{ f.readlines()}
        \ControlFlowTok{for}\NormalTok{ item }\KeywordTok{in}\NormalTok{ tqdm(data, desc}\OperatorTok{=}\StringTok{"Processing"}\NormalTok{, unit}\OperatorTok{=}\StringTok{"lines"}\NormalTok{):}
\NormalTok{            item }\OperatorTok{=}\NormalTok{ json.loads(item)}
\NormalTok{            message }\OperatorTok{=}\NormalTok{ convert\_message(item[}\StringTok{\textquotesingle{}conversations\textquotesingle{}}\NormalTok{])}
\NormalTok{            sft.write(json.dumps(message, ensure\_ascii}\OperatorTok{=}\VariableTok{False}\NormalTok{) }\OperatorTok{+} \StringTok{\textquotesingle{}}\CharTok{\textbackslash{}n}\StringTok{\textquotesingle{}}\NormalTok{)}
\end{Highlighting}
\end{Shaded}

\subsubsection{5.3.2 训练 Tokenizer}\label{ux8badux7ec3-tokenizer-1}

首先,我们需要为文本处理训练一个Tokenizer。Tokenizer的作用是将文本转换为数字序列,以便模型能够理解和处理。我们使用的数据集是
\href{https://www.modelscope.cn/datasets/ddzhu123/seq-monkey/files}{出门问问序列猴子开源数据集}
,这个数据集包含了大量的中文文本数据,可以用于训练Tokenizer。

\begin{quote}
注:由于数据集较大,如果大家在自己本地电脑训练的话进度比较慢,所以在这里我们提供了一个已经训练好的Tokenizer,大家可以直接使用。如果大家想要自己训练的话,可以参考下面的代码。
\end{quote}

\begin{Shaded}
\begin{Highlighting}[]
\ExtensionTok{python}\NormalTok{ code/train\_tokenizer.py}
\end{Highlighting}
\end{Shaded}

\begin{Shaded}
\begin{Highlighting}[]
\ImportTok{import}\NormalTok{ random}
\ImportTok{import}\NormalTok{ json}
\ImportTok{import}\NormalTok{ os}
\ImportTok{from}\NormalTok{ transformers }\ImportTok{import}\NormalTok{ AutoTokenizer, PreTrainedTokenizerFast}
\ImportTok{from}\NormalTok{ tokenizers }\ImportTok{import}\NormalTok{ (}
\NormalTok{    decoders,}
\NormalTok{    models,}
\NormalTok{    pre\_tokenizers,}
\NormalTok{    trainers,}
\NormalTok{    Tokenizer,}
\NormalTok{)}
\ImportTok{from}\NormalTok{ tokenizers.normalizers }\ImportTok{import}\NormalTok{ NFKC}
\ImportTok{from}\NormalTok{ typing }\ImportTok{import}\NormalTok{ Generator}

\NormalTok{random.seed(}\DecValTok{42}\NormalTok{)}

\KeywordTok{def}\NormalTok{ read\_texts\_from\_jsonl(file\_path: }\BuiltInTok{str}\NormalTok{) }\OperatorTok{{-}\textgreater{}}\NormalTok{ Generator[}\BuiltInTok{str}\NormalTok{, }\VariableTok{None}\NormalTok{, }\VariableTok{None}\NormalTok{]:}
    \CommentTok{"""读取JSONL文件并安全提取文本数据"""}
    \ControlFlowTok{with} \BuiltInTok{open}\NormalTok{(file\_path, }\StringTok{\textquotesingle{}r\textquotesingle{}}\NormalTok{, encoding}\OperatorTok{=}\StringTok{\textquotesingle{}utf{-}8\textquotesingle{}}\NormalTok{) }\ImportTok{as}\NormalTok{ f:}
        \ControlFlowTok{for}\NormalTok{ line\_num, line }\KeywordTok{in} \BuiltInTok{enumerate}\NormalTok{(f, }\DecValTok{1}\NormalTok{):}
            \ControlFlowTok{try}\NormalTok{:}
\NormalTok{                data }\OperatorTok{=}\NormalTok{ json.loads(line)}
                \ControlFlowTok{if} \StringTok{\textquotesingle{}text\textquotesingle{}} \KeywordTok{not} \KeywordTok{in}\NormalTok{ data:}
                    \ControlFlowTok{raise} \PreprocessorTok{KeyError}\NormalTok{(}\SpecialStringTok{f"Missing \textquotesingle{}text\textquotesingle{} field in line }\SpecialCharTok{\{}\NormalTok{line\_num}\SpecialCharTok{\}}\SpecialStringTok{"}\NormalTok{)}
                \ControlFlowTok{yield}\NormalTok{ data[}\StringTok{\textquotesingle{}text\textquotesingle{}}\NormalTok{]}
            \ControlFlowTok{except}\NormalTok{ json.JSONDecodeError:}
                \BuiltInTok{print}\NormalTok{(}\SpecialStringTok{f"Error decoding JSON in line }\SpecialCharTok{\{}\NormalTok{line\_num}\SpecialCharTok{\}}\SpecialStringTok{"}\NormalTok{)}
                \ControlFlowTok{continue}
            \ControlFlowTok{except} \PreprocessorTok{KeyError} \ImportTok{as}\NormalTok{ e:}
                \BuiltInTok{print}\NormalTok{(e)}
                \ControlFlowTok{continue}

\KeywordTok{def}\NormalTok{ create\_tokenizer\_config(save\_dir: }\BuiltInTok{str}\NormalTok{) }\OperatorTok{{-}\textgreater{}} \VariableTok{None}\NormalTok{:}
    \CommentTok{"""创建完整的tokenizer配置文件"""}
\NormalTok{    config }\OperatorTok{=}\NormalTok{ \{}
        \StringTok{"add\_bos\_token"}\NormalTok{: }\VariableTok{False}\NormalTok{,}
        \StringTok{"add\_eos\_token"}\NormalTok{: }\VariableTok{False}\NormalTok{,}
        \StringTok{"add\_prefix\_space"}\NormalTok{: }\VariableTok{True}\NormalTok{,}
        \StringTok{"bos\_token"}\NormalTok{: }\StringTok{"\textless{}|im\_start|\textgreater{}"}\NormalTok{,}
        \StringTok{"eos\_token"}\NormalTok{: }\StringTok{"\textless{}|im\_end|\textgreater{}"}\NormalTok{,}
        \StringTok{"pad\_token"}\NormalTok{: }\StringTok{"\textless{}|im\_end|\textgreater{}"}\NormalTok{,}
        \StringTok{"unk\_token"}\NormalTok{: }\StringTok{"\textless{}unk\textgreater{}"}\NormalTok{,}
        \StringTok{"model\_max\_length"}\NormalTok{: }\DecValTok{1000000000000000019884624838656}\NormalTok{,}
        \StringTok{"clean\_up\_tokenization\_spaces"}\NormalTok{: }\VariableTok{False}\NormalTok{,}
        \StringTok{"tokenizer\_class"}\NormalTok{: }\StringTok{"PreTrainedTokenizerFast"}\NormalTok{,}
        \StringTok{"chat\_template"}\NormalTok{: (}
            \StringTok{"\{}\SpecialCharTok{\% f}\StringTok{or message in messages \%\}"}
            \StringTok{"\{}\SpecialCharTok{\% i}\StringTok{f message[\textquotesingle{}role\textquotesingle{}] == \textquotesingle{}system\textquotesingle{} \%\}"}
            \StringTok{"\textless{}|im\_start|\textgreater{}system}\CharTok{\textbackslash{}n}\SpecialCharTok{\{\{}\StringTok{ message[\textquotesingle{}content\textquotesingle{}] }\SpecialCharTok{\}\}}\StringTok{\textless{}|im\_end|\textgreater{}}\CharTok{\textbackslash{}n}\StringTok{"}
            \StringTok{"\{}\SpecialCharTok{\% e}\StringTok{lif message[\textquotesingle{}role\textquotesingle{}] == \textquotesingle{}user\textquotesingle{} \%\}"}
            \StringTok{"\textless{}|im\_start|\textgreater{}user}\CharTok{\textbackslash{}n}\SpecialCharTok{\{\{}\StringTok{ message[\textquotesingle{}content\textquotesingle{}] }\SpecialCharTok{\}\}}\StringTok{\textless{}|im\_end|\textgreater{}}\CharTok{\textbackslash{}n}\StringTok{"}
            \StringTok{"\{}\SpecialCharTok{\% e}\StringTok{lif message[\textquotesingle{}role\textquotesingle{}] == \textquotesingle{}assistant\textquotesingle{} \%\}"}
            \StringTok{"\textless{}|im\_start|\textgreater{}assistant}\CharTok{\textbackslash{}n}\SpecialCharTok{\{\{}\StringTok{ message[\textquotesingle{}content\textquotesingle{}] }\SpecialCharTok{\}\}}\StringTok{\textless{}|im\_end|\textgreater{}}\CharTok{\textbackslash{}n}\StringTok{"}
            \StringTok{"\{}\SpecialCharTok{\% e}\StringTok{ndif \%\}"}
            \StringTok{"\{}\SpecialCharTok{\% e}\StringTok{ndfor \%\}"}
            \StringTok{"\{}\SpecialCharTok{\% i}\StringTok{f add\_generation\_prompt \%\}"}
            \StringTok{"}\SpecialCharTok{\{\{}\StringTok{ \textquotesingle{}\textless{}|im\_start|\textgreater{}assistant}\CharTok{\textbackslash{}n}\StringTok{\textquotesingle{} }\SpecialCharTok{\}\}}\StringTok{"}
            \StringTok{"\{}\SpecialCharTok{\% e}\StringTok{ndif \%\}"}
\NormalTok{        )}
\NormalTok{    \}}

    \CommentTok{\# 保存主配置文件}
    \ControlFlowTok{with} \BuiltInTok{open}\NormalTok{(os.path.join(save\_dir, }\StringTok{"tokenizer\_config.json"}\NormalTok{), }\StringTok{"w"}\NormalTok{, encoding}\OperatorTok{=}\StringTok{"utf{-}8"}\NormalTok{) }\ImportTok{as}\NormalTok{ f:}
\NormalTok{        json.dump(config, f, ensure\_ascii}\OperatorTok{=}\VariableTok{False}\NormalTok{, indent}\OperatorTok{=}\DecValTok{4}\NormalTok{)}

    \CommentTok{\# 创建special\_tokens\_map.json}
\NormalTok{    special\_tokens\_map }\OperatorTok{=}\NormalTok{ \{}
        \StringTok{"bos\_token"}\NormalTok{: }\StringTok{"\textless{}|im\_start|\textgreater{}"}\NormalTok{,}
        \StringTok{"eos\_token"}\NormalTok{: }\StringTok{"\textless{}|im\_end|\textgreater{}"}\NormalTok{,}
        \StringTok{"unk\_token"}\NormalTok{: }\StringTok{"\textless{}unk\textgreater{}"}\NormalTok{,}
        \StringTok{"pad\_token"}\NormalTok{: }\StringTok{"\textless{}|im\_end|\textgreater{}"}\NormalTok{,}
        \StringTok{"additional\_special\_tokens"}\NormalTok{: [}\StringTok{"\textless{}s\textgreater{}"}\NormalTok{, }\StringTok{"\textless{}/s\textgreater{}"}\NormalTok{]}
\NormalTok{    \}}
    \ControlFlowTok{with} \BuiltInTok{open}\NormalTok{(os.path.join(save\_dir, }\StringTok{"special\_tokens\_map.json"}\NormalTok{), }\StringTok{"w"}\NormalTok{, encoding}\OperatorTok{=}\StringTok{"utf{-}8"}\NormalTok{) }\ImportTok{as}\NormalTok{ f:}
\NormalTok{        json.dump(special\_tokens\_map, f, ensure\_ascii}\OperatorTok{=}\VariableTok{False}\NormalTok{, indent}\OperatorTok{=}\DecValTok{4}\NormalTok{)}

\KeywordTok{def}\NormalTok{ train\_tokenizer(data\_path: }\BuiltInTok{str}\NormalTok{, save\_dir: }\BuiltInTok{str}\NormalTok{, vocab\_size: }\BuiltInTok{int} \OperatorTok{=} \DecValTok{8192}\NormalTok{) }\OperatorTok{{-}\textgreater{}} \VariableTok{None}\NormalTok{:}
    \CommentTok{"""训练并保存自定义tokenizer"""}
\NormalTok{    os.makedirs(save\_dir, exist\_ok}\OperatorTok{=}\VariableTok{True}\NormalTok{)}
    
    \CommentTok{\# 初始化tokenizer}
\NormalTok{    tokenizer }\OperatorTok{=}\NormalTok{ Tokenizer(models.BPE(unk\_token}\OperatorTok{=}\StringTok{"\textless{}unk\textgreater{}"}\NormalTok{))}
\NormalTok{    tokenizer.normalizer }\OperatorTok{=}\NormalTok{ NFKC()  }\CommentTok{\# 添加文本规范化}
\NormalTok{    tokenizer.pre\_tokenizer }\OperatorTok{=}\NormalTok{ pre\_tokenizers.ByteLevel(add\_prefix\_space}\OperatorTok{=}\VariableTok{False}\NormalTok{)}
\NormalTok{    tokenizer.decoder }\OperatorTok{=}\NormalTok{ decoders.ByteLevel()}

    \CommentTok{\# 配置特殊token}
\NormalTok{    special\_tokens }\OperatorTok{=}\NormalTok{ [}
        \StringTok{"\textless{}unk\textgreater{}"}\NormalTok{, }
        \StringTok{"\textless{}s\textgreater{}"}\NormalTok{, }
        \StringTok{"\textless{}/s\textgreater{}"}\NormalTok{, }
        \StringTok{"\textless{}|im\_start|\textgreater{}"}\NormalTok{, }
        \StringTok{"\textless{}|im\_end|\textgreater{}"}
\NormalTok{    ]}

    \CommentTok{\# 配置训练器}
\NormalTok{    trainer }\OperatorTok{=}\NormalTok{ trainers.BpeTrainer(}
\NormalTok{        vocab\_size}\OperatorTok{=}\NormalTok{vocab\_size,}
\NormalTok{        special\_tokens}\OperatorTok{=}\NormalTok{special\_tokens,}
\NormalTok{        min\_frequency}\OperatorTok{=}\DecValTok{2}\NormalTok{,  }\CommentTok{\# 提高低频词过滤}
\NormalTok{        show\_progress}\OperatorTok{=}\VariableTok{True}\NormalTok{,}
\NormalTok{        initial\_alphabet}\OperatorTok{=}\NormalTok{pre\_tokenizers.ByteLevel.alphabet()}
\NormalTok{    )}

    \CommentTok{\# 训练tokenizer}
    \BuiltInTok{print}\NormalTok{(}\SpecialStringTok{f"Training tokenizer with data from }\SpecialCharTok{\{}\NormalTok{data\_path}\SpecialCharTok{\}}\SpecialStringTok{"}\NormalTok{)}
\NormalTok{    texts }\OperatorTok{=}\NormalTok{ read\_texts\_from\_jsonl(data\_path)}
\NormalTok{    tokenizer.train\_from\_iterator(texts, trainer}\OperatorTok{=}\NormalTok{trainer, length}\OperatorTok{=}\NormalTok{os.path.getsize(data\_path))}

    \CommentTok{\# 验证特殊token映射}
    \ControlFlowTok{try}\NormalTok{:}
        \ControlFlowTok{assert}\NormalTok{ tokenizer.token\_to\_id(}\StringTok{"\textless{}unk\textgreater{}"}\NormalTok{) }\OperatorTok{==} \DecValTok{0}
        \ControlFlowTok{assert}\NormalTok{ tokenizer.token\_to\_id(}\StringTok{"\textless{}s\textgreater{}"}\NormalTok{) }\OperatorTok{==} \DecValTok{1}
        \ControlFlowTok{assert}\NormalTok{ tokenizer.token\_to\_id(}\StringTok{"\textless{}/s\textgreater{}"}\NormalTok{) }\OperatorTok{==} \DecValTok{2}
        \ControlFlowTok{assert}\NormalTok{ tokenizer.token\_to\_id(}\StringTok{"\textless{}|im\_start|\textgreater{}"}\NormalTok{) }\OperatorTok{==} \DecValTok{3}
        \ControlFlowTok{assert}\NormalTok{ tokenizer.token\_to\_id(}\StringTok{"\textless{}|im\_end|\textgreater{}"}\NormalTok{) }\OperatorTok{==} \DecValTok{4}
    \ControlFlowTok{except} \PreprocessorTok{AssertionError} \ImportTok{as}\NormalTok{ e:}
        \BuiltInTok{print}\NormalTok{(}\StringTok{"Special tokens mapping error:"}\NormalTok{, e)}
        \ControlFlowTok{raise}

    \CommentTok{\# 保存tokenizer文件}
\NormalTok{    tokenizer.save(os.path.join(save\_dir, }\StringTok{"tokenizer.json"}\NormalTok{))}
    
    \CommentTok{\# 创建配置文件}
\NormalTok{    create\_tokenizer\_config(save\_dir)}
    \BuiltInTok{print}\NormalTok{(}\SpecialStringTok{f"Tokenizer saved to }\SpecialCharTok{\{}\NormalTok{save\_dir}\SpecialCharTok{\}}\SpecialStringTok{"}\NormalTok{)}

\KeywordTok{def}\NormalTok{ eval\_tokenizer(tokenizer\_path: }\BuiltInTok{str}\NormalTok{) }\OperatorTok{{-}\textgreater{}} \VariableTok{None}\NormalTok{:}
    \CommentTok{"""评估tokenizer功能"""}
    \ControlFlowTok{try}\NormalTok{:}
\NormalTok{        tokenizer }\OperatorTok{=}\NormalTok{ AutoTokenizer.from\_pretrained(tokenizer\_path)}
    \ControlFlowTok{except} \PreprocessorTok{Exception} \ImportTok{as}\NormalTok{ e:}
        \BuiltInTok{print}\NormalTok{(}\SpecialStringTok{f"Error loading tokenizer: }\SpecialCharTok{\{}\NormalTok{e}\SpecialCharTok{\}}\SpecialStringTok{"}\NormalTok{)}
        \ControlFlowTok{return}

    \CommentTok{\# 测试基本属性}
    \BuiltInTok{print}\NormalTok{(}\StringTok{"}\CharTok{\textbackslash{}n}\StringTok{=== Tokenizer基本信息 ==="}\NormalTok{)}
    \BuiltInTok{print}\NormalTok{(}\SpecialStringTok{f"Vocab size: }\SpecialCharTok{\{}\BuiltInTok{len}\NormalTok{(tokenizer)}\SpecialCharTok{\}}\SpecialStringTok{"}\NormalTok{)}
    \BuiltInTok{print}\NormalTok{(}\SpecialStringTok{f"Special tokens: }\SpecialCharTok{\{}\NormalTok{tokenizer}\SpecialCharTok{.}\NormalTok{all\_special\_tokens}\SpecialCharTok{\}}\SpecialStringTok{"}\NormalTok{)}
    \BuiltInTok{print}\NormalTok{(}\SpecialStringTok{f"Special token IDs: }\SpecialCharTok{\{}\NormalTok{tokenizer}\SpecialCharTok{.}\NormalTok{all\_special\_ids}\SpecialCharTok{\}}\SpecialStringTok{"}\NormalTok{)}

    \CommentTok{\# 测试聊天模板}
\NormalTok{    messages }\OperatorTok{=}\NormalTok{ [}
\NormalTok{        \{}\StringTok{"role"}\NormalTok{: }\StringTok{"system"}\NormalTok{, }\StringTok{"content"}\NormalTok{: }\StringTok{"你是一个AI助手。"}\NormalTok{\},}
\NormalTok{        \{}\StringTok{"role"}\NormalTok{: }\StringTok{"user"}\NormalTok{, }\StringTok{"content"}\NormalTok{: }\StringTok{"How are you?"}\NormalTok{\},}
\NormalTok{        \{}\StringTok{"role"}\NormalTok{: }\StringTok{"assistant"}\NormalTok{, }\StringTok{"content"}\NormalTok{: }\StringTok{"I\textquotesingle{}m fine, thank you. and you?"}\NormalTok{\},}
\NormalTok{        \{}\StringTok{"role"}\NormalTok{: }\StringTok{"user"}\NormalTok{, }\StringTok{"content"}\NormalTok{: }\StringTok{"I\textquotesingle{}m good too."}\NormalTok{\},}
\NormalTok{        \{}\StringTok{"role"}\NormalTok{: }\StringTok{"assistant"}\NormalTok{, }\StringTok{"content"}\NormalTok{: }\StringTok{"That\textquotesingle{}s great to hear!"}\NormalTok{\},}
\NormalTok{    ]}
    
    \BuiltInTok{print}\NormalTok{(}\StringTok{"}\CharTok{\textbackslash{}n}\StringTok{=== 聊天模板测试 ==="}\NormalTok{)}
\NormalTok{    prompt }\OperatorTok{=}\NormalTok{ tokenizer.apply\_chat\_template(}
\NormalTok{        messages, }
\NormalTok{        tokenize}\OperatorTok{=}\VariableTok{False}\NormalTok{, }
        \CommentTok{\# add\_generation\_prompt=True}
\NormalTok{    )}
    \BuiltInTok{print}\NormalTok{(}\StringTok{"Generated prompt:}\CharTok{\textbackslash{}n}\StringTok{"}\NormalTok{, prompt, sep}\OperatorTok{=}\StringTok{""}\NormalTok{)}

    \CommentTok{\# 测试编码解码}
    \BuiltInTok{print}\NormalTok{(}\StringTok{"}\CharTok{\textbackslash{}n}\StringTok{=== 编码解码测试 ==="}\NormalTok{)}
\NormalTok{    encoded }\OperatorTok{=}\NormalTok{ tokenizer(prompt, truncation}\OperatorTok{=}\VariableTok{True}\NormalTok{, max\_length}\OperatorTok{=}\DecValTok{256}\NormalTok{)}
\NormalTok{    decoded }\OperatorTok{=}\NormalTok{ tokenizer.decode(encoded[}\StringTok{"input\_ids"}\NormalTok{], skip\_special\_tokens}\OperatorTok{=}\VariableTok{False}\NormalTok{)}
    \BuiltInTok{print}\NormalTok{(}\StringTok{"Decoded text matches original:"}\NormalTok{, decoded }\OperatorTok{==}\NormalTok{ prompt)}

    \CommentTok{\# 测试特殊token处理}
    \BuiltInTok{print}\NormalTok{(}\StringTok{"}\CharTok{\textbackslash{}n}\StringTok{=== 特殊token处理 ==="}\NormalTok{)}
\NormalTok{    test\_text }\OperatorTok{=} \StringTok{"\textless{}|im\_start|\textgreater{}user}\CharTok{\textbackslash{}n}\StringTok{Hello\textless{}|im\_end|\textgreater{}"}
\NormalTok{    encoded }\OperatorTok{=}\NormalTok{ tokenizer(test\_text).input\_ids}
\NormalTok{    decoded }\OperatorTok{=}\NormalTok{ tokenizer.decode(encoded)}
    \BuiltInTok{print}\NormalTok{(}\SpecialStringTok{f"Original: }\SpecialCharTok{\{}\NormalTok{test\_text}\SpecialCharTok{\}}\SpecialStringTok{"}\NormalTok{)}
    \BuiltInTok{print}\NormalTok{(}\SpecialStringTok{f"Decoded:  }\SpecialCharTok{\{}\NormalTok{decoded}\SpecialCharTok{\}}\SpecialStringTok{"}\NormalTok{)}
    \BuiltInTok{print}\NormalTok{(}\StringTok{"Special tokens preserved:"}\NormalTok{, decoded }\OperatorTok{==}\NormalTok{ test\_text)}

\KeywordTok{def}\NormalTok{ main():}
    \CommentTok{\# 配置路径}
\NormalTok{    data\_path }\OperatorTok{=} \StringTok{"your data path"}
\NormalTok{    save\_dir }\OperatorTok{=} \StringTok{"tokenizer\_k"}

    \CommentTok{\# 训练tokenizer}
\NormalTok{    train\_tokenizer(}
\NormalTok{        data\_path}\OperatorTok{=}\NormalTok{data\_path,}
\NormalTok{        save\_dir}\OperatorTok{=}\NormalTok{save\_dir,}
\NormalTok{        vocab\_size}\OperatorTok{=}\DecValTok{6144}
\NormalTok{    )}

    \CommentTok{\# 评估tokenizer}
\NormalTok{    eval\_tokenizer(save\_dir)}

\ControlFlowTok{if} \VariableTok{\_\_name\_\_} \OperatorTok{==} \StringTok{\textquotesingle{}\_\_main\_\_\textquotesingle{}}\NormalTok{:}
\NormalTok{    main()}
\end{Highlighting}
\end{Shaded}

训练完成之后可以可以使用 \texttt{eval\_tokenizer()} 测试 Tokenizer
的功能,确保 Tokenizer 正常工作。在这个函数中,我们首先加载训练好的
Tokenizer,然后测试了 Tokenizer
的基本属性、聊天模板、编码解码等功能。这些测试可以帮助我们验证 Tokenizer
的正确性,确保它能够正常工作。正确的输出为:

OUT:

\begin{verbatim}
=== Tokenizer基本信息 ===
Vocab size: 6144
Special tokens: ['<|im_start|>', '<|im_end|>', '<unk>', '<s>', '</s>']
Special token IDs: [3, 4, 0, 1, 2]

=== 聊天模板测试 ===
Generated prompt:
<|im_start|>system
你是一个AI助手。<|im_end|>
<|im_start|>user
How are you?<|im_end|>
<|im_start|>assistant
I'm fine, thank you. and you?<|im_end|>
<|im_start|>user
I'm good too.<|im_end|>
<|im_start|>assistant
That's great to hear!<|im_end|>


=== 编码解码测试 ===
Decoded text matches original: False

=== 特殊token处理 ===
Original: <|im_start|>user
Hello<|im_end|>
Decoded:  <|im_start|> user
Hello<|im_end|>
Special tokens preserved: False
\end{verbatim}

\subsubsection{5.3.3 Dataset}\label{dataset}

\paragraph{PretrainDataset}\label{pretraindataset}

在将数据送入到模型之前,我们还需要进行一些处理用于将文本数据转化为模型能够理解的Token。在这里我们使用的是Pytorch的Dataset类,用于加载数据集。我们定义了一个\texttt{PretrainDataset}类,用于加载已预处理好的数据集。我们继承了\texttt{torch.utils.data.IterableDataset}来定义该数据集,这使得我们可以更灵活、高效地处理数据。

\begin{Shaded}
\begin{Highlighting}[]
\ImportTok{from}\NormalTok{ torch.utils.data }\ImportTok{import}\NormalTok{ Dataset}

\KeywordTok{class}\NormalTok{ PretrainDataset(Dataset):}
    \KeywordTok{def} \FunctionTok{\_\_init\_\_}\NormalTok{(}\VariableTok{self}\NormalTok{, data\_path, tokenizer, max\_length}\OperatorTok{=}\DecValTok{512}\NormalTok{):}
        \BuiltInTok{super}\NormalTok{().}\FunctionTok{\_\_init\_\_}\NormalTok{()}
        \VariableTok{self}\NormalTok{.data\_path }\OperatorTok{=}\NormalTok{ data\_path}
        \VariableTok{self}\NormalTok{.tokenizer }\OperatorTok{=}\NormalTok{ tokenizer}
        \VariableTok{self}\NormalTok{.max\_length }\OperatorTok{=}\NormalTok{ max\_length}
        \VariableTok{self}\NormalTok{.padding }\OperatorTok{=} \DecValTok{0}
        \ControlFlowTok{with} \BuiltInTok{open}\NormalTok{(data\_path, }\StringTok{\textquotesingle{}r\textquotesingle{}}\NormalTok{, encoding}\OperatorTok{=}\StringTok{\textquotesingle{}utf{-}8\textquotesingle{}}\NormalTok{) }\ImportTok{as}\NormalTok{ f:}
            \VariableTok{self}\NormalTok{.data }\OperatorTok{=}\NormalTok{ f.readlines()}

    \KeywordTok{def} \FunctionTok{\_\_len\_\_}\NormalTok{(}\VariableTok{self}\NormalTok{):}
        \ControlFlowTok{return} \BuiltInTok{len}\NormalTok{(}\VariableTok{self}\NormalTok{.data)}

    \KeywordTok{def} \FunctionTok{\_\_getitem\_\_}\NormalTok{(}\VariableTok{self}\NormalTok{, index: }\BuiltInTok{int}\NormalTok{):}
\NormalTok{        sample }\OperatorTok{=}\NormalTok{ json.loads(}\VariableTok{self}\NormalTok{.data[index])}
\NormalTok{        text }\OperatorTok{=} \SpecialStringTok{f"}\SpecialCharTok{\{}\VariableTok{self}\SpecialCharTok{.}\NormalTok{tokenizer}\SpecialCharTok{.}\NormalTok{bos\_token}\SpecialCharTok{\}\{}\NormalTok{sample[}\StringTok{\textquotesingle{}text\textquotesingle{}}\NormalTok{]}\SpecialCharTok{\}}\SpecialStringTok{"}
\NormalTok{        input\_id }\OperatorTok{=} \VariableTok{self}\NormalTok{.tokenizer(text).data[}\StringTok{\textquotesingle{}input\_ids\textquotesingle{}}\NormalTok{][:}\VariableTok{self}\NormalTok{.max\_length]}
\NormalTok{        text\_len }\OperatorTok{=} \BuiltInTok{len}\NormalTok{(input\_id)}
        \CommentTok{\# 没满最大长度的剩余部分}
\NormalTok{        padding\_len }\OperatorTok{=} \VariableTok{self}\NormalTok{.max\_length }\OperatorTok{{-}}\NormalTok{ text\_len}
\NormalTok{        input\_id }\OperatorTok{=}\NormalTok{ input\_id }\OperatorTok{+}\NormalTok{ [}\VariableTok{self}\NormalTok{.padding] }\OperatorTok{*}\NormalTok{ padding\_len}
        \CommentTok{\# 0表示不计算损失}
\NormalTok{        loss\_mask }\OperatorTok{=}\NormalTok{ [}\DecValTok{1}\NormalTok{] }\OperatorTok{*}\NormalTok{ text\_len }\OperatorTok{+}\NormalTok{ [}\DecValTok{0}\NormalTok{] }\OperatorTok{*}\NormalTok{ padding\_len}

\NormalTok{        input\_id }\OperatorTok{=}\NormalTok{ np.array(input\_id)}
\NormalTok{        X }\OperatorTok{=}\NormalTok{ np.array(input\_id[:}\OperatorTok{{-}}\DecValTok{1}\NormalTok{]).astype(np.int64)}
\NormalTok{        Y }\OperatorTok{=}\NormalTok{ np.array(input\_id[}\DecValTok{1}\NormalTok{:]).astype(np.int64)}
\NormalTok{        loss\_mask }\OperatorTok{=}\NormalTok{ np.array(loss\_mask[}\DecValTok{1}\NormalTok{:]).astype(np.int64)}
        \ControlFlowTok{return}\NormalTok{ torch.from\_numpy(X), torch.from\_numpy(Y), torch.from\_numpy(loss\_mask)}
\end{Highlighting}
\end{Shaded}

在以上代码和图5.3可以看出,\texttt{Pretrain\ Dataset} 主要是将
\texttt{text} 通过 \texttt{tokenizer} 转换成 \texttt{input\_id},然后将
\texttt{input\_id} 拆分成 \texttt{X} 和 \texttt{Y},其中 \texttt{X} 为
\texttt{input\_id} 的前 n-1 个元素,\texttt{Y} 为 \texttt{input\_id}
的后 n-1 \texttt{个元素。loss\_mask}
主要是用来标记哪些位置需要计算损失,哪些位置不需要计算损失。

\begin{figure}[htbp]\centering
\includegraphics[width=1.0\textwidth]{https://raw.githubusercontent.com/datawhalechina/happy-llm/main/docs/images/5-images/pretrain_dataset.png}
\caption{图5.3 预训练损失函数计算}
\end{figure}

图中示例展示了当\texttt{max\_length=9}时的处理过程: -
\textbf{输入序列}:\texttt{{[}BOS,\ T1,\ T2,\ T3,\ T4,\ T5,\ T6,\ T7,\ EOS{]}}
- \textbf{样本拆分}: -
X:\texttt{{[}BOS,\ T1,\ T2,\ T3,\ T4,\ T5,\ T6,\ T7{]}} →
模型输入上下文 -
Y:\texttt{{[}T1,\ T2,\ T3,\ T4,\ T5,\ T6,\ T7,\ EOS{]}} → 模型预测目标
- \textbf{损失掩码}: -
有效位置:\texttt{{[}0,\ 1,\ 1,\ 1,\ 1,\ 1,\ 1,\ 1,\ 1{]}} →
仅对T1-EOS计算损失

\paragraph{SFTDataset}\label{sftdataset}

\texttt{SFTDataset}
其实是一个多轮对话数据集,我们的目标是让模型学会如何进行多轮对话。在这个阶段我们的输入是上一轮的对话内容,输出是当前轮的对话内容。

\begin{Shaded}
\begin{Highlighting}[]
\KeywordTok{class}\NormalTok{ SFTDataset(Dataset):}
    \KeywordTok{def} \FunctionTok{\_\_init\_\_}\NormalTok{(}\VariableTok{self}\NormalTok{, data\_path, tokenizer, max\_length}\OperatorTok{=}\DecValTok{512}\NormalTok{):}
        \BuiltInTok{super}\NormalTok{().}\FunctionTok{\_\_init\_\_}\NormalTok{()}
        \VariableTok{self}\NormalTok{.data\_path }\OperatorTok{=}\NormalTok{ data\_path}
        \VariableTok{self}\NormalTok{.tokenizer }\OperatorTok{=}\NormalTok{ tokenizer}
        \VariableTok{self}\NormalTok{.max\_length }\OperatorTok{=}\NormalTok{ max\_length}
        \VariableTok{self}\NormalTok{.padding }\OperatorTok{=} \DecValTok{0}
        \ControlFlowTok{with} \BuiltInTok{open}\NormalTok{(data\_path, }\StringTok{\textquotesingle{}r\textquotesingle{}}\NormalTok{, encoding}\OperatorTok{=}\StringTok{\textquotesingle{}utf{-}8\textquotesingle{}}\NormalTok{) }\ImportTok{as}\NormalTok{ f:}
            \VariableTok{self}\NormalTok{.data }\OperatorTok{=}\NormalTok{ f.readlines()}

    \KeywordTok{def} \FunctionTok{\_\_len\_\_}\NormalTok{(}\VariableTok{self}\NormalTok{):}
        \ControlFlowTok{return} \BuiltInTok{len}\NormalTok{(}\VariableTok{self}\NormalTok{.data)}

    \KeywordTok{def}\NormalTok{ generate\_loss\_mask(}\VariableTok{self}\NormalTok{, input\_ids):}
        \CommentTok{\# 生成 loss mask, 0 表示不计算损失, 1 表示计算损失}
\NormalTok{        mask }\OperatorTok{=}\NormalTok{ [}\DecValTok{0}\NormalTok{] }\OperatorTok{*} \BuiltInTok{len}\NormalTok{(input\_ids)}
\NormalTok{        a\_sequence }\OperatorTok{=}\NormalTok{ [}\DecValTok{3}\NormalTok{, }\DecValTok{1074}\NormalTok{, }\DecValTok{537}\NormalTok{, }\DecValTok{500}\NormalTok{, }\DecValTok{203}\NormalTok{]  }\CommentTok{\# \textless{}|im\_start|\textgreater{}assistant\textbackslash{}n}
\NormalTok{        a\_length }\OperatorTok{=} \BuiltInTok{len}\NormalTok{(a\_sequence)}
\NormalTok{        n }\OperatorTok{=} \BuiltInTok{len}\NormalTok{(input\_ids)}
\NormalTok{        i }\OperatorTok{=} \DecValTok{0}
        
        \ControlFlowTok{while}\NormalTok{ i }\OperatorTok{\textless{}=}\NormalTok{ n }\OperatorTok{{-}}\NormalTok{ a\_length:}
            \CommentTok{\# 检查当前位置是否匹配目标子序列}
\NormalTok{            match }\OperatorTok{=} \VariableTok{True}
            \ControlFlowTok{for}\NormalTok{ k }\KeywordTok{in} \BuiltInTok{range}\NormalTok{(a\_length):}
                \ControlFlowTok{if}\NormalTok{ input\_ids[i }\OperatorTok{+}\NormalTok{ k] }\OperatorTok{!=}\NormalTok{ a\_sequence[k]:}
\NormalTok{                    match }\OperatorTok{=} \VariableTok{False}
                    \ControlFlowTok{break}
            \ControlFlowTok{if}\NormalTok{ match:}
                \CommentTok{\# 从子序列结束的位置开始查找第一个4, 4 为 \textless{}|im\_end|\textgreater{} EOS id}
\NormalTok{                j }\OperatorTok{=} \VariableTok{None}
                \ControlFlowTok{for}\NormalTok{ idx }\KeywordTok{in} \BuiltInTok{range}\NormalTok{(i }\OperatorTok{+}\NormalTok{ a\_length, n):}
                    \ControlFlowTok{if}\NormalTok{ input\_ids[idx] }\OperatorTok{==} \DecValTok{4}\NormalTok{:}
\NormalTok{                        j }\OperatorTok{=}\NormalTok{ idx}
                        \ControlFlowTok{break}
                \ControlFlowTok{if}\NormalTok{ j }\KeywordTok{is} \KeywordTok{not} \VariableTok{None}\NormalTok{:}
\NormalTok{                    start }\OperatorTok{=}\NormalTok{ i }\OperatorTok{+}\NormalTok{ a\_length}
\NormalTok{                    end }\OperatorTok{=}\NormalTok{ j  }\CommentTok{\# 结束位置设为j(包含4)}
                    \CommentTok{\# 标记区间为1(包括start到end)}
                    \ControlFlowTok{if}\NormalTok{ start }\OperatorTok{\textless{}=}\NormalTok{ end:}
                        \ControlFlowTok{for}\NormalTok{ pos }\KeywordTok{in} \BuiltInTok{range}\NormalTok{(start, end }\OperatorTok{+} \DecValTok{1}\NormalTok{):}
                            \ControlFlowTok{if}\NormalTok{ pos }\OperatorTok{\textless{}} \BuiltInTok{len}\NormalTok{(mask):}
\NormalTok{                                mask[pos] }\OperatorTok{=} \DecValTok{1}
                \CommentTok{\# 跳过当前子序列,避免重叠匹配}
\NormalTok{                i }\OperatorTok{+=}\NormalTok{ a\_length}
            \ControlFlowTok{else}\NormalTok{:}
\NormalTok{                i }\OperatorTok{+=} \DecValTok{1}
        \ControlFlowTok{return}\NormalTok{ mask}

    \KeywordTok{def} \FunctionTok{\_\_getitem\_\_}\NormalTok{(}\VariableTok{self}\NormalTok{, index: }\BuiltInTok{int}\NormalTok{):}
\NormalTok{        sample }\OperatorTok{=}\NormalTok{ json.loads(}\VariableTok{self}\NormalTok{.data[index])}
\NormalTok{        text }\OperatorTok{=} \VariableTok{self}\NormalTok{.tokenizer.apply\_chat\_template(sample, tokenize}\OperatorTok{=}\VariableTok{False}\NormalTok{, add\_generation\_prompt}\OperatorTok{=}\VariableTok{False}\NormalTok{)}
\NormalTok{        input\_id }\OperatorTok{=} \VariableTok{self}\NormalTok{.tokenizer(text).data[}\StringTok{\textquotesingle{}input\_ids\textquotesingle{}}\NormalTok{][:}\VariableTok{self}\NormalTok{.max\_length]}
\NormalTok{        text\_len }\OperatorTok{=} \BuiltInTok{len}\NormalTok{(input\_id)}
        \CommentTok{\# 没满最大长度的剩余部分}
\NormalTok{        padding\_len }\OperatorTok{=} \VariableTok{self}\NormalTok{.max\_length }\OperatorTok{{-}}\NormalTok{ text\_len}
\NormalTok{        input\_id }\OperatorTok{=}\NormalTok{ input\_id }\OperatorTok{+}\NormalTok{ [}\VariableTok{self}\NormalTok{.padding] }\OperatorTok{*}\NormalTok{ padding\_len}
        \CommentTok{\# 0表示不计算损失}
\NormalTok{        loss\_mask }\OperatorTok{=} \VariableTok{self}\NormalTok{.generate\_loss\_mask(input\_id)}

\NormalTok{        input\_id }\OperatorTok{=}\NormalTok{ np.array(input\_id)}
\NormalTok{        X }\OperatorTok{=}\NormalTok{ np.array(input\_id[:}\OperatorTok{{-}}\DecValTok{1}\NormalTok{]).astype(np.int64)}
\NormalTok{        Y }\OperatorTok{=}\NormalTok{ np.array(input\_id[}\DecValTok{1}\NormalTok{:]).astype(np.int64)}
\NormalTok{        loss\_mask }\OperatorTok{=}\NormalTok{ np.array(loss\_mask[}\DecValTok{1}\NormalTok{:]).astype(np.int64)}
        \ControlFlowTok{return}\NormalTok{ torch.from\_numpy(X), torch.from\_numpy(Y), torch.from\_numpy(loss\_mask)}
\end{Highlighting}
\end{Shaded}

在 SFT
阶段,这里使用的是多轮对话数据集,所以就需要区分哪些位置需要计算损失,哪些位置不需要计算损失。在上面的代码中,我使用了一个
\texttt{generate\_loss\_mask} 函数来生成
\texttt{loss\_mask}。这个函数主要是用来生成 \texttt{loss\_mask},其中
\texttt{loss\_mask} 的生成规则是:当遇到
\texttt{\textbar{}\textless{}im\_start\textbar{}\textgreater{}assistant\textbackslash{}n}
时,就开始计算损失,直到遇到
\texttt{\textbar{}\textless{}im\_end\textbar{}\textgreater{}}
为止。这样就可以保证我们的模型在 SFT
阶段只计算当前轮的对话内容,如图5.4所示。

\begin{figure}[htbp]\centering
\includegraphics[width=0.9\textwidth]{https://raw.githubusercontent.com/datawhalechina/happy-llm/main/docs/images/5-images/sftdataset.png}
\caption{图5.4 SFT 损失函数计算}
\end{figure}

可以看到,其实 SFT Dataset 和 Pretrain Dataset 的 \texttt{X} 和
\texttt{Y} 是一样的,只是在 SFT Dataset 中我们需要生成一个
\texttt{loss\_mask} 来标记哪些位置需要计算损失,哪些位置不需要计算损失。
图中 \texttt{Input\ ids}
中的蓝色小方格就是AI的回答,所以是需要模型学习的地方。所以在
\texttt{loss\_mask}
中,蓝色小方格对应的位置是黄色,其他位置是灰色。在代码
\texttt{loss\_mask} 中的 1 对应的位置计算损失,0 对应的位置不计算损失。

\subsubsection{5.3.4 预训练}\label{ux9884ux8badux7ec3}

在数据预处理完成后,我们就可以开始训练模型了。我们使用的模型是一个和LLama2结构一样的
Decoder only
Transformer模型,使用Pytorch实现。相关代码在\texttt{code/k\_model.py}文件中。此处不再赘述,源码中有详细的中文注释,且我们在之前的文章中也有详细的介绍。

在模型这一部分可以重点看一下生成式模型是如何实现生成token的,可以查看\texttt{k\_model.py}文件中的\texttt{Transforerm}类中的\texttt{generate}方法。

\begin{Shaded}
\begin{Highlighting}[]
\AttributeTok{@torch.inference\_mode}\NormalTok{()}
    \KeywordTok{def}\NormalTok{ generate(}\VariableTok{self}\NormalTok{, idx, stop\_id}\OperatorTok{=}\VariableTok{None}\NormalTok{, max\_new\_tokens}\OperatorTok{=}\DecValTok{256}\NormalTok{, temperature}\OperatorTok{=}\FloatTok{1.0}\NormalTok{, top\_k}\OperatorTok{=}\VariableTok{None}\NormalTok{):}
        \CommentTok{"""}
\CommentTok{        给定输入序列 idx(形状为 (bz,seq\_len) 的长整型张量),通过多次生成新 token 来完成序列。}
\CommentTok{        在 model.eval() 模式下运行。效率较低的采样版本,没有使用键k/v cache。}
\CommentTok{        """}
\NormalTok{        index }\OperatorTok{=}\NormalTok{ idx.shape[}\DecValTok{1}\NormalTok{]}
        \ControlFlowTok{for}\NormalTok{ \_ }\KeywordTok{in} \BuiltInTok{range}\NormalTok{(max\_new\_tokens):}
            \CommentTok{\# 如果序列上下文过长,截断它到最大长度}
\NormalTok{            idx\_cond }\OperatorTok{=}\NormalTok{ idx }\ControlFlowTok{if}\NormalTok{ idx.size(}\DecValTok{1}\NormalTok{) }\OperatorTok{\textless{}=} \VariableTok{self}\NormalTok{.args.max\_seq\_len }\ControlFlowTok{else}\NormalTok{ idx[:, }\OperatorTok{{-}}\VariableTok{self}\NormalTok{.args.max\_seq\_len:]}
            
            \CommentTok{\# 前向传播获取序列中最后一个位置的 logits}
\NormalTok{            logits }\OperatorTok{=} \VariableTok{self}\NormalTok{(idx\_cond).logits}
\NormalTok{            logits }\OperatorTok{=}\NormalTok{ logits[:, }\OperatorTok{{-}}\DecValTok{1}\NormalTok{, :] }\CommentTok{\# 只保留最后一个时间步的输出}
            
            \ControlFlowTok{if}\NormalTok{ temperature }\OperatorTok{==} \FloatTok{0.0}\NormalTok{:}
                \CommentTok{\# 选择最有可能的索引}
\NormalTok{                \_, idx\_next }\OperatorTok{=}\NormalTok{ torch.topk(logits, k}\OperatorTok{=}\DecValTok{1}\NormalTok{, dim}\OperatorTok{={-}}\DecValTok{1}\NormalTok{)}
            \ControlFlowTok{else}\NormalTok{:}
                \CommentTok{\# 缩放 logits 并应用 softmax}
\NormalTok{                logits }\OperatorTok{=}\NormalTok{ logits }\OperatorTok{/}\NormalTok{ temperature}
                \ControlFlowTok{if}\NormalTok{ top\_k }\KeywordTok{is} \KeywordTok{not} \VariableTok{None}\NormalTok{:}
\NormalTok{                    v, \_ }\OperatorTok{=}\NormalTok{ torch.topk(logits, }\BuiltInTok{min}\NormalTok{(top\_k, logits.size(}\OperatorTok{{-}}\DecValTok{1}\NormalTok{)))}
\NormalTok{                    logits[logits }\OperatorTok{\textless{}}\NormalTok{ v[:, [}\OperatorTok{{-}}\DecValTok{1}\NormalTok{]]] }\OperatorTok{=} \OperatorTok{{-}}\BuiltInTok{float}\NormalTok{(}\StringTok{\textquotesingle{}Inf\textquotesingle{}}\NormalTok{)}
\NormalTok{                probs }\OperatorTok{=}\NormalTok{ F.softmax(logits, dim}\OperatorTok{={-}}\DecValTok{1}\NormalTok{)}
\NormalTok{                idx\_next }\OperatorTok{=}\NormalTok{ torch.multinomial(probs, num\_samples}\OperatorTok{=}\DecValTok{1}\NormalTok{)}
            

            \ControlFlowTok{if}\NormalTok{ idx\_next }\OperatorTok{==}\NormalTok{ stop\_id:}
                \ControlFlowTok{break}

            \CommentTok{\# 将采样的索引添加到序列中并继续}
\NormalTok{            idx }\OperatorTok{=}\NormalTok{ torch.cat((idx, idx\_next), dim}\OperatorTok{=}\DecValTok{1}\NormalTok{)}

        \ControlFlowTok{return}\NormalTok{ idx[:, index:] }\CommentTok{\# 只返回生成的token}
\end{Highlighting}
\end{Shaded}

在 \texttt{generate} 方法中,我们首先获取序列中最后一个位置的
\texttt{logits},然后基于这些 \texttt{logits} 生成新的
\texttt{token}。接着,生成的新 \texttt{token}
会被添加到序列中,模型随后会继续生成下一个
\texttt{token}。通过这种迭代过程,我们能够生成完整的文本。

接下来就是最重要的部分,训练模型!

\begin{quote}
注:在使用下面代码进行模型训练时,需要指定 \texttt{-\/-data\_path}
参数为预处理好的数据集路径,例如
\texttt{-\/-data\_path\ seq\_monkey\_datawhale.jsonl},也需要指定要用哪几张GPU进行训练,例如
\texttt{-\/-gpus\ 0,1}。
\end{quote}

\begin{Shaded}
\begin{Highlighting}[]
\KeywordTok{def}\NormalTok{ get\_lr(it, }\BuiltInTok{all}\NormalTok{):}
    \CommentTok{"""}
\CommentTok{    计算当前迭代的学习率,使用余弦退火调度策略}
\CommentTok{    }
\CommentTok{    学习率调度策略:}
\CommentTok{    1. Warmup阶段:学习率从0线性增长到目标学习率}
\CommentTok{    2. 余弦退火阶段:学习率按余弦函数衰减到最小学习率}
\CommentTok{    3. 超出训练步数后:保持最小学习率}
\CommentTok{    }
\CommentTok{    Args:}
\CommentTok{        it (int): 当前迭代步数}
\CommentTok{        all (int): 总迭代步数}
\CommentTok{        }
\CommentTok{    Returns:}
\CommentTok{        float: 当前步数对应的学习率}
\CommentTok{    """}
\NormalTok{    warmup\_iters }\OperatorTok{=}\NormalTok{ args.warmup\_iters  }\CommentTok{\# 预热迭代次数}
\NormalTok{    lr\_decay\_iters }\OperatorTok{=} \BuiltInTok{all}  \CommentTok{\# 学习率衰减的总迭代次数}
\NormalTok{    min\_lr }\OperatorTok{=}\NormalTok{ args.learning\_rate }\OperatorTok{/} \DecValTok{10}  \CommentTok{\# 最小学习率,为初始学习率的1/10}

    \CommentTok{\# Warmup阶段:线性增长}
    \ControlFlowTok{if}\NormalTok{ it }\OperatorTok{\textless{}}\NormalTok{ warmup\_iters:}
        \ControlFlowTok{return}\NormalTok{ args.learning\_rate }\OperatorTok{*}\NormalTok{ it }\OperatorTok{/}\NormalTok{ warmup\_iters}
    
    \CommentTok{\# 超出训练步数:保持最小学习率}
    \ControlFlowTok{if}\NormalTok{ it }\OperatorTok{\textgreater{}}\NormalTok{ lr\_decay\_iters:}
        \ControlFlowTok{return}\NormalTok{ min\_lr}
    
    \CommentTok{\# 余弦退火阶段}
\NormalTok{    decay\_ratio }\OperatorTok{=}\NormalTok{ (it }\OperatorTok{{-}}\NormalTok{ warmup\_iters) }\OperatorTok{/}\NormalTok{ (lr\_decay\_iters }\OperatorTok{{-}}\NormalTok{ warmup\_iters)}
    \ControlFlowTok{assert} \DecValTok{0} \OperatorTok{\textless{}=}\NormalTok{ decay\_ratio }\OperatorTok{\textless{}=} \DecValTok{1}
\NormalTok{    coeff }\OperatorTok{=} \FloatTok{0.5} \OperatorTok{*}\NormalTok{ (}\FloatTok{1.0} \OperatorTok{+}\NormalTok{ math.cos(math.pi }\OperatorTok{*}\NormalTok{ decay\_ratio))  }\CommentTok{\# 余弦系数}
    \ControlFlowTok{return}\NormalTok{ min\_lr }\OperatorTok{+}\NormalTok{ coeff }\OperatorTok{*}\NormalTok{ (args.learning\_rate }\OperatorTok{{-}}\NormalTok{ min\_lr)}

\KeywordTok{def}\NormalTok{ train\_epoch(epoch):}
    \CommentTok{"""}
\CommentTok{    训练一个epoch的函数}
\CommentTok{    }
\CommentTok{    实现了完整的训练循环,包括:}
\CommentTok{    1. 数据加载和设备转移}
\CommentTok{    2. 动态学习率调整}
\CommentTok{    3. 前向传播和损失计算}
\CommentTok{    4. 梯度累积和反向传播}
\CommentTok{    5. 梯度裁剪和优化器更新}
\CommentTok{    6. 日志记录和模型保存}
\CommentTok{    }
\CommentTok{    Args:}
\CommentTok{        epoch (int): 当前epoch编号}
\CommentTok{    """}
\NormalTok{    start\_time }\OperatorTok{=}\NormalTok{ time.time()  }\CommentTok{\# 记录开始时间}
    
    \CommentTok{\# 遍历数据加载器中的每个batch}
    \ControlFlowTok{for}\NormalTok{ step, (X, Y, loss\_mask) }\KeywordTok{in} \BuiltInTok{enumerate}\NormalTok{(train\_loader):}
        \CommentTok{\# 将数据转移到指定设备(GPU/CPU)}
\NormalTok{        X }\OperatorTok{=}\NormalTok{ X.to(args.device)  }\CommentTok{\# 输入序列}
\NormalTok{        Y }\OperatorTok{=}\NormalTok{ Y.to(args.device)  }\CommentTok{\# 目标序列}
\NormalTok{        loss\_mask }\OperatorTok{=}\NormalTok{ loss\_mask.to(args.device)  }\CommentTok{\# 损失掩码,用于忽略padding token}

        \CommentTok{\# 计算当前步骤的学习率}
\NormalTok{        lr }\OperatorTok{=}\NormalTok{ get\_lr(epoch }\OperatorTok{*}\NormalTok{ iter\_per\_epoch }\OperatorTok{+}\NormalTok{ step, args.epochs }\OperatorTok{*}\NormalTok{ iter\_per\_epoch)}
        \CommentTok{\# 更新优化器中所有参数组的学习率}
        \ControlFlowTok{for}\NormalTok{ param\_group }\KeywordTok{in}\NormalTok{ optimizer.param\_groups:}
\NormalTok{            param\_group[}\StringTok{\textquotesingle{}lr\textquotesingle{}}\NormalTok{] }\OperatorTok{=}\NormalTok{ lr}

        \CommentTok{\# 使用混合精度训练上下文}
        \ControlFlowTok{with}\NormalTok{ ctx:}
            \CommentTok{\# 前向传播}
\NormalTok{            out }\OperatorTok{=}\NormalTok{ model(X, Y)}
            \CommentTok{\# 计算损失并除以累积步数(用于梯度累积)}
\NormalTok{            loss }\OperatorTok{=}\NormalTok{ out.last\_loss }\OperatorTok{/}\NormalTok{ args.accumulation\_steps}
            \CommentTok{\# 将loss\_mask展平为一维}
\NormalTok{            loss\_mask }\OperatorTok{=}\NormalTok{ loss\_mask.view(}\OperatorTok{{-}}\DecValTok{1}\NormalTok{)}
            \CommentTok{\# 应用掩码计算有效损失(忽略padding位置)}
\NormalTok{            loss }\OperatorTok{=}\NormalTok{ torch.}\BuiltInTok{sum}\NormalTok{(loss }\OperatorTok{*}\NormalTok{ loss\_mask) }\OperatorTok{/}\NormalTok{ loss\_mask.}\BuiltInTok{sum}\NormalTok{()}

        \CommentTok{\# 使用scaler进行混合精度的反向传播}
\NormalTok{        scaler.scale(loss).backward()}

        \CommentTok{\# 每accumulation\_steps步执行一次优化器更新}
        \ControlFlowTok{if}\NormalTok{ (step }\OperatorTok{+} \DecValTok{1}\NormalTok{) }\OperatorTok{\%}\NormalTok{ args.accumulation\_steps }\OperatorTok{==} \DecValTok{0}\NormalTok{:}
            \CommentTok{\# 取消梯度缩放,准备梯度裁剪}
\NormalTok{            scaler.unscale\_(optimizer)}
            \CommentTok{\# 梯度裁剪,防止梯度爆炸}
\NormalTok{            torch.nn.utils.clip\_grad\_norm\_(model.parameters(), args.grad\_clip)}

            \CommentTok{\# 执行优化器步骤}
\NormalTok{            scaler.step(optimizer)}
            \CommentTok{\# 更新scaler的缩放因子}
\NormalTok{            scaler.update()}

            \CommentTok{\# 清零梯度,set\_to\_none=True可以节省内存}
\NormalTok{            optimizer.zero\_grad(set\_to\_none}\OperatorTok{=}\VariableTok{True}\NormalTok{)}

        \CommentTok{\# 每log\_interval步记录一次日志}
        \ControlFlowTok{if}\NormalTok{ step }\OperatorTok{\%}\NormalTok{ args.log\_interval }\OperatorTok{==} \DecValTok{0}\NormalTok{:}
\NormalTok{            spend\_time }\OperatorTok{=}\NormalTok{ time.time() }\OperatorTok{{-}}\NormalTok{ start\_time}
            \CommentTok{\# 打印训练进度信息}
\NormalTok{            Logger(}
                \StringTok{\textquotesingle{}Epoch:[}\SpecialCharTok{\{\}}\StringTok{/}\SpecialCharTok{\{\}}\StringTok{](}\SpecialCharTok{\{\}}\StringTok{/}\SpecialCharTok{\{\}}\StringTok{) loss:}\SpecialCharTok{\{:.3f\}}\StringTok{ lr:}\SpecialCharTok{\{:.7f\}}\StringTok{ epoch\_Time:}\SpecialCharTok{\{\}}\StringTok{min;\textquotesingle{}}\NormalTok{.}\BuiltInTok{format}\NormalTok{(}
\NormalTok{                    epoch }\OperatorTok{+} \DecValTok{1}\NormalTok{,}
\NormalTok{                    args.epochs,}
\NormalTok{                    step,}
\NormalTok{                    iter\_per\_epoch,}
\NormalTok{                    loss.item() }\OperatorTok{*}\NormalTok{ args.accumulation\_steps,  }\CommentTok{\# 恢复真实的loss值}
\NormalTok{                    optimizer.param\_groups[}\OperatorTok{{-}}\DecValTok{1}\NormalTok{][}\StringTok{\textquotesingle{}lr\textquotesingle{}}\NormalTok{],}
\NormalTok{                    spend\_time }\OperatorTok{/}\NormalTok{ (step }\OperatorTok{+} \DecValTok{1}\NormalTok{) }\OperatorTok{*}\NormalTok{ iter\_per\_epoch }\OperatorTok{//} \DecValTok{60} \OperatorTok{{-}}\NormalTok{ spend\_time }\OperatorTok{//} \DecValTok{60}\NormalTok{))}
            
            \CommentTok{\# 如果启用SwanLab,记录训练指标}
            \ControlFlowTok{if}\NormalTok{ args.use\_swanlab:}
\NormalTok{                swanlab.log(\{}
                    \StringTok{"loss"}\NormalTok{: loss.item() }\OperatorTok{*}\NormalTok{ args.accumulation\_steps,}
                    \StringTok{"lr"}\NormalTok{: optimizer.param\_groups[}\OperatorTok{{-}}\DecValTok{1}\NormalTok{][}\StringTok{\textquotesingle{}lr\textquotesingle{}}\NormalTok{]}
\NormalTok{                \})}

        \CommentTok{\# 每save\_interval步保存一次模型}
        \ControlFlowTok{if}\NormalTok{ (step }\OperatorTok{+} \DecValTok{1}\NormalTok{) }\OperatorTok{\%}\NormalTok{ args.save\_interval }\OperatorTok{==} \DecValTok{0}\NormalTok{:}
\NormalTok{            model.}\BuiltInTok{eval}\NormalTok{()  }\CommentTok{\# 切换到评估模式}
            \CommentTok{\# 构建检查点文件名}
\NormalTok{            ckp }\OperatorTok{=} \SpecialStringTok{f\textquotesingle{}}\SpecialCharTok{\{}\NormalTok{args}\SpecialCharTok{.}\NormalTok{save\_dir}\SpecialCharTok{\}}\SpecialStringTok{/pretrain\_}\SpecialCharTok{\{}\NormalTok{lm\_config}\SpecialCharTok{.}\NormalTok{dim}\SpecialCharTok{\}}\SpecialStringTok{\_}\SpecialCharTok{\{}\NormalTok{lm\_config}\SpecialCharTok{.}\NormalTok{n\_layers}\SpecialCharTok{\}}\SpecialStringTok{\_}\SpecialCharTok{\{}\NormalTok{lm\_config}\SpecialCharTok{.}\NormalTok{vocab\_size}\SpecialCharTok{\}}\SpecialStringTok{.pth\textquotesingle{}}

            \CommentTok{\# 处理多卡保存:如果是DataParallel模型,需要访问.module属性}
\NormalTok{            state\_dict }\OperatorTok{=}\NormalTok{ model.module.state\_dict() }\ControlFlowTok{if} \BuiltInTok{isinstance}\NormalTok{(model, torch.nn.DataParallel) }\ControlFlowTok{else}\NormalTok{ model.state\_dict()}
\NormalTok{            torch.save(state\_dict, ckp)}
\NormalTok{            model.train()  }\CommentTok{\# 切换回训练模式}
        
        \CommentTok{\# 每20000步保存一个带步数标记的检查点}
        \ControlFlowTok{if}\NormalTok{ (step }\OperatorTok{+} \DecValTok{1}\NormalTok{) }\OperatorTok{\%} \DecValTok{20000} \OperatorTok{==} \DecValTok{0}\NormalTok{:}
\NormalTok{            model.}\BuiltInTok{eval}\NormalTok{()}
            \CommentTok{\# 构建带步数的检查点文件名}
\NormalTok{            ckp }\OperatorTok{=} \SpecialStringTok{f\textquotesingle{}}\SpecialCharTok{\{}\NormalTok{args}\SpecialCharTok{.}\NormalTok{save\_dir}\SpecialCharTok{\}}\SpecialStringTok{/pretrain\_}\SpecialCharTok{\{}\NormalTok{lm\_config}\SpecialCharTok{.}\NormalTok{dim}\SpecialCharTok{\}}\SpecialStringTok{\_}\SpecialCharTok{\{}\NormalTok{lm\_config}\SpecialCharTok{.}\NormalTok{n\_layers}\SpecialCharTok{\}}\SpecialStringTok{\_}\SpecialCharTok{\{}\NormalTok{lm\_config}\SpecialCharTok{.}\NormalTok{vocab\_size}\SpecialCharTok{\}}\SpecialStringTok{\_step}\SpecialCharTok{\{}\NormalTok{step}\OperatorTok{+}\DecValTok{1}\SpecialCharTok{\}}\SpecialStringTok{.pth\textquotesingle{}}

            \CommentTok{\# 保存模型状态字典}
\NormalTok{            state\_dict }\OperatorTok{=}\NormalTok{ model.module.state\_dict() }\ControlFlowTok{if} \BuiltInTok{isinstance}\NormalTok{(model, torch.nn.DataParallel) }\ControlFlowTok{else}\NormalTok{ model.state\_dict()}
\NormalTok{            torch.save(state\_dict, ckp)}
\NormalTok{            model.train()}


\KeywordTok{def}\NormalTok{ init\_model():}
    \CommentTok{"""}
\CommentTok{    初始化模型和分词器}
\CommentTok{    }
\CommentTok{    功能包括:}
\CommentTok{    1. 加载预训练的分词器}
\CommentTok{    2. 创建Transformer模型}
\CommentTok{    3. 设置多GPU并行训练(如果可用)}
\CommentTok{    4. 将模型移动到指定设备}
\CommentTok{    5. 统计并打印模型参数量}
\CommentTok{    }
\CommentTok{    Returns:}
\CommentTok{        tuple: (model, tokenizer) 初始化后的模型和分词器}
\CommentTok{    """}
    \KeywordTok{def}\NormalTok{ count\_parameters(model):}
        \CommentTok{"""}
\CommentTok{        统计模型中可训练参数的数量}
\CommentTok{        }
\CommentTok{        Args:}
\CommentTok{            model: PyTorch模型}
\CommentTok{            }
\CommentTok{        Returns:}
\CommentTok{            int: 可训练参数总数}
\CommentTok{        """}
        \ControlFlowTok{return} \BuiltInTok{sum}\NormalTok{(p.numel() }\ControlFlowTok{for}\NormalTok{ p }\KeywordTok{in}\NormalTok{ model.parameters() }\ControlFlowTok{if}\NormalTok{ p.requires\_grad)}

    \CommentTok{\# 从本地路径加载预训练的分词器}
\NormalTok{    tokenizer }\OperatorTok{=}\NormalTok{ AutoTokenizer.from\_pretrained(}\StringTok{\textquotesingle{}./tokenizer\_k/\textquotesingle{}}\NormalTok{)}

    \CommentTok{\# 根据配置创建Transformer模型}
\NormalTok{    model }\OperatorTok{=}\NormalTok{ Transformer(lm\_config)}
    
    \CommentTok{\# 多卡初始化:检查可用GPU数量并设置DataParallel}
\NormalTok{    num\_gpus }\OperatorTok{=}\NormalTok{ torch.cuda.device\_count()}
    \ControlFlowTok{if}\NormalTok{ num\_gpus }\OperatorTok{\textgreater{}} \DecValTok{1}\NormalTok{:}
\NormalTok{        Logger(}\SpecialStringTok{f"Using }\SpecialCharTok{\{}\NormalTok{num\_gpus}\SpecialCharTok{\}}\SpecialStringTok{ GPUs with DataParallel!"}\NormalTok{)}
        \CommentTok{\# 使用DataParallel包装模型以支持多GPU训练}
\NormalTok{        model }\OperatorTok{=}\NormalTok{ torch.nn.DataParallel(model)}
    
    \CommentTok{\# 将模型移动到指定设备(GPU或CPU)}
\NormalTok{    model }\OperatorTok{=}\NormalTok{ model.to(args.device)}
    
    \CommentTok{\# 计算并打印模型参数量(以百万为单位)}
\NormalTok{    Logger(}\SpecialStringTok{f\textquotesingle{}LLM总参数量:}\SpecialCharTok{\{}\NormalTok{count\_parameters(model) }\OperatorTok{/} \FloatTok{1e6}\SpecialCharTok{:.3f\}}\SpecialStringTok{ 百万\textquotesingle{}}\NormalTok{)}
    \ControlFlowTok{return}\NormalTok{ model, tokenizer}


\ControlFlowTok{if} \VariableTok{\_\_name\_\_} \OperatorTok{==} \StringTok{"\_\_main\_\_"}\NormalTok{:}
    \CommentTok{\# ==================== 命令行参数解析 ====================}
\NormalTok{    parser }\OperatorTok{=}\NormalTok{ argparse.ArgumentParser(description}\OperatorTok{=}\StringTok{"Tiny{-}LLM Pretraining"}\NormalTok{)}
    
    \CommentTok{\# 基础训练参数}
\NormalTok{    parser.add\_argument(}\StringTok{"{-}{-}out\_dir"}\NormalTok{, }\BuiltInTok{type}\OperatorTok{=}\BuiltInTok{str}\NormalTok{, default}\OperatorTok{=}\StringTok{"base\_model\_215M"}\NormalTok{, }\BuiltInTok{help}\OperatorTok{=}\StringTok{"模型输出目录"}\NormalTok{)}
\NormalTok{    parser.add\_argument(}\StringTok{"{-}{-}epochs"}\NormalTok{, }\BuiltInTok{type}\OperatorTok{=}\BuiltInTok{int}\NormalTok{, default}\OperatorTok{=}\DecValTok{1}\NormalTok{, }\BuiltInTok{help}\OperatorTok{=}\StringTok{"训练轮数"}\NormalTok{)}
\NormalTok{    parser.add\_argument(}\StringTok{"{-}{-}batch\_size"}\NormalTok{, }\BuiltInTok{type}\OperatorTok{=}\BuiltInTok{int}\NormalTok{, default}\OperatorTok{=}\DecValTok{64}\NormalTok{, }\BuiltInTok{help}\OperatorTok{=}\StringTok{"批次大小"}\NormalTok{)}
\NormalTok{    parser.add\_argument(}\StringTok{"{-}{-}learning\_rate"}\NormalTok{, }\BuiltInTok{type}\OperatorTok{=}\BuiltInTok{float}\NormalTok{, default}\OperatorTok{=}\FloatTok{2e{-}4}\NormalTok{, }\BuiltInTok{help}\OperatorTok{=}\StringTok{"学习率"}\NormalTok{)}
\NormalTok{    parser.add\_argument(}\StringTok{"{-}{-}device"}\NormalTok{, }\BuiltInTok{type}\OperatorTok{=}\BuiltInTok{str}\NormalTok{, default}\OperatorTok{=}\StringTok{"cuda:0"} \ControlFlowTok{if}\NormalTok{ torch.cuda.is\_available() }\ControlFlowTok{else} \StringTok{"cpu"}\NormalTok{, }\BuiltInTok{help}\OperatorTok{=}\StringTok{"训练设备"}\NormalTok{)}
\NormalTok{    parser.add\_argument(}\StringTok{"{-}{-}dtype"}\NormalTok{, }\BuiltInTok{type}\OperatorTok{=}\BuiltInTok{str}\NormalTok{, default}\OperatorTok{=}\StringTok{"bfloat16"}\NormalTok{, }\BuiltInTok{help}\OperatorTok{=}\StringTok{"数据类型"}\NormalTok{)}
    
    \CommentTok{\# 实验跟踪和数据加载参数}
\NormalTok{    parser.add\_argument(}\StringTok{"{-}{-}use\_swanlab"}\NormalTok{, action}\OperatorTok{=}\StringTok{"store\_true"}\NormalTok{, }\BuiltInTok{help}\OperatorTok{=}\StringTok{"是否使用SwanLab进行实验跟踪"}\NormalTok{)}
\NormalTok{    parser.add\_argument(}\StringTok{"{-}{-}num\_workers"}\NormalTok{, }\BuiltInTok{type}\OperatorTok{=}\BuiltInTok{int}\NormalTok{, default}\OperatorTok{=}\DecValTok{8}\NormalTok{, }\BuiltInTok{help}\OperatorTok{=}\StringTok{"数据加载的工作进程数"}\NormalTok{)}
\NormalTok{    parser.add\_argument(}\StringTok{"{-}{-}data\_path"}\NormalTok{, }\BuiltInTok{type}\OperatorTok{=}\BuiltInTok{str}\NormalTok{, default}\OperatorTok{=}\StringTok{"./seq\_monkey\_datawhale.jsonl"}\NormalTok{, }\BuiltInTok{help}\OperatorTok{=}\StringTok{"训练数据路径"}\NormalTok{)}
    
    \CommentTok{\# 训练优化参数}
\NormalTok{    parser.add\_argument(}\StringTok{"{-}{-}accumulation\_steps"}\NormalTok{, }\BuiltInTok{type}\OperatorTok{=}\BuiltInTok{int}\NormalTok{, default}\OperatorTok{=}\DecValTok{8}\NormalTok{, }\BuiltInTok{help}\OperatorTok{=}\StringTok{"梯度累积步数"}\NormalTok{)}
\NormalTok{    parser.add\_argument(}\StringTok{"{-}{-}grad\_clip"}\NormalTok{, }\BuiltInTok{type}\OperatorTok{=}\BuiltInTok{float}\NormalTok{, default}\OperatorTok{=}\FloatTok{1.0}\NormalTok{, }\BuiltInTok{help}\OperatorTok{=}\StringTok{"梯度裁剪阈值"}\NormalTok{)}
\NormalTok{    parser.add\_argument(}\StringTok{"{-}{-}warmup\_iters"}\NormalTok{, }\BuiltInTok{type}\OperatorTok{=}\BuiltInTok{int}\NormalTok{, default}\OperatorTok{=}\DecValTok{0}\NormalTok{, }\BuiltInTok{help}\OperatorTok{=}\StringTok{"学习率预热迭代次数"}\NormalTok{)}
    
    \CommentTok{\# 日志和保存参数}
\NormalTok{    parser.add\_argument(}\StringTok{"{-}{-}log\_interval"}\NormalTok{, }\BuiltInTok{type}\OperatorTok{=}\BuiltInTok{int}\NormalTok{, default}\OperatorTok{=}\DecValTok{100}\NormalTok{, }\BuiltInTok{help}\OperatorTok{=}\StringTok{"日志记录间隔"}\NormalTok{)}
\NormalTok{    parser.add\_argument(}\StringTok{"{-}{-}save\_interval"}\NormalTok{, }\BuiltInTok{type}\OperatorTok{=}\BuiltInTok{int}\NormalTok{, default}\OperatorTok{=}\DecValTok{1000}\NormalTok{, }\BuiltInTok{help}\OperatorTok{=}\StringTok{"模型保存间隔"}\NormalTok{)}
    
    \CommentTok{\# 多GPU训练参数}
\NormalTok{    parser.add\_argument(}\StringTok{"{-}{-}gpus"}\NormalTok{, }\BuiltInTok{type}\OperatorTok{=}\BuiltInTok{str}\NormalTok{, default}\OperatorTok{=}\StringTok{\textquotesingle{}0,1,2,3,4,5,6,7\textquotesingle{}}\NormalTok{, }\BuiltInTok{help}\OperatorTok{=}\StringTok{"使用的GPU ID,用逗号分隔 (例如: \textquotesingle{}0,1,2\textquotesingle{})"}\NormalTok{)}

\NormalTok{    args }\OperatorTok{=}\NormalTok{ parser.parse\_args()}

    \CommentTok{\# ==================== GPU环境设置 ====================}
    \CommentTok{\# 设置可见的GPU设备}
    \ControlFlowTok{if}\NormalTok{ args.gpus }\KeywordTok{is} \KeywordTok{not} \VariableTok{None}\NormalTok{:}
\NormalTok{        os.environ[}\StringTok{"CUDA\_VISIBLE\_DEVICES"}\NormalTok{] }\OperatorTok{=}\NormalTok{ args.gpus}
        \CommentTok{\# 自动设置主设备为第一个可用GPU}
        \ControlFlowTok{if}\NormalTok{ torch.cuda.is\_available():}
\NormalTok{            args.device }\OperatorTok{=} \StringTok{"cuda:0"}
        \ControlFlowTok{else}\NormalTok{:}
\NormalTok{            args.device }\OperatorTok{=} \StringTok{"cpu"}

    \CommentTok{\# ==================== 实验跟踪初始化 ====================}
    \ControlFlowTok{if}\NormalTok{ args.use\_swanlab:}
        \CommentTok{\# 注意:使用前需要先登录 swanlab.login(api\_key=\textquotesingle{}your key\textquotesingle{})}
\NormalTok{        run }\OperatorTok{=}\NormalTok{ swanlab.init(}
\NormalTok{            project}\OperatorTok{=}\StringTok{"Happy{-}LLM"}\NormalTok{,  }\CommentTok{\# 项目名称}
\NormalTok{            experiment\_name}\OperatorTok{=}\StringTok{"Pretrain{-}215M"}\NormalTok{,  }\CommentTok{\# 实验名称}
\NormalTok{            config}\OperatorTok{=}\NormalTok{args,  }\CommentTok{\# 保存所有超参数}
\NormalTok{        )}

    \CommentTok{\# ==================== 模型配置 ====================}
    \CommentTok{\# 定义语言模型的配置参数}
\NormalTok{    lm\_config }\OperatorTok{=}\NormalTok{ ModelConfig(}
\NormalTok{        dim}\OperatorTok{=}\DecValTok{1024}\NormalTok{,      }\CommentTok{\# 模型维度}
\NormalTok{        n\_layers}\OperatorTok{=}\DecValTok{18}\NormalTok{,   }\CommentTok{\# Transformer层数}
\NormalTok{    )}

    \CommentTok{\# ==================== 训练环境设置 ====================}
\NormalTok{    max\_seq\_len }\OperatorTok{=}\NormalTok{ lm\_config.max\_seq\_len  }\CommentTok{\# 最大序列长度}
\NormalTok{    args.save\_dir }\OperatorTok{=}\NormalTok{ os.path.join(args.out\_dir)  }\CommentTok{\# 模型保存目录}
    
    \CommentTok{\# 创建必要的目录}
\NormalTok{    os.makedirs(args.save\_dir, exist\_ok}\OperatorTok{=}\VariableTok{True}\NormalTok{)}
\NormalTok{    os.makedirs(args.out\_dir, exist\_ok}\OperatorTok{=}\VariableTok{True}\NormalTok{)}
    
    \CommentTok{\# 设置随机种子以确保结果可复现}
\NormalTok{    torch.manual\_seed(}\DecValTok{42}\NormalTok{)}
    
    \CommentTok{\# 确定设备类型(用于选择合适的上下文管理器)}
\NormalTok{    device\_type }\OperatorTok{=} \StringTok{"cuda"} \ControlFlowTok{if} \StringTok{"cuda"} \KeywordTok{in}\NormalTok{ args.device }\ControlFlowTok{else} \StringTok{"cpu"}

    \CommentTok{\# 设置混合精度训练的上下文管理器}
    \CommentTok{\# CPU训练时使用nullcontext,GPU训练时使用autocast}
\NormalTok{    ctx }\OperatorTok{=}\NormalTok{ nullcontext() }\ControlFlowTok{if}\NormalTok{ device\_type }\OperatorTok{==} \StringTok{"cpu"} \ControlFlowTok{else}\NormalTok{ torch.cuda.amp.autocast()}

    \CommentTok{\# ==================== 模型和数据初始化 ====================}
    \CommentTok{\# 初始化模型和分词器}
\NormalTok{    model, tokenizer }\OperatorTok{=}\NormalTok{ init\_model()}
    
    \CommentTok{\# 创建训练数据集}
\NormalTok{    train\_ds }\OperatorTok{=}\NormalTok{ PretrainDataset(args.data\_path, tokenizer, max\_length}\OperatorTok{=}\NormalTok{max\_seq\_len)}
    
    \CommentTok{\# 创建数据加载器}
\NormalTok{    train\_loader }\OperatorTok{=}\NormalTok{ DataLoader(}
\NormalTok{        train\_ds,}
\NormalTok{        batch\_size}\OperatorTok{=}\NormalTok{args.batch\_size,  }\CommentTok{\# 批次大小}
\NormalTok{        pin\_memory}\OperatorTok{=}\VariableTok{True}\NormalTok{,             }\CommentTok{\# 将数据加载到固定内存中,加速GPU传输}
\NormalTok{        drop\_last}\OperatorTok{=}\VariableTok{False}\NormalTok{,             }\CommentTok{\# 不丢弃最后一个不完整的批次}
\NormalTok{        shuffle}\OperatorTok{=}\VariableTok{True}\NormalTok{,                }\CommentTok{\# 随机打乱数据}
\NormalTok{        num\_workers}\OperatorTok{=}\NormalTok{args.num\_workers }\CommentTok{\# 数据加载的并行工作进程数}
\NormalTok{    )}

    \CommentTok{\# ==================== 优化器和训练组件初始化 ====================}
    \CommentTok{\# 初始化混合精度训练的梯度缩放器}
    \CommentTok{\# 只有在使用float16或bfloat16时才启用}
\NormalTok{    scaler }\OperatorTok{=}\NormalTok{ torch.cuda.amp.GradScaler(enabled}\OperatorTok{=}\NormalTok{(args.dtype }\KeywordTok{in}\NormalTok{ [}\StringTok{\textquotesingle{}float16\textquotesingle{}}\NormalTok{, }\StringTok{\textquotesingle{}bfloat16\textquotesingle{}}\NormalTok{]))}
    
    \CommentTok{\# 初始化Adam优化器}
\NormalTok{    optimizer }\OperatorTok{=}\NormalTok{ optim.Adam(model.parameters(), lr}\OperatorTok{=}\NormalTok{args.learning\_rate)}

    \CommentTok{\# ==================== 开始训练 ====================}
    \CommentTok{\# 计算每个epoch的迭代次数}
\NormalTok{    iter\_per\_epoch }\OperatorTok{=} \BuiltInTok{len}\NormalTok{(train\_loader)}
    
    \CommentTok{\# 开始训练循环}
    \ControlFlowTok{for}\NormalTok{ epoch }\KeywordTok{in} \BuiltInTok{range}\NormalTok{(args.epochs):}
\NormalTok{        train\_epoch(epoch)}
\end{Highlighting}
\end{Shaded}

\subsubsection{5.3.5 SFT 训练}\label{sft-ux8badux7ec3}

SFT 训练和预训练的代码基本一样,只是导入的 Dataset
不一样。在这里我们使用的是 SFTDataset,用于多轮对话的训练。

\begin{Shaded}
\begin{Highlighting}[]
\ImportTok{import}\NormalTok{ os}
\ImportTok{import}\NormalTok{ platform}
\ImportTok{import}\NormalTok{ argparse}
\ImportTok{import}\NormalTok{ time}
\ImportTok{import}\NormalTok{ warnings}
\ImportTok{import}\NormalTok{ math}
\ImportTok{import}\NormalTok{ pandas }\ImportTok{as}\NormalTok{ pd}
\ImportTok{import}\NormalTok{ torch}
\ImportTok{from}\NormalTok{ torch }\ImportTok{import}\NormalTok{ optim}
\ImportTok{from}\NormalTok{ torch.utils.data }\ImportTok{import}\NormalTok{ DataLoader}
\ImportTok{from}\NormalTok{ contextlib }\ImportTok{import}\NormalTok{ nullcontext}

\ImportTok{from}\NormalTok{ transformers }\ImportTok{import}\NormalTok{ AutoTokenizer}

\ImportTok{from}\NormalTok{ k\_model }\ImportTok{import}\NormalTok{ ModelConfig, Transformer}
\ImportTok{from}\NormalTok{ dataset }\ImportTok{import}\NormalTok{ SFTDataset}

\ImportTok{import}\NormalTok{ swanlab}

\CommentTok{\# 忽略警告}
\NormalTok{warnings.filterwarnings(}\StringTok{\textquotesingle{}ignore\textquotesingle{}}\NormalTok{)}


\KeywordTok{def}\NormalTok{ Logger(content):}
    \CommentTok{"""日志记录器"""}
    \BuiltInTok{print}\NormalTok{(content)}

\KeywordTok{def}\NormalTok{ get\_lr(it, }\BuiltInTok{all}\NormalTok{):}
    \CommentTok{"""获取学习率"""}
    \CommentTok{\# 1) linear warmup for warmup\_iters steps}
    \CommentTok{\# 1) 预热迭代的线性预热}
\NormalTok{    warmup\_iters }\OperatorTok{=}\NormalTok{ args.warmup\_iters}
\NormalTok{    lr\_decay\_iters }\OperatorTok{=} \BuiltInTok{all}
\NormalTok{    min\_lr }\OperatorTok{=}\NormalTok{ args.learning\_rate }\OperatorTok{/} \DecValTok{10}

    \ControlFlowTok{if}\NormalTok{ it }\OperatorTok{\textless{}}\NormalTok{ warmup\_iters:}
        \ControlFlowTok{return}\NormalTok{ args.learning\_rate }\OperatorTok{*}\NormalTok{ it }\OperatorTok{/}\NormalTok{ warmup\_iters}
    
    \CommentTok{\# 2) if it \textgreater{} lr\_decay\_iters, return min learning rate}
    \CommentTok{\# 2) 如果迭代次数超过学习率衰减迭代次数,则返回最小学习率}
    \ControlFlowTok{if}\NormalTok{ it }\OperatorTok{\textgreater{}}\NormalTok{ lr\_decay\_iters:}
        \ControlFlowTok{return}\NormalTok{ min\_lr}
    
    \CommentTok{\# 3) in between, use cosine decay down to min learning rate}
    \CommentTok{\# 3) 在两者之间,使用余弦衰减至最小学习率}
\NormalTok{    decay\_ratio }\OperatorTok{=}\NormalTok{ (it }\OperatorTok{{-}}\NormalTok{ warmup\_iters) }\OperatorTok{/}\NormalTok{ (lr\_decay\_iters }\OperatorTok{{-}}\NormalTok{ warmup\_iters)}
    \ControlFlowTok{assert} \DecValTok{0} \OperatorTok{\textless{}=}\NormalTok{ decay\_ratio }\OperatorTok{\textless{}=} \DecValTok{1}
\NormalTok{    coeff }\OperatorTok{=} \FloatTok{0.5} \OperatorTok{*}\NormalTok{ (}\FloatTok{1.0} \OperatorTok{+}\NormalTok{ math.cos(math.pi }\OperatorTok{*}\NormalTok{ decay\_ratio))}
    \ControlFlowTok{return}\NormalTok{ min\_lr }\OperatorTok{+}\NormalTok{ coeff }\OperatorTok{*}\NormalTok{ (args.learning\_rate }\OperatorTok{{-}}\NormalTok{ min\_lr)}

\KeywordTok{def}\NormalTok{ train\_epoch(epoch):}
    \CommentTok{"""训练一个epoch"""}
\NormalTok{    start\_time }\OperatorTok{=}\NormalTok{ time.time()}
    \ControlFlowTok{for}\NormalTok{ step, (X, Y, loss\_mask) }\KeywordTok{in} \BuiltInTok{enumerate}\NormalTok{(train\_loader):}
\NormalTok{        X }\OperatorTok{=}\NormalTok{ X.to(args.device)}
\NormalTok{        Y }\OperatorTok{=}\NormalTok{ Y.to(args.device)}
\NormalTok{        loss\_mask }\OperatorTok{=}\NormalTok{ loss\_mask.to(args.device)}

        \CommentTok{\# 获取学习率并更新优化器}
\NormalTok{        lr }\OperatorTok{=}\NormalTok{ get\_lr(epoch }\OperatorTok{*}\NormalTok{ iter\_per\_epoch }\OperatorTok{+}\NormalTok{ step, args.epochs }\OperatorTok{*}\NormalTok{ iter\_per\_epoch)}
        \ControlFlowTok{for}\NormalTok{ param\_group }\KeywordTok{in}\NormalTok{ optimizer.param\_groups:}
\NormalTok{            param\_group[}\StringTok{\textquotesingle{}lr\textquotesingle{}}\NormalTok{] }\OperatorTok{=}\NormalTok{ lr}

        \CommentTok{\# 前向传播}
        \ControlFlowTok{with}\NormalTok{ ctx:}
\NormalTok{            out }\OperatorTok{=}\NormalTok{ model(X, Y)}
\NormalTok{            loss }\OperatorTok{=}\NormalTok{ out.last\_loss }\OperatorTok{/}\NormalTok{ args.accumulation\_steps}
\NormalTok{            loss\_mask }\OperatorTok{=}\NormalTok{ loss\_mask.view(}\OperatorTok{{-}}\DecValTok{1}\NormalTok{)}
\NormalTok{            loss }\OperatorTok{=}\NormalTok{ torch.}\BuiltInTok{sum}\NormalTok{(loss }\OperatorTok{*}\NormalTok{ loss\_mask) }\OperatorTok{/}\NormalTok{ loss\_mask.}\BuiltInTok{sum}\NormalTok{()}

        \CommentTok{\# 反向传播}
\NormalTok{        scaler.scale(loss).backward()}

        \CommentTok{\# 更新权重}
        \ControlFlowTok{if}\NormalTok{ (step }\OperatorTok{+} \DecValTok{1}\NormalTok{) }\OperatorTok{\%}\NormalTok{ args.accumulation\_steps }\OperatorTok{==} \DecValTok{0}\NormalTok{:}
\NormalTok{            scaler.unscale\_(optimizer)}
\NormalTok{            torch.nn.utils.clip\_grad\_norm\_(model.parameters(), args.grad\_clip)}

\NormalTok{            scaler.step(optimizer)}
\NormalTok{            scaler.update()}

\NormalTok{            optimizer.zero\_grad(set\_to\_none}\OperatorTok{=}\VariableTok{True}\NormalTok{)}

        \CommentTok{\# 打印日志}
        \ControlFlowTok{if}\NormalTok{ step }\OperatorTok{\%}\NormalTok{ args.log\_interval }\OperatorTok{==} \DecValTok{0}\NormalTok{:}
\NormalTok{            spend\_time }\OperatorTok{=}\NormalTok{ time.time() }\OperatorTok{{-}}\NormalTok{ start\_time}
\NormalTok{            Logger(}
                \StringTok{\textquotesingle{}Epoch:[}\SpecialCharTok{\{\}}\StringTok{/}\SpecialCharTok{\{\}}\StringTok{](}\SpecialCharTok{\{\}}\StringTok{/}\SpecialCharTok{\{\}}\StringTok{) loss:}\SpecialCharTok{\{:.3f\}}\StringTok{ lr:}\SpecialCharTok{\{:.7f\}}\StringTok{ epoch\_Time:}\SpecialCharTok{\{\}}\StringTok{min:\textquotesingle{}}\NormalTok{.}\BuiltInTok{format}\NormalTok{(}
\NormalTok{                    epoch }\OperatorTok{+} \DecValTok{1}\NormalTok{,}
\NormalTok{                    args.epochs,}
\NormalTok{                    step,}
\NormalTok{                    iter\_per\_epoch,}
\NormalTok{                    loss.item() }\OperatorTok{*}\NormalTok{ args.accumulation\_steps,}
\NormalTok{                    optimizer.param\_groups[}\OperatorTok{{-}}\DecValTok{1}\NormalTok{][}\StringTok{\textquotesingle{}lr\textquotesingle{}}\NormalTok{],}
\NormalTok{                    spend\_time }\OperatorTok{/}\NormalTok{ (step }\OperatorTok{+} \DecValTok{1}\NormalTok{) }\OperatorTok{*}\NormalTok{ iter\_per\_epoch }\OperatorTok{//} \DecValTok{60} \OperatorTok{{-}}\NormalTok{ spend\_time }\OperatorTok{//} \DecValTok{60}\NormalTok{))}
            \ControlFlowTok{if}\NormalTok{ args.use\_swanlab:}
\NormalTok{                swanlab.log(\{}
                    \StringTok{"loss"}\NormalTok{: loss.item() }\OperatorTok{*}\NormalTok{ args.accumulation\_steps,}
                    \StringTok{"lr"}\NormalTok{: optimizer.param\_groups[}\OperatorTok{{-}}\DecValTok{1}\NormalTok{][}\StringTok{\textquotesingle{}lr\textquotesingle{}}\NormalTok{]}
\NormalTok{                \})}

        \CommentTok{\# 保存模型}
        \ControlFlowTok{if}\NormalTok{ (step }\OperatorTok{+} \DecValTok{1}\NormalTok{) }\OperatorTok{\%}\NormalTok{ args.save\_interval }\OperatorTok{==} \DecValTok{0}\NormalTok{:}
\NormalTok{            model.}\BuiltInTok{eval}\NormalTok{()}
\NormalTok{            ckp }\OperatorTok{=} \SpecialStringTok{f\textquotesingle{}}\SpecialCharTok{\{}\NormalTok{args}\SpecialCharTok{.}\NormalTok{save\_dir}\SpecialCharTok{\}}\SpecialStringTok{/sft\_dim}\SpecialCharTok{\{}\NormalTok{lm\_config}\SpecialCharTok{.}\NormalTok{dim}\SpecialCharTok{\}}\SpecialStringTok{\_layers}\SpecialCharTok{\{}\NormalTok{lm\_config}\SpecialCharTok{.}\NormalTok{n\_layers}\SpecialCharTok{\}}\SpecialStringTok{\_vocab\_size}\SpecialCharTok{\{}\NormalTok{lm\_config}\SpecialCharTok{.}\NormalTok{vocab\_size}\SpecialCharTok{\}}\SpecialStringTok{.pth\textquotesingle{}}

            \CommentTok{\# 处理多卡保存}
\NormalTok{            state\_dict }\OperatorTok{=}\NormalTok{ model.module.state\_dict() }\ControlFlowTok{if} \BuiltInTok{isinstance}\NormalTok{(model, torch.nn.DataParallel) }\ControlFlowTok{else}\NormalTok{ model.state\_dict()}
\NormalTok{            torch.save(state\_dict, ckp)}
\NormalTok{            model.train()}
        
        \CommentTok{\# 定期保存模型}
        \ControlFlowTok{if}\NormalTok{ (step }\OperatorTok{+} \DecValTok{1}\NormalTok{) }\OperatorTok{\%} \DecValTok{20000} \OperatorTok{==} \DecValTok{0}\NormalTok{:}
\NormalTok{            model.}\BuiltInTok{eval}\NormalTok{()}
\NormalTok{            ckp }\OperatorTok{=} \SpecialStringTok{f\textquotesingle{}}\SpecialCharTok{\{}\NormalTok{args}\SpecialCharTok{.}\NormalTok{save\_dir}\SpecialCharTok{\}}\SpecialStringTok{/sft\_dim}\SpecialCharTok{\{}\NormalTok{lm\_config}\SpecialCharTok{.}\NormalTok{dim}\SpecialCharTok{\}}\SpecialStringTok{\_layers}\SpecialCharTok{\{}\NormalTok{lm\_config}\SpecialCharTok{.}\NormalTok{n\_layers}\SpecialCharTok{\}}\SpecialStringTok{\_vocab\_size}\SpecialCharTok{\{}\NormalTok{lm\_config}\SpecialCharTok{.}\NormalTok{vocab\_size}\SpecialCharTok{\}}\SpecialStringTok{\_step}\SpecialCharTok{\{}\NormalTok{step}\OperatorTok{+}\DecValTok{1}\SpecialCharTok{\}}\SpecialStringTok{.pth\textquotesingle{}}

\NormalTok{            state\_dict }\OperatorTok{=}\NormalTok{ model.module.state\_dict() }\ControlFlowTok{if} \BuiltInTok{isinstance}\NormalTok{(model, torch.nn.DataParallel) }\ControlFlowTok{else}\NormalTok{ model.state\_dict()}
\NormalTok{            torch.save(state\_dict, ckp)}
\NormalTok{            model.train()}


\KeywordTok{def}\NormalTok{ init\_model():}
    \CommentTok{"""初始化模型"""}
    \KeywordTok{def}\NormalTok{ count\_parameters(model):}
        \CommentTok{"""计算模型参数量"""}
        \ControlFlowTok{return} \BuiltInTok{sum}\NormalTok{(p.numel() }\ControlFlowTok{for}\NormalTok{ p }\KeywordTok{in}\NormalTok{ model.parameters() }\ControlFlowTok{if}\NormalTok{ p.requires\_grad)}

    \CommentTok{\# 加载分词器}
\NormalTok{    tokenizer }\OperatorTok{=}\NormalTok{ AutoTokenizer.from\_pretrained(}\StringTok{\textquotesingle{}./tokenizer\_k/\textquotesingle{}}\NormalTok{)}

    \CommentTok{\# 初始化模型}
\NormalTok{    model }\OperatorTok{=}\NormalTok{ Transformer(lm\_config)}

    \CommentTok{\# 加载预训练权重}
\NormalTok{    ckp }\OperatorTok{=} \StringTok{\textquotesingle{}./base\_model\_215M/pretrain\_1024\_18\_6144.pth\textquotesingle{}}
\NormalTok{    state\_dict }\OperatorTok{=}\NormalTok{ torch.load(ckp, map\_location}\OperatorTok{=}\NormalTok{args.device)}
\NormalTok{    unwanted\_prefix }\OperatorTok{=} \StringTok{\textquotesingle{}\_orig\_mod.\textquotesingle{}}
    \ControlFlowTok{for}\NormalTok{ k, v }\KeywordTok{in} \BuiltInTok{list}\NormalTok{(state\_dict.items()):}
        \ControlFlowTok{if}\NormalTok{ k.startswith(unwanted\_prefix):}
\NormalTok{            state\_dict[k[}\BuiltInTok{len}\NormalTok{(unwanted\_prefix):]] }\OperatorTok{=}\NormalTok{ state\_dict.pop(k)}
\NormalTok{    model.load\_state\_dict(state\_dict, strict}\OperatorTok{=}\VariableTok{False}\NormalTok{)}
    
    \CommentTok{\# 多卡初始化}
\NormalTok{    num\_gpus }\OperatorTok{=}\NormalTok{ torch.cuda.device\_count()}
    \ControlFlowTok{if}\NormalTok{ num\_gpus }\OperatorTok{\textgreater{}} \DecValTok{1}\NormalTok{:}
\NormalTok{        Logger(}\SpecialStringTok{f"Using }\SpecialCharTok{\{}\NormalTok{num\_gpus}\SpecialCharTok{\}}\SpecialStringTok{ GPUs with DataParallel!"}\NormalTok{)}
\NormalTok{        model }\OperatorTok{=}\NormalTok{ torch.nn.DataParallel(model)}
    
\NormalTok{    model }\OperatorTok{=}\NormalTok{ model.to(args.device)}
\NormalTok{    Logger(}\SpecialStringTok{f\textquotesingle{}LLM总参数量:}\SpecialCharTok{\{}\NormalTok{count\_parameters(model) }\OperatorTok{/} \FloatTok{1e6}\SpecialCharTok{:.3f\}}\SpecialStringTok{ 百万\textquotesingle{}}\NormalTok{)}
    \ControlFlowTok{return}\NormalTok{ model, tokenizer}


\ControlFlowTok{if} \VariableTok{\_\_name\_\_} \OperatorTok{==} \StringTok{"\_\_main\_\_"}\NormalTok{:}
\NormalTok{    parser }\OperatorTok{=}\NormalTok{ argparse.ArgumentParser(description}\OperatorTok{=}\StringTok{"Tiny{-}LLM Pretraining"}\NormalTok{)}
\NormalTok{    parser.add\_argument(}\StringTok{"{-}{-}out\_dir"}\NormalTok{, }\BuiltInTok{type}\OperatorTok{=}\BuiltInTok{str}\NormalTok{, default}\OperatorTok{=}\StringTok{"sft\_model\_215M"}\NormalTok{, }\BuiltInTok{help}\OperatorTok{=}\StringTok{"输出目录"}\NormalTok{)}
\NormalTok{    parser.add\_argument(}\StringTok{"{-}{-}epochs"}\NormalTok{, }\BuiltInTok{type}\OperatorTok{=}\BuiltInTok{int}\NormalTok{, default}\OperatorTok{=}\DecValTok{1}\NormalTok{, }\BuiltInTok{help}\OperatorTok{=}\StringTok{"训练轮数"}\NormalTok{)}
\NormalTok{    parser.add\_argument(}\StringTok{"{-}{-}batch\_size"}\NormalTok{, }\BuiltInTok{type}\OperatorTok{=}\BuiltInTok{int}\NormalTok{, default}\OperatorTok{=}\DecValTok{64}\NormalTok{, }\BuiltInTok{help}\OperatorTok{=}\StringTok{"批处理大小"}\NormalTok{)}
\NormalTok{    parser.add\_argument(}\StringTok{"{-}{-}learning\_rate"}\NormalTok{, }\BuiltInTok{type}\OperatorTok{=}\BuiltInTok{float}\NormalTok{, default}\OperatorTok{=}\FloatTok{2e{-}4}\NormalTok{, }\BuiltInTok{help}\OperatorTok{=}\StringTok{"学习率"}\NormalTok{)}
\NormalTok{    parser.add\_argument(}\StringTok{"{-}{-}device"}\NormalTok{, }\BuiltInTok{type}\OperatorTok{=}\BuiltInTok{str}\NormalTok{, default}\OperatorTok{=}\StringTok{"cuda:0"} \ControlFlowTok{if}\NormalTok{ torch.cuda.is\_available() }\ControlFlowTok{else} \StringTok{"cpu"}\NormalTok{, }\BuiltInTok{help}\OperatorTok{=}\StringTok{"使用的设备"}\NormalTok{)}
\NormalTok{    parser.add\_argument(}\StringTok{"{-}{-}dtype"}\NormalTok{, }\BuiltInTok{type}\OperatorTok{=}\BuiltInTok{str}\NormalTok{, default}\OperatorTok{=}\StringTok{"bfloat16"}\NormalTok{, }\BuiltInTok{help}\OperatorTok{=}\StringTok{"数据类型"}\NormalTok{)}
\NormalTok{    parser.add\_argument(}\StringTok{"{-}{-}use\_swanlab"}\NormalTok{, action}\OperatorTok{=}\StringTok{"store\_true"}\NormalTok{, }\BuiltInTok{help}\OperatorTok{=}\StringTok{"是否使用SwanLab进行实验跟踪"}\NormalTok{)}
\NormalTok{    parser.add\_argument(}\StringTok{"{-}{-}num\_workers"}\NormalTok{, }\BuiltInTok{type}\OperatorTok{=}\BuiltInTok{int}\NormalTok{, default}\OperatorTok{=}\DecValTok{8}\NormalTok{, }\BuiltInTok{help}\OperatorTok{=}\StringTok{"数据加载的工作进程数"}\NormalTok{)}
\NormalTok{    parser.add\_argument(}\StringTok{"{-}{-}data\_path"}\NormalTok{, }\BuiltInTok{type}\OperatorTok{=}\BuiltInTok{str}\NormalTok{, default}\OperatorTok{=}\StringTok{"./BelleGroup\_sft.jsonl"}\NormalTok{, }\BuiltInTok{help}\OperatorTok{=}\StringTok{"训练数据路径"}\NormalTok{)}
\NormalTok{    parser.add\_argument(}\StringTok{"{-}{-}accumulation\_steps"}\NormalTok{, }\BuiltInTok{type}\OperatorTok{=}\BuiltInTok{int}\NormalTok{, default}\OperatorTok{=}\DecValTok{8}\NormalTok{, }\BuiltInTok{help}\OperatorTok{=}\StringTok{"梯度累积步数"}\NormalTok{)}
\NormalTok{    parser.add\_argument(}\StringTok{"{-}{-}grad\_clip"}\NormalTok{, }\BuiltInTok{type}\OperatorTok{=}\BuiltInTok{float}\NormalTok{, default}\OperatorTok{=}\FloatTok{1.0}\NormalTok{, }\BuiltInTok{help}\OperatorTok{=}\StringTok{"梯度裁剪阈值"}\NormalTok{)}
\NormalTok{    parser.add\_argument(}\StringTok{"{-}{-}warmup\_iters"}\NormalTok{, }\BuiltInTok{type}\OperatorTok{=}\BuiltInTok{int}\NormalTok{, default}\OperatorTok{=}\DecValTok{0}\NormalTok{, }\BuiltInTok{help}\OperatorTok{=}\StringTok{"预热迭代次数"}\NormalTok{)}
\NormalTok{    parser.add\_argument(}\StringTok{"{-}{-}log\_interval"}\NormalTok{, }\BuiltInTok{type}\OperatorTok{=}\BuiltInTok{int}\NormalTok{, default}\OperatorTok{=}\DecValTok{100}\NormalTok{, }\BuiltInTok{help}\OperatorTok{=}\StringTok{"日志记录间隔"}\NormalTok{)}
\NormalTok{    parser.add\_argument(}\StringTok{"{-}{-}save\_interval"}\NormalTok{, }\BuiltInTok{type}\OperatorTok{=}\BuiltInTok{int}\NormalTok{, default}\OperatorTok{=}\DecValTok{1000}\NormalTok{, }\BuiltInTok{help}\OperatorTok{=}\StringTok{"模型保存间隔"}\NormalTok{)}
    \CommentTok{\# 添加多卡参数}
\NormalTok{    parser.add\_argument(}\StringTok{"{-}{-}gpus"}\NormalTok{, }\BuiltInTok{type}\OperatorTok{=}\BuiltInTok{str}\NormalTok{, default}\OperatorTok{=}\StringTok{\textquotesingle{}0,1,2,3,4,5,6,7\textquotesingle{}}\NormalTok{, }\BuiltInTok{help}\OperatorTok{=}\StringTok{"逗号分隔的GPU ID (例如 \textquotesingle{}0,1,2\textquotesingle{})"}\NormalTok{)}

\NormalTok{    args }\OperatorTok{=}\NormalTok{ parser.parse\_args()}

    \CommentTok{\# 设置可见GPU}
    \ControlFlowTok{if}\NormalTok{ args.gpus }\KeywordTok{is} \KeywordTok{not} \VariableTok{None}\NormalTok{:}
\NormalTok{        os.environ[}\StringTok{"CUDA\_VISIBLE\_DEVICES"}\NormalTok{] }\OperatorTok{=}\NormalTok{ args.gpus}
        \CommentTok{\# 自动设置主设备为第一个GPU}
        \ControlFlowTok{if}\NormalTok{ torch.cuda.is\_available():}
\NormalTok{            args.device }\OperatorTok{=} \StringTok{"cuda:0"}
        \ControlFlowTok{else}\NormalTok{:}
\NormalTok{            args.device }\OperatorTok{=} \StringTok{"cpu"}

    \CommentTok{\# 初始化swanlab}
    \ControlFlowTok{if}\NormalTok{ args.use\_swanlab:}
\NormalTok{        run }\OperatorTok{=}\NormalTok{ swanlab.init(}
\NormalTok{            project}\OperatorTok{=}\StringTok{"Happy{-}LLM"}\NormalTok{,}
\NormalTok{            experiment\_name}\OperatorTok{=}\StringTok{"SFT{-}215M"}\NormalTok{,}
\NormalTok{            config}\OperatorTok{=}\NormalTok{args,}
\NormalTok{        )}

    \CommentTok{\# 模型配置}
\NormalTok{    lm\_config }\OperatorTok{=}\NormalTok{ ModelConfig(}
\NormalTok{        dim}\OperatorTok{=}\DecValTok{1024}\NormalTok{,}
\NormalTok{        n\_layers}\OperatorTok{=}\DecValTok{18}\NormalTok{,}
\NormalTok{    )}
\NormalTok{    max\_seq\_len }\OperatorTok{=}\NormalTok{ lm\_config.max\_seq\_len}
\NormalTok{    args.save\_dir }\OperatorTok{=}\NormalTok{ os.path.join(args.out\_dir)}
\NormalTok{    os.makedirs(args.save\_dir, exist\_ok}\OperatorTok{=}\VariableTok{True}\NormalTok{)}
\NormalTok{    os.makedirs(args.out\_dir, exist\_ok}\OperatorTok{=}\VariableTok{True}\NormalTok{)}
\NormalTok{    torch.manual\_seed(}\DecValTok{42}\NormalTok{)}
\NormalTok{    device\_type }\OperatorTok{=} \StringTok{"cuda"} \ControlFlowTok{if} \StringTok{"cuda"} \KeywordTok{in}\NormalTok{ args.device }\ControlFlowTok{else} \StringTok{"cpu"}

    \CommentTok{\# 上下文管理器}
\NormalTok{    ctx }\OperatorTok{=}\NormalTok{ nullcontext() }\ControlFlowTok{if}\NormalTok{ device\_type }\OperatorTok{==} \StringTok{"cpu"} \ControlFlowTok{else}\NormalTok{ torch.cuda.amp.autocast()}

    \CommentTok{\# 初始化模型和分词器}
\NormalTok{    model, tokenizer }\OperatorTok{=}\NormalTok{ init\_model()}
    
    \CommentTok{\# 创建数据集和数据加载器}
\NormalTok{    train\_ds }\OperatorTok{=}\NormalTok{ SFTDataset(args.data\_path, tokenizer, max\_length}\OperatorTok{=}\NormalTok{max\_seq\_len)}
\NormalTok{    train\_loader }\OperatorTok{=}\NormalTok{ DataLoader(}
\NormalTok{        train\_ds,}
\NormalTok{        batch\_size}\OperatorTok{=}\NormalTok{args.batch\_size,}
\NormalTok{        pin\_memory}\OperatorTok{=}\VariableTok{True}\NormalTok{,}
\NormalTok{        drop\_last}\OperatorTok{=}\VariableTok{False}\NormalTok{,}
\NormalTok{        shuffle}\OperatorTok{=}\VariableTok{True}\NormalTok{,}
\NormalTok{        num\_workers}\OperatorTok{=}\NormalTok{args.num\_workers}
\NormalTok{    )}

    \CommentTok{\# 缩放器和优化器}
\NormalTok{    scaler }\OperatorTok{=}\NormalTok{ torch.cuda.amp.GradScaler(enabled}\OperatorTok{=}\NormalTok{(args.dtype }\KeywordTok{in}\NormalTok{ [}\StringTok{\textquotesingle{}float16\textquotesingle{}}\NormalTok{, }\StringTok{\textquotesingle{}bfloat16\textquotesingle{}}\NormalTok{]))}
\NormalTok{    optimizer }\OperatorTok{=}\NormalTok{ optim.AdamW(model.parameters(), lr}\OperatorTok{=}\NormalTok{args.learning\_rate)}

    \CommentTok{\# 开始训练}
\NormalTok{    iter\_per\_epoch }\OperatorTok{=} \BuiltInTok{len}\NormalTok{(train\_loader)}
    \ControlFlowTok{for}\NormalTok{ epoch }\KeywordTok{in} \BuiltInTok{range}\NormalTok{(args.epochs):}
\NormalTok{        train\_epoch(epoch)}
\end{Highlighting}
\end{Shaded}

\subsubsection{5.3.6
使用模型生成文本}\label{ux4f7fux7528ux6a21ux578bux751fux6210ux6587ux672c}

在模型训练完成后,会在\texttt{output}目录下生成模型文件,这个文件就是我们训练好的模型。我们可以使用以下命令生成文本。

\begin{Shaded}
\begin{Highlighting}[]
\ExtensionTok{python}\NormalTok{ model\_sample.py}
\end{Highlighting}
\end{Shaded}

我们来看下\texttt{model\_sample.py}文件中的代码,这个文件中定义了一个\texttt{TextGenerator}类,用于生成文本。

\begin{Shaded}
\begin{Highlighting}[]
\ImportTok{import}\NormalTok{ os}
\ImportTok{import}\NormalTok{ pickle}
\ImportTok{from}\NormalTok{ contextlib }\ImportTok{import}\NormalTok{ nullcontext}
\ImportTok{import}\NormalTok{ torch}
\ImportTok{from}\NormalTok{ k\_model }\ImportTok{import}\NormalTok{ ModelConfig, Transformer}
\ImportTok{from}\NormalTok{ transformers }\ImportTok{import}\NormalTok{ AutoTokenizer, AutoModelForCausalLM}
\ImportTok{import}\NormalTok{ argparse}

\KeywordTok{class}\NormalTok{ TextGenerator:}
    \KeywordTok{def} \FunctionTok{\_\_init\_\_}\NormalTok{(}\VariableTok{self}\NormalTok{, }
\NormalTok{                 checkpoint}\OperatorTok{=}\StringTok{\textquotesingle{}./base\_model\_215M/pretrain\_1024\_18\_6144.pth\textquotesingle{}}\NormalTok{,  }\CommentTok{\# 模型检查点路径}
\NormalTok{                 tokenizer\_model\_path}\OperatorTok{=}\StringTok{\textquotesingle{}./tokenizer\_k/\textquotesingle{}}\NormalTok{,  }\CommentTok{\# 分词器模型路径}
\NormalTok{                 seed}\OperatorTok{=}\DecValTok{42}\NormalTok{,  }\CommentTok{\# 随机种子,确保可重复性}
\NormalTok{                 device}\OperatorTok{=}\VariableTok{None}\NormalTok{,  }\CommentTok{\# 设备,优先使用 CUDA,如果没有可用的 CUDA,则使用 CPU}
\NormalTok{                 dtype}\OperatorTok{=}\StringTok{"bfloat16"}\NormalTok{):  }\CommentTok{\# 数据类型,默认为 float32,可以选择 float16 或 bfloat16}
        \CommentTok{"""}
\CommentTok{        初始化 TextGenerator 类,加载模型、设置设备和分词器等。}
\CommentTok{        """}
        \CommentTok{\# 模型加载配置}
        \VariableTok{self}\NormalTok{.checkpoint }\OperatorTok{=}\NormalTok{ checkpoint  }\CommentTok{\# 保存的模型检查点路径}
        \VariableTok{self}\NormalTok{.tokenizer\_model\_path }\OperatorTok{=}\NormalTok{ tokenizer\_model\_path  }\CommentTok{\# 分词器模型文件路径}
        \VariableTok{self}\NormalTok{.seed }\OperatorTok{=}\NormalTok{ seed  }\CommentTok{\# 随机数种子,用于生成的可重复性}
        \VariableTok{self}\NormalTok{.device }\OperatorTok{=}\NormalTok{ device }\KeywordTok{or}\NormalTok{ (}\StringTok{\textquotesingle{}cuda:0\textquotesingle{}} \ControlFlowTok{if}\NormalTok{ torch.cuda.is\_available() }\ControlFlowTok{else} \StringTok{\textquotesingle{}cpu\textquotesingle{}}\NormalTok{)  }\CommentTok{\# 根据硬件条件选择设备}
        \VariableTok{self}\NormalTok{.dtype }\OperatorTok{=}\NormalTok{ dtype  }\CommentTok{\# 模型的浮点数类型}
        \VariableTok{self}\NormalTok{.device\_type }\OperatorTok{=} \StringTok{\textquotesingle{}cuda\textquotesingle{}} \ControlFlowTok{if} \StringTok{\textquotesingle{}cuda\textquotesingle{}} \KeywordTok{in} \VariableTok{self}\NormalTok{.device }\ControlFlowTok{else} \StringTok{\textquotesingle{}cpu\textquotesingle{}}  \CommentTok{\# 判断当前设备是否为 CUDA}
        
        \CommentTok{\# 设置随机种子,确保生成的可重复性}
\NormalTok{        torch.manual\_seed(seed)  }\CommentTok{\# 设置 CPU 随机种子}
\NormalTok{        torch.cuda.manual\_seed(seed)  }\CommentTok{\# 设置 CUDA 随机种子}
\NormalTok{        torch.backends.cuda.matmul.allow\_tf32 }\OperatorTok{=} \VariableTok{True}  \CommentTok{\# 允许 CUDA 使用 TF32 精度进行矩阵乘法运算}
\NormalTok{        torch.backends.cudnn.allow\_tf32 }\OperatorTok{=} \VariableTok{True}  \CommentTok{\# 允许 cuDNN 使用 TF32 精度加速}
        
        \CommentTok{\# 根据 dtype 选择适当的自动混合精度上下文}
\NormalTok{        ptdtype }\OperatorTok{=}\NormalTok{ \{}\StringTok{\textquotesingle{}float32\textquotesingle{}}\NormalTok{: torch.float32, }\StringTok{\textquotesingle{}bfloat16\textquotesingle{}}\NormalTok{: torch.bfloat16, }\StringTok{\textquotesingle{}float16\textquotesingle{}}\NormalTok{: torch.float16\}[}\VariableTok{self}\NormalTok{.dtype]}
        \VariableTok{self}\NormalTok{.ctx }\OperatorTok{=}\NormalTok{ nullcontext() }\ControlFlowTok{if} \VariableTok{self}\NormalTok{.device\_type }\OperatorTok{==} \StringTok{\textquotesingle{}cpu\textquotesingle{}} \ControlFlowTok{else}\NormalTok{ torch.amp.autocast(device\_type}\OperatorTok{=}\VariableTok{self}\NormalTok{.device\_type, dtype}\OperatorTok{=}\NormalTok{ptdtype)}
        
        \CommentTok{\# 加载模型检查点文件}
\NormalTok{        checkpoint\_dict }\OperatorTok{=}\NormalTok{ torch.load(}\VariableTok{self}\NormalTok{.checkpoint, map\_location}\OperatorTok{=}\VariableTok{self}\NormalTok{.device)  }\CommentTok{\# 加载模型参数 \# 初始化模型参数}
        \VariableTok{self}\NormalTok{.model }\OperatorTok{=}\NormalTok{ Transformer(ModelConfig(dim}\OperatorTok{=}\DecValTok{1024}\NormalTok{, n\_layers}\OperatorTok{=}\DecValTok{18}\NormalTok{))  }\CommentTok{\# 实例化 Transformer 模型}
\NormalTok{        sunwanted\_prefix }\OperatorTok{=} \StringTok{\textquotesingle{}\_orig\_mod.\textquotesingle{}}
        \ControlFlowTok{for}\NormalTok{ k, v }\KeywordTok{in} \BuiltInTok{list}\NormalTok{(checkpoint\_dict.items()):}
            \ControlFlowTok{if}\NormalTok{ k.startswith(sunwanted\_prefix):}
\NormalTok{                checkpoint\_dict[k[}\BuiltInTok{len}\NormalTok{(sunwanted\_prefix):]] }\OperatorTok{=}\NormalTok{ checkpoint\_dict.pop(k)}
        \VariableTok{self}\NormalTok{.model.load\_state\_dict(checkpoint\_dict, strict}\OperatorTok{=}\VariableTok{False}\NormalTok{)}
        
        \CommentTok{\# 计算模型参数量}
\NormalTok{        num\_params }\OperatorTok{=} \BuiltInTok{sum}\NormalTok{(p.numel() }\ControlFlowTok{for}\NormalTok{ p }\KeywordTok{in} \VariableTok{self}\NormalTok{.model.parameters() }\ControlFlowTok{if}\NormalTok{ p.requires\_grad)}
        \BuiltInTok{print}\NormalTok{(}\SpecialStringTok{f"Model has }\SpecialCharTok{\{}\NormalTok{num\_params }\OperatorTok{/} \FloatTok{1e6}\SpecialCharTok{:.3f\}}\SpecialStringTok{ M parameters."}\NormalTok{)}
        \CommentTok{\# 设置模型为评估模式(evaluation mode),防止训练模式下的 dropout 等操作影响结果}
        \VariableTok{self}\NormalTok{.model.}\BuiltInTok{eval}\NormalTok{()}
        \CommentTok{\# 将模型放置到正确的设备上(GPU 或 CPU)}
        \VariableTok{self}\NormalTok{.model.to(}\VariableTok{self}\NormalTok{.device)}
        \CommentTok{\# 初始化分词器}
        \VariableTok{self}\NormalTok{.tokenizer }\OperatorTok{=}\NormalTok{ AutoTokenizer.from\_pretrained(}\VariableTok{self}\NormalTok{.tokenizer\_model\_path)  }\CommentTok{\# 根据指定的路径加载分词器}

    \KeywordTok{def}\NormalTok{ chat\_template(}\VariableTok{self}\NormalTok{, prompt):}
\NormalTok{        message }\OperatorTok{=}\NormalTok{ [}
\NormalTok{            \{}\StringTok{"role"}\NormalTok{: }\StringTok{"system"}\NormalTok{, }\StringTok{"content"}\NormalTok{: }\StringTok{"你是一个AI助手,你的名字叫小明。"}\NormalTok{\},}
\NormalTok{            \{}\StringTok{"role"}\NormalTok{: }\StringTok{"user"}\NormalTok{, }\StringTok{"content"}\NormalTok{: prompt\}}
\NormalTok{        ]}
        \ControlFlowTok{return} \VariableTok{self}\NormalTok{.tokenizer.apply\_chat\_template(message, tokenize}\OperatorTok{=}\VariableTok{False}\NormalTok{, add\_generation\_prompt}\OperatorTok{=}\VariableTok{True}\NormalTok{)}

    \KeywordTok{def}\NormalTok{ sft\_sample(}\VariableTok{self}\NormalTok{, }
\NormalTok{               start}\OperatorTok{=}\StringTok{"Hello!"}\NormalTok{,  }\CommentTok{\# 生成文本的起始提示词,可以是任意字符串}
\NormalTok{               num\_samples}\OperatorTok{=}\DecValTok{3}\NormalTok{,  }\CommentTok{\# 生成样本的数量,默认生成 3 个样本}
\NormalTok{               max\_new\_tokens}\OperatorTok{=}\DecValTok{256}\NormalTok{,  }\CommentTok{\# 每个样本生成的最大 token 数,默认最多生成 256 个 token}
\NormalTok{               temperature}\OperatorTok{=}\FloatTok{0.7}\NormalTok{,  }\CommentTok{\# 控制生成的随机性,1.0 为标准,值越大越随机}
\NormalTok{               top\_k}\OperatorTok{=}\DecValTok{300}\NormalTok{):  }\CommentTok{\# 保留概率最高的 top\_k 个 token,限制生成时的选择范围}
        \CommentTok{"""}
\CommentTok{        根据给定的起始文本生成样本。}
\CommentTok{        }
\CommentTok{        :param start: 生成文本的起始提示词}
\CommentTok{        :param num\_samples: 要生成的文本样本数}
\CommentTok{        :param max\_new\_tokens: 每个样本生成的最大 token 数}
\CommentTok{        :param temperature: 控制生成的随机性,值越小生成越确定,值越大生成越随机}
\CommentTok{        :param top\_k: 限制生成时选择的 token 范围}
\CommentTok{        :return: 生成的文本样本列表}
\CommentTok{        """}
\NormalTok{        start }\OperatorTok{=} \VariableTok{self}\NormalTok{.chat\_template(start)}
        \CommentTok{\# 将起始文本编码为 token id 序列}
\NormalTok{        start\_ids }\OperatorTok{=} \VariableTok{self}\NormalTok{.tokenizer(start).data[}\StringTok{\textquotesingle{}input\_ids\textquotesingle{}}\NormalTok{]}
        \CommentTok{\# print(\textquotesingle{}start\_ids:\textquotesingle{}, start\_ids)}
\NormalTok{        x }\OperatorTok{=}\NormalTok{ (torch.tensor(start\_ids, dtype}\OperatorTok{=}\NormalTok{torch.}\BuiltInTok{long}\NormalTok{, device}\OperatorTok{=}\VariableTok{self}\NormalTok{.device)[}\VariableTok{None}\NormalTok{, ...])  }\CommentTok{\# 将编码后的 token id 转为 PyTorch 张量}
\NormalTok{        generated\_texts }\OperatorTok{=}\NormalTok{ []  }\CommentTok{\# 用于保存生成的文本样本}
        \ControlFlowTok{with}\NormalTok{ torch.no\_grad():  }\CommentTok{\# 禁用梯度计算,提升效率}
            \ControlFlowTok{with} \VariableTok{self}\NormalTok{.ctx:  }\CommentTok{\# 进入自动混合精度的上下文(如果是 GPU 并使用 float16 时)}
                \ControlFlowTok{for}\NormalTok{ k }\KeywordTok{in} \BuiltInTok{range}\NormalTok{(num\_samples):  }\CommentTok{\# 循环生成指定数量的样本}
\NormalTok{                    y }\OperatorTok{=} \VariableTok{self}\NormalTok{.model.generate(x, }\VariableTok{self}\NormalTok{.tokenizer.eos\_token\_id, max\_new\_tokens, temperature}\OperatorTok{=}\NormalTok{temperature, top\_k}\OperatorTok{=}\NormalTok{top\_k)  }\CommentTok{\# 生成文本}
\NormalTok{                    generated\_texts.append(}\VariableTok{self}\NormalTok{.tokenizer.decode(y[}\DecValTok{0}\NormalTok{].tolist()))  }\CommentTok{\# 解码生成的 token 序列为可读文本}
        \ControlFlowTok{return}\NormalTok{ generated\_texts  }\CommentTok{\# 返回生成的文本样本}


    \KeywordTok{def}\NormalTok{ pretrain\_sample(}\VariableTok{self}\NormalTok{, }
\NormalTok{               start}\OperatorTok{=}\StringTok{"Hello!"}\NormalTok{,  }\CommentTok{\# 生成文本的起始提示词,可以是任意字符串}
\NormalTok{               num\_samples}\OperatorTok{=}\DecValTok{3}\NormalTok{,  }\CommentTok{\# 生成样本的数量,默认生成 3 个样本}
\NormalTok{               max\_new\_tokens}\OperatorTok{=}\DecValTok{256}\NormalTok{,  }\CommentTok{\# 每个样本生成的最大 token 数,默认最多生成 256 个 token}
\NormalTok{               temperature}\OperatorTok{=}\FloatTok{0.7}\NormalTok{,  }\CommentTok{\# 控制生成的随机性,1.0 为标准,值越大越随机}
\NormalTok{               top\_k}\OperatorTok{=}\DecValTok{300}\NormalTok{):  }\CommentTok{\# 保留概率最高的 top\_k 个 token,限制生成时的选择范围}
        \CommentTok{"""}
\CommentTok{        根据给定的起始文本生成样本。}
\CommentTok{        }
\CommentTok{        :param start: 生成文本的起始提示词}
\CommentTok{        :param num\_samples: 要生成的文本样本数}
\CommentTok{        :param max\_new\_tokens: 每个样本生成的最大 token 数}
\CommentTok{        :param temperature: 控制生成的随机性,值越小生成越确定,值越大生成越随机}
\CommentTok{        :param top\_k: 限制生成时选择的 token 范围}
\CommentTok{        :return: 生成的文本样本列表}
\CommentTok{        """}
        \CommentTok{\# 如果 start 是以 \textquotesingle{}FILE:\textquotesingle{} 开头,表示从文件中读取起始文本}
        \ControlFlowTok{if}\NormalTok{ start.startswith(}\StringTok{\textquotesingle{}FILE:\textquotesingle{}}\NormalTok{):}
            \ControlFlowTok{with} \BuiltInTok{open}\NormalTok{(start[}\DecValTok{5}\NormalTok{:], }\StringTok{\textquotesingle{}r\textquotesingle{}}\NormalTok{, encoding}\OperatorTok{=}\StringTok{\textquotesingle{}utf{-}8\textquotesingle{}}\NormalTok{) }\ImportTok{as}\NormalTok{ f:}
\NormalTok{                start }\OperatorTok{=}\NormalTok{ f.read()  }\CommentTok{\# 读取文件内容作为起始文本}
        
        \CommentTok{\# 将起始文本编码为 token id 序列}
\NormalTok{        start\_ids }\OperatorTok{=} \VariableTok{self}\NormalTok{.tokenizer(start).data[}\StringTok{\textquotesingle{}input\_ids\textquotesingle{}}\NormalTok{]}
        \CommentTok{\# print(\textquotesingle{}start\_ids:\textquotesingle{}, start\_ids)}
\NormalTok{        x }\OperatorTok{=}\NormalTok{ (torch.tensor(start\_ids, dtype}\OperatorTok{=}\NormalTok{torch.}\BuiltInTok{long}\NormalTok{, device}\OperatorTok{=}\VariableTok{self}\NormalTok{.device)[}\VariableTok{None}\NormalTok{, ...])  }\CommentTok{\# 将编码后的 token id 转为 PyTorch 张量}
        \CommentTok{\# print(x.shape)}
\NormalTok{        generated\_texts }\OperatorTok{=}\NormalTok{ []  }\CommentTok{\# 用于保存生成的文本样本}
        \ControlFlowTok{with}\NormalTok{ torch.no\_grad():  }\CommentTok{\# 禁用梯度计算,提升效率}
            \ControlFlowTok{with} \VariableTok{self}\NormalTok{.ctx:  }\CommentTok{\# 进入自动混合精度的上下文(如果是 GPU 并使用 float16 时)}
                \ControlFlowTok{for}\NormalTok{ k }\KeywordTok{in} \BuiltInTok{range}\NormalTok{(num\_samples):  }\CommentTok{\# 循环生成指定数量的样本}
\NormalTok{                    y }\OperatorTok{=} \VariableTok{self}\NormalTok{.model.generate(x, max\_new\_tokens}\OperatorTok{=}\NormalTok{max\_new\_tokens, temperature}\OperatorTok{=}\NormalTok{temperature, top\_k}\OperatorTok{=}\NormalTok{top\_k)  }\CommentTok{\# 生成文本}
\NormalTok{                    generated\_texts.append(}\VariableTok{self}\NormalTok{.tokenizer.decode(y[}\DecValTok{0}\NormalTok{].tolist()))  }\CommentTok{\# 解码生成的 token 序列为可读文本}
        
        \ControlFlowTok{return}\NormalTok{ generated\_texts  }\CommentTok{\# 返回生成的文本样本}
    
\ControlFlowTok{if} \VariableTok{\_\_name\_\_} \OperatorTok{==} \StringTok{"\_\_main\_\_"}\NormalTok{:}
    \BuiltInTok{print}\NormalTok{(}\StringTok{"{-}{-}{-}{-}{-}{-}{-}{-}{-}{-}{-}{-}{-}{-}{-}{-}{-}{-}{-} Pretrain Sample {-}{-}{-}{-}{-}{-}{-}{-}{-}{-}{-}{-}{-}{-}{-}{-}{-}{-}{-} }\CharTok{\textbackslash{}n}\StringTok{"}\NormalTok{)}

\NormalTok{    pretrain\_prompt\_datas }\OperatorTok{=}\NormalTok{ [}
        \StringTok{\textquotesingle{}\textless{}|im\_start|\textgreater{}北京大学是\textquotesingle{}}\NormalTok{,}
        \StringTok{\textquotesingle{}\textless{}|im\_start|\textgreater{}中国矿业大学(北京)地球科学与测绘工程学院\textquotesingle{}}\NormalTok{,}
\NormalTok{    ]}

\NormalTok{    generator }\OperatorTok{=}\NormalTok{ TextGenerator(checkpoint}\OperatorTok{=}\StringTok{\textquotesingle{}./base\_model\_215M/pretrain\_1024\_18\_6144.pth\textquotesingle{}}\NormalTok{)  }\CommentTok{\# 初始化生成器}
    \ControlFlowTok{for}\NormalTok{ i }\KeywordTok{in} \BuiltInTok{range}\NormalTok{(}\BuiltInTok{len}\NormalTok{(pretrain\_prompt\_datas)):}
\NormalTok{        samples }\OperatorTok{=}\NormalTok{ generator.pretrain\_sample(start}\OperatorTok{=}\NormalTok{pretrain\_prompt\_datas[i], num\_samples}\OperatorTok{=}\DecValTok{1}\NormalTok{, max\_new\_tokens}\OperatorTok{=}\DecValTok{120}\NormalTok{, temperature}\OperatorTok{=}\FloatTok{0.75}\NormalTok{)}
        \BuiltInTok{print}\NormalTok{(}\SpecialStringTok{f"}\CharTok{\textbackslash{}n}\SpecialStringTok{Sample }\SpecialCharTok{\{}\NormalTok{i}\OperatorTok{+}\DecValTok{1}\SpecialCharTok{\}}\SpecialStringTok{:}\CharTok{\textbackslash{}n}\SpecialCharTok{\{}\NormalTok{pretrain\_prompt\_datas[i]}\SpecialCharTok{\}\{}\NormalTok{samples[}\DecValTok{0}\NormalTok{]}\SpecialCharTok{\}}\CharTok{\textbackslash{}n}\SpecialCharTok{\{}\StringTok{\textquotesingle{}{-}\textquotesingle{}}\OperatorTok{*}\DecValTok{20}\SpecialCharTok{\}}\SpecialStringTok{"}\NormalTok{)  }\CommentTok{\# 打印生成的样本并用分隔线分割}

    \BuiltInTok{print}\NormalTok{(}\StringTok{"}\CharTok{\textbackslash{}n}\StringTok{ {-}{-}{-}{-}{-}{-}{-}{-}{-}{-}{-}{-}{-}{-}{-}{-}{-}{-}{-} SFT Sample {-}{-}{-}{-}{-}{-}{-}{-}{-}{-}{-}{-}{-}{-}{-}{-}{-}{-}{-} }\CharTok{\textbackslash{}n}\StringTok{"}\NormalTok{)}

\NormalTok{    sft\_prompt\_datas }\OperatorTok{=}\NormalTok{ [}
        \StringTok{\textquotesingle{}你好呀\textquotesingle{}}\NormalTok{,}
        \StringTok{"中国的首都是哪里?"}\NormalTok{,}
        \StringTok{"1+12等于多少?"}\NormalTok{,}
        \StringTok{"你是谁?"}
\NormalTok{    ]}
\NormalTok{    generator }\OperatorTok{=}\NormalTok{ TextGenerator(checkpoint}\OperatorTok{=}\StringTok{\textquotesingle{}./sft\_model\_215M/sft\_dim1024\_layers18\_vocab\_size6144.pth\textquotesingle{}}\NormalTok{)  }\CommentTok{\# 初始化生成器}
    \ControlFlowTok{for}\NormalTok{ i }\KeywordTok{in} \BuiltInTok{range}\NormalTok{(}\BuiltInTok{len}\NormalTok{(sft\_prompt\_datas)):}
\NormalTok{        samples }\OperatorTok{=}\NormalTok{ generator.sft\_sample(start}\OperatorTok{=}\NormalTok{sft\_prompt\_datas[i], num\_samples}\OperatorTok{=}\DecValTok{1}\NormalTok{, max\_new\_tokens}\OperatorTok{=}\DecValTok{128}\NormalTok{, temperature}\OperatorTok{=}\FloatTok{0.6}\NormalTok{)}
        \BuiltInTok{print}\NormalTok{(}\SpecialStringTok{f"}\CharTok{\textbackslash{}n}\SpecialStringTok{Sample }\SpecialCharTok{\{}\NormalTok{i}\OperatorTok{+}\DecValTok{1}\SpecialCharTok{\}}\SpecialStringTok{:}\CharTok{\textbackslash{}n}\SpecialStringTok{Question: }\SpecialCharTok{\{}\NormalTok{sft\_prompt\_datas[i]}\SpecialCharTok{\}}\SpecialStringTok{ }\CharTok{\textbackslash{}n}\SpecialStringTok{AI answer: }\SpecialCharTok{\{}\NormalTok{samples[}\DecValTok{0}\NormalTok{]}\SpecialCharTok{\}}\CharTok{\textbackslash{}n}\SpecialCharTok{\{}\StringTok{\textquotesingle{}{-}\textquotesingle{}}\OperatorTok{*}\DecValTok{20}\SpecialCharTok{\}}\SpecialStringTok{"}\NormalTok{)  }\CommentTok{\# 打印生成的样本并用分隔线分割}
\end{Highlighting}
\end{Shaded}

最后我们来看一下模型输出的结果:

\begin{verbatim}
------------------- SFT Sample ------------------- 

Model has 215.127 M parameters.

Sample 1:
Question: 你好呀 
AI answer: 你好!有什么我可以帮你的吗?
--------------------

Sample 2:
Question: 中国的首都是哪里? 
AI answer: 中国的首都是北京。
--------------------

Sample 3:
Question: 1+1等于多少? 
AI answer: 1+1等于2。
--------------------
------------------- Pretrain Sample ------------------- 

Model has 215.127 M parameters.

Sample 1:
<|im_start|>北京大学是中国最早建立的研究型大学之一,是我国最早设置研究生院的高校之一,是第一、二国教育委员会师资培训基地;北京大学是第一、二所国立大学,其校名与北京大学相同。
北京大学录取标准:本科三批1万元,本科一批1万元,本科一批2000元,专科一批2000元,高中起点:非本科一批
--------------------

Sample 2:
<|im_start|>中国矿业大学(北京)地球科学与测绘工程学院副教授黄河流域地质学科带头人古建平教授为大家介绍世界地质变化的概念及工作经验。
古建平教授介绍了最近几年的植物学和地质学的基本概念,尤其是树都黄河、松涛、暗河等都有地质学工作者的身影,其中树都黄河以分布面积最大,是树都黄河中华砂岩公园的主景区。
黄河内蒙古
--------------------
\end{verbatim}

到这里,我们的模型就训绽完成了,恭喜你训练了一个属于你自己的大模型。

\begin{quote}
大家在训练的时候可以将 batch
调的低一些,这样可以减少显存的占用,避免显存不足的问题。当然这样会增加训练时间,可以根据自己的显卡显存大小来调整
batch 的大小。实测 Pretrain batch 为 4 的情况下只需要 7G
显存,训练时长预计 533 小时。作者是在 8卡4090
上进行训练的,预训练一共耗时 46 小时,SFT 阶段在 BelleGroup
350万条中文指令训练 24 小时。
\end{quote}

作者也在魔搭平台上传了本章节训来的模型,如果大家的设备不足以训练大模型,大家也可以在魔搭平台下载模型进行调试和模型体验。模型下载地址如下:

\begin{quote}
\emph{ModelScope
模型下载地址:\href{https://www.modelscope.cn/collections/Happy-LLM-e98b91b10b684a}{🤖
ModelScope}}\\
\emph{ModelScope
创空间体验地址:\href{https://www.modelscope.cn/studios/kmno4zx/happy_llm_215M_sft}{🤖
创空间}}
\end{quote}

\textbf{参考资料}

{[}1{]} Andrej Karpathy. (2023). \emph{llama2.c: Fullstack Llama 2 LLM
solution in pure C}. GitHub repository.
https://github.com/karpathy/llama2.c

{[}2{]} Andrej Karpathy. (2023). \emph{llm.c: GPT-2/GPT-3 pretraining in
C/CUDA}. GitHub repository. https://github.com/karpathy/llm.c

{[}3{]} Hugging Face. (2023). \emph{Tokenizers documentation}.
https://huggingface.co/docs/tokenizers/index

{[}4{]} Skywork Team. (2023). \emph{SkyPile-150B: A large-scale
bilingual dataset}. Hugging Face dataset.
https://huggingface.co/datasets/Skywork/SkyPile-150B

{[}5{]} BelleGroup. (2022). \emph{train\_3.5M\_CN: Chinese dialogue
dataset}. Hugging Face dataset.
https://huggingface.co/datasets/BelleGroup/train\_3.5M\_CN

{[}6{]} Jingyao Gong. (2023). \emph{minimind: Minimalist LLM
implementation}. GitHub repository.
https://github.com/jingyaogong/minimind

{[}7{]} Mobvoi. (2023). \emph{seq-monkey-data: Llama2 training/inference
data}. GitHub repository. https://github.com/mobvoi/seq-monkey-data

\end{document}


\end{document}
