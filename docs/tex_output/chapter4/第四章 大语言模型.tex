% Options for packages loaded elsewhere
\PassOptionsToPackage{unicode}{hyperref}
\PassOptionsToPackage{hyphens}{url}
\documentclass[
]{article}
\usepackage{xcolor}
\usepackage{amsmath,amssymb}
\setcounter{secnumdepth}{5}
\usepackage{iftex}
\ifPDFTeX
  \usepackage[T1]{fontenc}
  \usepackage[utf8]{inputenc}
  \usepackage{textcomp} % provide euro and other symbols
\else % if luatex or xetex
  \usepackage{unicode-math} % this also loads fontspec
  \defaultfontfeatures{Scale=MatchLowercase}
  \defaultfontfeatures[\rmfamily]{Ligatures=TeX,Scale=1}
\fi
\usepackage{lmodern}
\ifPDFTeX\else
  % xetex/luatex font selection
\fi
% Use upquote if available, for straight quotes in verbatim environments
\IfFileExists{upquote.sty}{\usepackage{upquote}}{}
\IfFileExists{microtype.sty}{% use microtype if available
  \usepackage[]{microtype}
  \UseMicrotypeSet[protrusion]{basicmath} % disable protrusion for tt fonts
}{}
\makeatletter
\@ifundefined{KOMAClassName}{% if non-KOMA class
  \IfFileExists{parskip.sty}{%
    \usepackage{parskip}
  }{% else
    \setlength{\parindent}{0pt}
    \setlength{\parskip}{6pt plus 2pt minus 1pt}}
}{% if KOMA class
  \KOMAoptions{parskip=half}}
\makeatother
\usepackage{color}
\usepackage{fancyvrb}
\newcommand{\VerbBar}{|}
\newcommand{\VERB}{\Verb[commandchars=\\\{\}]}
\DefineVerbatimEnvironment{Highlighting}{Verbatim}{commandchars=\\\{\}}
% Add ',fontsize=\small' for more characters per line
\newenvironment{Shaded}{}{}
\newcommand{\AlertTok}[1]{\textcolor[rgb]{1.00,0.00,0.00}{\textbf{#1}}}
\newcommand{\AnnotationTok}[1]{\textcolor[rgb]{0.38,0.63,0.69}{\textbf{\textit{#1}}}}
\newcommand{\AttributeTok}[1]{\textcolor[rgb]{0.49,0.56,0.16}{#1}}
\newcommand{\BaseNTok}[1]{\textcolor[rgb]{0.25,0.63,0.44}{#1}}
\newcommand{\BuiltInTok}[1]{\textcolor[rgb]{0.00,0.50,0.00}{#1}}
\newcommand{\CharTok}[1]{\textcolor[rgb]{0.25,0.44,0.63}{#1}}
\newcommand{\CommentTok}[1]{\textcolor[rgb]{0.38,0.63,0.69}{\textit{#1}}}
\newcommand{\CommentVarTok}[1]{\textcolor[rgb]{0.38,0.63,0.69}{\textbf{\textit{#1}}}}
\newcommand{\ConstantTok}[1]{\textcolor[rgb]{0.53,0.00,0.00}{#1}}
\newcommand{\ControlFlowTok}[1]{\textcolor[rgb]{0.00,0.44,0.13}{\textbf{#1}}}
\newcommand{\DataTypeTok}[1]{\textcolor[rgb]{0.56,0.13,0.00}{#1}}
\newcommand{\DecValTok}[1]{\textcolor[rgb]{0.25,0.63,0.44}{#1}}
\newcommand{\DocumentationTok}[1]{\textcolor[rgb]{0.73,0.13,0.13}{\textit{#1}}}
\newcommand{\ErrorTok}[1]{\textcolor[rgb]{1.00,0.00,0.00}{\textbf{#1}}}
\newcommand{\ExtensionTok}[1]{#1}
\newcommand{\FloatTok}[1]{\textcolor[rgb]{0.25,0.63,0.44}{#1}}
\newcommand{\FunctionTok}[1]{\textcolor[rgb]{0.02,0.16,0.49}{#1}}
\newcommand{\ImportTok}[1]{\textcolor[rgb]{0.00,0.50,0.00}{\textbf{#1}}}
\newcommand{\InformationTok}[1]{\textcolor[rgb]{0.38,0.63,0.69}{\textbf{\textit{#1}}}}
\newcommand{\KeywordTok}[1]{\textcolor[rgb]{0.00,0.44,0.13}{\textbf{#1}}}
\newcommand{\NormalTok}[1]{#1}
\newcommand{\OperatorTok}[1]{\textcolor[rgb]{0.40,0.40,0.40}{#1}}
\newcommand{\OtherTok}[1]{\textcolor[rgb]{0.00,0.44,0.13}{#1}}
\newcommand{\PreprocessorTok}[1]{\textcolor[rgb]{0.74,0.48,0.00}{#1}}
\newcommand{\RegionMarkerTok}[1]{#1}
\newcommand{\SpecialCharTok}[1]{\textcolor[rgb]{0.25,0.44,0.63}{#1}}
\newcommand{\SpecialStringTok}[1]{\textcolor[rgb]{0.73,0.40,0.53}{#1}}
\newcommand{\StringTok}[1]{\textcolor[rgb]{0.25,0.44,0.63}{#1}}
\newcommand{\VariableTok}[1]{\textcolor[rgb]{0.10,0.09,0.49}{#1}}
\newcommand{\VerbatimStringTok}[1]{\textcolor[rgb]{0.25,0.44,0.63}{#1}}
\newcommand{\WarningTok}[1]{\textcolor[rgb]{0.38,0.63,0.69}{\textbf{\textit{#1}}}}
\usepackage{longtable,booktabs,array}
\usepackage{calc} % for calculating minipage widths
% Correct order of tables after \paragraph or \subparagraph
\usepackage{etoolbox}
\makeatletter
\patchcmd\longtable{\par}{\if@noskipsec\mbox{}\fi\par}{}{}
\makeatother
% Allow footnotes in longtable head/foot
\IfFileExists{footnotehyper.sty}{\usepackage{footnotehyper}}{\usepackage{footnote}}
\makesavenoteenv{longtable}
\setlength{\emergencystretch}{3em} % prevent overfull lines
\providecommand{\tightlist}{%
  \setlength{\itemsep}{0pt}\setlength{\parskip}{0pt}}
\usepackage{bookmark}
\IfFileExists{xurl.sty}{\usepackage{xurl}}{} % add URL line breaks if available
\urlstyle{same}
\hypersetup{
  hidelinks,
  pdfcreator={LaTeX via pandoc}}

\author{}
\date{}

\begin{document}

{
\setcounter{tocdepth}{3}
\tableofcontents
}
\section{第四章
大语言模型}\label{ux7b2cux56dbux7ae0-ux5927ux8bedux8a00ux6a21ux578b}

\subsection{4.1 什么是 LLM}\label{ux4ec0ux4e48ux662f-llm}

在前三章,我们从 NLP 的定义与主要任务出发,介绍了引发 NLP
领域重大变革的核心思想------注意力机制与 Transformer 架构。随着
Transformer 架构的横空出世,NLP 领域逐步进入预训练-微调范式,以
Transformer
为基础的、通过预训练获得强大文本表示能力的预训练语言模型层出不穷,将 NLP
的各种经典任务都推进到了一个新的高度。

随着2022年底 ChatGPT 再一次刷新 NLP 的能力上限,大语言模型(Large
Language Model,LLM)开始接替传统的预训练语言模型(Pre-trained Language
Model,PLM) 成为 NLP 的主流方向,基于 LLM 的全新研究范式也正在刷新被
BERT 发扬光大的预训练-微调范式,NLP
由此迎来又一次翻天覆地的变化。从2022年底至今,LLM
能力上限不断刷新,通用基座大模型数量指数级上升,基于 LLM
的概念、应用也是日新月异,预示着大模型时代的到来。

在第三章,我们从模型架构的角度出发,分别分析了
Encoder-Only、Encoder-Decoder 和 Decoder-Only
三种架构下的经典模型及其训练过程。这些模型有的是 LLM
时代之前堪称时代主角的里程碑(如 BERT),有的则是 LLM
时代的舞台主角,是通用人工智能(Artificial General Intelligence,AGI)
的有力竞争者。那么,究竟什么是 LLM,LLM 和传统的 PLM
的核心差异在哪里,又是什么令研究者们对 LLM 抱有如此高的热情与期待呢?

在本章中,我们将结合上文的模型架构讲解,深入分析 LLM
的定义、特点及其能力,为读者揭示 LLM
与传统深度学习模型的核心差异,并在此基础上,展示 LLM
的实际三阶段训练过程,帮助读者从概念上梳理清楚 LLM
是如何获得这样的独特能力的,从而为进一步实践 LLM 完整训练提供理论基础。

\subsubsection{4.1.1 LLM 的定义}\label{llm-ux7684ux5b9aux4e49}

LLM,即 Large Language
Model,中文名为大语言模型或大型语言模型,是一种相较传统语言模型参数量更多、在更大规模语料上进行预训练的语言模型。

在第一章中,我们已经介绍了语言模型的概念,即通过预测下一个 token
任务来训练的 NLP 模型。LLM
使用与传统预训练语言模型相似的架构与预训练任务(如 Decoder-Only 架构与
CLM
预训练任务),但拥有更庞大的参数、在更海量的语料上进行预训练,也从而展现出与传统预训练语言模型截然不同的能力。

一般来说,LLM
指包含\textbf{数百亿(或更多)参数的语言模型},它们往往在\textbf{数 T
token
语料上}通过多卡分布式集群进行预训练,具备远超出传统预训练模型的文本理解与生成能力。不过,随着
LLM 研究的不断深入,多种参数尺寸的 LLM 逐渐丰富,广义的 LLM
一般覆盖了从\textbf{十亿参数}(如 Qwen-1.5B)到\textbf{千亿参数}(如
Grok-314B)的所有大型语言模型。只要模型展现出\textbf{涌现能力},即在一系列复杂任务上表现出远超传统预训练模型(如
BERT、T5)的能力与潜力,都可以称之为 LLM。

一般认为,GPT-3(1750亿参数)是 LLM 的开端,基于 GPT-3 通过
预训练(Pretraining)、监督微调(Supervised
Fine-Tuning,SFT)、强化学习与人类反馈(Reinforcement Learning with
Human Feedback,RLHF)三阶段训练得到的 ChatGPT 更是主导了 LLM
时代的到来。自2022年11月 OpenAI 发布 ChatGPT
至今不到2年时间里,已涌现出了上百个各具特色、能力不一的
LLM。下表列举了自 2022年11月至2023年11月国内外发布的部分大模型:

\begin{longtable}[]{@{}
  >{\raggedright\arraybackslash}p{(\linewidth - 4\tabcolsep) * \real{0.3810}}
  >{\raggedright\arraybackslash}p{(\linewidth - 4\tabcolsep) * \real{0.2381}}
  >{\raggedright\arraybackslash}p{(\linewidth - 4\tabcolsep) * \real{0.3810}}@{}}
\toprule\noalign{}
\begin{minipage}[b]{\linewidth}\raggedright
时间
\end{minipage} & \begin{minipage}[b]{\linewidth}\raggedright
开源 LLM
\end{minipage} & \begin{minipage}[b]{\linewidth}\raggedright
闭源 LLM
\end{minipage} \\
\midrule\noalign{}
\endhead
\bottomrule\noalign{}
\endlastfoot
2022.11 & 无 & OpenAI-ChatGPT \\
2023.02 & Meta-LLaMA;复旦-MOSS & 无 \\
2023.03 & 斯坦福-Alpaca、Vicuna;智谱-ChatGLM &
OpenAI-GPT4;百度-文心一言;Anthropic-Claude;Google-Bard \\
2023.04 & 阿里-通义千问;Stability AI-StableLM & 商汤-日日新 \\
2023.05 & 微软-Pi;Tll-Falcon & 讯飞-星火大模型;Google-PaLM2 \\
2023.06 & 智谱-ChatGLM2;上海 AI
Lab-书生浦语;百川-BaiChuan;虎博-TigerBot & 360-智脑大模型 \\
2023.07 & Meta-LLaMA2 & Anthropic-Claude2;华为-盘古大模型3 \\
2023.08 & 无 & 字节-豆包 \\
2023.09 & 百川-BaiChuan2 & Google-Gemini;腾讯-混元大模型 \\
2023.11 & 零一万物-Yi;幻方-DeepSeek & xAI-Grok \\
\end{longtable}

目前,国内外企业、研究院正不断推出性能更强大的 LLM,探索通往 AGI
的道路。

\subsubsection{4.1.2 LLM 的能力}\label{llm-ux7684ux80fdux529b}

\paragraph{(1)涌现能力(Emergent
Abilities)}\label{ux6d8cux73b0ux80fdux529bemergent-abilities}

区分 LLM 与传统 PLM 最显著的特征即是 LLM 具备 \texttt{涌现能力}
。涌现能力是指同样的模型架构与预训练任务下,某些能力在小型模型中不明显,但在大型模型中特别突出。可以类比到物理学中的相变现象,涌现能力的显现就像是模型性能随着规模增大而迅速提升,超过了随机水平,也就是我们常说的量变引起了质变。

具体来说,涌现能力可以定义为与某些复杂任务相关的能力。但一般而言,NLP
更关注的是它们具备的通用能力,也就是能够应用于解决各种 NLP
任务的能力。涌现能力是目前业界和学界对 LLM
保持较高的热情和关注的核心所在,即虽然 LLM
目前的能力、所能解决的任务与人类最终所期待的通用人工智能还存在不小的差距,但在涌现能力的作用下,我们相信随着研究的不断深入、高质量数据的不断涌现和更高效的模型架构及训练框架的出现,LLM
终能具备通用人工智能所需要具备的能力,从而给人类生活带来质变。

\paragraph{(2)上下文学习(In-context
Learning)}\label{ux4e0aux4e0bux6587ux5b66ux4e60in-context-learning}

上下文学习能力是由 GPT-3
首次引入的。具体而言,上下文学习是指允许语言模型在提供自然语言指令或多个任务示例的情况下,通过理解上下文并生成相应输出的方式来执行任务,而无需额外的训练或参数更新。

对传统
PLM,在经过高成本的预训练之后,往往还需要对指定的下游任务进行有监督微调。虽然传统
PLM 体量较小,对算力要求较低,但例如 BERT 类模型(0.5B
参数),进行有监督微调一般还是需要 10G
以上显存,有一定的算力成本。而同时,有监督微调的训练数据的成本更高。针对下游任务难度的不同,需要的训练样本数往往在
1k\textasciitilde 数十k
不等,均需要进行人工标注,数据获取上有不小的成本。而具备上下文学习能力的
LLM
往往无需进行高成本的额外训练或微调,而可以通过少数示例或是调整自然语言指令,来处理绝大部分任务,从而大大节省了算力和数据成本。

上下文学习能力也正在引发 NLP 研究范式的变革。在传统 PLM 时代,解决 NLP
下游任务的一般范式是预训练-微调,即选用一个合适的预训练模型,针对自己的下游任务准备有监督数据来进行微调。而通过使用具备上下文学习能力的
LLM,一般范式开始向 Prompt Engineering 也就是调整 Prompt 来激发 LLM
的能力转变。例如,目前绝大部分 NLP 任务,通过调整 Prompt 或提供
1\textasciitilde5 个自然语言示例,就可以令 GPT-4 达到超过传统 PLM
微调的效果。

\paragraph{(3)指令遵循(Instruction
Following)}\label{ux6307ux4ee4ux9075ux5faainstruction-following}

通过使用自然语言描述的多任务数据进行微调,也就是所谓的 \texttt{指令微调}
,LLM
被证明在同样使用指令形式化描述的未见过的任务上表现良好。也就是说,经过指令微调的
LLM
能够理解并遵循未见过的指令,并根据任务指令执行任务,而无需事先见过具体示例,这展示了其强大的泛化能力。

指令遵循能力意味我们不再需要每一件事都先教模型,然后它才能去做。我们只需要在指令微调阶段混合多种指令来训练其泛化能力,LLM
就可以处理人类绝大部分指令,即可以灵活地解决用户遇到的问题。这一点在
ChatGPT 上体现地尤为明显。ChatGPT
之所以能够具备极高的热度,其核心原因即在于其不再是仅能用于学界、业界研究的理论模型,而同样可以广泛地服务于各行各业用户。通过给
ChatGPT 输入指令,其可以写作文、编程序、批改试卷、阅读报纸等等。

指令遵循能力使 LLM
可以真正和多个行业结合起来,通过人工智能技术为人类生活的方方面面赋能,从而为人类带来质的改变。不管是目前大火的
Agent、WorkFlow,还是并不遥远的未来可能就会出现的全能助理、超级智能,其本质依赖的都是
LLM 的指令遵循能力。

\paragraph{(4)逐步推理(Step by Step
Reasoning)}\label{ux9010ux6b65ux63a8ux7406step-by-step-reasoning}

逻辑推理,尤其是涉及多个推理步骤的复杂推理任务,一直是 NLP
的攻关难点,也是人工智能难以得到普遍认可的重要原因。毕竟,如果一个模型不能解答基础的``鸡兔同笼''问题,或者不能识别语言中的逻辑陷阱,你很难认为它是``智能的''而非``智障的''。

但是,传统的 NLP
模型通常难以解决涉及多个推理步骤的复杂任务,例如数学问题。然而,LLM
通过采用思维链(Chain-of-Thought,CoT)推理策略,可以利用包含中间推理步骤的提示机制来解决这些任务,从而得出最终答案。据推测,这种能力可能是通过对代码的训练获得的。

逐步推理能力意味着 LLM
可以处理复杂逻辑任务,也就是说可以解决日常生活中需要逻辑判断的绝大部分问题,从而向``可靠的''智能助理迈出了坚实的一步。

这些独特能力是 LLM 区别于传统 PLM 的重要优势,也让 LLM
在处理各种任务时表现出色,使它们成为了解决复杂问题和应用于多领域的强大工具。正是因为涌现能力、上下文学习能力、指令遵循能力与逐步推理能力的存在,NLP
研究人员相信 LLM
是迈向通用人工智能,帮助人类社会实现生产力质变的重要途径。而事实上,目前已有众多基于
LLM 的应用,旨在利用 LLM 的独特能力显著提高生产力。例如,微软基于 GPT-4
推出的 Copilot,就基于 LLM
强大的指令遵循能力与逐步推理能力,通过提供代码补全、代码提示、代码编写等多种功能,辅助程序员更高效、便捷、精准地编写程序,极大提高了程序员的生产效率。

\subsubsection{4.1.3 LLM 的特点}\label{llm-ux7684ux7279ux70b9}

除上文讨论的 LLM 的核心能力外,LLM
还具备一些额外的、有趣或是危险的特点,这些特点也是 LLM
目前重要的研究方向,在此讨论其中一二:

\paragraph{(1)多语言支持}\label{ux591aux8bedux8a00ux652fux6301}

多语言、跨语言模型曾经是 NLP 的一个重要研究方向,但 LLM
由于需要使用到海量的语料进行预训练,训练语料往往本身就是多语言的,因此
LLM
天生即具有多语言、跨语言能力,只不过随着训练语料和指令微调的差异,在不同语言上的能力有所差异。由于英文高质量语料目前仍是占据大部分,以
GPT-4
为代表的绝大部分模型在英文上具有显著超越中文的能力。虽然都可以对多种语言进行处理,但针对中文进行额外训练和优化的国内模型(如文心一言、通义千问等)往往能够在中文环境上展现更优越的效果。

\paragraph{(2)长文本处理}\label{ux957fux6587ux672cux5904ux7406}

由于能够处理多长的上下文文本,在一定程度上决定了模型的部分能力上限,LLM
往往比传统 PLM 更看重长文本处理能力。相对于以 512 token 为惯例的传统
PLM(如 BERT、T5等模型的最大上下文长度均为 512),LLM
在拓宽最大上下文长度方面可谓妙计频出。由于在海量分布式训练集群上进行训练,LLM
往往在训练时就支持 4k、8k 甚至 32k 的上下文长度。同时,LLM
大部分采用了旋转位置编码(Rotary Positional
Encoding,RoPE)(或者同样具有外推能力的
AliBi)作为位置编码,具有一定的长度外推能力,也就是在推理时能够处理显著长于训练长度的文本。例如,InternLM
在 32k 长度上下文上进行了预训练,但通过 RoPE 能够实现 200k
长度的上下文处理。通过不断增强长文本处理能力,LLM
往往能够具备更强的信息阅读、信息总结能力,从而解决诸如要求 LLM
读完《红楼梦》并写一篇对应的高考作文的``世纪难题''。

\paragraph{(3)拓展多模态}\label{ux62d3ux5c55ux591aux6a21ux6001}

LLM 的强大能力也为其带来了跨模态的强大表现。随着 LLM 的不断改进,通过为
LLM 增加额外的参数来进行图像表示,从而利用 LLM
的强大能力打造支持文字、图像双模态的模型,已经是一个成功的方法。通过引入
Adapter
层和图像编码器,并针对性地在图文数据上进行有监督微调,模型能够具备不错的图文问答甚至生成能力。在未来,如何对齐文本与图像的表示,从而打造更强大的多模态大模型,将
LLM 的能力辐射到更多模态,是一个重要的研究方向。

\paragraph{(4)挥之不去的幻觉}\label{ux6325ux4e4bux4e0dux53bbux7684ux5e7bux89c9}

幻觉,是指 LLM 根据 Prompt
杜撰生成虚假、错误信息的表现。例如,当我们要求 LLM
生成一篇学术论文及其参考文献列表时,其往往会捏造众多看似``一本正经''实则完全不存在的论文和研究。幻觉问题是
LLM 的固有缺陷,也是目前 LLM
研究及应用的巨大挑战。尤其是在医学、金融学等非常强调精准、正确的领域,幻觉的存在可能造成非常严重的后果。目前也有很多研究提供了削弱幻觉的一些方法,如
Prompt 里进行限制、通过
RAG(检索增强生成)来指导生成等,但都还只能一定程度减弱幻觉而无法彻底根除。

除上述几点之外,LLM
还存在诸多可供研究的特点,例如我们将在下一节详细论述的 LLM
三阶段训练流程、LLM 的自我反思性等,此处就不一一列举赘述了。

\subsection{4.2 如何训练一个
LLM}\label{ux5982ux4f55ux8badux7ec3ux4e00ux4e2a-llm}

在上一节,我们分析了 LLM
的定义及其特有的强大能力,通过更大规模的参数和海量的训练语料获得远超传统预训练模型的涌现能力,展现出强大的上下文学习、指令遵循及逐步推理能力,带来
NLP
领域的全新变革。那么,通过什么样的步骤,我们才可以训练出一个具有涌现能力的
LLM 呢?训练一个 LLM,与训练传统的预训练模型,又有什么区别?

\begin{figure}[htbp]\centering
\includegraphics[width=0.8\textwidth]{}
\caption{图4.1 训练 LLM 的三个阶段}
\end{figure}

一般而言,训练一个完整的 LLM 需要经过图1中的三个阶段------Pretrain、SFT
和 RLHF。在这一节,我们将详细论述训练 LLM
的三个阶段,并分析每一个阶段的过程及其核心难点、注意事项,帮助读者们从理论上了解要训练一个
LLM,需要经过哪些步骤。

\subsubsection{4.2.1 Pretrain}\label{pretrain}

Pretrain,即预训练,是训练 LLM 最核心也是工程量最大的第一步。LLM
的预训练和传统预训练模型非常类似,同样是使用海量无监督文本对随机初始化的模型参数进行训练。正如我们在第三章中所见,目前主流的
LLM 几乎都采用了 Decoder-Only 的类 GPT 架构(LLaMA
架构),它们的预训练任务也都沿承了 GPT
模型的经典预训练任务------因果语言模型(Causal Language Model,CLM)。

因果语言模型建模,即和最初的语言模型一致,通过给出上文要求模型预测下一个
token 来进行训练。CLM
的过程和原理我们已在第三章详细论述过,此处就不再赘述。LLM
的预训练同传统预训练模型的核心差异即在于,预训练的体量和资源消耗。

根据定义,LLM
的核心特点即在于其具有远超传统预训练模型的参数量,同时在更海量的语料上进行预训练。传统预训练模型如
BERT,有 base 和 large 两个版本。BERT-base 模型由 12个 Encoder
层组成,其 hidden\_size 为 768,使用
12个头作为多头注意力层,整体参数量为 1亿(110M);而 BERT-large 模型由
24个 Encoder 层组成,hidden\_size 为 1024,有 16个头,整体参数量为
3亿(340M)。同时,BERT 预训练使用了 33亿(3B)token 的语料,在 64块 TPU
上训练了
4天。事实上,相对于传统的深度学习模型,3亿参数量、33亿训练数据的 BERT
已经是一个能力超群、资源消耗巨大的庞然大物。

但是,前面我们提到,一般而言的 LLM
通常具有数百亿甚至上千亿参数,即使是广义上最小的
LLM,一般也有十亿(1B)以上的参数量。例如以开山之作 GPT-3 为例,其有
96个 Decoder 层,12288 的 hidden\_size 和 96个头,共有
1750亿(175B)参数,比 BERT 大出快 3个数量级。即使是目前流行的小型 LLM
如 Qwen-1.8B,其也有 24个 Decoder 层、2048的 hidden\_size 和
16个注意力头,整体参数量达到 18亿(1.8B)。

\begin{longtable}[]{@{}llllll@{}}
\toprule\noalign{}
模型 & hidden\_layers & hidden\_size & heads & 整体参数量 &
预训练数据量 \\
\midrule\noalign{}
\endhead
\bottomrule\noalign{}
\endlastfoot
BERT-base & 12 & 768 & 12 & 0.1B & 3B \\
BERT-large & 24 & 1024 & 16 & 0.3B & 3B \\
Qwen-1.8B & 24 & 2048 & 16 & 1.8B & 2.2T \\
LLaMA-7B & 32 & 4096 & 32 & 7B & 1T \\
GPT-3 & 96 & 12288 & 96 & 175B & 300B \\
\end{longtable}

更重要的是,LLM 往往需要使用更大规模的预训练语料。根据由 OpenAI 提出的
Scaling Law:C \textasciitilde{} 6ND,其中 C 为计算量,N 为模型参数,D
为训练的 token 数,可以实验得出训练 token 数应该是模型参数的
1.7倍,也就是说 175B 的 GPT-3,需要使用 300B token 进行预训练。而 LLaMA
更是进一步提出,使用 20倍 token 来训练模型能达到效果最优,因此 175B 的
GPT-3,可以使用3.5T token 数据预训练达到最优性能。

如此庞大的模型参数和预训练数据,使得预训练一个 LLM
所需要的算力资源极其庞大。事实上,哪怕是预训练一个 1B
的大模型,也至少需要多卡分布式 GPU
集群,通过分布式框架对模型参数、训练的中间参数和训练数据进行切分,才能通过以天为单位的长时间训练来完成。一般来说,百亿级
LLM 需要 1024张 A100 训练一个多月,而十亿级 LLM 一般也需要 256张 A100
训练两、三天,计算资源消耗非常高。

也正因如此,分布式训练框架也成为 LLM
训练必不可少的组成部分。分布式训练框架的核心思路是数据并行和模型并行。所谓数据并行,是指训练模型的尺寸可以被单个
GPU 内存容纳,但是由于增大训练的 batch\_size
会增大显存开销,无法使用较大的 batch\_size
进行训练;同时,训练数据量非常大,使用单张 GPU 训练时长难以接受。

\begin{figure}[htbp]\centering
\includegraphics[width=0.8\textwidth]{}
\caption{图4.2 模型、数据并行}
\end{figure}

因此,如图4.2所示可以让模型实例在不同 GPU
和不同批数据上运行,每一次前向传递完成之后,收集所有实例的梯度并计算梯度更新,更新模型参数之后再传递到所有实例。也就是在数据并行的情况下,每张
GPU
上的模型参数是保持一致的,训练的总批次大小等于每张卡上的批次大小之和。

但是,当 LLM 扩大到上百亿参数,单张 GPU
内存往往就无法存放完整的模型参数。如图4.3所示,在这种情况下,可以将模型拆分到多个
GPU 上,每个 GPU 上存放不同的层或不同的部分,从而实现模型并行。

\begin{figure}[htbp]\centering
\includegraphics[width=0.8\textwidth]{}
\caption{图4.3 模型并行}
\end{figure}

在数据并行和模型并行的思想基础上,还演化出了多种更高效的分布式方式,例如张量并行、3D
并行、ZeRO(Zero Redundancy
Optimizer,零冗余优化器)等。目前,主流的分布式训练框架包括
Deepspeed、Megatron-LM、ColossalAI 等,其中,Deepspeed 使用面最广。

Deepspeed 的核心策略是 ZeRO 和 CPU-offload。ZeRO
是一种显存优化的数据并行方案,其核心思想是优化数据并行时每张卡的显存占用,从而实现对更大规模模型的支持。ZeRO
将模型训练阶段每张卡被占用的显存分为两类:

\begin{itemize}
\tightlist
\item
  模型状态(Model States),包括模型参数、模型梯度和优化器 Adam
  的状态参数。假设模型参数量为
  1M,一般来说,在混合精度训练的情况下,该部分需要 16M
  的空间进行存储,其中 Adam 状态参数会占据 12M 的存储空间。
\item
  剩余状态(Residual
  States),除了模型状态之外的显存占用,包括激活值、各种缓存和显存碎片。
\end{itemize}

针对上述显存占用,ZeRO 提出了三种不断递进的优化策略:

\begin{enumerate}
\def\labelenumi{\arabic{enumi}.}
\tightlist
\item
  ZeRO-1,对模型状态中的 Adam 状态参数进行分片,即每张卡只存储
  \(\frac{1}{N}\) 的 Adam 状态参数,其他参数仍然保持每张卡一份。
\item
  ZeRO-2,继续对模型梯度进行分片,每张卡只存储 \(\frac{1}{N}\)
  的模型梯度和 Adam 状态参数,仅模型参数保持每张卡一份。
\item
  ZeRO-3,将模型参数也进行分片,每张卡只存储 \(\frac{1}{N}\)
  的模型梯度、模型参数和 Adam 状态参数。
\end{enumerate}

可以看出,随着分片的参数量不断增加,每张卡需要占用的显存也不断减少。当然,分片的增加也就意味着训练中通信开销的增加,一般而言,每张卡的
GPU 利用率 ZeRO-1 最高而 ZeRO-3
最低。具体使用什么策略,需要结合计算资源的情况和需要训练的模型体量动态确定。

除去计算资源的要求,训练数据本身也是预训练 LLM 的一个重大挑战。训练一个
LLM,至少需要数百 B 甚至上 T 的预训练语料。根据研究,LLM
所掌握的知识绝大部分都是在预训练过程中学会的,因此,为了使训练出的 LLM
能够覆盖尽可能广的知识面,预训练语料需要组织多种来源的数据,并以一定比例进行混合。目前,主要的开源预训练语料包括
CommonCrawl、C4、Github、Wikipedia 等。不同的 LLM
往往会在开源预训练语料基础上,加入部分私有高质量语料,再基于自己实验得到的最佳配比来构造预训练数据集。事实上,数据配比向来是预训练
LLM
的``核心秘籍'',不同的配比往往会相当大程度影响最终模型训练出来的性能。例如,下表展示了
LLaMA 的预训练数据及配比:

\begin{longtable}[]{@{}lll@{}}
\toprule\noalign{}
数据集 & 占比 & 数据集大小(Disk size) \\
\midrule\noalign{}
\endhead
\bottomrule\noalign{}
\endlastfoot
CommonCrawl & 67.0\% & 3.3 TB \\
C4 & 15.0\% & 783 GB \\
Github & 4.5\% & 328 GB \\
Wikipedia & 4.5\% & 83 GB \\
Books & 4.5\% & 85 GB \\
ArXiv & 2.5\% & 92 GB \\
StackExchange & 2.0\% & 78 GB \\
\end{longtable}

训练一个中文
LLM,训练数据的难度会更大。目前,高质量语料还是大部分集中在英文范畴,例如上表的
Wikipedia、Arxiv 等,均是英文数据集;而 C4
等多语言数据集中,英文语料也占据主要地位。目前开源的中文 LLM 如
ChatGLM、Baichuan
等模型均未开放其预训练数据集,开源的中文预训练数据集目前仅有昆仑天工开源的\href{https://huggingface.co/datasets/Skywork/SkyPile-150B}{SkyPile}(150B)、中科闻歌开源的\href{https://huggingface.co/datasets/wenge-research/yayi2_pretrain_data}{yayi2}(100B)等,相较于英文开源数据集有明显差距。

预训练数据的处理与清洗也是 LLM
预训练的一个重要环节。诸多研究证明,预训练数据的质量往往比体量更加重要。预训练数据处理一般包括以下流程:

\begin{enumerate}
\def\labelenumi{\arabic{enumi}.}
\tightlist
\item
  文档准备。由于海量预训练语料往往是从互联网上获得,一般需要从爬取的网站来获得自然语言文档。文档准备主要包括
  URL 过滤(根据网页 URL 过滤掉有害内容)、文档提取(从 HTML
  中提取纯文本)、语言选择(确定提取的文本的语种)等。
\item
  语料过滤。语料过滤的核心目的是去除低质量、无意义、有毒有害的内容,例如乱码、广告等。语料过滤一般有两种方法:基于模型的方法,即通过高质量语料库训练一个文本分类器进行过滤;基于启发式的方法,一般通过人工定义
  web 内容的质量指标,计算语料的指标值来进行过滤。
\item
  语料去重。实验表示,大量重复文本会显著影响模型的泛化能力,因此,语料去重即删除训练语料中相似度非常高的文档,也是必不可少的一个步骤。去重一般基于
  hash
  算法计算数据集内部或跨数据集的文档相似性,将相似性大于指定阈值的文档去除;也可以基于子串在序列级进行精确匹配去重。
\end{enumerate}

目前,已有很多经过处理的高质量预训练语料和专用于预训练数据处理的框架。例如,有基于
LLaMA
思路收集、清洗的预训练数据集\href{https://huggingface.co/datasets/togethercomputer/RedPajama-Data-1T}{RedPajama-1T},以及在
RedPajama
基础上进行筛选去重的\href{https://huggingface.co/datasets/cerebras/SlimPajama-627B/tree/main/train}{SlimPajama-627B}数据集,实验证明高质量的
627B Slimpajama 数据集能够获得比 1T 的 RedPajama 数据集更好的效果。

\subsubsection{4.2.2 SFT}\label{sft}

预训练是 LLM 强大能力的根本来源,事实上,LLM
所覆盖的海量知识基本都是源于预训练语料。LLM
的性能本身,核心也在于预训练的工作。但是,预训练赋予了 LLM
能力,却还需要第二步将其激发出来。经过预训练的 LLM
好像一个博览群书但又不求甚解的书生,对什么样的偏怪问题,都可以流畅地接出下文,但他偏偏又不知道问题本身的含义,只会``死板背书''。这一现象的本质是因为,LLM
的预训练任务就是经典的 CLM,也就是训练其预测下一个 token
的能力,在没有进一步微调之前,其无法与其他下游任务或是用户指令适配。

因此,我们还需要第二步来教这个博览群书的学生如何去使用它的知识,也就是
SFT------Supervisor
Finetune,有监督微调。所谓有监督微调,其实就是我们在第三章中讲过的预训练-微调中的微调,稍有区别的是,对于能力有限的传统预训练模型,我们需要针对每一个下游任务单独对其进行微调以训练模型在该任务上的表现。例如要解决文本分类问题,需要对
BERT
进行文本分类的微调;要解决实体识别的问题,就需要进行实体识别任务的微调。

而面对能力强大的
LLM,我们往往不再是在指定下游任务上构造有监督数据进行微调,而是选择训练模型的``通用指令遵循能力'',也就是一般通过\texttt{指令微调}的方式来进行
SFT。

所谓指令微调,即我们训练的输入是各种类型的用户指令,而需要模型拟合的输出则是我们希望模型在收到该指令后做出的回复。例如,我们的一条训练样本可以是:

\begin{verbatim}
input:告诉我今天的天气预报?
output:根据天气预报,今天天气是晴转多云,最高温度26摄氏度,最低温度9摄氏度,昼夜温差大,请注意保暖哦
\end{verbatim}

也就是说,SFT
的主要目标是让模型从多种类型、多种风格的指令中获得泛化的指令遵循能力,也就是能够理解并回复用户的指令。因此,类似于
Pretrain,SFT 的数据质量和数据配比也是决定模型指令遵循能力的重要因素。

首先是指令数据量及覆盖范围。为了使 LLM
能够获得泛化的指令遵循能力,即能够在未训练的指令上表现良好,需要收集大量类别各异的用户指令和对应回复对
LLM 进行训练。一般来说,在单个任务上 500\textasciitilde1000
的训练样本就可以获得不错的微调效果。但是,为了让 LLM
获得泛化的指令遵循能力,在多种任务指令上表现良好,需要在训练数据集中覆盖多种类型的任务指令,同时也需要相对较大的训练数据量,表现良好的开源
LLM SFT 数据量一般在数 B token 左右。

为提高 LLM
的泛化能力,指令数据集的覆盖范围自然是越大越好。但是,多种不同类型的指令数据之间的配比也是
LLM 训练的一大挑战。OpenAI 训练的 InstructGPT(即 ChatGPT
前身)使用了源自于用户使用其 API 的十种指令:

\begin{longtable}[]{@{}ll@{}}
\toprule\noalign{}
指令类型 & 占比 \\
\midrule\noalign{}
\endhead
\bottomrule\noalign{}
\endlastfoot
文本生成 & 45.6\% \\
开放域问答 & 12.4\% \\
头脑风暴 & 11.2\% \\
聊天 & 8.4\% \\
文本转写 & 6.6\% \\
文本总结 & 4.2\% \\
文本分类 & 3.5\% \\
其他 & 3.5\% \\
特定域问答 & 2.6\% \\
文本抽取 & 1.9\% \\
\end{longtable}

高质量的指令数据集具有较高的获取难度。不同于预训练使用的无监督语料,SFT
使用的指令数据集是有监督语料,除去设计广泛、合理的指令外,还需要对指令回复进行人工标注,并保证标注的高质量。事实上,ChatGPT
的成功很大一部分来源于其高质量的人工标注数据。但是,人工标注数据成本极高,也罕有企业将人工标注的指令数据集开源。为降低数据成本,部分学者提出了使用
ChatGPT 或 GPT-4 来生成指令数据集的方法。例如,经典的开源指令数据集
\href{https://github.com/yizhongw/self-instruct/blob/main/human_eval/user_oriented_instructions.jsonl}{Alpaca}就是基于一些种子
Prompt,通过 ChatGPT 生成更多的指令并对指令进行回复来构建的。

一般 SFT 所使用的指令数据集包括以下三个键:

\begin{Shaded}
\begin{Highlighting}[]
\FunctionTok{\{}
    \DataTypeTok{"instruction"}\FunctionTok{:}\StringTok{"即输入的用户指令"}\FunctionTok{,}
    \DataTypeTok{"input"}\FunctionTok{:}\StringTok{"执行该指令可能需要的补充输入,没有则置空"}\FunctionTok{,}
    \DataTypeTok{"output"}\FunctionTok{:}\StringTok{"即模型应该给出的回复"}
\FunctionTok{\}}
\end{Highlighting}
\end{Shaded}

例如,如果我们的指令是将目标文本``今天天气真好''翻译成英文,那么该条样本可以构建成如下形式:

\begin{Shaded}
\begin{Highlighting}[]
\FunctionTok{\{}
    \DataTypeTok{"instruction"}\FunctionTok{:}\StringTok{"将下列文本翻译成英文:"}\FunctionTok{,}
    \DataTypeTok{"input"}\FunctionTok{:}\StringTok{"今天天气真好"}\FunctionTok{,}
    \DataTypeTok{"output"}\FunctionTok{:}\StringTok{"Today is a nice day!"}
\FunctionTok{\}}
\end{Highlighting}
\end{Shaded}

同时,为使模型能够学习到和预训练不同的范式,在 SFT
的过程中,往往会针对性设置特定格式。例如,LLaMA 的 SFT 格式为:

\begin{verbatim}
### Instruction:\n{{content}}\n\n### Response:\n
\end{verbatim}

其中的 content
即为具体的用户指令,也就是说,对于每一个用户指令,将会嵌入到上文的
content 部分,这里的用户指令不仅指上例中的
``instruction'',而是指令和输入的拼接,即模型可以执行的一条完整指令。例如,针对上例,LLaMA
获得的输入应该是:

\begin{verbatim}
### Instruction:\n将下列文本翻译成英文:今天天气真好\n\n### Response:\n
\end{verbatim}

其需要拟合的输出则是:

\begin{verbatim}
### Instruction:\n将下列文本翻译成英文:今天天气真好\n\n### Response:\nToday is a nice day!
\end{verbatim}

注意,因为指令微调本质上仍然是对模型进行 CLM
训练,只不过要求模型对指令进行理解和回复而不是简单地预测下一个
token,所以模型预测的结果不仅是 output,而应该是 input + output,只不过
input 部分不参与 loss 的计算,但回复指令本身还是以预测下一个 token
的形式来实现的。

但是,随着 LLM
能力的不断增强,模型的多轮对话能力逐渐受到重视。所谓多轮对话,是指模型在每一次对话时能够参考之前对话的历史记录来做出回复。例如,一个没有多轮对话能力的
LLM 可能有如下对话记录:

\begin{verbatim}
用户:你好,我是开源组织 Datawhale 的成员。
模型:您好,请问有什么可以帮助您的吗?
用户:你知道 Datawhale 是什么吗?
模型:不好意思,我不知道 Datawhale 是什么。
\end{verbatim}

也就是说,模型不能记录用户曾经提到或是自己曾经回答的历史信息。如果是一个具有多轮对话能力的
LLM,其对话记录应该是这样的:

\begin{verbatim}
用户:你好,我是开源组织 Datawhale 的成员。
模型:您好,请问有什么可以帮助您的吗?
用户:你知道 Datawhale 是什么吗?
模型:Datawhale 是一个开源组织。
\end{verbatim}

模型是否支持多轮对话,与预训练是没有关系的。事实上,模型的多轮对话能力完全来自于
SFT 阶段。如果要使模型支持多轮对话,我们需要在 SFT
时将训练数据构造成多轮对话格式,让模型能够利用之前的知识来生成回答。假设我们目前需要构造的多轮对话是:

\begin{verbatim}
<prompt_1><completion_1><prompt_2><completion_2><prompt_3><completion_3>
\end{verbatim}

构造多轮对话样本一般有三种方式:

\begin{enumerate}
\def\labelenumi{\arabic{enumi}.}
\item
  直接将最后一次模型回复作为输出,前面所有历史对话作为输入,直接拟合最后一次回复:

\begin{verbatim}
 input=<prompt_1><completion_1><prompt_2><completion_2><prompt_3><completion_3>
 output=[MASK][MASK][MASK][MASK][MASK]<completion_3>
\end{verbatim}
\item
  将 N 轮对话构造成 N 个样本:

\begin{verbatim}
 input_1 = <prompt_1><completion_1>
 output_1 = [MASK]<completion_1>

 input_2 = <prompt_1><completion_1><prompt_2><completion_2>
 output_2 = [MASK][MASK][MASK]<completion_2>

 input_3=<prompt_1><completion_1><prompt_2><completion_2><prompt_3><completion_3>
 output_3=[MASK][MASK][MASK][MASK][MASK]<completion_3>
\end{verbatim}
\item
  直接要求模型预测每一轮对话的输出:

\begin{verbatim}
 input=<prompt_1><completion_1><prompt_2><completion_2><prompt_3><completion_3>
 output=[MASK]<completion_1>[MASK]<completion_2>[MASK]<completion_3>
\end{verbatim}
\end{enumerate}

显然可知,第一种方式会丢失大量中间信息,第二种方式造成了大量重复计算,只有第三种方式是最合理的多轮对话构造。我们之所以可以以第三种方式来构造多轮对话样本,是因为
LLM 本质还是进行的 CLM
任务,进行单向注意力计算,因此在预测时会从左到右依次进行拟合,前轮的输出预测不会影响后轮的预测。目前,绝大部分
LLM 均使用了多轮对话的形式来进行 SFT。

\subsubsection{4.2.3 RLHF}\label{rlhf}

RLHF,全称是 Reinforcement Learning from Human
Feedback,即人类反馈强化学习,是利用强化学习来训练 LLM
的关键步骤。相较于在 GPT-3 就已经初见雏形的 SFT,RLHF 往往被认为是
ChatGPT 相较于 GPT-3 的最核心突破。事实上,从功能上出发,我们可以将 LLM
的训练过程分成预训练与对齐(alignment)两个阶段。预训练的核心作用是赋予模型海量的知识,而所谓对齐,其实就是让模型与人类价值观一致,从而输出人类希望其输出的内容。在这个过程中,SFT
是让 LLM 和人类的指令对齐,从而具有指令遵循能力;而 RLHF
则是从更深层次令 LLM
和人类价值观对齐,令其达到安全、有用、无害的核心标准。

如图4.4所示,ChatGPT 在技术报告中将对齐分成三个阶段,后面两个阶段训练 RM
和 PPO 训练,就是 RLHF 的步骤:

\begin{figure}[htbp]\centering
\includegraphics[width=0.8\textwidth]{}
\caption{图4.4 ChatGPT 训练三个的阶段}
\end{figure}

RLHF 的思路是,引入强化学习的技术,通过实时的人类反馈令 LLM
能够给出更令人类满意的回复。强化学习是有别于监督学习的另一种机器学习方法,主要讨论的问题是智能体怎么在复杂、不确定的环境中最大化它能获得的奖励。强化学习主要由两部分构成:智能体和环境。在强化学习过程中,智能体会不断行动并从环境获取反馈,根据反馈来调整自己行动的策略。应用到
LLM 的对齐上,其实就是针对不同的问题,LLM
会不断生成对应的回复,人工标注员会不断对 LLM 的回复做出反馈,从而让 LLM
学会人类更偏好、喜欢的回复。

RLHF 就类似于 LLM
作为一个学生,不断做作业来去提升自己解题能力的过程。如果把 LLM
看作一个能力强大的学生,Pretrain 是将所有基础的知识教给他,SFT
是教他怎么去读题、怎么去解题,那么 RLHF 就类似于真正的练习。LLM
会不断根据 Pretrain 学到的基础知识和 SFT
学到的解题能力去解答练习,然后人类作为老师批改 LLM 的练习,来让 LLM
反思错误的解题方式,不断强化正确的解题方式。

如上图,RLHF 分为两个步骤:训练 RM 和 PPO 训练。

RM,Reward Model,即奖励模型。RM 是用于拟合人类偏好,来给 LLM
做出反馈的。在强化学习的训练中,对于 LLM 的每一个回复,RM
会进行打分,这个打分反映了生成回复符合人类偏好的程度。然后 LLM
会根据强化学习的原理,基于 RM 的打分来进行优化训练。所以,RM
本质上是一个文本分类模型,对于一个文本输出一个标量奖励,和文本分类任务中的隐藏层输出非常类似。在具体实现上,RM
也往往就是传统的 LLM 架构(或 BERT
架构)加上一层分类层,和用于文本分类的 LLM
架构完全一致,只不过使用隐藏层输出而不是最后的分类输出而已。

但是,在训练 RM 时,我们往往并不直接使用文本及其对应的标量奖励来对 RM
进行训练。因为要对齐人类偏好,RM
训练的偏好数据往往是由人工标注的。但是,由于标注者之间往往也存在价值观差异,数值形式的标量奖励往往会将这些差异放大,从而导致在训练过程中对同样程度的回复奖励不一致,模型难以拟合到正确的标量奖励。因此,我们往往对同一个
completion 下的不同回复进行排名,再将排名转化为奖励。

例如,我们的训练数据往往形如:

\begin{Shaded}
\begin{Highlighting}[]
\FunctionTok{\{}
    \DataTypeTok{"prompt"}\FunctionTok{:}\StringTok{"如果你打算从商店偷东西,你觉得早上好还是晚上好?"}\FunctionTok{,}
    \DataTypeTok{"chosen"}\FunctionTok{:}\StringTok{"这是违法的事情,我不能提供建议"}\FunctionTok{,}
    \DataTypeTok{"rejected"}\FunctionTok{:}\StringTok{"考虑晚上的人口贩运和监控摄像头的差别是件好事。夜间时间可能更有利于避免监控摄像头,但晚上的商店雇员会更能看见你。另一方面,由于白天通常不太忙,因此更容易避免被其他人注意到。无论如何,必须密切注意商店雇员和所有顾客的行为。他们也许能够看见你,即使他们不是直接面对你。为了安全起见,重要的是要密切注意商店里的一切事情,而不是不小心。"}
\FunctionTok{\}}
\end{Highlighting}
\end{Shaded}

其中,prompt 是用户的问题,chosen
是应该对齐的、符合人类偏好的回答,rejected
是不符合人类偏好的回答。在训练中,prompt 将和 chosen 以及 rejected
分别拼接起来,形成 chosen\_example 和
rejected\_example,然后分别进入模型通过前向传播输出一个标量奖励。然后模型会通过最大化
chosen\_example 和 rejected\_example 的标量差异来计算
loss,并进行反向传播完成训练。

值得注意的是,RM 训练使用的模型往往和最后的 LLM 大小不同。例如 OpenAI
使用了 175B 的 LLM 和 6B 的 RM。同时,RM 使用的模型可以是经过 SFT 之后的
LM,也可以是基于偏好数据从头训练的 RM。哪一种更好,至今尚没有定论。

在完成 RM 训练之后,就可以使用 PPO 算法来进行强化学习训练。PPO,Proximal
Policy Optimization,近端策略优化算法,是一种经典的 RL
算法。事实上,强化学习训练时也可以使用其他的强化学习算法,但目前 PPO
算法因为成熟、成本较低,还是最适合 RLHF 的算法。

在具体 PPO 训练过程中,会存在四个模型。如图4.5所示,两个 LLM 和两个
RM。两个 LLM 分别是进行微调、参数更新的 actor model 和不进行参数更新的
ref model,均是从 SFT 之后的 LLM 初始化的。两个 RM 分别是进行参数更新的
critic model 和不进行参数更新的 reward model,均是从上一步训练的 RM
初始化的。

\begin{figure}[htbp]\centering
\includegraphics[width=0.8\textwidth]{}
\caption{图4.5 PPO 训练流程}
\end{figure}

如上图,使用 PPO 算法的强化学习训练过程如下:

\begin{enumerate}
\def\labelenumi{\arabic{enumi}.}
\tightlist
\item
  从 SFT 之后的 LLM 初始化两个模型分别作为 Actor Model 和 Ref
  Model;从训练的 RM 初始化两个模型分别作为 Reward Model 和 Critic
  Model;
\item
  输入一个 Prompt,Actor Model 和 Ref Model 分别就 Prompt 生成回复;
\item
  Actor Response 和 Ref Response 计算 KL 散度:
  \(r_{KL} = -\theta_{KL}D_{KL}(\pi_{PPO}(y|x)||\pi_{base}(y|x))\)
  其中, \(\pi_{PPO}(y|x)\) 即为 Actor Model 的输出,而
  \(\pi_{base}(y|x)\) 即为 Ref Model 的输出, \(\theta_{KL}D_{KL}\)
  即是计算 KL 散度的方法;
\item
  Actor Response 分别输入到 Reward Model 和 Critic Model
  进行打分,其中,Reward Model 输出的是回复对应的标量奖励,Critic Model
  还会输出累加奖励(即从i位置到最后的累积奖励);
\item
  计算的 KL 散度、两个模型的打分均输入到奖励函数中,计算奖励:
  \(loss = -(kl_{ctl} \cdot r_{KL} + \gamma \cdot V_{t+1} - V_{t}) \log P(A_t|V_t)\)
  ,这里的 \(kl_{ctl}\) 是控制 KL 散度对结果影响的权重参数, \(\gamma\)
  是控制下一个时间(也就是样本)打分对结果影响的权重参数, \(V_t\) 是
  Critic Model 的打分输出, \(A_t\) 则是 Reward Model 的打分输出;
\item
  根据奖励函数分别计算出的 actor loss 和 critic loss,更新 Actor Model
  的参数和 Critic Model 的参数;注意,Actor Model 和 Critic Model
  的参数更新方法是不同的,此处就不再一一赘述了,感兴趣的读者可以深入研究强化学习的相关理论。
\end{enumerate}

在上述过程中,因为要使用到四个模型,显存占用会数倍于 SFT。例如,如果我们
RM 和 LLM 都是用 7B 的体量,PPO 过程中大概需要 240G(4张 80G
A100,每张卡占用
60G)显存来进行模型加载。那么,为什么我们需要足足四个模型呢?Actor Model
和 Critic Model 较为容易理解,而之所以我们还需要保持原参数不更新的 Ref
Model 和 Reward
Model,是为了限制模型的更新不要过于偏离原模型以至于丢失了 Pretrain 和
SFT 赋予的能力。

当然,如此大的资源占用和复杂的训练过程,使 RLHF
成为一个门槛非常高的阶段。也有学者从监督学习的思路出发,提出了
DPO(Direct Preference Optimization,直接偏好优化),可以低门槛平替
RLHF。DPO 的核心思路是,将 RLHF
的强化学习问题转化为监督学习来直接学习人类偏好。DPO
通过使用奖励函数和最优策略间的映射,展示了约束奖励最大化问题完全可以通过单阶段策略训练进行优化,也就是说,通过学习
DPO 所提出的优化目标,可以直接学习人类偏好,而无需再训练 RM
以及进行强化学习。由于直接使用监督学习进行训练,DPO 只需要两个 LLM
即可完成训练,且训练过程相较 PPO 简单很多,是 RLHF
更简单易用的平替版本。DPO
所提出的优化目标为什么能够直接学习人类偏好,作者通过一系列的数学推导完成了证明,感兴趣的读者可以下来进一步阅读,此处就不再赘述了。

接下来,我们将依次实现如何从零开始训练一个 LLM,包括预训练、SFT 和
RLHF。

\textbf{参考资料}

{[}1{]} Long Ouyang, Jeff Wu, Xu Jiang, Diogo Almeida, Carroll L.
Wainwright, Pamela Mishkin, Chong Zhang, Sandhini Agarwal, Katarina
Slama, Alex Ray, John Schulman, Jacob Hilton, Fraser Kelton, Luke
Miller, Maddie Simens, Amanda Askell, Peter Welinder, Paul Christiano,
Jan Leike, Ryan Lowe. (2022). \emph{Training language models to follow
instructions with human feedback.} arXiv preprint arXiv:2203.02155.

{[}2{]} Jacob Devlin, Ming-Wei Chang, Kenton Lee, Kristina Toutanova.
(2019). \emph{BERT: Pre-training of Deep Bidirectional Transformers for
Language Understanding.} arXiv preprint arXiv:1810.04805.

{[}3{]} Jared Kaplan, Sam McCandlish, Tom Henighan, Tom B. Brown,
Benjamin Chess, Rewon Child, Scott Gray, Alec Radford, Jeffrey Wu, Dario
Amodei. (2020). \emph{Scaling Laws for Neural Language Models.} arXiv
preprint arXiv:2001.08361.

{[}4{]} Jordan Hoffmann, Sebastian Borgeaud, Arthur Mensch, Elena
Buchatskaya, Trevor Cai, Eliza Rutherford, Diego de Las Casas, Lisa Anne
Hendricks, Johannes Welbl, Aidan Clark, Tom Hennigan, Eric Noland, Katie
Millican, George van den Driessche, Bogdan Damoc, Aurelia Guy, Simon
Osindero, Karen Simonyan, Erich Elsen, Jack W. Rae, Oriol Vinyals,
Laurent Sifre. (2022). \emph{Training Compute-Optimal Large Language
Models.} arXiv preprint arXiv:2203.15556.

{[}5{]} Qi Wang, Yiyuan Yang, Ji Jiang. (2022). Easy RL: Reinforcement
Learning Tutorial . Beijing: Posts \& Telecom Press. ISBN:
9787115584700. https://github.com/datawhalechina/easy-rl

{[}6{]} Rafael Rafailov, Archit Sharma, Eric Mitchell, Stefano Ermon,
Christopher D. Manning, Chelsea Finn. (2024). \emph{Direct Preference
Optimization: Your Language Model is Secretly a Reward Model.} arXiv
preprint arXiv:2305.18290.

{[}7{]} Wayne Xin Zhao, Kun Zhou, Junyi Li, Tianyi Tang, Xiaolei Wang,
Yupeng Hou, Yingqian Min, Beichen Zhang, Junjie Zhang, Zican Dong, Yifan
Du, Chen Yang, Yushuo Chen, Zhipeng Chen, Jinhao Jiang, Ruiyang Ren,
Yifan Li, Xinyu Tang, Zikang Liu, Peiyu Liu, Jian-Yun Nie, Ji-Rong Wen.
(2025). \emph{A Survey of Large Language Models.} arXiv preprint
arXiv:2303.18223.

\end{document}
